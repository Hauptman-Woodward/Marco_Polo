%% Generated by Sphinx.
\def\sphinxdocclass{report}
\documentclass[letterpaper,10pt,english]{sphinxmanual}
\ifdefined\pdfpxdimen
   \let\sphinxpxdimen\pdfpxdimen\else\newdimen\sphinxpxdimen
\fi \sphinxpxdimen=.75bp\relax

\PassOptionsToPackage{warn}{textcomp}
\usepackage[utf8]{inputenc}
\ifdefined\DeclareUnicodeCharacter
% support both utf8 and utf8x syntaxes
  \ifdefined\DeclareUnicodeCharacterAsOptional
    \def\sphinxDUC#1{\DeclareUnicodeCharacter{"#1}}
  \else
    \let\sphinxDUC\DeclareUnicodeCharacter
  \fi
  \sphinxDUC{00A0}{\nobreakspace}
  \sphinxDUC{2500}{\sphinxunichar{2500}}
  \sphinxDUC{2502}{\sphinxunichar{2502}}
  \sphinxDUC{2514}{\sphinxunichar{2514}}
  \sphinxDUC{251C}{\sphinxunichar{251C}}
  \sphinxDUC{2572}{\textbackslash}
\fi
\usepackage{cmap}
\usepackage[T1]{fontenc}
\usepackage{amsmath,amssymb,amstext}
\usepackage{babel}



\usepackage{times}
\expandafter\ifx\csname T@LGR\endcsname\relax
\else
% LGR was declared as font encoding
  \substitutefont{LGR}{\rmdefault}{cmr}
  \substitutefont{LGR}{\sfdefault}{cmss}
  \substitutefont{LGR}{\ttdefault}{cmtt}
\fi
\expandafter\ifx\csname T@X2\endcsname\relax
  \expandafter\ifx\csname T@T2A\endcsname\relax
  \else
  % T2A was declared as font encoding
    \substitutefont{T2A}{\rmdefault}{cmr}
    \substitutefont{T2A}{\sfdefault}{cmss}
    \substitutefont{T2A}{\ttdefault}{cmtt}
  \fi
\else
% X2 was declared as font encoding
  \substitutefont{X2}{\rmdefault}{cmr}
  \substitutefont{X2}{\sfdefault}{cmss}
  \substitutefont{X2}{\ttdefault}{cmtt}
\fi


\usepackage[Bjarne]{fncychap}
\usepackage{sphinx}

\fvset{fontsize=\small}
\usepackage{geometry}


% Include hyperref last.
\usepackage{hyperref}
% Fix anchor placement for figures with captions.
\usepackage{hypcap}% it must be loaded after hyperref.
% Set up styles of URL: it should be placed after hyperref.
\urlstyle{same}

\addto\captionsenglish{\renewcommand{\contentsname}{Contents:}}

\usepackage{sphinxmessages}
\setcounter{tocdepth}{0}



\title{Polo Documentation}
\date{Jun 15, 2020}
\release{0.0.1}
\author{Ethan T. Holleman}
\newcommand{\sphinxlogo}{\vbox{}}
\renewcommand{\releasename}{Release}
\makeindex
\begin{document}

\pagestyle{empty}
\sphinxmaketitle
\pagestyle{plain}
\sphinxtableofcontents
\pagestyle{normal}
\phantomsection\label{\detokenize{index::doc}}



\chapter{Polo Quickstart Guide}
\label{\detokenize{Quickstart:polo-quickstart-guide}}\label{\detokenize{Quickstart::doc}}

\section{Windows}
\label{\detokenize{Quickstart:windows}}
Do this for Windows machines


\section{Linux}
\label{\detokenize{Quickstart:linux}}
Do this for linux (Ubuntu)


\chapter{FAQs}
\label{\detokenize{FAQS:faqs}}\label{\detokenize{FAQS::doc}}
A place to include frequently asked questions


\chapter{User’s Guide}
\label{\detokenize{user_guide:user-s-guide}}\label{\detokenize{user_guide::doc}}
If you have yet to install Polo on your machine, head to the installation
guide here.


\section{Importing and Opening Image Data}
\label{\detokenize{user_guide:importing-and-opening-image-data}}
The first step to using Polo is adding your own data. Polo organizes data into “runs”, which
consist of a set of related screening images. Polo organizes runs into three different
categories which are described below.
\begin{itemize}
\item {} 
HWI Screening Runs

\item {} 
Non\sphinxhyphen{}HWI Screening Runs

\item {} 
Raw Image Collections

\end{itemize}


\subsection{Getting Your Data (via FTP)}
\label{\detokenize{user_guide:getting-your-data-via-ftp}}
If you do not already have your data downloaded to your local machine, Polo
includes an FTP file browser which utilizes Python’s ftplib package. For
HWI users this allows you to download your screening images from the
HWI server without leaving the application.

To open the FTP file browser, on the menubar navigate to Import \sphinxhyphen{}\textgreater{} Images
\sphinxhyphen{}\textgreater{} From FTP. Enter your credentials in the new window.Once you are
connected to a server files available to download will be listed in the browser menu.
Select the checkboxes next to the files you wish to download, or select
all files in a directory at once. Pressing “Download Selected Files”
will start the download.


\subsection{Using the Run Importer}
\label{\detokenize{user_guide:using-the-run-importer}}
To import a new run into Polo for classification head to the menubar and
click Import \sphinxhyphen{}\textgreater{} Images \sphinxhyphen{}\textgreater{} New Run. This will open a new window that will
look similar to the one below.

The selection box under “Import Type” lists the three run types you can
import. Only one type may be imported at a time. Fill in the fields listed
for the run type you wish to import and hit “submit run” to import your
image data into Polo. Once you select a directory Polo will fill in fields
automatically based on any metadata it finds. This is done most extensively
for HWI Screening Runs and unless you know what you are doing it is
recommended to use the Polo suggested settings.


\subsection{Importing a Saved Run}
\label{\detokenize{user_guide:importing-a-saved-run}}
One of the advantages of Polo is the introduction of the .xtal file format. Xtal
is a json file that can store all your run data, including images, classification and
other metadata in a single, portable format. If you have an xtal file from a previously
saved run you can load it into Polo by navigating to the menubar and selecting
Import \sphinxhyphen{}\textgreater{} Images \sphinxhyphen{}\textgreater{} From Saved Run. A file browser will open and you can select
your .xtal file and pick up where you left off.


\subsection{Opening a Run}
\label{\detokenize{user_guide:opening-a-run}}
Once a run has been successfully imported, the run name will appear in the
loaded runs list, like in the image below.

If it is a new run double clicking on the run name will run the MARCO model
on the run’s images. You can check on MARCO’s progress by using the
Classification Progress bar located just below the Loaded Runs list.

Once a run has been classified you can load it into the current view
by double clicking on it again.


\section{Using the Slideshow View}
\label{\detokenize{user_guide:using-the-slideshow-view}}
The slideshow view is the main Polo user interface. It allows you to view
your screening images, label them and filter them by MARCO classifications,
your own classifications or both.


\subsection{Navigation}
\label{\detokenize{user_guide:navigation}}
Once you have images loaded in you can cycle through images by
pressing the next or previous image buttons under the navigation
panel. You can also use the right or left arrowkeys respectively.

If you are viewing an HWI Screening Run you will be able to toogle
between ordering images by index (well numbe) or by cocktail number. Also
for HWI runs, you can navigate directly to an image by well number by using
the By Well Number selector also in the Navigation panel.

It is also posssible to move between runs that may be of the same protein
sample but imaged at a different time or with a different optical
technology. See the Advanced Settings section for more details on
these features.


\subsection{Classification}
\label{\detokenize{user_guide:classification}}
Arguebly the most important Polo feature is the ability to easily
label your screening images. To label the currently displayed image
press the button in the classification panel with your desired label.
You can classify images as Crystal, Precipitate, Clear or Other.
See this link for more details on each classifiction. To increase
your speed you can classify images using keyboard shortcuts which are
listed below.
\begin{itemize}
\item {} 
1: Crystal

\item {} 
2: Precipitate

\item {} 
3: Clear

\item {} 
4: Other

\end{itemize}

Classifiying an image will automatically move you to the next image in
the slideshow.


\subsection{Filtering}
\label{\detokenize{user_guide:filtering}}
Using the checkboxes under the Image Filter’s panel in the lower
right corner of the window will allow you to filter the kinds of
images in your current slideshow. For example if you only wanted to
see images that MARCO has classified as Crystal you could check the Crystal box
under Image Types and MARCO under the Classifier panel. If you had checked Human
instead only images that you have classified as Crystal would be shown.

You can reset the slideshow to include all images by selecting all checkboxes
or no boxes and pressing submit filters.


\subsection{Image Metadata}
\label{\detokenize{user_guide:image-metadata}}
Image metadata will be displayed in the Image Details and Cocktail Details
windows when it is available. Image details will give you basic information about
the image currently being displayed, such as well number, imaging technology
imaging date and current classifications. If you are viewing an HWI run
the chemical conditions the current image was plated in will be displayed in the
Cocktail Details window.


\section{Using the Table View}
\label{\detokenize{user_guide:using-the-table-view}}

\section{Using Plot Functions}
\label{\detokenize{user_guide:using-plot-functions}}

\section{Editing Run Data After Loading}
\label{\detokenize{user_guide:editing-run-data-after-loading}}

\subsection{Editing Run Date}
\label{\detokenize{user_guide:editing-run-date}}

\subsection{Editing Run Name}
\label{\detokenize{user_guide:editing-run-name}}

\subsection{Adding Annotations}
\label{\detokenize{user_guide:adding-annotations}}

\subsection{Deleting a Run}
\label{\detokenize{user_guide:deleting-a-run}}

\section{Saving a Run}
\label{\detokenize{user_guide:saving-a-run}}

\subsection{.xtal File Format}
\label{\detokenize{user_guide:xtal-file-format}}

\section{Exporting a Run}
\label{\detokenize{user_guide:exporting-a-run}}

\subsection{HTML and PDF Reports}
\label{\detokenize{user_guide:html-and-pdf-reports}}

\subsection{CSV Exports}
\label{\detokenize{user_guide:csv-exports}}

\chapter{polo package}
\label{\detokenize{polo:polo-package}}\label{\detokenize{polo::doc}}

\section{Subpackages}
\label{\detokenize{polo:subpackages}}

\subsection{polo.crystallography package}
\label{\detokenize{polo.crystallography:polo-crystallography-package}}\label{\detokenize{polo.crystallography::doc}}

\subsubsection{Submodules}
\label{\detokenize{polo.crystallography:submodules}}

\subsubsection{polo.crystallography.broke module}
\label{\detokenize{polo.crystallography:module-polo.crystallography.broke}}\label{\detokenize{polo.crystallography:polo-crystallography-broke-module}}\index{module@\spxentry{module}!polo.crystallography.broke@\spxentry{polo.crystallography.broke}}\index{polo.crystallography.broke@\spxentry{polo.crystallography.broke}!module@\spxentry{module}}

\subsubsection{polo.crystallography.cocktail module}
\label{\detokenize{polo.crystallography:module-polo.crystallography.cocktail}}\label{\detokenize{polo.crystallography:polo-crystallography-cocktail-module}}\index{module@\spxentry{module}!polo.crystallography.cocktail@\spxentry{polo.crystallography.cocktail}}\index{polo.crystallography.cocktail@\spxentry{polo.crystallography.cocktail}!module@\spxentry{module}}\index{Cocktail (class in polo.crystallography.cocktail)@\spxentry{Cocktail}\spxextra{class in polo.crystallography.cocktail}}

\begin{fulllineitems}
\phantomsection\label{\detokenize{polo.crystallography:polo.crystallography.cocktail.Cocktail}}\pysiglinewithargsret{\sphinxbfcode{\sphinxupquote{class }}\sphinxcode{\sphinxupquote{polo.crystallography.cocktail.}}\sphinxbfcode{\sphinxupquote{Cocktail}}}{\emph{\DUrole{n}{number}\DUrole{o}{=}\DUrole{default_value}{None}}, \emph{\DUrole{n}{well\_assignment}\DUrole{o}{=}\DUrole{default_value}{None}}, \emph{\DUrole{n}{commercial\_code}\DUrole{o}{=}\DUrole{default_value}{None}}, \emph{\DUrole{n}{pH}\DUrole{o}{=}\DUrole{default_value}{None}}, \emph{\DUrole{n}{reagents}\DUrole{o}{=}\DUrole{default_value}{None}}}{}
Bases: \sphinxcode{\sphinxupquote{object}}
\index{add\_reagent() (polo.crystallography.cocktail.Cocktail method)@\spxentry{add\_reagent()}\spxextra{polo.crystallography.cocktail.Cocktail method}}

\begin{fulllineitems}
\phantomsection\label{\detokenize{polo.crystallography:polo.crystallography.cocktail.Cocktail.add_reagent}}\pysiglinewithargsret{\sphinxbfcode{\sphinxupquote{add\_reagent}}}{\emph{\DUrole{n}{new\_reagent}}}{}
Adds a reagent to the existing list of reagents stored in the
reagents attribute.
\begin{quote}\begin{description}
\item[{Parameters}] \leavevmode
\sphinxstyleliteralstrong{\sphinxupquote{new\_reagent}} ({\hyperref[\detokenize{polo.crystallography:polo.crystallography.cocktail.Reagent}]{\sphinxcrossref{\sphinxstyleliteralemphasis{\sphinxupquote{Reagent}}}}}) \textendash{} Reagent to add to this cocktail

\end{description}\end{quote}

\end{fulllineitems}

\index{cocktail\_index() (polo.crystallography.cocktail.Cocktail property)@\spxentry{cocktail\_index()}\spxextra{polo.crystallography.cocktail.Cocktail property}}

\begin{fulllineitems}
\phantomsection\label{\detokenize{polo.crystallography:polo.crystallography.cocktail.Cocktail.cocktail_index}}\pysigline{\sphinxbfcode{\sphinxupquote{property }}\sphinxbfcode{\sphinxupquote{cocktail\_index}}}
Attempt to pull out the cocktail number from the cocktail number
string. Dependent on consistent formating between cocktail menus that
I have not currently varrified.
\begin{quote}\begin{description}
\item[{Returns}] \leavevmode
cocktail number

\item[{Return type}] \leavevmode
int

\end{description}\end{quote}

\end{fulllineitems}

\index{well\_assignment() (polo.crystallography.cocktail.Cocktail property)@\spxentry{well\_assignment()}\spxextra{polo.crystallography.cocktail.Cocktail property}}

\begin{fulllineitems}
\phantomsection\label{\detokenize{polo.crystallography:polo.crystallography.cocktail.Cocktail.well_assignment}}\pysigline{\sphinxbfcode{\sphinxupquote{property }}\sphinxbfcode{\sphinxupquote{well\_assignment}}}
Return the current well assignment for this cocktail
\begin{quote}\begin{description}
\item[{Returns}] \leavevmode
well assignment

\item[{Return type}] \leavevmode
int

\end{description}\end{quote}

\end{fulllineitems}


\end{fulllineitems}

\index{Reagent (class in polo.crystallography.cocktail)@\spxentry{Reagent}\spxextra{class in polo.crystallography.cocktail}}

\begin{fulllineitems}
\phantomsection\label{\detokenize{polo.crystallography:polo.crystallography.cocktail.Reagent}}\pysiglinewithargsret{\sphinxbfcode{\sphinxupquote{class }}\sphinxcode{\sphinxupquote{polo.crystallography.cocktail.}}\sphinxbfcode{\sphinxupquote{Reagent}}}{\emph{\DUrole{n}{chemical\_additive}\DUrole{o}{=}\DUrole{default_value}{None}}, \emph{\DUrole{n}{concentration}\DUrole{o}{=}\DUrole{default_value}{None}}, \emph{\DUrole{n}{chemical\_formula}\DUrole{o}{=}\DUrole{default_value}{None}}, \emph{\DUrole{n}{stock\_con}\DUrole{o}{=}\DUrole{default_value}{None}}}{}
Bases: \sphinxcode{\sphinxupquote{object}}
\index{chemical\_formula() (polo.crystallography.cocktail.Reagent property)@\spxentry{chemical\_formula()}\spxextra{polo.crystallography.cocktail.Reagent property}}

\begin{fulllineitems}
\phantomsection\label{\detokenize{polo.crystallography:polo.crystallography.cocktail.Reagent.chemical_formula}}\pysigline{\sphinxbfcode{\sphinxupquote{property }}\sphinxbfcode{\sphinxupquote{chemical\_formula}}}
Return the current chemical formula for this Reagent. It is not
certain that a reagent will have a chemical formula.
\begin{quote}\begin{description}
\item[{Returns}] \leavevmode
Chemical formula

\item[{Return type}] \leavevmode
molmass.Formula

\end{description}\end{quote}

\end{fulllineitems}

\index{concentration() (polo.crystallography.cocktail.Reagent property)@\spxentry{concentration()}\spxextra{polo.crystallography.cocktail.Reagent property}}

\begin{fulllineitems}
\phantomsection\label{\detokenize{polo.crystallography:polo.crystallography.cocktail.Reagent.concentration}}\pysigline{\sphinxbfcode{\sphinxupquote{property }}\sphinxbfcode{\sphinxupquote{concentration}}}
Return the current concentration for this reagent. Concentration
utimately refers back to a condition in a specific screening well.
\begin{quote}\begin{description}
\item[{Returns}] \leavevmode
Chemical concentration

\item[{Return type}] \leavevmode
{\hyperref[\detokenize{polo.crystallography:polo.crystallography.cocktail.SignedValue}]{\sphinxcrossref{SignedValue}}}

\end{description}\end{quote}

\end{fulllineitems}

\index{molar\_mass() (polo.crystallography.cocktail.Reagent property)@\spxentry{molar\_mass()}\spxextra{polo.crystallography.cocktail.Reagent property}}

\begin{fulllineitems}
\phantomsection\label{\detokenize{polo.crystallography:polo.crystallography.cocktail.Reagent.molar_mass}}\pysigline{\sphinxbfcode{\sphinxupquote{property }}\sphinxbfcode{\sphinxupquote{molar\_mass}}}
Attempt to calculate the molar mass of this reagent. Closely related
to the molarity property. See its docstring for why this is not always
possible. Return False if cannot be calculated.
\begin{quote}\begin{description}
\item[{Returns}] \leavevmode
Molarity or False

\item[{Return type}] \leavevmode
{\hyperref[\detokenize{polo.crystallography:polo.crystallography.cocktail.SignedValue}]{\sphinxcrossref{SignedValue}}} or False

\end{description}\end{quote}

\end{fulllineitems}

\index{molarity() (polo.crystallography.cocktail.Reagent property)@\spxentry{molarity()}\spxextra{polo.crystallography.cocktail.Reagent property}}

\begin{fulllineitems}
\phantomsection\label{\detokenize{polo.crystallography:polo.crystallography.cocktail.Reagent.molarity}}\pysigline{\sphinxbfcode{\sphinxupquote{property }}\sphinxbfcode{\sphinxupquote{molarity}}}
Attempt to calculate the molarity of this reagent at its current
concentration. This calculation is not certain to return a molarity
because HWI cocktail menu files use a variety of units to describe
chemical concentrations, including \%w/v or \%v/v. Currently, Polo is
not able to convert \%v/v units to molarity as it would require knowing
both the molar mass of the reagent and its density. If the reagent
concentration cannot be converted to mols / liter then returns false.
\begin{quote}\begin{description}
\item[{Returns}] \leavevmode
molarity or False

\item[{Return type}] \leavevmode
{\hyperref[\detokenize{polo.crystallography:polo.crystallography.cocktail.SignedValue}]{\sphinxcrossref{SignedValue}}} or Bool

\end{description}\end{quote}

\end{fulllineitems}

\index{peg\_parser() (polo.crystallography.cocktail.Reagent method)@\spxentry{peg\_parser()}\spxextra{polo.crystallography.cocktail.Reagent method}}

\begin{fulllineitems}
\phantomsection\label{\detokenize{polo.crystallography:polo.crystallography.cocktail.Reagent.peg_parser}}\pysiglinewithargsret{\sphinxbfcode{\sphinxupquote{peg\_parser}}}{\emph{\DUrole{n}{peg\_string}}}{}
Attempts to pull out a molar mass from a chemical name since the
molar mass is often included in the name of PEG species. A string is
considered to be a potential PEG species if it contains ‘PEG’ or
‘Polyethylene glycol’ in it.
\begin{quote}\begin{description}
\item[{Parameters}] \leavevmode
\sphinxstyleliteralstrong{\sphinxupquote{peg\_string}} (\sphinxstyleliteralemphasis{\sphinxupquote{str}}) \textendash{} String to look for PEG species in

\item[{Returns}] \leavevmode
molar mass if found to be valid PEG species, False otherwise.

\item[{Return type}] \leavevmode
float or Bool

\end{description}\end{quote}

\end{fulllineitems}

\index{stock\_volume() (polo.crystallography.cocktail.Reagent method)@\spxentry{stock\_volume()}\spxextra{polo.crystallography.cocktail.Reagent method}}

\begin{fulllineitems}
\phantomsection\label{\detokenize{polo.crystallography:polo.crystallography.cocktail.Reagent.stock_volume}}\pysiglinewithargsret{\sphinxbfcode{\sphinxupquote{stock\_volume}}}{\emph{\DUrole{n}{target\_volume}}}{}
Attempt to calculate the required amount of stock solution to
produce the reagent’s set concentration in the given target\_volume.
Stock concentration is taken from the stock\_con attribute. If
stock\_con is not set or the molarity of the reagent can not be
calculated this method will return False.
\begin{quote}\begin{description}
\item[{Parameters}] \leavevmode
\sphinxstyleliteralstrong{\sphinxupquote{target\_volume}} ({\hyperref[\detokenize{polo.crystallography:polo.crystallography.cocktail.SignedValue}]{\sphinxcrossref{\sphinxstyleliteralemphasis{\sphinxupquote{SignedValue}}}}}) \textendash{} Volume in which stock will be diluted into

\item[{Returns}] \leavevmode
Volume of stock or False

\item[{Return type}] \leavevmode
{\hyperref[\detokenize{polo.crystallography:polo.crystallography.cocktail.SignedValue}]{\sphinxcrossref{SignedValue}}} or False

\end{description}\end{quote}

\end{fulllineitems}

\index{units (polo.crystallography.cocktail.Reagent attribute)@\spxentry{units}\spxextra{polo.crystallography.cocktail.Reagent attribute}}

\begin{fulllineitems}
\phantomsection\label{\detokenize{polo.crystallography:polo.crystallography.cocktail.Reagent.units}}\pysigline{\sphinxbfcode{\sphinxupquote{units}}\sphinxbfcode{\sphinxupquote{ = {[}\textquotesingle{}M\textquotesingle{}, \textquotesingle{}(w/v)\textquotesingle{}, \textquotesingle{}(v/v)\textquotesingle{}{]}}}}
\end{fulllineitems}


\end{fulllineitems}

\index{SignedValue (class in polo.crystallography.cocktail)@\spxentry{SignedValue}\spxextra{class in polo.crystallography.cocktail}}

\begin{fulllineitems}
\phantomsection\label{\detokenize{polo.crystallography:polo.crystallography.cocktail.SignedValue}}\pysiglinewithargsret{\sphinxbfcode{\sphinxupquote{class }}\sphinxcode{\sphinxupquote{polo.crystallography.cocktail.}}\sphinxbfcode{\sphinxupquote{SignedValue}}}{\emph{\DUrole{n}{value}\DUrole{o}{=}\DUrole{default_value}{None}}, \emph{\DUrole{n}{units}\DUrole{o}{=}\DUrole{default_value}{None}}}{}
Bases: \sphinxcode{\sphinxupquote{object}}
\index{make\_from\_string() (polo.crystallography.cocktail.SignedValue class method)@\spxentry{make\_from\_string()}\spxextra{polo.crystallography.cocktail.SignedValue class method}}

\begin{fulllineitems}
\phantomsection\label{\detokenize{polo.crystallography:polo.crystallography.cocktail.SignedValue.make_from_string}}\pysiglinewithargsret{\sphinxbfcode{\sphinxupquote{classmethod }}\sphinxbfcode{\sphinxupquote{make\_from\_string}}}{\emph{\DUrole{n}{string}}}{}
\end{fulllineitems}

\index{micro() (polo.crystallography.cocktail.SignedValue property)@\spxentry{micro()}\spxextra{polo.crystallography.cocktail.SignedValue property}}

\begin{fulllineitems}
\phantomsection\label{\detokenize{polo.crystallography:polo.crystallography.cocktail.SignedValue.micro}}\pysigline{\sphinxbfcode{\sphinxupquote{property }}\sphinxbfcode{\sphinxupquote{micro}}}
\end{fulllineitems}

\index{milli() (polo.crystallography.cocktail.SignedValue property)@\spxentry{milli()}\spxextra{polo.crystallography.cocktail.SignedValue property}}

\begin{fulllineitems}
\phantomsection\label{\detokenize{polo.crystallography:polo.crystallography.cocktail.SignedValue.milli}}\pysigline{\sphinxbfcode{\sphinxupquote{property }}\sphinxbfcode{\sphinxupquote{milli}}}
\end{fulllineitems}

\index{nano() (polo.crystallography.cocktail.SignedValue property)@\spxentry{nano()}\spxextra{polo.crystallography.cocktail.SignedValue property}}

\begin{fulllineitems}
\phantomsection\label{\detokenize{polo.crystallography:polo.crystallography.cocktail.SignedValue.nano}}\pysigline{\sphinxbfcode{\sphinxupquote{property }}\sphinxbfcode{\sphinxupquote{nano}}}
\end{fulllineitems}

\index{set\_from\_string() (polo.crystallography.cocktail.SignedValue method)@\spxentry{set\_from\_string()}\spxextra{polo.crystallography.cocktail.SignedValue method}}

\begin{fulllineitems}
\phantomsection\label{\detokenize{polo.crystallography:polo.crystallography.cocktail.SignedValue.set_from_string}}\pysiglinewithargsret{\sphinxbfcode{\sphinxupquote{set\_from\_string}}}{\emph{\DUrole{n}{string}}}{}
\end{fulllineitems}

\index{supported\_units (polo.crystallography.cocktail.SignedValue attribute)@\spxentry{supported\_units}\spxextra{polo.crystallography.cocktail.SignedValue attribute}}

\begin{fulllineitems}
\phantomsection\label{\detokenize{polo.crystallography:polo.crystallography.cocktail.SignedValue.supported_units}}\pysigline{\sphinxbfcode{\sphinxupquote{supported\_units}}\sphinxbfcode{\sphinxupquote{ = \{\textquotesingle{}L\textquotesingle{}, \textquotesingle{}M\textquotesingle{}, \textquotesingle{}X\textquotesingle{}, \textquotesingle{}v/v\textquotesingle{}, \textquotesingle{}w/v\textquotesingle{}\}}}}
\end{fulllineitems}

\index{units() (polo.crystallography.cocktail.SignedValue property)@\spxentry{units()}\spxextra{polo.crystallography.cocktail.SignedValue property}}

\begin{fulllineitems}
\phantomsection\label{\detokenize{polo.crystallography:polo.crystallography.cocktail.SignedValue.units}}\pysigline{\sphinxbfcode{\sphinxupquote{property }}\sphinxbfcode{\sphinxupquote{units}}}
\end{fulllineitems}

\index{value() (polo.crystallography.cocktail.SignedValue property)@\spxentry{value()}\spxextra{polo.crystallography.cocktail.SignedValue property}}

\begin{fulllineitems}
\phantomsection\label{\detokenize{polo.crystallography:polo.crystallography.cocktail.SignedValue.value}}\pysigline{\sphinxbfcode{\sphinxupquote{property }}\sphinxbfcode{\sphinxupquote{value}}}
\end{fulllineitems}


\end{fulllineitems}



\subsubsection{polo.crystallography.image module}
\label{\detokenize{polo.crystallography:module-polo.crystallography.image}}\label{\detokenize{polo.crystallography:polo-crystallography-image-module}}\index{module@\spxentry{module}!polo.crystallography.image@\spxentry{polo.crystallography.image}}\index{polo.crystallography.image@\spxentry{polo.crystallography.image}!module@\spxentry{module}}\index{Image (class in polo.crystallography.image)@\spxentry{Image}\spxextra{class in polo.crystallography.image}}

\begin{fulllineitems}
\phantomsection\label{\detokenize{polo.crystallography:polo.crystallography.image.Image}}\pysiglinewithargsret{\sphinxbfcode{\sphinxupquote{class }}\sphinxcode{\sphinxupquote{polo.crystallography.image.}}\sphinxbfcode{\sphinxupquote{Image}}}{\emph{\DUrole{n}{path}\DUrole{o}{=}\DUrole{default_value}{None}}, \emph{\DUrole{n}{bites}\DUrole{o}{=}\DUrole{default_value}{None}}, \emph{\DUrole{n}{well\_number}\DUrole{o}{=}\DUrole{default_value}{None}}, \emph{\DUrole{n}{human\_class}\DUrole{o}{=}\DUrole{default_value}{None}}, \emph{\DUrole{n}{machine\_class}\DUrole{o}{=}\DUrole{default_value}{None}}, \emph{\DUrole{n}{prediction\_dict}\DUrole{o}{=}\DUrole{default_value}{None}}, \emph{\DUrole{n}{plate\_id}\DUrole{o}{=}\DUrole{default_value}{None}}, \emph{\DUrole{n}{date}\DUrole{o}{=}\DUrole{default_value}{None}}, \emph{\DUrole{n}{cocktail}\DUrole{o}{=}\DUrole{default_value}{None}}, \emph{\DUrole{n}{spectrum}\DUrole{o}{=}\DUrole{default_value}{None}}, \emph{\DUrole{n}{previous\_image}\DUrole{o}{=}\DUrole{default_value}{None}}, \emph{\DUrole{n}{next\_image}\DUrole{o}{=}\DUrole{default_value}{None}}, \emph{\DUrole{n}{alt\_image}\DUrole{o}{=}\DUrole{default_value}{None}}}{}
Bases: \sphinxcode{\sphinxupquote{object}}
\index{DEFAULT\_IMAGE (polo.crystallography.image.Image attribute)@\spxentry{DEFAULT\_IMAGE}\spxextra{polo.crystallography.image.Image attribute}}

\begin{fulllineitems}
\phantomsection\label{\detokenize{polo.crystallography:polo.crystallography.image.Image.DEFAULT_IMAGE}}\pysigline{\sphinxbfcode{\sphinxupquote{DEFAULT\_IMAGE}}\sphinxbfcode{\sphinxupquote{ = \textquotesingle{}\textquotesingle{}}}}
\end{fulllineitems}

\index{classify\_image() (polo.crystallography.image.Image method)@\spxentry{classify\_image()}\spxextra{polo.crystallography.image.Image method}}

\begin{fulllineitems}
\phantomsection\label{\detokenize{polo.crystallography:polo.crystallography.image.Image.classify_image}}\pysiglinewithargsret{\sphinxbfcode{\sphinxupquote{classify\_image}}}{}{}
\end{fulllineitems}

\index{encode\_base64() (polo.crystallography.image.Image method)@\spxentry{encode\_base64()}\spxextra{polo.crystallography.image.Image method}}

\begin{fulllineitems}
\phantomsection\label{\detokenize{polo.crystallography:polo.crystallography.image.Image.encode_base64}}\pysiglinewithargsret{\sphinxbfcode{\sphinxupquote{encode\_base64}}}{}{}
\end{fulllineitems}

\index{get\_cocktail\_number() (polo.crystallography.image.Image method)@\spxentry{get\_cocktail\_number()}\spxextra{polo.crystallography.image.Image method}}

\begin{fulllineitems}
\phantomsection\label{\detokenize{polo.crystallography:polo.crystallography.image.Image.get_cocktail_number}}\pysiglinewithargsret{\sphinxbfcode{\sphinxupquote{get\_cocktail\_number}}}{}{}
Returns the numerical cocktail number as an integer. If it does not
exist returns 0.

\end{fulllineitems}

\index{get\_pixel\_map() (polo.crystallography.image.Image method)@\spxentry{get\_pixel\_map()}\spxextra{polo.crystallography.image.Image method}}

\begin{fulllineitems}
\phantomsection\label{\detokenize{polo.crystallography:polo.crystallography.image.Image.get_pixel_map}}\pysiglinewithargsret{\sphinxbfcode{\sphinxupquote{get\_pixel\_map}}}{}{}
\end{fulllineitems}

\index{get\_tool\_tip() (polo.crystallography.image.Image method)@\spxentry{get\_tool\_tip()}\spxextra{polo.crystallography.image.Image method}}

\begin{fulllineitems}
\phantomsection\label{\detokenize{polo.crystallography:polo.crystallography.image.Image.get_tool_tip}}\pysiglinewithargsret{\sphinxbfcode{\sphinxupquote{get\_tool\_tip}}}{}{}
\end{fulllineitems}

\index{resize() (polo.crystallography.image.Image method)@\spxentry{resize()}\spxextra{polo.crystallography.image.Image method}}

\begin{fulllineitems}
\phantomsection\label{\detokenize{polo.crystallography:polo.crystallography.image.Image.resize}}\pysiglinewithargsret{\sphinxbfcode{\sphinxupquote{resize}}}{\emph{\DUrole{n}{x}}, \emph{\DUrole{n}{y}}, \emph{\DUrole{n}{preserve\_aspect}\DUrole{o}{=}\DUrole{default_value}{True}}}{}
Resizes an image given x and y resolution. Copy is true will copy
the image instead of overwriting it.

\end{fulllineitems}


\end{fulllineitems}



\subsubsection{polo.crystallography.make\_screen module}
\label{\detokenize{polo.crystallography:module-polo.crystallography.make_screen}}\label{\detokenize{polo.crystallography:polo-crystallography-make-screen-module}}\index{module@\spxentry{module}!polo.crystallography.make\_screen@\spxentry{polo.crystallography.make\_screen}}\index{polo.crystallography.make\_screen@\spxentry{polo.crystallography.make\_screen}!module@\spxentry{module}}\index{calculate\_row() (in module polo.crystallography.make\_screen)@\spxentry{calculate\_row()}\spxextra{in module polo.crystallography.make\_screen}}

\begin{fulllineitems}
\phantomsection\label{\detokenize{polo.crystallography:polo.crystallography.make_screen.calculate_row}}\pysiglinewithargsret{\sphinxcode{\sphinxupquote{polo.crystallography.make\_screen.}}\sphinxbfcode{\sphinxupquote{calculate\_row}}}{\emph{\DUrole{n}{volume}}, \emph{\DUrole{n}{x\_con}}, \emph{\DUrole{n}{x\_stock\_con}}, \emph{\DUrole{n}{y\_stock\_con}}, \emph{\DUrole{n}{y\_con}}, \emph{\DUrole{n}{y\_step}}, \emph{\DUrole{n}{steps}}}{}
\end{fulllineitems}

\index{get\_dilution() (in module polo.crystallography.make\_screen)@\spxentry{get\_dilution()}\spxextra{in module polo.crystallography.make\_screen}}

\begin{fulllineitems}
\phantomsection\label{\detokenize{polo.crystallography:polo.crystallography.make_screen.get_dilution}}\pysiglinewithargsret{\sphinxcode{\sphinxupquote{polo.crystallography.make\_screen.}}\sphinxbfcode{\sphinxupquote{get\_dilution}}}{\emph{\DUrole{n}{target\_con}}, \emph{\DUrole{n}{stock\_con}}}{}
\end{fulllineitems}

\index{get\_mols() (in module polo.crystallography.make\_screen)@\spxentry{get\_mols()}\spxextra{in module polo.crystallography.make\_screen}}

\begin{fulllineitems}
\phantomsection\label{\detokenize{polo.crystallography:polo.crystallography.make_screen.get_mols}}\pysiglinewithargsret{\sphinxcode{\sphinxupquote{polo.crystallography.make\_screen.}}\sphinxbfcode{\sphinxupquote{get\_mols}}}{\emph{\DUrole{n}{total\_volume}}, \emph{\DUrole{n}{target\_con}}}{}
\end{fulllineitems}

\index{get\_stock\_volume() (in module polo.crystallography.make\_screen)@\spxentry{get\_stock\_volume()}\spxextra{in module polo.crystallography.make\_screen}}

\begin{fulllineitems}
\phantomsection\label{\detokenize{polo.crystallography:polo.crystallography.make_screen.get_stock_volume}}\pysiglinewithargsret{\sphinxcode{\sphinxupquote{polo.crystallography.make\_screen.}}\sphinxbfcode{\sphinxupquote{get\_stock\_volume}}}{\emph{\DUrole{n}{target\_con}}, \emph{\DUrole{n}{total\_volume}}, \emph{\DUrole{n}{stock\_con}}}{}
\end{fulllineitems}

\index{get\_usable\_volume() (in module polo.crystallography.make\_screen)@\spxentry{get\_usable\_volume()}\spxextra{in module polo.crystallography.make\_screen}}

\begin{fulllineitems}
\phantomsection\label{\detokenize{polo.crystallography:polo.crystallography.make_screen.get_usable_volume}}\pysiglinewithargsret{\sphinxcode{\sphinxupquote{polo.crystallography.make\_screen.}}\sphinxbfcode{\sphinxupquote{get\_usable\_volume}}}{\emph{\DUrole{n}{total\_volume}}}{}
\end{fulllineitems}

\index{make\_tray() (in module polo.crystallography.make\_screen)@\spxentry{make\_tray()}\spxextra{in module polo.crystallography.make\_screen}}

\begin{fulllineitems}
\phantomsection\label{\detokenize{polo.crystallography:polo.crystallography.make_screen.make_tray}}\pysiglinewithargsret{\sphinxcode{\sphinxupquote{polo.crystallography.make\_screen.}}\sphinxbfcode{\sphinxupquote{make\_tray}}}{\emph{\DUrole{n}{total\_volume}}, \emph{\DUrole{n}{hit\_concentration}}, \emph{\DUrole{n}{total\_wells}}, \emph{\DUrole{n}{sample\_concentration}}, \emph{\DUrole{n}{sample\_volume\_per\_well}}, \emph{\DUrole{n}{stock\_cocentration}}, \emph{\DUrole{n}{step\_percent}}}{}
\end{fulllineitems}



\subsubsection{polo.crystallography.run module}
\label{\detokenize{polo.crystallography:module-polo.crystallography.run}}\label{\detokenize{polo.crystallography:polo-crystallography-run-module}}\index{module@\spxentry{module}!polo.crystallography.run@\spxentry{polo.crystallography.run}}\index{polo.crystallography.run@\spxentry{polo.crystallography.run}!module@\spxentry{module}}\index{HWIRun (class in polo.crystallography.run)@\spxentry{HWIRun}\spxextra{class in polo.crystallography.run}}

\begin{fulllineitems}
\phantomsection\label{\detokenize{polo.crystallography:polo.crystallography.run.HWIRun}}\pysiglinewithargsret{\sphinxbfcode{\sphinxupquote{class }}\sphinxcode{\sphinxupquote{polo.crystallography.run.}}\sphinxbfcode{\sphinxupquote{HWIRun}}}{\emph{\DUrole{n}{image\_dir}}, \emph{\DUrole{n}{run\_name}}, \emph{\DUrole{n}{cocktail\_dict}\DUrole{o}{=}\DUrole{default_value}{None}}, \emph{\DUrole{n}{images}\DUrole{o}{=}\DUrole{default_value}{{[}{]}}}, \emph{\DUrole{n}{plate\_id}\DUrole{o}{=}\DUrole{default_value}{None}}, \emph{\DUrole{n}{annotations}\DUrole{o}{=}\DUrole{default_value}{None}}, \emph{\DUrole{n}{save\_file\_path}\DUrole{o}{=}\DUrole{default_value}{None}}, \emph{\DUrole{n}{num\_wells}\DUrole{o}{=}\DUrole{default_value}{1536}}, \emph{\DUrole{n}{image\_spectrum}\DUrole{o}{=}\DUrole{default_value}{None}}, \emph{\DUrole{n}{date}\DUrole{o}{=}\DUrole{default_value}{None}}, \emph{\DUrole{n}{next\_run}\DUrole{o}{=}\DUrole{default_value}{None}}, \emph{\DUrole{n}{previous\_run}\DUrole{o}{=}\DUrole{default_value}{None}}, \emph{\DUrole{n}{alt\_spectrum}\DUrole{o}{=}\DUrole{default_value}{None}}, \emph{\DUrole{n}{number\_grid\_pages}\DUrole{o}{=}\DUrole{default_value}{None}}, \emph{\DUrole{n}{current\_grid\_page}\DUrole{o}{=}\DUrole{default_value}{1}}, \emph{\DUrole{n}{journal}\DUrole{o}{=}\DUrole{default_value}{None}}, \emph{\DUrole{n}{current\_image\_index}\DUrole{o}{=}\DUrole{default_value}{0}}, \emph{\DUrole{n}{current\_image}\DUrole{o}{=}\DUrole{default_value}{None}}, \emph{\DUrole{o}{*}\DUrole{n}{args}}, \emph{\DUrole{o}{**}\DUrole{n}{kwargs}}}{}
Bases: {\hyperref[\detokenize{polo.crystallography:polo.crystallography.run.Run}]{\sphinxcrossref{\sphinxcode{\sphinxupquote{polo.crystallography.run.Run}}}}}
\index{AllOWED\_PLOTS (polo.crystallography.run.HWIRun attribute)@\spxentry{AllOWED\_PLOTS}\spxextra{polo.crystallography.run.HWIRun attribute}}

\begin{fulllineitems}
\phantomsection\label{\detokenize{polo.crystallography:polo.crystallography.run.HWIRun.AllOWED_PLOTS}}\pysigline{\sphinxbfcode{\sphinxupquote{AllOWED\_PLOTS}}\sphinxbfcode{\sphinxupquote{ = {[}\textquotesingle{}Classification Counts\textquotesingle{}, \textquotesingle{}MARCO Accuracy\textquotesingle{}, \textquotesingle{}Classification Progress\textquotesingle{}, \textquotesingle{}Plate Heatmaps\textquotesingle{}{]}}}}
Child class of Run. Is used to represent runs from the HWI screening center
as images will have additional metadata like well and cocktail information.
Main difference is that HWIRuns will always contain 1536 images as that is
the number of wells in a HWI crystallization plate. Each well uses a
different chemcial cocktail which is described in the cocktail tsv file
included in the directory of images provided by HWI in each run.

\end{fulllineitems}

\index{add\_images\_from\_dir() (polo.crystallography.run.HWIRun method)@\spxentry{add\_images\_from\_dir()}\spxextra{polo.crystallography.run.HWIRun method}}

\begin{fulllineitems}
\phantomsection\label{\detokenize{polo.crystallography:polo.crystallography.run.HWIRun.add_images_from_dir}}\pysiglinewithargsret{\sphinxbfcode{\sphinxupquote{add\_images\_from\_dir}}}{}{}
Populates the images attribute with a list of images read from the
image\_dir location. Currently is dependent on having a cocktail
dictionary available. This is passed into the function and would
normally come from the most recently used HWI file that is
stored as a dictionary in the mainWindow object.

\end{fulllineitems}

\index{link\_to\_alt\_images() (polo.crystallography.run.HWIRun method)@\spxentry{link\_to\_alt\_images()}\spxextra{polo.crystallography.run.HWIRun method}}

\begin{fulllineitems}
\phantomsection\label{\detokenize{polo.crystallography:polo.crystallography.run.HWIRun.link_to_alt_images}}\pysiglinewithargsret{\sphinxbfcode{\sphinxupquote{link\_to\_alt\_images}}}{\emph{\DUrole{n}{other\_run}}}{}
Establish linked lists between this run and another run instance
that holds images of a different imaging spectrum / technology.
\begin{quote}\begin{description}
\item[{Parameters}] \leavevmode
\sphinxstyleliteralstrong{\sphinxupquote{other\_run}} ({\hyperref[\detokenize{polo.crystallography:polo.crystallography.run.Run}]{\sphinxcrossref{\sphinxstyleliteralemphasis{\sphinxupquote{Run}}}}}) \textendash{} A different run instance

\end{description}\end{quote}

\end{fulllineitems}

\index{link\_to\_alt\_spectrum() (polo.crystallography.run.HWIRun method)@\spxentry{link\_to\_alt\_spectrum()}\spxextra{polo.crystallography.run.HWIRun method}}

\begin{fulllineitems}
\phantomsection\label{\detokenize{polo.crystallography:polo.crystallography.run.HWIRun.link_to_alt_spectrum}}\pysiglinewithargsret{\sphinxbfcode{\sphinxupquote{link\_to\_alt\_spectrum}}}{\emph{\DUrole{n}{other\_run}}}{}
\end{fulllineitems}

\index{link\_to\_decendent() (polo.crystallography.run.HWIRun method)@\spxentry{link\_to\_decendent()}\spxextra{polo.crystallography.run.HWIRun method}}

\begin{fulllineitems}
\phantomsection\label{\detokenize{polo.crystallography:polo.crystallography.run.HWIRun.link_to_decendent}}\pysiglinewithargsret{\sphinxbfcode{\sphinxupquote{link\_to\_decendent}}}{\emph{\DUrole{n}{other\_run}}}{}
\end{fulllineitems}

\index{link\_to\_predecessor() (polo.crystallography.run.HWIRun method)@\spxentry{link\_to\_predecessor()}\spxextra{polo.crystallography.run.HWIRun method}}

\begin{fulllineitems}
\phantomsection\label{\detokenize{polo.crystallography:polo.crystallography.run.HWIRun.link_to_predecessor}}\pysiglinewithargsret{\sphinxbfcode{\sphinxupquote{link\_to\_predecessor}}}{\emph{\DUrole{n}{other\_run}}}{}
\end{fulllineitems}

\index{sort\_current\_images\_by\_cocktail() (polo.crystallography.run.HWIRun method)@\spxentry{sort\_current\_images\_by\_cocktail()}\spxextra{polo.crystallography.run.HWIRun method}}

\begin{fulllineitems}
\phantomsection\label{\detokenize{polo.crystallography:polo.crystallography.run.HWIRun.sort_current_images_by_cocktail}}\pysiglinewithargsret{\sphinxbfcode{\sphinxupquote{sort\_current\_images\_by\_cocktail}}}{}{}
Sorts the current slideshow images by cocktail number. Allows the user
to navigate by cocktail number and therefore similar chemical conditions
opposed to as by well number which is proxy for physical location in
the well. Oftentimes due to plate shape similar well numbers will be
in different family of chemcial conditions.

\end{fulllineitems}


\end{fulllineitems}

\index{Run (class in polo.crystallography.run)@\spxentry{Run}\spxextra{class in polo.crystallography.run}}

\begin{fulllineitems}
\phantomsection\label{\detokenize{polo.crystallography:polo.crystallography.run.Run}}\pysiglinewithargsret{\sphinxbfcode{\sphinxupquote{class }}\sphinxcode{\sphinxupquote{polo.crystallography.run.}}\sphinxbfcode{\sphinxupquote{Run}}}{\emph{\DUrole{n}{image\_dir}}, \emph{\DUrole{n}{run\_name}}, \emph{\DUrole{n}{images}\DUrole{o}{=}\DUrole{default_value}{None}}, \emph{\DUrole{n}{save\_file\_path}\DUrole{o}{=}\DUrole{default_value}{None}}, \emph{\DUrole{n}{log}\DUrole{o}{=}\DUrole{default_value}{None}}, \emph{\DUrole{n}{date}\DUrole{o}{=}\DUrole{default_value}{None}}, \emph{\DUrole{n}{image\_spectrum}\DUrole{o}{=}\DUrole{default_value}{None}}, \emph{\DUrole{n}{next\_run}\DUrole{o}{=}\DUrole{default_value}{None}}, \emph{\DUrole{n}{previous\_run}\DUrole{o}{=}\DUrole{default_value}{None}}, \emph{\DUrole{n}{alt\_spectrum}\DUrole{o}{=}\DUrole{default_value}{None}}, \emph{\DUrole{n}{journal}\DUrole{o}{=}\DUrole{default_value}{\{\}}}, \emph{\DUrole{n}{current\_image}\DUrole{o}{=}\DUrole{default_value}{None}}, \emph{\DUrole{n}{current\_image\_index}\DUrole{o}{=}\DUrole{default_value}{0}}, \emph{\DUrole{o}{*}\DUrole{n}{args}}, \emph{\DUrole{o}{**}\DUrole{n}{kwargs}}}{}
Bases: \sphinxcode{\sphinxupquote{object}}

Holds data relating to an individual screening run, or one plate of
images.
\index{AllOWED\_PLOTS (polo.crystallography.run.Run attribute)@\spxentry{AllOWED\_PLOTS}\spxextra{polo.crystallography.run.Run attribute}}

\begin{fulllineitems}
\phantomsection\label{\detokenize{polo.crystallography:polo.crystallography.run.Run.AllOWED_PLOTS}}\pysigline{\sphinxbfcode{\sphinxupquote{AllOWED\_PLOTS}}\sphinxbfcode{\sphinxupquote{ = {[}\textquotesingle{}Classification Counts\textquotesingle{}, \textquotesingle{}MARCO Accuracy\textquotesingle{}, \textquotesingle{}Classification Progress\textquotesingle{}{]}}}}
\end{fulllineitems}

\index{add\_images\_from\_dir() (polo.crystallography.run.Run method)@\spxentry{add\_images\_from\_dir()}\spxextra{polo.crystallography.run.Run method}}

\begin{fulllineitems}
\phantomsection\label{\detokenize{polo.crystallography:polo.crystallography.run.Run.add_images_from_dir}}\pysiglinewithargsret{\sphinxbfcode{\sphinxupquote{add\_images\_from\_dir}}}{}{}
Adds the contents of a directory to self.images
\begin{description}
\item[{TODO: Add validation for file types and content. Handle if user}] \leavevmode
gives a directory where there are no images or edge cases like
that.

\end{description}

\end{fulllineitems}

\index{add\_journal\_entry() (polo.crystallography.run.Run method)@\spxentry{add\_journal\_entry()}\spxextra{polo.crystallography.run.Run method}}

\begin{fulllineitems}
\phantomsection\label{\detokenize{polo.crystallography:polo.crystallography.run.Run.add_journal_entry}}\pysiglinewithargsret{\sphinxbfcode{\sphinxupquote{add\_journal\_entry}}}{\emph{\DUrole{n}{contents}}, \emph{\DUrole{n}{title}}}{}
\end{fulllineitems}

\index{encode\_images\_to\_base64() (polo.crystallography.run.Run method)@\spxentry{encode\_images\_to\_base64()}\spxextra{polo.crystallography.run.Run method}}

\begin{fulllineitems}
\phantomsection\label{\detokenize{polo.crystallography:polo.crystallography.run.Run.encode_images_to_base64}}\pysiglinewithargsret{\sphinxbfcode{\sphinxupquote{encode\_images\_to\_base64}}}{}{}
\end{fulllineitems}

\index{export\_to\_csv() (polo.crystallography.run.Run method)@\spxentry{export\_to\_csv()}\spxextra{polo.crystallography.run.Run method}}

\begin{fulllineitems}
\phantomsection\label{\detokenize{polo.crystallography:polo.crystallography.run.Run.export_to_csv}}\pysiglinewithargsret{\sphinxbfcode{\sphinxupquote{export\_to\_csv}}}{\emph{\DUrole{n}{output\_dir}}}{}
Exports run classifcation data to a csv table.

\end{fulllineitems}

\index{get\_cocktails() (polo.crystallography.run.Run method)@\spxentry{get\_cocktails()}\spxextra{polo.crystallography.run.Run method}}

\begin{fulllineitems}
\phantomsection\label{\detokenize{polo.crystallography:polo.crystallography.run.Run.get_cocktails}}\pysiglinewithargsret{\sphinxbfcode{\sphinxupquote{get\_cocktails}}}{}{}
Returns list of list of cocktails assigned to this run

\end{fulllineitems}

\index{get\_current\_hits() (polo.crystallography.run.Run method)@\spxentry{get\_current\_hits()}\spxextra{polo.crystallography.run.Run method}}

\begin{fulllineitems}
\phantomsection\label{\detokenize{polo.crystallography:polo.crystallography.run.Run.get_current_hits}}\pysiglinewithargsret{\sphinxbfcode{\sphinxupquote{get\_current\_hits}}}{}{}
\end{fulllineitems}

\index{get\_current\_table\_data() (polo.crystallography.run.Run method)@\spxentry{get\_current\_table\_data()}\spxextra{polo.crystallography.run.Run method}}

\begin{fulllineitems}
\phantomsection\label{\detokenize{polo.crystallography:polo.crystallography.run.Run.get_current_table_data}}\pysiglinewithargsret{\sphinxbfcode{\sphinxupquote{get\_current\_table\_data}}}{\emph{\DUrole{n}{image\_types}}, \emph{\DUrole{n}{human}\DUrole{o}{=}\DUrole{default_value}{True}}, \emph{\DUrole{n}{marco}\DUrole{o}{=}\DUrole{default_value}{False}}}{}
\end{fulllineitems}

\index{get\_heap\_map\_data() (polo.crystallography.run.Run method)@\spxentry{get\_heap\_map\_data()}\spxextra{polo.crystallography.run.Run method}}

\begin{fulllineitems}
\phantomsection\label{\detokenize{polo.crystallography:polo.crystallography.run.Run.get_heap_map_data}}\pysiglinewithargsret{\sphinxbfcode{\sphinxupquote{get\_heap\_map\_data}}}{}{}
Returns data that will be needed to render the heatmap
view of results

\end{fulllineitems}

\index{get\_human\_statistics() (polo.crystallography.run.Run method)@\spxentry{get\_human\_statistics()}\spxextra{polo.crystallography.run.Run method}}

\begin{fulllineitems}
\phantomsection\label{\detokenize{polo.crystallography:polo.crystallography.run.Run.get_human_statistics}}\pysiglinewithargsret{\sphinxbfcode{\sphinxupquote{get\_human\_statistics}}}{}{}
Returns stats that would be shown in the stats tab of the viewer.

\end{fulllineitems}

\index{get\_image\_table\_data() (polo.crystallography.run.Run method)@\spxentry{get\_image\_table\_data()}\spxextra{polo.crystallography.run.Run method}}

\begin{fulllineitems}
\phantomsection\label{\detokenize{polo.crystallography:polo.crystallography.run.Run.get_image_table_data}}\pysiglinewithargsret{\sphinxbfcode{\sphinxupquote{get\_image\_table\_data}}}{\emph{\DUrole{n}{image}}, \emph{\DUrole{n}{attributes}}}{}
\end{fulllineitems}

\index{get\_images\_by\_classification() (polo.crystallography.run.Run method)@\spxentry{get\_images\_by\_classification()}\spxextra{polo.crystallography.run.Run method}}

\begin{fulllineitems}
\phantomsection\label{\detokenize{polo.crystallography:polo.crystallography.run.Run.get_images_by_classification}}\pysiglinewithargsret{\sphinxbfcode{\sphinxupquote{get\_images\_by\_classification}}}{\emph{\DUrole{n}{human}\DUrole{o}{=}\DUrole{default_value}{True}}}{}
Create a dictionary of image classifcations. Keys are
each type of classification and values are list of
images with classification of the key. The human
boolean determines what classifier should be used to
determine the image type. Human = True sets the human
as the classifier and False sets MARCO as the classifier.

\end{fulllineitems}

\index{get\_table\_data() (polo.crystallography.run.Run method)@\spxentry{get\_table\_data()}\spxextra{polo.crystallography.run.Run method}}

\begin{fulllineitems}
\phantomsection\label{\detokenize{polo.crystallography:polo.crystallography.run.Run.get_table_data}}\pysiglinewithargsret{\sphinxbfcode{\sphinxupquote{get\_table\_data}}}{\emph{\DUrole{n}{image\_types}}, \emph{\DUrole{n}{human}}, \emph{\DUrole{n}{marco}}}{}
\end{fulllineitems}

\index{image\_filter\_engine() (polo.crystallography.run.Run method)@\spxentry{image\_filter\_engine()}\spxextra{polo.crystallography.run.Run method}}

\begin{fulllineitems}
\phantomsection\label{\detokenize{polo.crystallography:polo.crystallography.run.Run.image_filter_engine}}\pysiglinewithargsret{\sphinxbfcode{\sphinxupquote{image\_filter\_engine}}}{\emph{\DUrole{n}{image}}, \emph{\DUrole{n}{image\_types}}, \emph{\DUrole{n}{human}\DUrole{o}{=}\DUrole{default_value}{False}}, \emph{\DUrole{n}{marco}\DUrole{o}{=}\DUrole{default_value}{False}}}{}
\end{fulllineitems}

\index{image\_filter\_query() (polo.crystallography.run.Run method)@\spxentry{image\_filter\_query()}\spxextra{polo.crystallography.run.Run method}}

\begin{fulllineitems}
\phantomsection\label{\detokenize{polo.crystallography:polo.crystallography.run.Run.image_filter_query}}\pysiglinewithargsret{\sphinxbfcode{\sphinxupquote{image\_filter\_query}}}{\emph{\DUrole{n}{image\_types}}, \emph{\DUrole{n}{human}\DUrole{o}{=}\DUrole{default_value}{False}}, \emph{\DUrole{n}{marco}\DUrole{o}{=}\DUrole{default_value}{False}}}{}
\end{fulllineitems}


\end{fulllineitems}



\subsubsection{Module contents}
\label{\detokenize{polo.crystallography:module-polo.crystallography}}\label{\detokenize{polo.crystallography:module-contents}}\index{module@\spxentry{module}!polo.crystallography@\spxentry{polo.crystallography}}\index{polo.crystallography@\spxentry{polo.crystallography}!module@\spxentry{module}}

\subsection{polo.marco package}
\label{\detokenize{polo.marco:polo-marco-package}}\label{\detokenize{polo.marco::doc}}

\subsubsection{Submodules}
\label{\detokenize{polo.marco:submodules}}

\subsubsection{polo.marco.run\_marco module}
\label{\detokenize{polo.marco:module-polo.marco.run_marco}}\label{\detokenize{polo.marco:polo-marco-run-marco-module}}\index{module@\spxentry{module}!polo.marco.run\_marco@\spxentry{polo.marco.run\_marco}}\index{polo.marco.run\_marco@\spxentry{polo.marco.run\_marco}!module@\spxentry{module}}\index{classify\_image() (in module polo.marco.run\_marco)@\spxentry{classify\_image()}\spxextra{in module polo.marco.run\_marco}}

\begin{fulllineitems}
\phantomsection\label{\detokenize{polo.marco:polo.marco.run_marco.classify_image}}\pysiglinewithargsret{\sphinxcode{\sphinxupquote{polo.marco.run\_marco.}}\sphinxbfcode{\sphinxupquote{classify\_image}}}{\emph{\DUrole{n}{tf\_predictor}}, \emph{\DUrole{n}{image\_path}}}{}
Given a tensorflow predictor (the MARCO model) and the path to an image, 
runs the model on that image. Returns a tuple where the first item is the
classification with greatest confidence and the second is a dictionary where
keys are image classification types and values are model confidence for that
classification.

param: tf\_predictor: Tensorflow predictor object. Should be MARCO model         in ready to roll form.
param: image\_path: String. Path to an image that will be classified.

\end{fulllineitems}

\index{get\_images() (in module polo.marco.run\_marco)@\spxentry{get\_images()}\spxextra{in module polo.marco.run\_marco}}

\begin{fulllineitems}
\phantomsection\label{\detokenize{polo.marco:polo.marco.run_marco.get_images}}\pysiglinewithargsret{\sphinxcode{\sphinxupquote{polo.marco.run\_marco.}}\sphinxbfcode{\sphinxupquote{get\_images}}}{\emph{\DUrole{n}{images\_path}}}{}
returns a list of all file paths contained in the given directory.

\end{fulllineitems}

\index{load\_image() (in module polo.marco.run\_marco)@\spxentry{load\_image()}\spxextra{in module polo.marco.run\_marco}}

\begin{fulllineitems}
\phantomsection\label{\detokenize{polo.marco:polo.marco.run_marco.load_image}}\pysiglinewithargsret{\sphinxcode{\sphinxupquote{polo.marco.run\_marco.}}\sphinxbfcode{\sphinxupquote{load\_image}}}{\emph{\DUrole{n}{file\_path}}}{}
\end{fulllineitems}

\index{load\_images() (in module polo.marco.run\_marco)@\spxentry{load\_images()}\spxextra{in module polo.marco.run\_marco}}

\begin{fulllineitems}
\phantomsection\label{\detokenize{polo.marco:polo.marco.run_marco.load_images}}\pysiglinewithargsret{\sphinxcode{\sphinxupquote{polo.marco.run\_marco.}}\sphinxbfcode{\sphinxupquote{load\_images}}}{\emph{\DUrole{n}{file\_list}}}{}
Loads images from a list of paths (should be from get\_images) in format
that can be read by the tensorflow package

\end{fulllineitems}



\subsubsection{Module contents}
\label{\detokenize{polo.marco:module-polo.marco}}\label{\detokenize{polo.marco:module-contents}}\index{module@\spxentry{module}!polo.marco@\spxentry{polo.marco}}\index{polo.marco@\spxentry{polo.marco}!module@\spxentry{module}}

\subsection{polo.utils package}
\label{\detokenize{polo.utils:polo-utils-package}}\label{\detokenize{polo.utils::doc}}

\subsubsection{Submodules}
\label{\detokenize{polo.utils:submodules}}

\subsubsection{polo.utils.exceptions module}
\label{\detokenize{polo.utils:module-polo.utils.exceptions}}\label{\detokenize{polo.utils:polo-utils-exceptions-module}}\index{module@\spxentry{module}!polo.utils.exceptions@\spxentry{polo.utils.exceptions}}\index{polo.utils.exceptions@\spxentry{polo.utils.exceptions}!module@\spxentry{module}}\index{EmptyDirectoryError@\spxentry{EmptyDirectoryError}}

\begin{fulllineitems}
\phantomsection\label{\detokenize{polo.utils:polo.utils.exceptions.EmptyDirectoryError}}\pysigline{\sphinxbfcode{\sphinxupquote{exception }}\sphinxcode{\sphinxupquote{polo.utils.exceptions.}}\sphinxbfcode{\sphinxupquote{EmptyDirectoryError}}}
Bases: \sphinxcode{\sphinxupquote{Exception}}

Raised when attempting to load images from an empty directory

\end{fulllineitems}

\index{EmptyRunNameError@\spxentry{EmptyRunNameError}}

\begin{fulllineitems}
\phantomsection\label{\detokenize{polo.utils:polo.utils.exceptions.EmptyRunNameError}}\pysigline{\sphinxbfcode{\sphinxupquote{exception }}\sphinxcode{\sphinxupquote{polo.utils.exceptions.}}\sphinxbfcode{\sphinxupquote{EmptyRunNameError}}}
Bases: \sphinxcode{\sphinxupquote{Exception}}

Raised when reading in an HWI directory but it does not contain number
of images corresponding to number of wells.

\end{fulllineitems}

\index{ForbiddenImageTypeError@\spxentry{ForbiddenImageTypeError}}

\begin{fulllineitems}
\phantomsection\label{\detokenize{polo.utils:polo.utils.exceptions.ForbiddenImageTypeError}}\pysigline{\sphinxbfcode{\sphinxupquote{exception }}\sphinxcode{\sphinxupquote{polo.utils.exceptions.}}\sphinxbfcode{\sphinxupquote{ForbiddenImageTypeError}}}
Bases: \sphinxcode{\sphinxupquote{Exception}}

Raised when user attempts to load in an image that is not in the allowed
image types.

\end{fulllineitems}

\index{IncompletePlateError@\spxentry{IncompletePlateError}}

\begin{fulllineitems}
\phantomsection\label{\detokenize{polo.utils:polo.utils.exceptions.IncompletePlateError}}\pysigline{\sphinxbfcode{\sphinxupquote{exception }}\sphinxcode{\sphinxupquote{polo.utils.exceptions.}}\sphinxbfcode{\sphinxupquote{IncompletePlateError}}}
Bases: \sphinxcode{\sphinxupquote{Exception}}

Raised when reading in an HWI directory but it does not contain number
of images corresponding to number of wells.

\end{fulllineitems}

\index{InvalidCocktailFile@\spxentry{InvalidCocktailFile}}

\begin{fulllineitems}
\phantomsection\label{\detokenize{polo.utils:polo.utils.exceptions.InvalidCocktailFile}}\pysigline{\sphinxbfcode{\sphinxupquote{exception }}\sphinxcode{\sphinxupquote{polo.utils.exceptions.}}\sphinxbfcode{\sphinxupquote{InvalidCocktailFile}}}
Bases: \sphinxcode{\sphinxupquote{Exception}}

Raised when user attempts to load in a file containing well cocktail
information that does not confrom to existing formating standards.

\end{fulllineitems}

\index{NotASolutionError@\spxentry{NotASolutionError}}

\begin{fulllineitems}
\phantomsection\label{\detokenize{polo.utils:polo.utils.exceptions.NotASolutionError}}\pysigline{\sphinxbfcode{\sphinxupquote{exception }}\sphinxcode{\sphinxupquote{polo.utils.exceptions.}}\sphinxbfcode{\sphinxupquote{NotASolutionError}}}
Bases: \sphinxcode{\sphinxupquote{Exception}}

Raised when user attempts to load in an image that is not in the allowed
image types.

\end{fulllineitems}

\index{NotHWIDirectoryError@\spxentry{NotHWIDirectoryError}}

\begin{fulllineitems}
\phantomsection\label{\detokenize{polo.utils:polo.utils.exceptions.NotHWIDirectoryError}}\pysigline{\sphinxbfcode{\sphinxupquote{exception }}\sphinxcode{\sphinxupquote{polo.utils.exceptions.}}\sphinxbfcode{\sphinxupquote{NotHWIDirectoryError}}}
Bases: \sphinxcode{\sphinxupquote{Exception}}

Raised when user attempts to read in a directory as HWI but it does
not look like one.

TODO: Add utils function to determine when to raise this exception.

\end{fulllineitems}



\subsubsection{polo.utils.ftp\_utils module}
\label{\detokenize{polo.utils:module-polo.utils.ftp_utils}}\label{\detokenize{polo.utils:polo-utils-ftp-utils-module}}\index{module@\spxentry{module}!polo.utils.ftp\_utils@\spxentry{polo.utils.ftp\_utils}}\index{polo.utils.ftp\_utils@\spxentry{polo.utils.ftp\_utils}!module@\spxentry{module}}\index{catch\_ftp\_errors() (in module polo.utils.ftp\_utils)@\spxentry{catch\_ftp\_errors()}\spxextra{in module polo.utils.ftp\_utils}}

\begin{fulllineitems}
\phantomsection\label{\detokenize{polo.utils:polo.utils.ftp_utils.catch_ftp_errors}}\pysiglinewithargsret{\sphinxcode{\sphinxupquote{polo.utils.ftp\_utils.}}\sphinxbfcode{\sphinxupquote{catch\_ftp\_errors}}}{\emph{\DUrole{n}{funct}}}{}
General decorator function for catching any errors thrown by other
ftp\_utils functions

\end{fulllineitems}

\index{get\_cwd() (in module polo.utils.ftp\_utils)@\spxentry{get\_cwd()}\spxextra{in module polo.utils.ftp\_utils}}

\begin{fulllineitems}
\phantomsection\label{\detokenize{polo.utils:polo.utils.ftp_utils.get_cwd}}\pysiglinewithargsret{\sphinxcode{\sphinxupquote{polo.utils.ftp\_utils.}}\sphinxbfcode{\sphinxupquote{get\_cwd}}}{\emph{\DUrole{o}{*}\DUrole{n}{args}}, \emph{\DUrole{o}{**}\DUrole{n}{kwargs}}}{}
\end{fulllineitems}

\index{list\_dir() (in module polo.utils.ftp\_utils)@\spxentry{list\_dir()}\spxextra{in module polo.utils.ftp\_utils}}

\begin{fulllineitems}
\phantomsection\label{\detokenize{polo.utils:polo.utils.ftp_utils.list_dir}}\pysiglinewithargsret{\sphinxcode{\sphinxupquote{polo.utils.ftp\_utils.}}\sphinxbfcode{\sphinxupquote{list\_dir}}}{\emph{\DUrole{o}{*}\DUrole{n}{args}}, \emph{\DUrole{o}{**}\DUrole{n}{kwargs}}}{}
\end{fulllineitems}

\index{logon() (in module polo.utils.ftp\_utils)@\spxentry{logon()}\spxextra{in module polo.utils.ftp\_utils}}

\begin{fulllineitems}
\phantomsection\label{\detokenize{polo.utils:polo.utils.ftp_utils.logon}}\pysiglinewithargsret{\sphinxcode{\sphinxupquote{polo.utils.ftp\_utils.}}\sphinxbfcode{\sphinxupquote{logon}}}{\emph{\DUrole{o}{*}\DUrole{n}{args}}, \emph{\DUrole{o}{**}\DUrole{n}{kwargs}}}{}
\end{fulllineitems}

\index{traverse\_folder() (in module polo.utils.ftp\_utils)@\spxentry{traverse\_folder()}\spxextra{in module polo.utils.ftp\_utils}}

\begin{fulllineitems}
\phantomsection\label{\detokenize{polo.utils:polo.utils.ftp_utils.traverse_folder}}\pysiglinewithargsret{\sphinxcode{\sphinxupquote{polo.utils.ftp\_utils.}}\sphinxbfcode{\sphinxupquote{traverse\_folder}}}{\emph{\DUrole{n}{cwd}}, \emph{\DUrole{n}{tree}}, \emph{\DUrole{n}{ftp}}}{}
\end{fulllineitems}



\subsubsection{polo.utils.io\_utils module}
\label{\detokenize{polo.utils:module-polo.utils.io_utils}}\label{\detokenize{polo.utils:polo-utils-io-utils-module}}\index{module@\spxentry{module}!polo.utils.io\_utils@\spxentry{polo.utils.io\_utils}}\index{polo.utils.io\_utils@\spxentry{polo.utils.io\_utils}!module@\spxentry{module}}\index{BarTender (class in polo.utils.io\_utils)@\spxentry{BarTender}\spxextra{class in polo.utils.io\_utils}}

\begin{fulllineitems}
\phantomsection\label{\detokenize{polo.utils:polo.utils.io_utils.BarTender}}\pysiglinewithargsret{\sphinxbfcode{\sphinxupquote{class }}\sphinxcode{\sphinxupquote{polo.utils.io\_utils.}}\sphinxbfcode{\sphinxupquote{BarTender}}}{\emph{\DUrole{n}{cocktail\_dir}}, \emph{\DUrole{n}{cocktail\_meta}}}{}
Bases: \sphinxcode{\sphinxupquote{object}}

Class for organizing and accessing cocktail menus
\index{add\_menus\_from\_metadata() (polo.utils.io\_utils.BarTender method)@\spxentry{add\_menus\_from\_metadata()}\spxextra{polo.utils.io\_utils.BarTender method}}

\begin{fulllineitems}
\phantomsection\label{\detokenize{polo.utils:polo.utils.io_utils.BarTender.add_menus_from_metadata}}\pysiglinewithargsret{\sphinxbfcode{\sphinxupquote{add\_menus\_from\_metadata}}}{}{}
Adds menu objects to the menus attribute.

\end{fulllineitems}

\index{date\_range\_parser() (polo.utils.io\_utils.BarTender static method)@\spxentry{date\_range\_parser()}\spxextra{polo.utils.io\_utils.BarTender static method}}

\begin{fulllineitems}
\phantomsection\label{\detokenize{polo.utils:polo.utils.io_utils.BarTender.date_range_parser}}\pysiglinewithargsret{\sphinxbfcode{\sphinxupquote{static }}\sphinxbfcode{\sphinxupquote{date\_range\_parser}}}{\emph{\DUrole{n}{date\_range\_string}}}{}~\begin{description}
\item[{Utility function for converting the date ranges in the cocktail.units}] \leavevmode
metadata csv file to datetime objects using the \sphinxtitleref{datetime\_converter}
classmethod.

Date ranges should have the format

start date \sphinxhyphen{} end date

If the date range is for the most recent cocktail menu then there
will not be an end date and the format will be

start date \sphinxhyphen{}

\end{description}
\begin{quote}\begin{description}
\item[{Parameters}] \leavevmode
\sphinxstyleliteralstrong{\sphinxupquote{date\_range\_string}} (\sphinxstyleliteralemphasis{\sphinxupquote{str}}) \textendash{} string to pull dates out of

\item[{Returns}] \leavevmode
tuple of datetime objects, start date and end date

\item[{Return type}] \leavevmode
tuple

\end{description}\end{quote}

\end{fulllineitems}

\index{datetime\_converter() (polo.utils.io\_utils.BarTender static method)@\spxentry{datetime\_converter()}\spxextra{polo.utils.io\_utils.BarTender static method}}

\begin{fulllineitems}
\phantomsection\label{\detokenize{polo.utils:polo.utils.io_utils.BarTender.datetime_converter}}\pysiglinewithargsret{\sphinxbfcode{\sphinxupquote{static }}\sphinxbfcode{\sphinxupquote{datetime\_converter}}}{\emph{\DUrole{n}{date\_string}}}{}~\begin{description}
\item[{General utility function for converting strings to datetime objects..units}] \leavevmode
Attempts to convert the string by trying a couple of datetime.units
formats that are common in cocktail menu files and other.units
locations in the HWI file universe Polo runs across.

\end{description}
\begin{quote}\begin{description}
\item[{Parameters}] \leavevmode
\sphinxstyleliteralstrong{\sphinxupquote{date\_string}} (\sphinxstyleliteralemphasis{\sphinxupquote{str}}) \textendash{} string to convert to datetime

\item[{Returns}] \leavevmode
datetime object

\item[{Return type}] \leavevmode
datetime

\end{description}\end{quote}

\end{fulllineitems}

\index{get\_menu\_by\_date() (polo.utils.io\_utils.BarTender method)@\spxentry{get\_menu\_by\_date()}\spxextra{polo.utils.io\_utils.BarTender method}}

\begin{fulllineitems}
\phantomsection\label{\detokenize{polo.utils:polo.utils.io_utils.BarTender.get_menu_by_date}}\pysiglinewithargsret{\sphinxbfcode{\sphinxupquote{get\_menu\_by\_date}}}{\emph{\DUrole{n}{date}}, \emph{\DUrole{n}{type\_}}}{}
Get a menu instance who’s usage dates include the given date and
match the given screen type.

Screen types can either be ‘s’ for ‘soluble’ screens or ‘m’ for
membrane screens.
\begin{quote}\begin{description}
\item[{Parameters}] \leavevmode\begin{itemize}
\item {} 
\sphinxstyleliteralstrong{\sphinxupquote{date}} (\sphinxstyleliteralemphasis{\sphinxupquote{datetime}}) \textendash{} Date to search menus with

\item {} 
\sphinxstyleliteralstrong{\sphinxupquote{type}} (\sphinxstyleliteralemphasis{\sphinxupquote{str}}) \textendash{} Type of screen to return (soluble or membrane)

\end{itemize}

\item[{Returns}] \leavevmode
menu matching the given date and type

\item[{Return type}] \leavevmode
{\hyperref[\detokenize{polo.utils:polo.utils.io_utils.Menu}]{\sphinxcrossref{Menu}}}

\end{description}\end{quote}

\end{fulllineitems}

\index{get\_menu\_by\_path() (polo.utils.io\_utils.BarTender method)@\spxentry{get\_menu\_by\_path()}\spxextra{polo.utils.io\_utils.BarTender method}}

\begin{fulllineitems}
\phantomsection\label{\detokenize{polo.utils:polo.utils.io_utils.BarTender.get_menu_by_path}}\pysiglinewithargsret{\sphinxbfcode{\sphinxupquote{get\_menu\_by\_path}}}{\emph{\DUrole{n}{path}}}{}
Returns a menu by its file path, which is used as the key
for accessing the menus attribute normally.
\begin{quote}\begin{description}
\item[{Parameters}] \leavevmode
\sphinxstyleliteralstrong{\sphinxupquote{path}} (\sphinxstyleliteralemphasis{\sphinxupquote{str}}) \textendash{} file path of a menu csv file

\item[{Returns}] \leavevmode
Menu instance that is mapped to given path

\item[{Return type}] \leavevmode
{\hyperref[\detokenize{polo.utils:polo.utils.io_utils.Menu}]{\sphinxcrossref{Menu}}}

\end{description}\end{quote}

\end{fulllineitems}

\index{get\_menus\_by\_type() (polo.utils.io\_utils.BarTender method)@\spxentry{get\_menus\_by\_type()}\spxextra{polo.utils.io\_utils.BarTender method}}

\begin{fulllineitems}
\phantomsection\label{\detokenize{polo.utils:polo.utils.io_utils.BarTender.get_menus_by_type}}\pysiglinewithargsret{\sphinxbfcode{\sphinxupquote{get\_menus\_by\_type}}}{\emph{\DUrole{n}{type\_}}}{}
Returns all menus of a given screen type.

‘s’ for soluble screens and ‘m’ for membrane screens. No other
characters should be passed to \sphinxtitleref{type\_}.
\begin{quote}\begin{description}
\item[{Parameters}] \leavevmode
\sphinxstyleliteralstrong{\sphinxupquote{type}} (\sphinxstyleliteralemphasis{\sphinxupquote{str}}\sphinxstyleliteralemphasis{\sphinxupquote{ (}}\sphinxstyleliteralemphasis{\sphinxupquote{max length 1}}\sphinxstyleliteralemphasis{\sphinxupquote{)}}) \textendash{} Key for type of screen to return

\item[{Returns}] \leavevmode
list of menus of that screen type

\item[{Return type}] \leavevmode
list

\end{description}\end{quote}

\end{fulllineitems}


\end{fulllineitems}

\index{CocktailMenuReader (class in polo.utils.io\_utils)@\spxentry{CocktailMenuReader}\spxextra{class in polo.utils.io\_utils}}

\begin{fulllineitems}
\phantomsection\label{\detokenize{polo.utils:polo.utils.io_utils.CocktailMenuReader}}\pysiglinewithargsret{\sphinxbfcode{\sphinxupquote{class }}\sphinxcode{\sphinxupquote{polo.utils.io\_utils.}}\sphinxbfcode{\sphinxupquote{CocktailMenuReader}}}{\emph{\DUrole{n}{menu\_file}}, \emph{\DUrole{n}{delim}\DUrole{o}{=}\DUrole{default_value}{\textquotesingle{},\textquotesingle{}}}, \emph{\DUrole{o}{**}\DUrole{n}{kwargs}}}{}
Bases: \sphinxcode{\sphinxupquote{object}}
\index{cocktail\_map (polo.utils.io\_utils.CocktailMenuReader attribute)@\spxentry{cocktail\_map}\spxextra{polo.utils.io\_utils.CocktailMenuReader attribute}}

\begin{fulllineitems}
\phantomsection\label{\detokenize{polo.utils:polo.utils.io_utils.CocktailMenuReader.cocktail_map}}\pysigline{\sphinxbfcode{\sphinxupquote{cocktail\_map}}\sphinxbfcode{\sphinxupquote{ = \{0: \textquotesingle{}well\_assignment\textquotesingle{}, 1: \textquotesingle{}number\textquotesingle{}, 2: \textquotesingle{}commercial\_code\textquotesingle{}, 8: \textquotesingle{}pH\textquotesingle{}\}}}}
\end{fulllineitems}

\index{formula\_pos (polo.utils.io\_utils.CocktailMenuReader attribute)@\spxentry{formula\_pos}\spxextra{polo.utils.io\_utils.CocktailMenuReader attribute}}

\begin{fulllineitems}
\phantomsection\label{\detokenize{polo.utils:polo.utils.io_utils.CocktailMenuReader.formula_pos}}\pysigline{\sphinxbfcode{\sphinxupquote{formula\_pos}}\sphinxbfcode{\sphinxupquote{ = 4}}}
\end{fulllineitems}

\index{menu\_file() (polo.utils.io\_utils.CocktailMenuReader property)@\spxentry{menu\_file()}\spxextra{polo.utils.io\_utils.CocktailMenuReader property}}

\begin{fulllineitems}
\phantomsection\label{\detokenize{polo.utils:polo.utils.io_utils.CocktailMenuReader.menu_file}}\pysigline{\sphinxbfcode{\sphinxupquote{property }}\sphinxbfcode{\sphinxupquote{menu\_file}}}
Get the instances menu file
\begin{quote}\begin{description}
\item[{Returns}] \leavevmode
the menu file path

\item[{Return type}] \leavevmode
str or Path or IO

\end{description}\end{quote}

\end{fulllineitems}

\index{set\_cocktail\_map() (polo.utils.io\_utils.CocktailMenuReader class method)@\spxentry{set\_cocktail\_map()}\spxextra{polo.utils.io\_utils.CocktailMenuReader class method}}

\begin{fulllineitems}
\phantomsection\label{\detokenize{polo.utils:polo.utils.io_utils.CocktailMenuReader.set_cocktail_map}}\pysiglinewithargsret{\sphinxbfcode{\sphinxupquote{classmethod }}\sphinxbfcode{\sphinxupquote{set\_cocktail\_map}}}{\emph{\DUrole{n}{map}}}{}
Classmethod to edit the cocktail\_map. The cocktail map describes
where Cocktail level information is stored in a given cocktail menu
file row. It is a dictionary that maps specific indicies in a row to
the Cocktail attribute to set the value of the key index to.

The default cocktail\_map dictionary is below.

cocktail\_map = \{
0: ‘well\_assignment’,
1: ‘number’,
8: ‘pH’,
2: ‘commercial\_code’
\}

This tells instances of CocktailMenuReader to look at index 0 of a row
for the well\_assignment attribute of the Cocktail class, index 1 for
the number attribute of the Cocktail class, etc.
\begin{quote}\begin{description}
\item[{Parameters}] \leavevmode
\sphinxstyleliteralstrong{\sphinxupquote{map}} (\sphinxstyleliteralemphasis{\sphinxupquote{dict}}) \textendash{} Dictionary mapping csv row indicies to Cocktail object
attributes

\end{description}\end{quote}

\end{fulllineitems}

\index{set\_formula\_pos() (polo.utils.io\_utils.CocktailMenuReader class method)@\spxentry{set\_formula\_pos()}\spxextra{polo.utils.io\_utils.CocktailMenuReader class method}}

\begin{fulllineitems}
\phantomsection\label{\detokenize{polo.utils:polo.utils.io_utils.CocktailMenuReader.set_formula_pos}}\pysiglinewithargsret{\sphinxbfcode{\sphinxupquote{classmethod }}\sphinxbfcode{\sphinxupquote{set\_formula\_pos}}}{\emph{\DUrole{n}{pos}}}{}
Classmethod to change the formula\_pos attribute. The formula\_pos
describes the location (base 0) of the chemical formula in a row of
a cocktail menu file csv. For some reason, HWI cocktail menu files
will only have one chemical formula per row (cocktail) no matter
the number of reagents that composite that cocktail. This is why
its location is represented using an int instead of a dict.

Generally, formula\_pos should not be changed without a very good
reason as the position of the chemical formula is consistent across
all HWI cocktail menu files.
\begin{quote}\begin{description}
\item[{Parameters}] \leavevmode
\sphinxstyleliteralstrong{\sphinxupquote{pos}} (\sphinxstyleliteralemphasis{\sphinxupquote{int}}) \textendash{} Index where chemical formula can be found

\end{description}\end{quote}

\end{fulllineitems}


\end{fulllineitems}

\index{HtmlWriter (class in polo.utils.io\_utils)@\spxentry{HtmlWriter}\spxextra{class in polo.utils.io\_utils}}

\begin{fulllineitems}
\phantomsection\label{\detokenize{polo.utils:polo.utils.io_utils.HtmlWriter}}\pysiglinewithargsret{\sphinxbfcode{\sphinxupquote{class }}\sphinxcode{\sphinxupquote{polo.utils.io\_utils.}}\sphinxbfcode{\sphinxupquote{HtmlWriter}}}{\emph{\DUrole{n}{run}}, \emph{\DUrole{o}{**}\DUrole{n}{kwargs}}}{}
Bases: {\hyperref[\detokenize{polo.utils:polo.utils.io_utils.RunSerializer}]{\sphinxcrossref{\sphinxcode{\sphinxupquote{polo.utils.io\_utils.RunSerializer}}}}}
\index{finished\_writing() (polo.utils.io\_utils.HtmlWriter method)@\spxentry{finished\_writing()}\spxextra{polo.utils.io\_utils.HtmlWriter method}}

\begin{fulllineitems}
\phantomsection\label{\detokenize{polo.utils:polo.utils.io_utils.HtmlWriter.finished_writing}}\pysiglinewithargsret{\sphinxbfcode{\sphinxupquote{finished\_writing}}}{}{}
Method to connect to Qthread instance. Should not be called from
anything other than a Qthread instance.

\end{fulllineitems}

\index{make\_template() (polo.utils.io\_utils.HtmlWriter static method)@\spxentry{make\_template()}\spxextra{polo.utils.io\_utils.HtmlWriter static method}}

\begin{fulllineitems}
\phantomsection\label{\detokenize{polo.utils:polo.utils.io_utils.HtmlWriter.make_template}}\pysiglinewithargsret{\sphinxbfcode{\sphinxupquote{static }}\sphinxbfcode{\sphinxupquote{make\_template}}}{\emph{\DUrole{n}{template\_path}}}{}
Given a path to an html file to serve as a jinja2 template, read the
file and create a new template object.

\end{fulllineitems}

\index{write\_complete\_run() (polo.utils.io\_utils.HtmlWriter method)@\spxentry{write\_complete\_run()}\spxextra{polo.utils.io\_utils.HtmlWriter method}}

\begin{fulllineitems}
\phantomsection\label{\detokenize{polo.utils:polo.utils.io_utils.HtmlWriter.write_complete_run}}\pysiglinewithargsret{\sphinxbfcode{\sphinxupquote{write\_complete\_run}}}{\emph{\DUrole{n}{output\_path}}, \emph{\DUrole{n}{encode\_images}\DUrole{o}{=}\DUrole{default_value}{True}}}{}
\end{fulllineitems}

\index{write\_complete\_run\_on\_thread() (polo.utils.io\_utils.HtmlWriter method)@\spxentry{write\_complete\_run\_on\_thread()}\spxextra{polo.utils.io\_utils.HtmlWriter method}}

\begin{fulllineitems}
\phantomsection\label{\detokenize{polo.utils:polo.utils.io_utils.HtmlWriter.write_complete_run_on_thread}}\pysiglinewithargsret{\sphinxbfcode{\sphinxupquote{write\_complete\_run\_on\_thread}}}{\emph{\DUrole{n}{output\_path}}, \emph{\DUrole{n}{encode\_images}}}{}
Wrapper around \sphinxtitleref{write\_complete\_run} that executes on a seperate
Qthread.

\end{fulllineitems}

\index{write\_grid\_screen() (polo.utils.io\_utils.HtmlWriter method)@\spxentry{write\_grid\_screen()}\spxextra{polo.utils.io\_utils.HtmlWriter method}}

\begin{fulllineitems}
\phantomsection\label{\detokenize{polo.utils:polo.utils.io_utils.HtmlWriter.write_grid_screen}}\pysiglinewithargsret{\sphinxbfcode{\sphinxupquote{write\_grid\_screen}}}{\emph{\DUrole{n}{output\_path}}, \emph{\DUrole{n}{plate\_list}}, \emph{\DUrole{n}{well\_number}}, \emph{\DUrole{n}{x\_reagent}}, \emph{\DUrole{n}{y\_reagent}}, \emph{\DUrole{n}{well\_volume}}, \emph{\DUrole{n}{run\_name}\DUrole{o}{=}\DUrole{default_value}{None}}}{}
Write the contents of optimization grid screen to an html file
\begin{quote}\begin{description}
\item[{Parameters}] \leavevmode\begin{itemize}
\item {} 
\sphinxstyleliteralstrong{\sphinxupquote{output\_path}} (\sphinxstyleliteralemphasis{\sphinxupquote{str}}) \textendash{} Path to html file

\item {} 
\sphinxstyleliteralstrong{\sphinxupquote{plate\_list}} (\sphinxstyleliteralemphasis{\sphinxupquote{list}}) \textendash{} list containing grid screen data

\item {} 
\sphinxstyleliteralstrong{\sphinxupquote{well\_number}} (\sphinxstyleliteralemphasis{\sphinxupquote{int}}\sphinxstyleliteralemphasis{\sphinxupquote{ or }}\sphinxstyleliteralemphasis{\sphinxupquote{str}}) \textendash{} well number of hit screen is created from

\item {} 
\sphinxstyleliteralstrong{\sphinxupquote{x\_reagent}} ({\hyperref[\detokenize{polo.crystallography:polo.crystallography.cocktail.Reagent}]{\sphinxcrossref{\sphinxstyleliteralemphasis{\sphinxupquote{Reagent}}}}}) \textendash{} reagent varried in x direction

\item {} 
\sphinxstyleliteralstrong{\sphinxupquote{y\_reagent}} ({\hyperref[\detokenize{polo.crystallography:polo.crystallography.cocktail.Reagent}]{\sphinxcrossref{\sphinxstyleliteralemphasis{\sphinxupquote{Reagent}}}}}) \textendash{} reagent varried in y direction

\item {} 
\sphinxstyleliteralstrong{\sphinxupquote{well\_volume}} (\sphinxstyleliteralemphasis{\sphinxupquote{int}}\sphinxstyleliteralemphasis{\sphinxupquote{ or }}\sphinxstyleliteralemphasis{\sphinxupquote{str}}) \textendash{} Volume of well used in screen

\item {} 
\sphinxstyleliteralstrong{\sphinxupquote{run\_name}} (\sphinxstyleliteralemphasis{\sphinxupquote{str}}\sphinxstyleliteralemphasis{\sphinxupquote{, }}\sphinxstyleliteralemphasis{\sphinxupquote{optional}}) \textendash{} name of run, defaults to None

\end{itemize}

\end{description}\end{quote}

\end{fulllineitems}


\end{fulllineitems}

\index{Menu (class in polo.utils.io\_utils)@\spxentry{Menu}\spxextra{class in polo.utils.io\_utils}}

\begin{fulllineitems}
\phantomsection\label{\detokenize{polo.utils:polo.utils.io_utils.Menu}}\pysiglinewithargsret{\sphinxbfcode{\sphinxupquote{class }}\sphinxcode{\sphinxupquote{polo.utils.io\_utils.}}\sphinxbfcode{\sphinxupquote{Menu}}}{\emph{\DUrole{n}{path}}, \emph{\DUrole{n}{start\_date}}, \emph{\DUrole{n}{end\_date}}, \emph{\DUrole{n}{type\_}}}{}
Bases: \sphinxcode{\sphinxupquote{object}}
\index{cocktails() (polo.utils.io\_utils.Menu property)@\spxentry{cocktails()}\spxextra{polo.utils.io\_utils.Menu property}}

\begin{fulllineitems}
\phantomsection\label{\detokenize{polo.utils:polo.utils.io_utils.Menu.cocktails}}\pysigline{\sphinxbfcode{\sphinxupquote{property }}\sphinxbfcode{\sphinxupquote{cocktails}}}
Property to return the Menu instance’s cocktail dict
\begin{quote}\begin{description}
\item[{Returns}] \leavevmode
cocktail attribute

\item[{Return type}] \leavevmode
dict

\end{description}\end{quote}

\end{fulllineitems}

\index{path() (polo.utils.io\_utils.Menu property)@\spxentry{path()}\spxextra{polo.utils.io\_utils.Menu property}}

\begin{fulllineitems}
\phantomsection\label{\detokenize{polo.utils:polo.utils.io_utils.Menu.path}}\pysigline{\sphinxbfcode{\sphinxupquote{property }}\sphinxbfcode{\sphinxupquote{path}}}
Property to return the Menu instance’s path attribute
\begin{quote}\begin{description}
\item[{Returns}] \leavevmode
The path attribute

\item[{Return type}] \leavevmode
str or IO

\end{description}\end{quote}

\end{fulllineitems}


\end{fulllineitems}

\index{RunDeserializer (class in polo.utils.io\_utils)@\spxentry{RunDeserializer}\spxextra{class in polo.utils.io\_utils}}

\begin{fulllineitems}
\phantomsection\label{\detokenize{polo.utils:polo.utils.io_utils.RunDeserializer}}\pysiglinewithargsret{\sphinxbfcode{\sphinxupquote{class }}\sphinxcode{\sphinxupquote{polo.utils.io\_utils.}}\sphinxbfcode{\sphinxupquote{RunDeserializer}}}{\emph{\DUrole{n}{xtal\_path}}}{}
Bases: \sphinxcode{\sphinxupquote{object}}
\index{clean\_base64\_string() (polo.utils.io\_utils.RunDeserializer static method)@\spxentry{clean\_base64\_string()}\spxextra{polo.utils.io\_utils.RunDeserializer static method}}

\begin{fulllineitems}
\phantomsection\label{\detokenize{polo.utils:polo.utils.io_utils.RunDeserializer.clean_base64_string}}\pysiglinewithargsret{\sphinxbfcode{\sphinxupquote{static }}\sphinxbfcode{\sphinxupquote{clean\_base64\_string}}}{\emph{\DUrole{n}{string}}}{}
Image instances may contain byte strings that store their actual
crystallization image encoded as base64. Previously, these byte strings
were written directly into the json file as strings causing the b’
byte string identifier to be written along with the actual base64 data.
This method removes those artifacts if they are present and returns a
clean byte string with only the actual base64 data.
\begin{quote}\begin{description}
\item[{Parameters}] \leavevmode
\sphinxstyleliteralstrong{\sphinxupquote{string}} (\sphinxstyleliteralemphasis{\sphinxupquote{str}}) \textendash{} a string to interogate

\item[{Returns}] \leavevmode
byte string with non\sphinxhyphen{}data artifacts removed

\item[{Return type}] \leavevmode
bytes

\end{description}\end{quote}

\end{fulllineitems}

\index{dict\_to\_obj() (polo.utils.io\_utils.RunDeserializer static method)@\spxentry{dict\_to\_obj()}\spxextra{polo.utils.io\_utils.RunDeserializer static method}}

\begin{fulllineitems}
\phantomsection\label{\detokenize{polo.utils:polo.utils.io_utils.RunDeserializer.dict_to_obj}}\pysiglinewithargsret{\sphinxbfcode{\sphinxupquote{static }}\sphinxbfcode{\sphinxupquote{dict\_to\_obj}}}{\emph{\DUrole{n}{our\_dict}}}{}
Oposite of the obj\_to\_dict method in XtalWriter class, this method
takes a dictionary instance that has been previously serialized and
attempts to convert it back into an object instance.
\begin{quote}\begin{description}
\item[{Parameters}] \leavevmode
\sphinxstyleliteralstrong{\sphinxupquote{our\_dict}} (\sphinxstyleliteralemphasis{\sphinxupquote{dict}}) \textendash{} dictionary to convert back to object

\item[{Returns}] \leavevmode
an object

\item[{Return type}] \leavevmode
object

\end{description}\end{quote}

\end{fulllineitems}

\index{xtal\_header\_reader() (polo.utils.io\_utils.RunDeserializer method)@\spxentry{xtal\_header\_reader()}\spxextra{polo.utils.io\_utils.RunDeserializer method}}

\begin{fulllineitems}
\phantomsection\label{\detokenize{polo.utils:polo.utils.io_utils.RunDeserializer.xtal_header_reader}}\pysiglinewithargsret{\sphinxbfcode{\sphinxupquote{xtal\_header\_reader}}}{\emph{\DUrole{n}{xtal\_file\_io}}}{}
Reads the header section of an open xtal file. Should always be
called before reading the json content of an xtal file. Note than
xtal files must always have a line of equal signs before the json
content even if there is no header content otherwise this method will
read one line into the json content causing the json reader to
throw an error.
\begin{quote}\begin{description}
\item[{Parameters}] \leavevmode
\sphinxstyleliteralstrong{\sphinxupquote{xtal\_file\_io}} (\sphinxstyleliteralemphasis{\sphinxupquote{TextIoWrapper}}) \textendash{} xtal file currently being read

\item[{Returns}] \leavevmode
xtal header contents

\item[{Return type}] \leavevmode
list

\end{description}\end{quote}

\end{fulllineitems}

\index{xtal\_to\_run() (polo.utils.io\_utils.RunDeserializer method)@\spxentry{xtal\_to\_run()}\spxextra{polo.utils.io\_utils.RunDeserializer method}}

\begin{fulllineitems}
\phantomsection\label{\detokenize{polo.utils:polo.utils.io_utils.RunDeserializer.xtal_to_run}}\pysiglinewithargsret{\sphinxbfcode{\sphinxupquote{xtal\_to\_run}}}{}{}
Method that actually does the heavy lifting of converting the json
contents of xtal files back into Run instances. Currently is pretty
brittle and ugly so looking for a more flexible recursive solution.
\begin{quote}\begin{description}
\item[{Raises}] \leavevmode
\sphinxstyleliteralstrong{\sphinxupquote{e}} \textendash{} Error raised while reading the xtal file

\item[{Returns}] \leavevmode
The contents of xtal file as a Run instance

\item[{Return type}] \leavevmode
{\hyperref[\detokenize{polo.crystallography:polo.crystallography.run.Run}]{\sphinxcrossref{Run}}}

\end{description}\end{quote}

\end{fulllineitems}


\end{fulllineitems}

\index{RunSerializer (class in polo.utils.io\_utils)@\spxentry{RunSerializer}\spxextra{class in polo.utils.io\_utils}}

\begin{fulllineitems}
\phantomsection\label{\detokenize{polo.utils:polo.utils.io_utils.RunSerializer}}\pysiglinewithargsret{\sphinxbfcode{\sphinxupquote{class }}\sphinxcode{\sphinxupquote{polo.utils.io\_utils.}}\sphinxbfcode{\sphinxupquote{RunSerializer}}}{\emph{\DUrole{n}{run}}}{}
Bases: \sphinxcode{\sphinxupquote{object}}
\index{make\_message\_box() (polo.utils.io\_utils.RunSerializer static method)@\spxentry{make\_message\_box()}\spxextra{polo.utils.io\_utils.RunSerializer static method}}

\begin{fulllineitems}
\phantomsection\label{\detokenize{polo.utils:polo.utils.io_utils.RunSerializer.make_message_box}}\pysiglinewithargsret{\sphinxbfcode{\sphinxupquote{static }}\sphinxbfcode{\sphinxupquote{make\_message\_box}}}{\emph{\DUrole{n}{message}}, \emph{\DUrole{n}{icon}\DUrole{o}{=}\DUrole{default_value}{1}}, \emph{\DUrole{n}{buttons}\DUrole{o}{=}\DUrole{default_value}{1024}}, \emph{\DUrole{n}{connected\_function}\DUrole{o}{=}\DUrole{default_value}{None}}}{}
Return a QMessageBox instance to show to the user.
\begin{quote}\begin{description}
\item[{Parameters}] \leavevmode\begin{itemize}
\item {} 
\sphinxstyleliteralstrong{\sphinxupquote{message}} (\sphinxstyleliteralemphasis{\sphinxupquote{str}}) \textendash{} Message to be displayed to the user

\item {} 
\sphinxstyleliteralstrong{\sphinxupquote{icon}} (\sphinxstyleliteralemphasis{\sphinxupquote{QMessageBoxIcon}}\sphinxstyleliteralemphasis{\sphinxupquote{, }}\sphinxstyleliteralemphasis{\sphinxupquote{optional}}) \textendash{} Icon for message box, defaults to QtWidgets.QMessageBox.Information

\item {} 
\sphinxstyleliteralstrong{\sphinxupquote{buttons}} (\sphinxstyleliteralemphasis{\sphinxupquote{{[}}}\sphinxstyleliteralemphasis{\sphinxupquote{type}}\sphinxstyleliteralemphasis{\sphinxupquote{{]}}}\sphinxstyleliteralemphasis{\sphinxupquote{, }}\sphinxstyleliteralemphasis{\sphinxupquote{optional}}) \textendash{} {[}description{]}, defaults to QtWidgets.QMessageBox.Ok

\item {} 
\sphinxstyleliteralstrong{\sphinxupquote{connected\_function}} (\sphinxstyleliteralemphasis{\sphinxupquote{{[}}}\sphinxstyleliteralemphasis{\sphinxupquote{type}}\sphinxstyleliteralemphasis{\sphinxupquote{{]}}}\sphinxstyleliteralemphasis{\sphinxupquote{, }}\sphinxstyleliteralemphasis{\sphinxupquote{optional}}) \textendash{} {[}description{]}, defaults to None

\end{itemize}

\item[{Returns}] \leavevmode
{[}description{]}

\item[{Return type}] \leavevmode
{[}type{]}

\end{description}\end{quote}

\end{fulllineitems}

\index{make\_thread() (polo.utils.io\_utils.RunSerializer class method)@\spxentry{make\_thread()}\spxextra{polo.utils.io\_utils.RunSerializer class method}}

\begin{fulllineitems}
\phantomsection\label{\detokenize{polo.utils:polo.utils.io_utils.RunSerializer.make_thread}}\pysiglinewithargsret{\sphinxbfcode{\sphinxupquote{classmethod }}\sphinxbfcode{\sphinxupquote{make\_thread}}}{\emph{\DUrole{n}{job\_function}}, \emph{\DUrole{o}{**}\DUrole{n}{kwargs}}}{}
Creates a new qthread object. The job function is the
function the thread will execute and and arguements that the job
function requires should be passed has keyword arguements. These are
stored as a dictionary in the new thread object until the thread is
activated and they are passed as arguements.

\end{fulllineitems}

\index{path\_suffix\_checker() (polo.utils.io\_utils.RunSerializer static method)@\spxentry{path\_suffix\_checker()}\spxextra{polo.utils.io\_utils.RunSerializer static method}}

\begin{fulllineitems}
\phantomsection\label{\detokenize{polo.utils:polo.utils.io_utils.RunSerializer.path_suffix_checker}}\pysiglinewithargsret{\sphinxbfcode{\sphinxupquote{static }}\sphinxbfcode{\sphinxupquote{path\_suffix\_checker}}}{\emph{\DUrole{n}{path}}, \emph{\DUrole{n}{desired\_suffix}}}{}
Check is a file path has a desired suffix, if not then replace the
current suffix with the desired suffix. Useful for checking filenames
that are taken from user input.
\begin{quote}\begin{description}
\item[{Parameters}] \leavevmode
\sphinxstyleliteralstrong{\sphinxupquote{desired\_suffix}} \textendash{} File extension for given file path.

\end{description}\end{quote}

\end{fulllineitems}

\index{path\_validator() (polo.utils.io\_utils.RunSerializer static method)@\spxentry{path\_validator()}\spxextra{polo.utils.io\_utils.RunSerializer static method}}

\begin{fulllineitems}
\phantomsection\label{\detokenize{polo.utils:polo.utils.io_utils.RunSerializer.path_validator}}\pysiglinewithargsret{\sphinxbfcode{\sphinxupquote{static }}\sphinxbfcode{\sphinxupquote{path\_validator}}}{\emph{\DUrole{n}{path}}, \emph{\DUrole{n}{parent}\DUrole{o}{=}\DUrole{default_value}{False}}}{}
Tests to ensure a path exists. Passing parent = True will check for
the existance of the parent directory of the path.

\end{fulllineitems}


\end{fulllineitems}

\index{XtalWriter (class in polo.utils.io\_utils)@\spxentry{XtalWriter}\spxextra{class in polo.utils.io\_utils}}

\begin{fulllineitems}
\phantomsection\label{\detokenize{polo.utils:polo.utils.io_utils.XtalWriter}}\pysiglinewithargsret{\sphinxbfcode{\sphinxupquote{class }}\sphinxcode{\sphinxupquote{polo.utils.io\_utils.}}\sphinxbfcode{\sphinxupquote{XtalWriter}}}{\emph{\DUrole{n}{run}}, \emph{\DUrole{o}{**}\DUrole{n}{kwargs}}}{}
Bases: {\hyperref[\detokenize{polo.utils:polo.utils.io_utils.RunSerializer}]{\sphinxcrossref{\sphinxcode{\sphinxupquote{polo.utils.io\_utils.RunSerializer}}}}}
\index{clean\_run\_for\_save() (polo.utils.io\_utils.XtalWriter method)@\spxentry{clean\_run\_for\_save()}\spxextra{polo.utils.io\_utils.XtalWriter method}}

\begin{fulllineitems}
\phantomsection\label{\detokenize{polo.utils:polo.utils.io_utils.XtalWriter.clean_run_for_save}}\pysiglinewithargsret{\sphinxbfcode{\sphinxupquote{clean\_run\_for\_save}}}{}{}
Remove circular references from a Run instance to avoid errors
when serializing using json. Uses the run stored in the run attribute
\begin{quote}\begin{description}
\item[{Returns}] \leavevmode
The cleaned run

\item[{Return type}] \leavevmode
{\hyperref[\detokenize{polo.crystallography:polo.crystallography.run.Run}]{\sphinxcrossref{Run}}}

\end{description}\end{quote}

\end{fulllineitems}

\index{file\_ext (polo.utils.io\_utils.XtalWriter attribute)@\spxentry{file\_ext}\spxextra{polo.utils.io\_utils.XtalWriter attribute}}

\begin{fulllineitems}
\phantomsection\label{\detokenize{polo.utils:polo.utils.io_utils.XtalWriter.file_ext}}\pysigline{\sphinxbfcode{\sphinxupquote{file\_ext}}\sphinxbfcode{\sphinxupquote{ = \textquotesingle{}.xtal\textquotesingle{}}}}
\end{fulllineitems}

\index{finished\_save() (polo.utils.io\_utils.XtalWriter method)@\spxentry{finished\_save()}\spxextra{polo.utils.io\_utils.XtalWriter method}}

\begin{fulllineitems}
\phantomsection\label{\detokenize{polo.utils:polo.utils.io_utils.XtalWriter.finished_save}}\pysiglinewithargsret{\sphinxbfcode{\sphinxupquote{finished\_save}}}{}{}
\end{fulllineitems}

\index{header\_flag (polo.utils.io\_utils.XtalWriter attribute)@\spxentry{header\_flag}\spxextra{polo.utils.io\_utils.XtalWriter attribute}}

\begin{fulllineitems}
\phantomsection\label{\detokenize{polo.utils:polo.utils.io_utils.XtalWriter.header_flag}}\pysigline{\sphinxbfcode{\sphinxupquote{header\_flag}}\sphinxbfcode{\sphinxupquote{ = \textquotesingle{}\textless{}\textgreater{}\textquotesingle{}}}}
\end{fulllineitems}

\index{header\_line (polo.utils.io\_utils.XtalWriter attribute)@\spxentry{header\_line}\spxextra{polo.utils.io\_utils.XtalWriter attribute}}

\begin{fulllineitems}
\phantomsection\label{\detokenize{polo.utils:polo.utils.io_utils.XtalWriter.header_line}}\pysigline{\sphinxbfcode{\sphinxupquote{header\_line}}\sphinxbfcode{\sphinxupquote{ = \textquotesingle{}\{\}\{\}:\{\}.unitsn\textquotesingle{}}}}
\end{fulllineitems}

\index{json\_encoder() (polo.utils.io\_utils.XtalWriter static method)@\spxentry{json\_encoder()}\spxextra{polo.utils.io\_utils.XtalWriter static method}}

\begin{fulllineitems}
\phantomsection\label{\detokenize{polo.utils:polo.utils.io_utils.XtalWriter.json_encoder}}\pysiglinewithargsret{\sphinxbfcode{\sphinxupquote{static }}\sphinxbfcode{\sphinxupquote{json\_encoder}}}{\emph{\DUrole{n}{obj}}}{}
Use instead of the defauly json encoder. If the encoded object
is from a module within Polo will include a module and class
identifier so it can be more easily deserialized when loaded
back into the program.
\begin{quote}\begin{description}
\item[{Param}] \leavevmode
obj: An object to serialize to json.

\item[{Returns}] \leavevmode
A dictionary or string version of passed object

\end{description}\end{quote}

\end{fulllineitems}

\index{run\_to\_dict() (polo.utils.io\_utils.XtalWriter method)@\spxentry{run\_to\_dict()}\spxextra{polo.utils.io\_utils.XtalWriter method}}

\begin{fulllineitems}
\phantomsection\label{\detokenize{polo.utils:polo.utils.io_utils.XtalWriter.run_to_dict}}\pysiglinewithargsret{\sphinxbfcode{\sphinxupquote{run\_to\_dict}}}{}{}
Create a json string from the run stored in the run attribute.
\begin{quote}\begin{description}
\item[{Returns}] \leavevmode
Run instance serialized to json

\item[{Return type}] \leavevmode
str

\end{description}\end{quote}

\end{fulllineitems}

\index{write\_xtal\_file() (polo.utils.io\_utils.XtalWriter method)@\spxentry{write\_xtal\_file()}\spxextra{polo.utils.io\_utils.XtalWriter method}}

\begin{fulllineitems}
\phantomsection\label{\detokenize{polo.utils:polo.utils.io_utils.XtalWriter.write_xtal_file}}\pysiglinewithargsret{\sphinxbfcode{\sphinxupquote{write\_xtal\_file}}}{\emph{\DUrole{n}{output\_path}}}{}
Method to serialize run object to xtal file format.
\begin{quote}\begin{description}
\item[{Parameters}] \leavevmode
\sphinxstyleliteralstrong{\sphinxupquote{output\_path}} (\sphinxstyleliteralemphasis{\sphinxupquote{str}}) \textendash{} Xtal file path

\item[{Returns}] \leavevmode
path to xtal file

\item[{Return type}] \leavevmode
str

\end{description}\end{quote}

\end{fulllineitems}

\index{write\_xtal\_file\_on\_thread() (polo.utils.io\_utils.XtalWriter method)@\spxentry{write\_xtal\_file\_on\_thread()}\spxextra{polo.utils.io\_utils.XtalWriter method}}

\begin{fulllineitems}
\phantomsection\label{\detokenize{polo.utils:polo.utils.io_utils.XtalWriter.write_xtal_file_on_thread}}\pysiglinewithargsret{\sphinxbfcode{\sphinxupquote{write\_xtal\_file\_on\_thread}}}{\emph{\DUrole{n}{output\_path}}}{}
Wrapper method around \sphinxtitleref{write\_xtal\_file} that executes on a Qthread
instance to prevent freezing the GUI when saving large xtal files
\begin{quote}\begin{description}
\item[{Parameters}] \leavevmode
\sphinxstyleliteralstrong{\sphinxupquote{output\_path}} (\sphinxstyleliteralemphasis{\sphinxupquote{str}}) \textendash{} Path to xtal file

\end{description}\end{quote}

\end{fulllineitems}

\index{xtal\_header() (polo.utils.io\_utils.XtalWriter property)@\spxentry{xtal\_header()}\spxextra{polo.utils.io\_utils.XtalWriter property}}

\begin{fulllineitems}
\phantomsection\label{\detokenize{polo.utils:polo.utils.io_utils.XtalWriter.xtal_header}}\pysigline{\sphinxbfcode{\sphinxupquote{property }}\sphinxbfcode{\sphinxupquote{xtal\_header}}}
Creates the header for xtal file when called. Header lines are
indicated as such by the string in the header\_line constant,
which should be ‘\textless{}\textgreater{}’. The last line of the header will be a row
of equal signs and then the actual json content begins on the
next line.

\end{fulllineitems}


\end{fulllineitems}

\index{check\_for\_missing\_images() (in module polo.utils.io\_utils)@\spxentry{check\_for\_missing\_images()}\spxextra{in module polo.utils.io\_utils}}

\begin{fulllineitems}
\phantomsection\label{\detokenize{polo.utils:polo.utils.io_utils.check_for_missing_images}}\pysiglinewithargsret{\sphinxcode{\sphinxupquote{polo.utils.io\_utils.}}\sphinxbfcode{\sphinxupquote{check\_for\_missing\_images}}}{\emph{\DUrole{n}{dir\_path}}, \emph{\DUrole{n}{expected\_image\_count}}}{}
\end{fulllineitems}

\index{datetime\_converter() (in module polo.utils.io\_utils)@\spxentry{datetime\_converter()}\spxextra{in module polo.utils.io\_utils}}

\begin{fulllineitems}
\phantomsection\label{\detokenize{polo.utils:polo.utils.io_utils.datetime_converter}}\pysiglinewithargsret{\sphinxcode{\sphinxupquote{polo.utils.io\_utils.}}\sphinxbfcode{\sphinxupquote{datetime\_converter}}}{\emph{\DUrole{n}{date\_string}}}{}
\end{fulllineitems}

\index{directory\_validator() (in module polo.utils.io\_utils)@\spxentry{directory\_validator()}\spxextra{in module polo.utils.io\_utils}}

\begin{fulllineitems}
\phantomsection\label{\detokenize{polo.utils:polo.utils.io_utils.directory_validator}}\pysiglinewithargsret{\sphinxcode{\sphinxupquote{polo.utils.io\_utils.}}\sphinxbfcode{\sphinxupquote{directory\_validator}}}{\emph{\DUrole{n}{dir\_path}}}{}
\end{fulllineitems}

\index{export\_run\_to\_csv() (in module polo.utils.io\_utils)@\spxentry{export\_run\_to\_csv()}\spxextra{in module polo.utils.io\_utils}}

\begin{fulllineitems}
\phantomsection\label{\detokenize{polo.utils:polo.utils.io_utils.export_run_to_csv}}\pysiglinewithargsret{\sphinxcode{\sphinxupquote{polo.utils.io\_utils.}}\sphinxbfcode{\sphinxupquote{export\_run\_to\_csv}}}{\emph{\DUrole{n}{run}}, \emph{\DUrole{n}{output\_path}}}{}
\end{fulllineitems}

\index{get\_available\_cocktails() (in module polo.utils.io\_utils)@\spxentry{get\_available\_cocktails()}\spxextra{in module polo.utils.io\_utils}}

\begin{fulllineitems}
\phantomsection\label{\detokenize{polo.utils:polo.utils.io_utils.get_available_cocktails}}\pysiglinewithargsret{\sphinxcode{\sphinxupquote{polo.utils.io\_utils.}}\sphinxbfcode{\sphinxupquote{get\_available\_cocktails}}}{}{}
Returns a list of cocktail csv files that are included with the Polo
program. List is sorted from most recently used cocktails to least 
recently used. The year the cocktails were first used are the first two
characters of the csv file name if downloaded from HWI website.

TODO: Inculde metadata file on available cocktails that gives when they
were used at the screening center. Add another function to read in that
metadata and create dialog so user can select cocktail that would go with
their screen.

NOTE: Cocktail TSV file had unicode error in it when read using UTF\sphinxhyphen{}8
incoding but one on webiste did not. Need to ask Bowman about that one.

\end{fulllineitems}

\index{get\_cocktail\_number\_as\_int() (in module polo.utils.io\_utils)@\spxentry{get\_cocktail\_number\_as\_int()}\spxextra{in module polo.utils.io\_utils}}

\begin{fulllineitems}
\phantomsection\label{\detokenize{polo.utils:polo.utils.io_utils.get_cocktail_number_as_int}}\pysiglinewithargsret{\sphinxcode{\sphinxupquote{polo.utils.io\_utils.}}\sphinxbfcode{\sphinxupquote{get\_cocktail\_number\_as\_int}}}{\emph{\DUrole{n}{cocktail\_number\_string}}}{}
\end{fulllineitems}

\index{if\_dir\_not\_exists\_make() (in module polo.utils.io\_utils)@\spxentry{if\_dir\_not\_exists\_make()}\spxextra{in module polo.utils.io\_utils}}

\begin{fulllineitems}
\phantomsection\label{\detokenize{polo.utils:polo.utils.io_utils.if_dir_not_exists_make}}\pysiglinewithargsret{\sphinxcode{\sphinxupquote{polo.utils.io\_utils.}}\sphinxbfcode{\sphinxupquote{if\_dir\_not\_exists\_make}}}{\emph{\DUrole{n}{parent\_dir}}, \emph{\DUrole{n}{child\_dir}\DUrole{o}{=}\DUrole{default_value}{None}}}{}
If only parent\_dir is given attempts to make that directory. If parent
and child are given tries to make a directory child\_dir within parent dir.

\end{fulllineitems}

\index{list\_dir\_abs() (in module polo.utils.io\_utils)@\spxentry{list\_dir\_abs()}\spxextra{in module polo.utils.io\_utils}}

\begin{fulllineitems}
\phantomsection\label{\detokenize{polo.utils:polo.utils.io_utils.list_dir_abs}}\pysiglinewithargsret{\sphinxcode{\sphinxupquote{polo.utils.io\_utils.}}\sphinxbfcode{\sphinxupquote{list\_dir\_abs}}}{\emph{\DUrole{n}{parent\_dir}}, \emph{\DUrole{n}{allowed}\DUrole{o}{=}\DUrole{default_value}{False}}}{}
\end{fulllineitems}

\index{make\_dict\_from\_run\_via\_json() (in module polo.utils.io\_utils)@\spxentry{make\_dict\_from\_run\_via\_json()}\spxextra{in module polo.utils.io\_utils}}

\begin{fulllineitems}
\phantomsection\label{\detokenize{polo.utils:polo.utils.io_utils.make_dict_from_run_via_json}}\pysiglinewithargsret{\sphinxcode{\sphinxupquote{polo.utils.io\_utils.}}\sphinxbfcode{\sphinxupquote{make\_dict\_from\_run\_via\_json}}}{\emph{\DUrole{n}{run}}}{}
\end{fulllineitems}

\index{parse\_HWI\_filename\_meta() (in module polo.utils.io\_utils)@\spxentry{parse\_HWI\_filename\_meta()}\spxextra{in module polo.utils.io\_utils}}

\begin{fulllineitems}
\phantomsection\label{\detokenize{polo.utils:polo.utils.io_utils.parse_HWI_filename_meta}}\pysiglinewithargsret{\sphinxcode{\sphinxupquote{polo.utils.io\_utils.}}\sphinxbfcode{\sphinxupquote{parse\_HWI\_filename\_meta}}}{\emph{\DUrole{n}{HWI\_image\_file}}}{}
HWI images have a standard file nameing schema that gives info about when
they are taken and well number and that kind of thing. This function returns
that data

\end{fulllineitems}

\index{parse\_cocktail\_csv() (in module polo.utils.io\_utils)@\spxentry{parse\_cocktail\_csv()}\spxextra{in module polo.utils.io\_utils}}

\begin{fulllineitems}
\phantomsection\label{\detokenize{polo.utils:polo.utils.io_utils.parse_cocktail_csv}}\pysiglinewithargsret{\sphinxcode{\sphinxupquote{polo.utils.io\_utils.}}\sphinxbfcode{\sphinxupquote{parse\_cocktail\_csv}}}{\emph{\DUrole{n}{file\_path}}}{}
\end{fulllineitems}

\index{parse\_cocktail\_metadata() (in module polo.utils.io\_utils)@\spxentry{parse\_cocktail\_metadata()}\spxextra{in module polo.utils.io\_utils}}

\begin{fulllineitems}
\phantomsection\label{\detokenize{polo.utils:polo.utils.io_utils.parse_cocktail_metadata}}\pysiglinewithargsret{\sphinxcode{\sphinxupquote{polo.utils.io\_utils.}}\sphinxbfcode{\sphinxupquote{parse\_cocktail\_metadata}}}{}{}
\end{fulllineitems}

\index{parse\_hwi\_dir\_metadata() (in module polo.utils.io\_utils)@\spxentry{parse\_hwi\_dir\_metadata()}\spxextra{in module polo.utils.io\_utils}}

\begin{fulllineitems}
\phantomsection\label{\detokenize{polo.utils:polo.utils.io_utils.parse_hwi_dir_metadata}}\pysiglinewithargsret{\sphinxcode{\sphinxupquote{polo.utils.io\_utils.}}\sphinxbfcode{\sphinxupquote{parse\_hwi\_dir\_metadata}}}{\emph{\DUrole{n}{dir\_name}}}{}
\end{fulllineitems}

\index{parse\_reagents() (in module polo.utils.io\_utils)@\spxentry{parse\_reagents()}\spxextra{in module polo.utils.io\_utils}}

\begin{fulllineitems}
\phantomsection\label{\detokenize{polo.utils:polo.utils.io_utils.parse_reagents}}\pysiglinewithargsret{\sphinxcode{\sphinxupquote{polo.utils.io\_utils.}}\sphinxbfcode{\sphinxupquote{parse\_reagents}}}{\emph{\DUrole{n}{row}}}{}
\end{fulllineitems}

\index{read\_import\_descriptors() (in module polo.utils.io\_utils)@\spxentry{read\_import\_descriptors()}\spxextra{in module polo.utils.io\_utils}}

\begin{fulllineitems}
\phantomsection\label{\detokenize{polo.utils:polo.utils.io_utils.read_import_descriptors}}\pysiglinewithargsret{\sphinxcode{\sphinxupquote{polo.utils.io\_utils.}}\sphinxbfcode{\sphinxupquote{read\_import\_descriptors}}}{}{}
\end{fulllineitems}

\index{run\_name\_validator() (in module polo.utils.io\_utils)@\spxentry{run\_name\_validator()}\spxextra{in module polo.utils.io\_utils}}

\begin{fulllineitems}
\phantomsection\label{\detokenize{polo.utils:polo.utils.io_utils.run_name_validator}}\pysiglinewithargsret{\sphinxcode{\sphinxupquote{polo.utils.io\_utils.}}\sphinxbfcode{\sphinxupquote{run\_name\_validator}}}{\emph{\DUrole{n}{new\_run\_name}}, \emph{\DUrole{n}{current\_run\_names}}}{}
\end{fulllineitems}

\index{write\_screen\_html() (in module polo.utils.io\_utils)@\spxentry{write\_screen\_html()}\spxextra{in module polo.utils.io\_utils}}

\begin{fulllineitems}
\phantomsection\label{\detokenize{polo.utils:polo.utils.io_utils.write_screen_html}}\pysiglinewithargsret{\sphinxcode{\sphinxupquote{polo.utils.io\_utils.}}\sphinxbfcode{\sphinxupquote{write\_screen\_html}}}{\emph{\DUrole{n}{plate\_list}}, \emph{\DUrole{n}{well\_number}}, \emph{\DUrole{n}{run\_name}}, \emph{\DUrole{n}{x\_reagent}}, \emph{\DUrole{n}{y\_reagent}}, \emph{\DUrole{n}{well\_volume}}, \emph{\DUrole{n}{output\_path}}}{}
\end{fulllineitems}



\subsubsection{polo.utils.math\_utils module}
\label{\detokenize{polo.utils:module-polo.utils.math_utils}}\label{\detokenize{polo.utils:polo-utils-math-utils-module}}\index{module@\spxentry{module}!polo.utils.math\_utils@\spxentry{polo.utils.math\_utils}}\index{polo.utils.math\_utils@\spxentry{polo.utils.math\_utils}!module@\spxentry{module}}\index{best\_aspect\_ratio() (in module polo.utils.math\_utils)@\spxentry{best\_aspect\_ratio()}\spxextra{in module polo.utils.math\_utils}}

\begin{fulllineitems}
\phantomsection\label{\detokenize{polo.utils:polo.utils.math_utils.best_aspect_ratio}}\pysiglinewithargsret{\sphinxcode{\sphinxupquote{polo.utils.math\_utils.}}\sphinxbfcode{\sphinxupquote{best\_aspect\_ratio}}}{\emph{\DUrole{n}{w}}, \emph{\DUrole{n}{h}}, \emph{\DUrole{n}{n}}}{}
\end{fulllineitems}

\index{factors() (in module polo.utils.math\_utils)@\spxentry{factors()}\spxextra{in module polo.utils.math\_utils}}

\begin{fulllineitems}
\phantomsection\label{\detokenize{polo.utils:polo.utils.math_utils.factors}}\pysiglinewithargsret{\sphinxcode{\sphinxupquote{polo.utils.math\_utils.}}\sphinxbfcode{\sphinxupquote{factors}}}{\emph{\DUrole{n}{n}}}{}
\end{fulllineitems}

\index{get\_cell\_image\_dims() (in module polo.utils.math\_utils)@\spxentry{get\_cell\_image\_dims()}\spxextra{in module polo.utils.math\_utils}}

\begin{fulllineitems}
\phantomsection\label{\detokenize{polo.utils:polo.utils.math_utils.get_cell_image_dims}}\pysiglinewithargsret{\sphinxcode{\sphinxupquote{polo.utils.math\_utils.}}\sphinxbfcode{\sphinxupquote{get\_cell\_image\_dims}}}{\emph{\DUrole{n}{w}}, \emph{\DUrole{n}{h}}, \emph{\DUrole{n}{n}}}{}
\end{fulllineitems}

\index{get\_image\_cell\_size() (in module polo.utils.math\_utils)@\spxentry{get\_image\_cell\_size()}\spxextra{in module polo.utils.math\_utils}}

\begin{fulllineitems}
\phantomsection\label{\detokenize{polo.utils:polo.utils.math_utils.get_image_cell_size}}\pysiglinewithargsret{\sphinxcode{\sphinxupquote{polo.utils.math\_utils.}}\sphinxbfcode{\sphinxupquote{get\_image\_cell\_size}}}{\emph{\DUrole{n}{cell\_aspect}}, \emph{\DUrole{n}{w}}, \emph{\DUrole{n}{h}}}{}
\end{fulllineitems}



\subsubsection{Module contents}
\label{\detokenize{polo.utils:module-polo.utils}}\label{\detokenize{polo.utils:module-contents}}\index{module@\spxentry{module}!polo.utils@\spxentry{polo.utils}}\index{polo.utils@\spxentry{polo.utils}!module@\spxentry{module}}

\section{Module contents}
\label{\detokenize{polo:module-polo}}\label{\detokenize{polo:module-contents}}\index{module@\spxentry{module}!polo@\spxentry{polo}}\index{polo@\spxentry{polo}!module@\spxentry{module}}\index{make\_default\_logger() (in module polo)@\spxentry{make\_default\_logger()}\spxextra{in module polo}}

\begin{fulllineitems}
\phantomsection\label{\detokenize{polo:polo.make_default_logger}}\pysiglinewithargsret{\sphinxcode{\sphinxupquote{polo.}}\sphinxbfcode{\sphinxupquote{make\_default\_logger}}}{\emph{\DUrole{n}{name}}}{}
\end{fulllineitems}



\chapter{Indices and tables}
\label{\detokenize{index:indices-and-tables}}\begin{itemize}
\item {} 
\DUrole{xref,std,std-ref}{genindex}

\item {} 
\DUrole{xref,std,std-ref}{modindex}

\item {} 
\DUrole{xref,std,std-ref}{search}

\end{itemize}


\renewcommand{\indexname}{Python Module Index}
\begin{sphinxtheindex}
\let\bigletter\sphinxstyleindexlettergroup
\bigletter{p}
\item\relax\sphinxstyleindexentry{polo}\sphinxstyleindexpageref{polo:\detokenize{module-polo}}
\item\relax\sphinxstyleindexentry{polo.crystallography}\sphinxstyleindexpageref{polo.crystallography:\detokenize{module-polo.crystallography}}
\item\relax\sphinxstyleindexentry{polo.crystallography.broke}\sphinxstyleindexpageref{polo.crystallography:\detokenize{module-polo.crystallography.broke}}
\item\relax\sphinxstyleindexentry{polo.crystallography.cocktail}\sphinxstyleindexpageref{polo.crystallography:\detokenize{module-polo.crystallography.cocktail}}
\item\relax\sphinxstyleindexentry{polo.crystallography.image}\sphinxstyleindexpageref{polo.crystallography:\detokenize{module-polo.crystallography.image}}
\item\relax\sphinxstyleindexentry{polo.crystallography.make\_screen}\sphinxstyleindexpageref{polo.crystallography:\detokenize{module-polo.crystallography.make_screen}}
\item\relax\sphinxstyleindexentry{polo.crystallography.run}\sphinxstyleindexpageref{polo.crystallography:\detokenize{module-polo.crystallography.run}}
\item\relax\sphinxstyleindexentry{polo.marco}\sphinxstyleindexpageref{polo.marco:\detokenize{module-polo.marco}}
\item\relax\sphinxstyleindexentry{polo.marco.run\_marco}\sphinxstyleindexpageref{polo.marco:\detokenize{module-polo.marco.run_marco}}
\item\relax\sphinxstyleindexentry{polo.utils}\sphinxstyleindexpageref{polo.utils:\detokenize{module-polo.utils}}
\item\relax\sphinxstyleindexentry{polo.utils.exceptions}\sphinxstyleindexpageref{polo.utils:\detokenize{module-polo.utils.exceptions}}
\item\relax\sphinxstyleindexentry{polo.utils.ftp\_utils}\sphinxstyleindexpageref{polo.utils:\detokenize{module-polo.utils.ftp_utils}}
\item\relax\sphinxstyleindexentry{polo.utils.io\_utils}\sphinxstyleindexpageref{polo.utils:\detokenize{module-polo.utils.io_utils}}
\item\relax\sphinxstyleindexentry{polo.utils.math\_utils}\sphinxstyleindexpageref{polo.utils:\detokenize{module-polo.utils.math_utils}}
\end{sphinxtheindex}

\renewcommand{\indexname}{Index}
\printindex
\end{document}