%% Generated by Sphinx.
\def\sphinxdocclass{report}
\documentclass[letterpaper,10pt,english]{sphinxmanual}
\ifdefined\pdfpxdimen
   \let\sphinxpxdimen\pdfpxdimen\else\newdimen\sphinxpxdimen
\fi \sphinxpxdimen=.75bp\relax

\PassOptionsToPackage{warn}{textcomp}
\usepackage[utf8]{inputenc}
\ifdefined\DeclareUnicodeCharacter
% support both utf8 and utf8x syntaxes
  \ifdefined\DeclareUnicodeCharacterAsOptional
    \def\sphinxDUC#1{\DeclareUnicodeCharacter{"#1}}
  \else
    \let\sphinxDUC\DeclareUnicodeCharacter
  \fi
  \sphinxDUC{00A0}{\nobreakspace}
  \sphinxDUC{2500}{\sphinxunichar{2500}}
  \sphinxDUC{2502}{\sphinxunichar{2502}}
  \sphinxDUC{2514}{\sphinxunichar{2514}}
  \sphinxDUC{251C}{\sphinxunichar{251C}}
  \sphinxDUC{2572}{\textbackslash}
\fi
\usepackage{cmap}
\usepackage[T1]{fontenc}
\usepackage{amsmath,amssymb,amstext}
\usepackage{babel}



\usepackage{times}
\expandafter\ifx\csname T@LGR\endcsname\relax
\else
% LGR was declared as font encoding
  \substitutefont{LGR}{\rmdefault}{cmr}
  \substitutefont{LGR}{\sfdefault}{cmss}
  \substitutefont{LGR}{\ttdefault}{cmtt}
\fi
\expandafter\ifx\csname T@X2\endcsname\relax
  \expandafter\ifx\csname T@T2A\endcsname\relax
  \else
  % T2A was declared as font encoding
    \substitutefont{T2A}{\rmdefault}{cmr}
    \substitutefont{T2A}{\sfdefault}{cmss}
    \substitutefont{T2A}{\ttdefault}{cmtt}
  \fi
\else
% X2 was declared as font encoding
  \substitutefont{X2}{\rmdefault}{cmr}
  \substitutefont{X2}{\sfdefault}{cmss}
  \substitutefont{X2}{\ttdefault}{cmtt}
\fi


\usepackage[Bjarne]{fncychap}
\usepackage{sphinx}

\fvset{fontsize=\small}
\usepackage{geometry}


% Include hyperref last.
\usepackage{hyperref}
% Fix anchor placement for figures with captions.
\usepackage{hypcap}% it must be loaded after hyperref.
% Set up styles of URL: it should be placed after hyperref.
\urlstyle{same}

\addto\captionsenglish{\renewcommand{\contentsname}{Contents:}}

\usepackage{sphinxmessages}
\setcounter{tocdepth}{1}



\title{Polo}
\date{Jul 23, 2020}
\release{0.0.5}
\author{Ethan Holleman}
\newcommand{\sphinxlogo}{\vbox{}}
\renewcommand{\releasename}{Release}
\makeindex
\begin{document}

\pagestyle{empty}
\sphinxmaketitle
\pagestyle{plain}
\sphinxtableofcontents
\pagestyle{normal}
\phantomsection\label{\detokenize{index::doc}}


Polo is a python GUI build using the Qt library for
high throughput crysta.lization screening users.


\chapter{About}
\label{\detokenize{about:about}}\label{\detokenize{about::doc}}

\section{Background}
\label{\detokenize{about:background}}
One of the largest hurtles to obtaining X\sphinxhyphen{}ray diffraction
data from biological samples is growing large, high quality crystals.

Currently, there is not way to reliably predict successful crystallization
conditions based on protein sequence alone and so high\sphinxhyphen{}throughput approaches
are very appealing. High\sphinxhyphen{}throughput crystallization screens test a large
chemical space using hundreds of different crystallization cocktails at the
nano\sphinxhyphen{}drop scale. Successful conditions can then be scaled up and optimized to
grow larger crystals.

The high\sphinxhyphen{}throughput crystallization screening center at the Hauptman\sphinxhyphen{}Woodward
Medical Research Institute provides this high\sphinxhyphen{}throughput screening service to
users; offering 1536 condition screens for both soluble and membrane protein
samples. Each plate is imaged over a period of two months in using both
visible light microscopy and UV\sphinxhyphen{}TPEF photography.

This high\sphinxhyphen{}throughput produces a large volume of images that must be
sorted through in order to pick out the best condition; a task that can be
very tedious and repetitious.

In 2019 Bruno \sphinxstyleemphasis{et al} published \sphinxhref{https://journals.plos.org/plosone/article?id=10.1371/journal.pone.0198883}{Classification of crystallization outcomes using deep convolutional neural networks}
which included a CNN model that could accurately classify crystallization screening
images, opening the door to automating this process. The MARCO model has been
used in large scale projects such as the \sphinxhref{http://xtuition.org/}{xtution database}
but has not been utilized in a average\sphinxhyphen{}user oriented graphical program.

Polo is therefore designed to incorporate the benefits of the MARCO model
and integrate the functionality of established crystallization image
labeling software such as \sphinxhref{https://hwi.buffalo.edu/wp-content/uploads/2016/11/MsjManual-0\_1\_1\_3.pdf}{MacroscopeJ}
to create a GUI targeted for HWI and other high\sphinxhyphen{}throughput crystallization screening
users that incorporates all the tools needed to go from raw crystallization images
to designing optimization screens without the need to install any dependencies.


\section{Features}
\label{\detokenize{about:features}}

\subsection{Automatic Image Classification}
\label{\detokenize{about:automatic-image-classification}}
Using the MARCO model Polo can cut down the time you spend looking
through your crystallization images by identifying wells likely to
contain a protein crystal. This can reduce the total number of
images that need to be considered from thousands to hundreds.


\subsection{Multiple Image View Modes}
\label{\detokenize{about:multiple-image-view-modes}}
Polo allows you to view your crystallization images in a variety of
ways that make it easier to identify true crystal hits. Images can be
viewed individually or in grids of up to 96.
If a sample has been imaged at multiple points in time it is easy
to create timeline views that allow you to assess the effectiveness
of a screening condition over time. Additionally, if your samples
have been imaged with photographic technologies outside of visible
light microscopy it is easy to compare these images to verify
the presence of protein crystals.


\subsection{Integrated FTP Browser}
\label{\detokenize{about:integrated-ftp-browser}}
Polo includes an simple FTP browser that allows you to download image
files from a remote server directly into Polo without then need to
install other software such as FileZilla. Polo is also packaged with
unrar for Windows and Mac.


\subsection{Data Management}
\label{\detokenize{about:data-management}}
Your image classifications are easily saved and managed via the xtal file format. Xtal files are similar to mso files created by MacroscopeJ and encode your image classifications, MARCO classifications, cocktail formulation and other metadata. In addition, xtal files increase portability
by encoding your screening images directly into the file along side your
metadata. This allows your classifications to be easily shared with
one file to anyone else with Polo on their computer.

Polo also has options to export your runs to csv files without
encoding your screening images or to HTML reports for a more
visual way to share your results with those who do not use Polo.


\subsection{Open Source Code Base}
\label{\detokenize{about:open-source-code-base}}
Polo is written in Python and is licensed under the
GNU 3.X license. This allows for modification and use of any of the
Polo source code. If you wish to change modify or extend any of
Polo’s functionality you are free to do so. Additionally,
documentation is available at THIS LINK.


\chapter{Installation Guide}
\label{\detokenize{install:installation-guide}}\label{\detokenize{install::doc}}
Please visit \sphinxhref{https://github.com/EthanHolleman/Marco\_Polo/releases}{the GitHub Release page}
and follow the install instructions for the latest release for your operating
system.

More details coming soon.


\chapter{FAQs}
\label{\detokenize{FAQS:faqs}}\label{\detokenize{FAQS::doc}}
A place to include frequently asked questions


\chapter{User’s Guide}
\label{\detokenize{user_guide:user-s-guide}}\label{\detokenize{user_guide::doc}}
If you have yet to install Polo on your machine, head to the installation
guide here.


\section{Importing and Opening Image Data}
\label{\detokenize{user_guide:importing-and-opening-image-data}}
The first step to using Polo is adding your own data. Polo organizes data into “runs”, which
consist of a set of related screening images. Polo organizes runs into three different
categories which are described below.
\begin{itemize}
\item {} 
HWI Screening Runs

\item {} 
Non\sphinxhyphen{}HWI Screening Runs

\item {} 
Raw Image Collections

\end{itemize}


\subsection{Getting Your Data (via FTP)}
\label{\detokenize{user_guide:getting-your-data-via-ftp}}
If you do not already have your data downloaded to your local machine, Polo
includes an FTP file browser which utilizes Python’s ftplib package. For
HWI users this allows you to download your screening images from the
HWI server without leaving the application.

To open the FTP file browser, on the menubar navigate to Import \sphinxhyphen{}\textgreater{} Images
\sphinxhyphen{}\textgreater{} From FTP. Enter your credentials in the new window. Once you are
connected to a server files available to download will be listed in the browser menu.

\noindent{\hspace*{\fill}\sphinxincludegraphics{{ftp_browser}.png}\hspace*{\fill}}

Select the checkboxes next to the files you wish to download, or select
all files in a directory at once. Pressing “Download Selected Files”
will start the download.

Since these are large files, Polo will download them in the background so you
can continue to the program normally. Once all files are downloaded Polo will
notify you that your download is complete. Closing Polo before this will
result in an incomplete download.


\subsection{Importing Images from a Directory}
\label{\detokenize{user_guide:importing-images-from-a-directory}}
Once you have your images on your local machine you can import them into
Polo by using the Run Importer tool. It can be opened by navigating to the
menu bar and selecting Import \sphinxhyphen{}\textgreater{} Images \sphinxhyphen{}\textgreater{} From Directory. A window like
the one below will then open.

\noindent\sphinxincludegraphics{{run_importer}.png}

If you are importing HWI screening images select \sphinxstylestrong{HWI Screening Run} from
the \sphinxstylestrong{Import Type} window (Beta testers do this). Select \sphinxstylestrong{Browse} to open
a file browser and select the directory containing the images you want to
import.

Other import types are included for compatibility with other
high\sphinxhyphen{}throughput protocols but metadata for these import modes is out of
the scope of the program. This means runs imported as these types will have
fewer available features.

However, since HWI has standard naming conventions Polo will suggest settings for your
run including what cocktail file to use, the date of imaging, the spectrum
and the number of wells. However, if you want to change any of these settings
you are free to do so.

Once you are happy with your import click \sphinxstylestrong{Submit Run} to load your
images into Polo.


\subsection{Importing a Saved Run}
\label{\detokenize{user_guide:importing-a-saved-run}}
One of the advantages of Polo is the introduction of the .xtal file format.
Xtal is a json like file that can store all the data relating to an individual
screening run in a single file with no other dependencies. Xtal files store
images, classifications, cocktail data and other annotations made while using
the Polo program. For more information you can visit \DUrole{xref,std,std-ref}{Xtal File Format}.

If you have xtal file ready for import you can load it into Polo by navigating
to the menu bar and selecting \sphinxstylestrong{Import} \sphinxhyphen{}\textgreater{} \sphinxstylestrong{Images} \sphinxhyphen{}\textgreater{} \sphinxstylestrong{From Saved Run}.
This will open a file browser and allow you to select the xtal file you wish
to import.


\subsection{Opening a Run}
\label{\detokenize{user_guide:opening-a-run}}
Once a run has been successfully imported, the run name will appear in the
\sphinxstylestrong{Loaded Runs} list, like in the image below.

If it is a new run double clicking on the run name will run the MARCO model
on the run’s images. You can check on MARCO’s progress by using the
\sphinxstylestrong{Classification Progress} bar located just below the Loaded Runs list. Polo
will also attempt to estimate the time remaining in your classification job.
\begin{quote}

\noindent\sphinxincludegraphics{{mid_class}.png}
\end{quote}

Once a run has been classified you can load it into the current view
by double clicking on it again. This will set the selected run as your
\sphinxstylestrong{Current Run} and all actions will be taken in reference to this run.
You can change your current run by double clicking another loaded run in the
\sphinxstylestrong{Loaded Runs} tab.
\begin{quote}

\noindent\sphinxincludegraphics{{loaded_image}.png}
\end{quote}


\section{Using the Slideshow View}
\label{\detokenize{user_guide:using-the-slideshow-view}}
The slideshow view is the main Polo user interface. It allows you to view
your screening images, label them and filter them by MARCO classifications,
your own classifications or both.
\begin{quote}

\noindent\sphinxincludegraphics{{loaded_image}.png}
\end{quote}


\subsection{Basic Navigation}
\label{\detokenize{user_guide:basic-navigation}}
Once you have images loaded in you can cycle through them by
pressing the \sphinxstylestrong{Next} or \sphinxstylestrong{Previous} image buttons under the navigation
panel. You can also use the right or left arrowkeys respectively. If you are
viewing an HWI screening run you can also navigate directly to a specific
well number by entering the well number into the \sphinxstylestrong{By Well Number} box.


\subsection{Classification}
\label{\detokenize{user_guide:classification}}
Arguably the most important Polo feature is the ability to easily
label your screening images. To label the currently displayed image
press the button in the \sphinxstylestrong{Classification} panel with your desired label.
You can classify images as Crystals, Precipitate, Clear or Other. To increase
your speed you can classify images using keyboard shortcuts which are
listed below.
\begin{itemize}
\item {} 
1: Crystal

\item {} 
2: Precipitate

\item {} 
3: Clear

\item {} 
4: Other

\end{itemize}

Classifying an image will automatically move you to the next image in
the slideshow.


\subsection{Filtering}
\label{\detokenize{user_guide:filtering}}
Using the checkboxes under the \sphinxstylestrong{Image Filters} panel in the lower
right corner of the window will allow you to filter the kinds of
images in your current slideshow. For example if you only wanted to
see images that MARCO has classified as Crystal you could check the Crystal box
under \sphinxstylestrong{Image Types} and MARCO under the \sphinxstylestrong{Classifier} panel. If you had checked Human
instead only images that you have classified as Crystal would be shown.

You can reset the slideshow to include all images by selecting all checkboxes
or no boxes and pressing submit filters.


\subsection{Image Metadata}
\label{\detokenize{user_guide:image-metadata}}
Image metadata will be displayed in the \sphinxstylestrong{Image Details} and \sphinxstylestrong{Cocktail Details}
windows when it is available. Image details will give you basic information about
the image currently being displayed, such as well number, imaging technology
imaging date and current classifications. If you are viewing an HWI run
the chemical conditions the current image was plated in will be displayed in the
\sphinxstylestrong{Cocktail Details} window.


\section{Using the Plate Viewer}
\label{\detokenize{user_guide:using-the-plate-viewer}}
To view multiple images in a grid you can utilize the \sphinxstylestrong{Plate Viewer}, which
can be found under the \sphinxstylestrong{Plate Viewer} tab. When you first open it up, it will
look something like the image below.
\begin{quote}

\noindent\sphinxincludegraphics{{plate_view}.png}
\end{quote}


\subsection{Basic Controls}
\label{\detokenize{user_guide:basic-controls}}
Assuming you have a run loaded and selected press \sphinxstylestrong{Reload Current View} button
to load in some images with the current settings.
\begin{quote}

\noindent\sphinxincludegraphics{{plate_24}.png}
\end{quote}

You can adjust the number of images shown in the grid by using the
\sphinxstylestrong{Images Per Plate} combo box and pressing \sphinxstylestrong{Reload Current View}.
\begin{quote}

\noindent\sphinxincludegraphics{{plate_96}.png}
\end{quote}

You can navigate to the next or previous view by using the \sphinxstylestrong{Next} or
\sphinxstylestrong{Previous} buttons respectively.


\subsection{Image Filtering and Coloring}
\label{\detokenize{user_guide:image-filtering-and-coloring}}
The Plateviewer window allows you to highlight images by either their MARCO
classification or the one you have given them. This allows you to find true
hits faster. First we will look at how to color images by their classification.

Open the \sphinxstylestrong{Image Coloring} tab on the bottom of the Plateviewer window.
Use the combo boxes to assign a color to each classification type. The
\sphinxstylestrong{MARCO} and \sphinxstylestrong{Human} radiobuttons tell Polo what classifier to use.
After you have picked out a color scheme switch back to the \sphinxstylestrong{Plate View} tab
select the \sphinxstylestrong{Apply Image Colors} checkbox and
press the \sphinxstylestrong{Apply Plate Settings} button to color your images.

\begin{sphinxadmonition}{note}{Note:}
The \sphinxstylestrong{Apply Image Colors} checkbox must be selected for colorings to be applied.
\end{sphinxadmonition}

A 96 well view with colors applied to all MARCO classifications.
\begin{quote}

\noindent\sphinxincludegraphics{{colo_96}.png}
\end{quote}

A disappointing 96 well view with only the Crystal classified images colored blue.
\begin{quote}

\noindent\sphinxincludegraphics{{one_colo_96}.png}
\end{quote}

Similarly, images can also be emphasized and deemphasized by either their
human or MARCO classification. Use the checkboxes under the \sphinxstylestrong{Image Filtering}
tab to select which images to emphasize. Then switch back to the \sphinxstylestrong{Plate View} tab
and check the \sphinxstylestrong{Apply Image Filters} box and hit \sphinxstylestrong{Apply Plate Settings}.

The same view as above but only selecting for MARCO classified crystal images.
\begin{quote}

\noindent\sphinxincludegraphics{{filter_96}.png}
\end{quote}


\subsection{Detail View}
\label{\detokenize{user_guide:detail-view}}
If you see an image that looks interesting you can select it and it will be
opened in a popout window like the one below. Here you can view details about
the image and assign it a classification using the buttons in the \sphinxstylestrong{Human Classification}
panel.
\begin{quote}

\noindent\sphinxincludegraphics{{pop}.png}
\end{quote}


\section{Using the Table View}
\label{\detokenize{user_guide:using-the-table-view}}
The table view allows you to view a run in a spreadsheet like view and does
not display images.
\begin{quote}

\noindent\sphinxincludegraphics{{table}.png}
\end{quote}


\subsection{Filtering}
\label{\detokenize{user_guide:id1}}
Just like the \sphinxstylestrong{Slideshow Viewer} and \sphinxstylestrong{Plate Viewer} tabs you can filter
what data is shown to you in the table view. For those familiar with SQL this
is like a SELECT / WHERE statement. Press the \sphinxstylestrong{Apply Settings} button to
apply your currently selected filters.
\begin{quote}

\noindent\sphinxincludegraphics{{filter_table}.png}
\end{quote}


\section{Using Plot Functions}
\label{\detokenize{user_guide:using-plot-functions}}
Polo utilizes the Matplotlib python library to provide a few diagnostic
plots to give you more information on your screening run. Once a run is loaded
in plots can be viewed by selecting the \sphinxstylestrong{Plots} tab.


\subsection{Classification Counts}
\label{\detokenize{user_guide:classification-counts}}
Visualize human and MARCO classifications by image type.
\begin{quote}

\noindent\sphinxincludegraphics{{class_counts}.png}
\end{quote}


\subsection{MARCO Accuracy}
\label{\detokenize{user_guide:marco-accuracy}}
View a basic bar graph of MARCO classification accuracy. Correct classifications
are those where the model applied the same classification as the human.
\begin{quote}

\noindent\sphinxincludegraphics{{mark_acc}.png}
\end{quote}


\subsection{Plate Heatmaps}
\label{\detokenize{user_guide:plate-heatmaps}}
View the MARCO model confidence for each image classification across
all images in your screening run.
\begin{quote}

\noindent\sphinxincludegraphics{{heat}.png}
\end{quote}


\subsection{Saving a Plot}
\label{\detokenize{user_guide:saving-a-plot}}
If you wish, you can save a plot as an image to your machine by
using the plot control icons at the bottom of the \sphinxstylestrong{Plots} tab. Select the
floppy disc to save the current plot.


\section{Using the Optimize Tool}
\label{\detokenize{user_guide:using-the-optimize-tool}}
Once you have identified crystal containing wells, you can easily
design optimization screens using the optimize tool. The optimize tool
automatically creates screens around the cocktail conditions a crystal hit
grown in.

Currently all conditions come directly from HWI screening conditions and changing
the conditions manually is not currently supported.

For more information on
optimization protocols you can read this document provided by HWI,

\sphinxhref{https://hwi.buffalo.edu/wp-content/uploads/2017/10/ReproduceCrystalLeads.pdf}{Reproducing HWI HTCSC crystallization screening hits}
\begin{quote}

\noindent\sphinxincludegraphics{{optimize}.png}
\end{quote}


\subsection{Setting Up Your Plate}
\label{\detokenize{user_guide:setting-up-your-plate}}
The first thing to do is to tell Polo what the plate you will be using for
your optimization screen. Set the number of wells in your plate by adjusting
the plate dimensions.

For example a standard 24 well plate could have 6 wells on X axis and 4 on
the Y or vice versa. Which value you assign to which axis is arbitrary as
long as you assign the reagent you want to screen for on that same axis.

You should also adjust the well volume by using the \sphinxstylestrong{Well Volume} and associated
units combo boxes. This volume will be used as the maximum volume for each well.


\subsection{Selecting Reagents}
\label{\detokenize{user_guide:selecting-reagents}}
Once your plate is set up you are ready to select reagents to screen for.
Wells you have classified as crystal containing are listed in the \sphinxstylestrong{Hit Well}
combo box under the \sphinxstylestrong{Reagent Controls} panel. Selecting a well will display
its containing reagents under the \sphinxstylestrong{Assign Reagents} combo boxes of the
\sphinxstylestrong{X Reagent} and \sphinxstylestrong{Y Reagent} tabs.

\begin{sphinxadmonition}{note}{Note:}
A reagent assigned under the \sphinxstylestrong{X Reagent tab} will be varied on the X\sphinxhyphen{}axis of the plate and the reagent assigned under the \sphinxstylestrong{Y Reagent} tab will be varied on the Y\sphinxhyphen{}axis of the plate.
\end{sphinxadmonition}

Since any given plate only has two axis only two reagents can be screened for,
but many crystallization cocktail contain three or more reagents. Reagents not
assigned to either the x or y reagent will be considered constant and will
be present in your final screen but always at the same concentration.

Once you have assigned your reagents set a stock concentration for each reagent and
select a percentage to the x and y reagents by. This is always in reference to the
hit concentration for the particular reagent.

For example if you selected citric acid as your x reagent and it was present
in the crystallization cocktail at a concentration of 0.5 M the concentration
of citric acid in your optimization screen will be varied in reference to
0.5 M by the percentage selected in the \sphinxstylestrong{Vary each well by} spin box.

Additionally, the hit concentration is always the center most well of the
axis. Meaning if there are 6 wells on your plate’s x\sphinxhyphen{}axis you can expect
the concentration of citric acis in the 3rd well to be 0.5 M (using the
example above).


\subsection{Viewing Your Optimization Screen}
\label{\detokenize{user_guide:viewing-your-optimization-screen}}
After everything is set up, press \sphinxstylestrong{Show Screen} to display your
current optimization screen. It will look something like the image below.
\begin{quote}

\noindent\sphinxincludegraphics{{screen}.png}
\end{quote}

If it does not fill the screen slightly adjust the size of the Polo window and
it should snap into place.


\subsection{Exporting A Screen}
\label{\detokenize{user_guide:exporting-a-screen}}
After setting up your screen you can export it to an html file to print
and / or share with your friends. To do so just hit the \sphinxstylestrong{Export} button.
This will open a file browser and you can select where you would like to
save the file. You can then open the file with any browser to view it or
save it as a pdf.


\section{Advanced Tools}
\label{\detokenize{user_guide:advanced-tools}}
Advanced tools go a bit beyond image classification and viewing and allow you
to take advantage of the fact a single screening run will likely be imaged at
multiple points in time and with multiple imaging technologies. All tools
described below can be found by navigating to the menu bar and selecting
\sphinxstylestrong{Advanced Tools}.


\subsection{Time Resolved Runs}
\label{\detokenize{user_guide:time-resolved-runs}}
Usually, the sample plate will be imaged at multiple points in time. Polo allows
you to link these runs together using the \sphinxstylestrong{Add Time Resolved} tool under
\sphinxstylestrong{Advanced Tools}. Polo will automatically determine the order of the runs
based on their date and then allow you to navigate between them in both
the Slideshow Viewer tab and Plate Viewer tab using the \sphinxstylestrong{Next Date} and
\sphinxstylestrong{Previous Date} buttons.


\subsection{Adding an Alternative Spectrum}
\label{\detokenize{user_guide:adding-an-alternative-spectrum}}
Oftentimes, the same plate will be imaged with a photographic technology
besides microscopy in order to verify the presence of crystals in a sample.
If you have images taken with such a technology you can identify them as such
in the \sphinxstylestrong{Run Importer} window (link here).

Then navigate to \sphinxstylestrong{Advanced Tools} and select \sphinxstylestrong{Add Spectrum} which will
open a dialog like the one shown below.

Valid runs will automatically be assigned to an image category based on the
spectrum they were assigned at import. Select one run from each category that
you would like to link and when finished press \sphinxstylestrong{Submit Assignment}.

You will then be able to swap between these images in both the Slideshow Viewer
and Plate Viewers. This can be very useful for verifying a crystal is in fact
a protein crystal and not salt.


\section{Saving a Run}
\label{\detokenize{user_guide:saving-a-run}}
Once you have invested time into picking out the best screening conditions you
are going to want to save your work and share it with others. The best way to
do this in Polo is by saving your current run as a .xtal file.


\subsection{.xtal File Format}
\label{\detokenize{user_guide:xtal-file-format}}
.xtal is a json based format for saving screening runs after they have been
loaded into and processed using Polo. It will maintain all of your image
classifications and metadata in a single file. Additionally, the actual
screening images are encoded directly into the xtal file which makes sharing
your data to another Polo user very easy as it only requires sharing the
single .xtal file.


\subsection{Actually Saving Your Run}
\label{\detokenize{user_guide:actually-saving-your-run}}
To save your Run from Polo, navigate to the menu bar and and select
\sphinxstylestrong{File} \sphinxhyphen{}\textgreater{} \sphinxstylestrong{Save} or \sphinxstylestrong{Save As}. If you have not saved your current run
to a xtal file before a file browser will automatically be opened and you
can specify a location and name for your save file.

\begin{sphinxadmonition}{note}{Note:}
Do not close Polo while a run is being saved!
\end{sphinxadmonition}

After the save is complete Polo will notify you with a popup telling you it
is safe to close the program and your save as completed. You can now share
your xtal file with anyone else using Polo!


\section{Exporting a Run}
\label{\detokenize{user_guide:exporting-a-run}}
In addition to saving screening runs as xtal files, Polo supports exporting
to formats which can be viewed outside of the Polo application. Currently
available export options are described below.


\subsection{HTML Reports}
\label{\detokenize{user_guide:html-reports}}

\subsection{CSV Exports}
\label{\detokenize{user_guide:csv-exports}}

\chapter{Beta Testers Guide}
\label{\detokenize{beta_testers:beta-testers-guide}}\label{\detokenize{beta_testers::doc}}

\section{Thank You!}
\label{\detokenize{beta_testers:thank-you}}
Thank you for taking some time out of your day to help me test Polo. It
means a lot. On this page you will find info on how to get started,
where to find test data, what to do with it and how to report any bugs that
you find.


\section{Getting Polo Beta Version}
\label{\detokenize{beta_testers:getting-polo-beta-version}}
Visit the {\hyperref[\detokenize{install:installation-guide}]{\sphinxcrossref{\DUrole{std,std-ref}{Installation Guide}}}} to learn more about getting the latest
Polo release for your system. For test releases test data will be packaged
along with the installer to help you start using Polo. You also might want to
skim over the {\hyperref[\detokenize{about:background}]{\sphinxcrossref{\DUrole{std,std-ref}{Background}}}} section of the About page to get a better idea
of what Polo’s use cases are. The {\hyperref[\detokenize{user_guide:user-s-guide}]{\sphinxcrossref{\DUrole{std,std-ref}{User’s Guide}}}} goes into more detail
about the specific functions of each tool included in Polo.


\section{What Now?}
\label{\detokenize{beta_testers:what-now}}
Now that you have got some data you can start working with it. Here is a list
of things to do with your data and links to sections of the documentation
on how to do it.
\begin{itemize}
\item {} 
Open the xtal file: {\hyperref[\detokenize{user_guide:importing-images-from-a-directory}]{\sphinxcrossref{\DUrole{std,std-ref}{Importing Images from a Directory}}}}

\item {} 
Open the folder of images and classify them: {\hyperref[\detokenize{user_guide:importing-a-saved-run}]{\sphinxcrossref{\DUrole{std,std-ref}{Importing a Saved Run}}}}

\item {} 
Classify images using the Slideshow Viewer: {\hyperref[\detokenize{user_guide:using-the-slideshow-view}]{\sphinxcrossref{\DUrole{std,std-ref}{Using the Slideshow View}}}}

\item {} 
Classify images using the Plate Viewer: {\hyperref[\detokenize{user_guide:using-the-plate-viewer}]{\sphinxcrossref{\DUrole{std,std-ref}{Using the Plate Viewer}}}}

\item {} 
Create an optimization screen: {\hyperref[\detokenize{user_guide:using-the-optimize-tool}]{\sphinxcrossref{\DUrole{std,std-ref}{Using the Optimize Tool}}}}

\item {} 
Do something unexpected in an attempt to break the program (\sphinxstylestrong{Highly Recommended!})

\end{itemize}

For more ideas and help getting started check out the {\hyperref[\detokenize{user_guide:user-s-guide}]{\sphinxcrossref{\DUrole{std,std-ref}{User’s Guide}}}}


\section{I Found a Bug}
\label{\detokenize{beta_testers:i-found-a-bug}}
Thats great! If you have some GitHub knowhow you can report it on the issues
\sphinxhref{https://github.com/EthanHolleman/Marco\_Polo/issues}{on the issues}
page of the Polo repository or fill out \sphinxhref{https://forms.gle/GfMT72z7Z6AzJqS86}{this google form}
Thanks for your help!


\chapter{polo package}
\label{\detokenize{polo:polo-package}}\label{\detokenize{polo::doc}}

\section{Subpackages}
\label{\detokenize{polo:subpackages}}

\subsection{polo.crystallography package}
\label{\detokenize{polo.crystallography:polo-crystallography-package}}\label{\detokenize{polo.crystallography::doc}}

\subsubsection{Submodules}
\label{\detokenize{polo.crystallography:submodules}}

\subsubsection{polo.crystallography.cocktail module}
\label{\detokenize{polo.crystallography:module-polo.crystallography.cocktail}}\label{\detokenize{polo.crystallography:polo-crystallography-cocktail-module}}\index{module@\spxentry{module}!polo.crystallography.cocktail@\spxentry{polo.crystallography.cocktail}}\index{polo.crystallography.cocktail@\spxentry{polo.crystallography.cocktail}!module@\spxentry{module}}\index{Cocktail (class in polo.crystallography.cocktail)@\spxentry{Cocktail}\spxextra{class in polo.crystallography.cocktail}}

\begin{fulllineitems}
\phantomsection\label{\detokenize{polo.crystallography:polo.crystallography.cocktail.Cocktail}}\pysiglinewithargsret{\sphinxbfcode{\sphinxupquote{class }}\sphinxcode{\sphinxupquote{polo.crystallography.cocktail.}}\sphinxbfcode{\sphinxupquote{Cocktail}}}{\emph{\DUrole{n}{number}\DUrole{o}{=}\DUrole{default_value}{None}}, \emph{\DUrole{n}{well\_assignment}\DUrole{o}{=}\DUrole{default_value}{None}}, \emph{\DUrole{n}{commercial\_code}\DUrole{o}{=}\DUrole{default_value}{None}}, \emph{\DUrole{n}{pH}\DUrole{o}{=}\DUrole{default_value}{None}}, \emph{\DUrole{n}{reagents}\DUrole{o}{=}\DUrole{default_value}{{[}{]}}}}{}
Bases: \sphinxcode{\sphinxupquote{object}}

Cocktail instances are used to hold a collection of 
{\hyperref[\detokenize{polo.crystallography:polo.crystallography.cocktail.Reagent}]{\sphinxcrossref{\sphinxcode{\sphinxupquote{Reagent}}}}}
instances that form one chemical cocktail. 
Cocktails also hold other metadata including their commercial code,
the cocktail pH and the well they are assigned to.
Currently, cocktails are only supported for HWIRuns.
\begin{quote}\begin{description}
\item[{Parameters}] \leavevmode\begin{itemize}
\item {} 
\sphinxstyleliteralstrong{\sphinxupquote{number}} (\sphinxstyleliteralemphasis{\sphinxupquote{str}}\sphinxstyleliteralemphasis{\sphinxupquote{, }}\sphinxstyleliteralemphasis{\sphinxupquote{optional}}) \textendash{} The cocktail number, defaults to None

\item {} 
\sphinxstyleliteralstrong{\sphinxupquote{well\_assignment}} (\sphinxstyleliteralemphasis{\sphinxupquote{int}}\sphinxstyleliteralemphasis{\sphinxupquote{, }}\sphinxstyleliteralemphasis{\sphinxupquote{optional}}) \textendash{} Well number in the screening plate this
cocktail belongs to, defaults to None

\item {} 
\sphinxstyleliteralstrong{\sphinxupquote{commercial\_code}} (\sphinxstyleliteralemphasis{\sphinxupquote{str}}\sphinxstyleliteralemphasis{\sphinxupquote{, }}\sphinxstyleliteralemphasis{\sphinxupquote{optional}}) \textendash{} Commercial code of cocktail, defaults to None

\item {} 
\sphinxstyleliteralstrong{\sphinxupquote{pH}} (\sphinxstyleliteralemphasis{\sphinxupquote{float}}\sphinxstyleliteralemphasis{\sphinxupquote{, }}\sphinxstyleliteralemphasis{\sphinxupquote{optional}}) \textendash{} pH of the cocktail, defaults to None

\item {} 
\sphinxstyleliteralstrong{\sphinxupquote{reagents}} ({\hyperref[\detokenize{polo.crystallography:polo.crystallography.cocktail.Reagent}]{\sphinxcrossref{\sphinxstyleliteralemphasis{\sphinxupquote{Reagent}}}}}\sphinxstyleliteralemphasis{\sphinxupquote{, }}\sphinxstyleliteralemphasis{\sphinxupquote{optional}}) \textendash{} list of reagent instances that make up the contents
of the cocktail, defaults to None

\end{itemize}

\end{description}\end{quote}
\index{add\_reagent() (polo.crystallography.cocktail.Cocktail method)@\spxentry{add\_reagent()}\spxextra{polo.crystallography.cocktail.Cocktail method}}

\begin{fulllineitems}
\phantomsection\label{\detokenize{polo.crystallography:polo.crystallography.cocktail.Cocktail.add_reagent}}\pysiglinewithargsret{\sphinxbfcode{\sphinxupquote{add\_reagent}}}{\emph{\DUrole{n}{new\_reagent}}}{}
Adds a reagent to the existing list of reagents referenced by the
:attr:{\color{red}\bfseries{}\textasciigrave{}}\textasciitilde{}polo.crystallography.cocktail.Cocktail.reagents attribute.
\begin{quote}\begin{description}
\item[{Parameters}] \leavevmode
\sphinxstyleliteralstrong{\sphinxupquote{new\_reagent}} ({\hyperref[\detokenize{polo.crystallography:polo.crystallography.cocktail.Reagent}]{\sphinxcrossref{\sphinxstyleliteralemphasis{\sphinxupquote{Reagent}}}}}) \textendash{} Reagent to add to this cocktail

\end{description}\end{quote}

\end{fulllineitems}

\index{cocktail\_index() (polo.crystallography.cocktail.Cocktail property)@\spxentry{cocktail\_index()}\spxextra{polo.crystallography.cocktail.Cocktail property}}

\begin{fulllineitems}
\phantomsection\label{\detokenize{polo.crystallography:polo.crystallography.cocktail.Cocktail.cocktail_index}}\pysigline{\sphinxbfcode{\sphinxupquote{property }}\sphinxbfcode{\sphinxupquote{cocktail\_index}}}
Attempt to pull out the cocktail number as an integer
from the \sphinxcode{\sphinxupquote{number}} attribute.
This property is dependent on consistent formating 
between cocktail menus that has not checked at this time.
\begin{quote}\begin{description}
\item[{Returns}] \leavevmode
Cocktail number

\item[{Return type}] \leavevmode
int

\end{description}\end{quote}

\end{fulllineitems}

\index{well\_assignment() (polo.crystallography.cocktail.Cocktail property)@\spxentry{well\_assignment()}\spxextra{polo.crystallography.cocktail.Cocktail property}}

\begin{fulllineitems}
\phantomsection\label{\detokenize{polo.crystallography:polo.crystallography.cocktail.Cocktail.well_assignment}}\pysigline{\sphinxbfcode{\sphinxupquote{property }}\sphinxbfcode{\sphinxupquote{well\_assignment}}}
Return the current well assignment for this Cocktail.
\begin{quote}\begin{description}
\item[{Returns}] \leavevmode
well assignment

\item[{Return type}] \leavevmode
int

\end{description}\end{quote}

\end{fulllineitems}


\end{fulllineitems}

\index{Reagent (class in polo.crystallography.cocktail)@\spxentry{Reagent}\spxextra{class in polo.crystallography.cocktail}}

\begin{fulllineitems}
\phantomsection\label{\detokenize{polo.crystallography:polo.crystallography.cocktail.Reagent}}\pysiglinewithargsret{\sphinxbfcode{\sphinxupquote{class }}\sphinxcode{\sphinxupquote{polo.crystallography.cocktail.}}\sphinxbfcode{\sphinxupquote{Reagent}}}{\emph{\DUrole{n}{chemical\_additive}\DUrole{o}{=}\DUrole{default_value}{None}}, \emph{\DUrole{n}{concentration}\DUrole{o}{=}\DUrole{default_value}{None}}, \emph{\DUrole{n}{chemical\_formula}\DUrole{o}{=}\DUrole{default_value}{None}}, \emph{\DUrole{n}{stock\_con}\DUrole{o}{=}\DUrole{default_value}{None}}}{}
Bases: \sphinxcode{\sphinxupquote{object}}

Reagent instances represent one specific kind of chemical compound at
a specific concentration. Multiple Reagents make up a cocktail. Reagents
are generally created from the contents of HWI cocktail csv files which
describe all 1536 cocktails and the reagents that compose them in one file.
The cocktail csv files can be found in the \sphinxtitleref{data} directory.
\begin{quote}\begin{description}
\item[{Parameters}] \leavevmode\begin{itemize}
\item {} 
\sphinxstyleliteralstrong{\sphinxupquote{chemical\_additive}} (\sphinxstyleliteralemphasis{\sphinxupquote{str}}\sphinxstyleliteralemphasis{\sphinxupquote{, }}\sphinxstyleliteralemphasis{\sphinxupquote{optional}}) \textendash{} Name of the chemical reagent,
defaults to None

\item {} 
\sphinxstyleliteralstrong{\sphinxupquote{concentration}} ({\hyperref[\detokenize{polo.crystallography:polo.crystallography.cocktail.UnitValue}]{\sphinxcrossref{\sphinxstyleliteralemphasis{\sphinxupquote{UnitValue}}}}}\sphinxstyleliteralemphasis{\sphinxupquote{, }}\sphinxstyleliteralemphasis{\sphinxupquote{optional}}) \textendash{} Concentration of the reagent,
defaults to None

\item {} 
\sphinxstyleliteralstrong{\sphinxupquote{chemical\_formula}} (\sphinxstyleliteralemphasis{\sphinxupquote{Formula}}\sphinxstyleliteralemphasis{\sphinxupquote{, }}\sphinxstyleliteralemphasis{\sphinxupquote{optional}}) \textendash{} Chemical formula for this reagent, defaults
to None

\item {} 
\sphinxstyleliteralstrong{\sphinxupquote{stock\_con}} ({\hyperref[\detokenize{polo.crystallography:polo.crystallography.cocktail.UnitValue}]{\sphinxcrossref{\sphinxstyleliteralemphasis{\sphinxupquote{UnitValue}}}}}\sphinxstyleliteralemphasis{\sphinxupquote{, }}\sphinxstyleliteralemphasis{\sphinxupquote{optional}}) \textendash{} Concentration of this reagent’s stock solution,
defaults to None

\end{itemize}

\end{description}\end{quote}
\index{chemical\_formula() (polo.crystallography.cocktail.Reagent property)@\spxentry{chemical\_formula()}\spxextra{polo.crystallography.cocktail.Reagent property}}

\begin{fulllineitems}
\phantomsection\label{\detokenize{polo.crystallography:polo.crystallography.cocktail.Reagent.chemical_formula}}\pysigline{\sphinxbfcode{\sphinxupquote{property }}\sphinxbfcode{\sphinxupquote{chemical\_formula}}}
The chemical formula for of this Reagent. Not all
Reagents have available chemical formulas as cocktail csv files do not
include formulas for all reagents. See the setter method for more details
on how chemical formulas are converted from strings to \sphinxtitleref{Formula} objects.
\begin{quote}\begin{description}
\item[{Returns}] \leavevmode
Chemical formula

\item[{Return type}] \leavevmode
Formula

\end{description}\end{quote}

\end{fulllineitems}

\index{concentration() (polo.crystallography.cocktail.Reagent property)@\spxentry{concentration()}\spxextra{polo.crystallography.cocktail.Reagent property}}

\begin{fulllineitems}
\phantomsection\label{\detokenize{polo.crystallography:polo.crystallography.cocktail.Reagent.concentration}}\pysigline{\sphinxbfcode{\sphinxupquote{property }}\sphinxbfcode{\sphinxupquote{concentration}}}
The current concentration of this Reagent. Concentration
ultimately refers back to a condition in a specific screening well.
\begin{quote}\begin{description}
\item[{Returns}] \leavevmode
Chemical concentration

\item[{Return type}] \leavevmode
{\hyperref[\detokenize{polo.crystallography:polo.crystallography.cocktail.UnitValue}]{\sphinxcrossref{UnitValue}}}

\end{description}\end{quote}

\end{fulllineitems}

\index{molar\_mass() (polo.crystallography.cocktail.Reagent property)@\spxentry{molar\_mass()}\spxextra{polo.crystallography.cocktail.Reagent property}}

\begin{fulllineitems}
\phantomsection\label{\detokenize{polo.crystallography:polo.crystallography.cocktail.Reagent.molar_mass}}\pysigline{\sphinxbfcode{\sphinxupquote{property }}\sphinxbfcode{\sphinxupquote{molar\_mass}}}
Attempt to calculate the molar mass of this reagent. Closely related
to the molarity property. The molar mass of the reagent cannot be
calculated for all HWI reagents.
\begin{quote}\begin{description}
\item[{Returns}] \leavevmode
Molar mass of the Reagent if it is calculable, False otherwise.

\item[{Return type}] \leavevmode
{\hyperref[\detokenize{polo.crystallography:polo.crystallography.cocktail.UnitValue}]{\sphinxcrossref{UnitValue}}} or bool

\end{description}\end{quote}

\end{fulllineitems}

\index{molarity() (polo.crystallography.cocktail.Reagent property)@\spxentry{molarity()}\spxextra{polo.crystallography.cocktail.Reagent property}}

\begin{fulllineitems}
\phantomsection\label{\detokenize{polo.crystallography:polo.crystallography.cocktail.Reagent.molarity}}\pysigline{\sphinxbfcode{\sphinxupquote{property }}\sphinxbfcode{\sphinxupquote{molarity}}}
Attempt to calculate the molarity of this reagent at its current
concentration. This calculation is not certain to return a value
as HWI cocktail menu files use a variety of units to describe
chemical concentrations, including \%w/v or \%v/v.

\%w/v is defined as grams of colute per 100 ml of solution * 100. This can
be converted to molarity when the molar mass of the reagent is known.

\%v/v is defined as the volume of solute over the total volume of solution
* 100. The density of the reagent is required to convert \%w/v to molarity
which is not included in HWI cocktail menu files. This makes conversion
from \%w/v out of reach for now.

If the reagent concentration cannot be converted to molarity then
this function will return False.
\begin{quote}\begin{description}
\item[{Returns}] \leavevmode
molarity or False

\item[{Return type}] \leavevmode
{\hyperref[\detokenize{polo.crystallography:polo.crystallography.cocktail.UnitValue}]{\sphinxcrossref{UnitValue}}} or Bool

\end{description}\end{quote}

\end{fulllineitems}

\index{peg\_parser() (polo.crystallography.cocktail.Reagent method)@\spxentry{peg\_parser()}\spxextra{polo.crystallography.cocktail.Reagent method}}

\begin{fulllineitems}
\phantomsection\label{\detokenize{polo.crystallography:polo.crystallography.cocktail.Reagent.peg_parser}}\pysiglinewithargsret{\sphinxbfcode{\sphinxupquote{peg\_parser}}}{\emph{\DUrole{n}{peg\_string}}}{}
Attempts to pull out a molar mass from a PEG species since the
molar mass is often included in the name of PEG species. A string is
considered to be a potential PEG species if it contains ‘PEG’ or
‘Polyethylene glycol’ in it.
\begin{quote}\begin{description}
\item[{Parameters}] \leavevmode
\sphinxstyleliteralstrong{\sphinxupquote{peg\_string}} (\sphinxstyleliteralemphasis{\sphinxupquote{str}}) \textendash{} String to look for PEG species in

\item[{Returns}] \leavevmode
molar mass if found to be valid PEG species, False otherwise.

\item[{Return type}] \leavevmode
float or Bool

\end{description}\end{quote}

\end{fulllineitems}

\index{stock\_volume() (polo.crystallography.cocktail.Reagent method)@\spxentry{stock\_volume()}\spxextra{polo.crystallography.cocktail.Reagent method}}

\begin{fulllineitems}
\phantomsection\label{\detokenize{polo.crystallography:polo.crystallography.cocktail.Reagent.stock_volume}}\pysiglinewithargsret{\sphinxbfcode{\sphinxupquote{stock\_volume}}}{\emph{\DUrole{n}{target\_volume}}}{}
Attempt to calculate the required amount of stock solution to
produce the Reagent’s set concentration in the given \sphinxtitleref{target\_volume}
argument. Stock concentration is taken from the 
\sphinxcode{\sphinxupquote{stock\_con}} attribute. If
\sphinxcode{\sphinxupquote{stock\_con}} is not
set or the molarity of the reagent can not be calculated this method
will return False.
\begin{quote}\begin{description}
\item[{Parameters}] \leavevmode
\sphinxstyleliteralstrong{\sphinxupquote{target\_volume}} ({\hyperref[\detokenize{polo.crystallography:polo.crystallography.cocktail.UnitValue}]{\sphinxcrossref{\sphinxstyleliteralemphasis{\sphinxupquote{UnitValue}}}}}) \textendash{} Volume in which stock will be diluted into

\item[{Returns}] \leavevmode
Volume of stock or False

\item[{Return type}] \leavevmode
{\hyperref[\detokenize{polo.crystallography:polo.crystallography.cocktail.UnitValue}]{\sphinxcrossref{UnitValue}}} or False

\end{description}\end{quote}

\end{fulllineitems}

\index{units (polo.crystallography.cocktail.Reagent attribute)@\spxentry{units}\spxextra{polo.crystallography.cocktail.Reagent attribute}}

\begin{fulllineitems}
\phantomsection\label{\detokenize{polo.crystallography:polo.crystallography.cocktail.Reagent.units}}\pysigline{\sphinxbfcode{\sphinxupquote{units}}\sphinxbfcode{\sphinxupquote{ = {[}\textquotesingle{}M\textquotesingle{}, \textquotesingle{}(w/v)\textquotesingle{}, \textquotesingle{}(v/v)\textquotesingle{}{]}}}}
\end{fulllineitems}


\end{fulllineitems}

\index{UnitValue (class in polo.crystallography.cocktail)@\spxentry{UnitValue}\spxextra{class in polo.crystallography.cocktail}}

\begin{fulllineitems}
\phantomsection\label{\detokenize{polo.crystallography:polo.crystallography.cocktail.UnitValue}}\pysiglinewithargsret{\sphinxbfcode{\sphinxupquote{class }}\sphinxcode{\sphinxupquote{polo.crystallography.cocktail.}}\sphinxbfcode{\sphinxupquote{UnitValue}}}{\emph{\DUrole{n}{value}\DUrole{o}{=}\DUrole{default_value}{None}}, \emph{\DUrole{n}{units}\DUrole{o}{=}\DUrole{default_value}{None}}}{}
Bases: \sphinxcode{\sphinxupquote{object}}

UnitValues are used to help handle numbers with units. They
are not the most robust but help to keep things more organized.
UnitValues can be created by either passing values to the
\sphinxtitleref{values} and \sphinxtitleref{units} args explicitly or by calling the classmethod
{\hyperref[\detokenize{polo.crystallography:polo.crystallography.cocktail.UnitValue.make_from_string}]{\sphinxcrossref{\sphinxcode{\sphinxupquote{make\_from\_string()}}}}}
which will use regex to pull out supported units and
values.
\index{make\_from\_string() (polo.crystallography.cocktail.UnitValue class method)@\spxentry{make\_from\_string()}\spxextra{polo.crystallography.cocktail.UnitValue class method}}

\begin{fulllineitems}
\phantomsection\label{\detokenize{polo.crystallography:polo.crystallography.cocktail.UnitValue.make_from_string}}\pysiglinewithargsret{\sphinxbfcode{\sphinxupquote{classmethod }}\sphinxbfcode{\sphinxupquote{make\_from\_string}}}{\emph{\DUrole{n}{string}}}{}
Create a \sphinxtitleref{UnitValue} from a string containing a value and a unit.
Utilizes the \sphinxcode{\sphinxupquote{polo.unit\_regex}} expression 
to pull out the units.

\begin{sphinxVerbatim}[commandchars=\\\{\}]
\PYG{n}{unit\PYGZus{}string} \PYG{o}{=} \PYG{l+s+s1}{\PYGZsq{}}\PYG{l+s+s1}{10.0 M}\PYG{l+s+s1}{\PYGZsq{}}  \PYG{c+c1}{\PYGZsh{} concentration of 10 molar}
\PYG{n}{sv} \PYG{o}{=} \PYG{n}{UnitValue}\PYG{o}{.}\PYG{n}{make\PYGZus{}from\PYGZus{}string}\PYG{p}{(}\PYG{n}{unit\PYGZus{}string}\PYG{p}{)}
\PYG{c+c1}{\PYGZsh{} sv.value = 10 sv.units = \PYGZsq{}M\PYGZsq{}}
\end{sphinxVerbatim}
\begin{quote}\begin{description}
\item[{Parameters}] \leavevmode
\sphinxstyleliteralstrong{\sphinxupquote{string}} (\sphinxstyleliteralemphasis{\sphinxupquote{str}}) \textendash{} The string to extract the UnitValue from

\item[{Returns}] \leavevmode
UnitValue instance

\item[{Return type}] \leavevmode
{\hyperref[\detokenize{polo.crystallography:polo.crystallography.cocktail.UnitValue}]{\sphinxcrossref{UnitValue}}}

\end{description}\end{quote}

\end{fulllineitems}

\index{round() (polo.crystallography.cocktail.UnitValue method)@\spxentry{round()}\spxextra{polo.crystallography.cocktail.UnitValue method}}

\begin{fulllineitems}
\phantomsection\label{\detokenize{polo.crystallography:polo.crystallography.cocktail.UnitValue.round}}\pysiglinewithargsret{\sphinxbfcode{\sphinxupquote{round}}}{\emph{\DUrole{n}{digits}}}{}
\end{fulllineitems}

\index{saved\_scalers (polo.crystallography.cocktail.UnitValue attribute)@\spxentry{saved\_scalers}\spxextra{polo.crystallography.cocktail.UnitValue attribute}}

\begin{fulllineitems}
\phantomsection\label{\detokenize{polo.crystallography:polo.crystallography.cocktail.UnitValue.saved_scalers}}\pysigline{\sphinxbfcode{\sphinxupquote{saved\_scalers}}\sphinxbfcode{\sphinxupquote{ = \{\textquotesingle{}c\textquotesingle{}: 0.01, \textquotesingle{}m\textquotesingle{}: 0.001, \textquotesingle{}u\textquotesingle{}: 1e\sphinxhyphen{}06\}}}}
\end{fulllineitems}

\index{scale() (polo.crystallography.cocktail.UnitValue method)@\spxentry{scale()}\spxextra{polo.crystallography.cocktail.UnitValue method}}

\begin{fulllineitems}
\phantomsection\label{\detokenize{polo.crystallography:polo.crystallography.cocktail.UnitValue.scale}}\pysiglinewithargsret{\sphinxbfcode{\sphinxupquote{scale}}}{\emph{\DUrole{n}{scale\_key}}}{}
Scale the {\hyperref[\detokenize{polo.crystallography:polo.crystallography.cocktail.UnitValue.value}]{\sphinxcrossref{\sphinxcode{\sphinxupquote{value}}}}}
using a key character that exists in the 
{\hyperref[\detokenize{polo.crystallography:polo.crystallography.cocktail.UnitValue.saved_scalers}]{\sphinxcrossref{\sphinxcode{\sphinxupquote{saved\_scalers}}}}}
dictionary. First converts the value to its
base unit and then divides by the \sphinxtitleref{scale\_key} argument value. 
The \sphinxtitleref{scale\_key} can be thought of as a SI prefix for a base unit.

\begin{sphinxVerbatim}[commandchars=\\\{\}]
\PYG{c+c1}{\PYGZsh{} self.saved\PYGZus{}scalers = \PYGZob{}\PYGZsq{}u\PYGZsq{}: 1e\PYGZhy{}6, \PYGZsq{}m\PYGZsq{}: 1e\PYGZhy{}3, \PYGZsq{}c\PYGZsq{}: 1e\PYGZhy{}2\PYGZcb{}}
\PYG{n}{v\PYGZus{}one} \PYG{o}{=} \PYG{n}{UnitValue}\PYG{p}{(}\PYG{l+m+mi}{10}\PYG{p}{,} \PYG{l+s+s1}{\PYGZsq{}}\PYG{l+s+s1}{L}\PYG{l+s+s1}{\PYGZsq{}}\PYG{p}{)}  \PYG{c+c1}{\PYGZsh{} value of 10 liters}
\PYG{n}{v\PYGZus{}one} \PYG{o}{=} \PYG{n}{v\PYGZus{}one}\PYG{o}{.}\PYG{n}{scale}\PYG{p}{(}\PYG{l+s+s1}{\PYGZsq{}}\PYG{l+s+s1}{u}\PYG{l+s+s1}{\PYGZsq{}}\PYG{p}{)}  \PYG{c+c1}{\PYGZsh{} get v\PYGZus{}one in microliters}
\end{sphinxVerbatim}
\begin{quote}\begin{description}
\item[{Parameters}] \leavevmode
\sphinxstyleliteralstrong{\sphinxupquote{scale\_key}} (\sphinxstyleliteralemphasis{\sphinxupquote{str}}) \textendash{} Character in {\hyperref[\detokenize{polo.crystallography:polo.crystallography.cocktail.UnitValue.saved_scalers}]{\sphinxcrossref{\sphinxcode{\sphinxupquote{saved\_scalers}}}}}
to convert value to.

\item[{Returns}] \leavevmode
UnitValue converted to scale\_key unit prefix

\item[{Return type}] \leavevmode
{\hyperref[\detokenize{polo.crystallography:polo.crystallography.cocktail.UnitValue}]{\sphinxcrossref{UnitValue}}}

\end{description}\end{quote}

\end{fulllineitems}

\index{set\_from\_string() (polo.crystallography.cocktail.UnitValue method)@\spxentry{set\_from\_string()}\spxextra{polo.crystallography.cocktail.UnitValue method}}

\begin{fulllineitems}
\phantomsection\label{\detokenize{polo.crystallography:polo.crystallography.cocktail.UnitValue.set_from_string}}\pysiglinewithargsret{\sphinxbfcode{\sphinxupquote{set\_from\_string}}}{\emph{\DUrole{n}{string}}}{}
\end{fulllineitems}

\index{to\_base() (polo.crystallography.cocktail.UnitValue method)@\spxentry{to\_base()}\spxextra{polo.crystallography.cocktail.UnitValue method}}

\begin{fulllineitems}
\phantomsection\label{\detokenize{polo.crystallography:polo.crystallography.cocktail.UnitValue.to_base}}\pysiglinewithargsret{\sphinxbfcode{\sphinxupquote{to\_base}}}{}{}
Converts the {\hyperref[\detokenize{polo.crystallography:polo.crystallography.cocktail.UnitValue.value}]{\sphinxcrossref{\sphinxcode{\sphinxupquote{value}}}}}
to the base unit, if it is not already in the base unit.
\begin{quote}\begin{description}
\item[{Returns}] \leavevmode
UnitValue converted to base unit

\item[{Return type}] \leavevmode
{\hyperref[\detokenize{polo.crystallography:polo.crystallography.cocktail.UnitValue}]{\sphinxcrossref{UnitValue}}}

\end{description}\end{quote}

\end{fulllineitems}

\index{value() (polo.crystallography.cocktail.UnitValue property)@\spxentry{value()}\spxextra{polo.crystallography.cocktail.UnitValue property}}

\begin{fulllineitems}
\phantomsection\label{\detokenize{polo.crystallography:polo.crystallography.cocktail.UnitValue.value}}\pysigline{\sphinxbfcode{\sphinxupquote{property }}\sphinxbfcode{\sphinxupquote{value}}}
\end{fulllineitems}


\end{fulllineitems}



\subsubsection{polo.crystallography.image module}
\label{\detokenize{polo.crystallography:module-polo.crystallography.image}}\label{\detokenize{polo.crystallography:polo-crystallography-image-module}}\index{module@\spxentry{module}!polo.crystallography.image@\spxentry{polo.crystallography.image}}\index{polo.crystallography.image@\spxentry{polo.crystallography.image}!module@\spxentry{module}}\index{Image (class in polo.crystallography.image)@\spxentry{Image}\spxextra{class in polo.crystallography.image}}

\begin{fulllineitems}
\phantomsection\label{\detokenize{polo.crystallography:polo.crystallography.image.Image}}\pysiglinewithargsret{\sphinxbfcode{\sphinxupquote{class }}\sphinxcode{\sphinxupquote{polo.crystallography.image.}}\sphinxbfcode{\sphinxupquote{Image}}}{\emph{\DUrole{n}{path}\DUrole{o}{=}\DUrole{default_value}{None}}, \emph{\DUrole{n}{bites}\DUrole{o}{=}\DUrole{default_value}{None}}, \emph{\DUrole{n}{well\_number}\DUrole{o}{=}\DUrole{default_value}{None}}, \emph{\DUrole{n}{human\_class}\DUrole{o}{=}\DUrole{default_value}{None}}, \emph{\DUrole{n}{machine\_class}\DUrole{o}{=}\DUrole{default_value}{None}}, \emph{\DUrole{n}{prediction\_dict}\DUrole{o}{=}\DUrole{default_value}{None}}, \emph{\DUrole{n}{plate\_id}\DUrole{o}{=}\DUrole{default_value}{None}}, \emph{\DUrole{n}{date}\DUrole{o}{=}\DUrole{default_value}{None}}, \emph{\DUrole{n}{cocktail}\DUrole{o}{=}\DUrole{default_value}{None}}, \emph{\DUrole{n}{spectrum}\DUrole{o}{=}\DUrole{default_value}{None}}, \emph{\DUrole{n}{previous\_image}\DUrole{o}{=}\DUrole{default_value}{None}}, \emph{\DUrole{n}{next\_image}\DUrole{o}{=}\DUrole{default_value}{None}}, \emph{\DUrole{n}{alt\_image}\DUrole{o}{=}\DUrole{default_value}{None}}, \emph{\DUrole{n}{favorite}\DUrole{o}{=}\DUrole{default_value}{False}}, \emph{\DUrole{n}{parent}\DUrole{o}{=}\DUrole{default_value}{None}}, \emph{\DUrole{o}{**}\DUrole{n}{kwargs}}}{}
Bases: \sphinxcode{\sphinxupquote{PyQt5.QtGui.QPixmap}}

Image objects hold the data relating to one image in a particular
screening well of a paricular run. Images encode the actual image file as
a file path to the image if it available on the local machine, as 
as a base64 encoded image held in memory or both.
\begin{quote}\begin{description}
\item[{Parameters}] \leavevmode\begin{itemize}
\item {} 
\sphinxstyleliteralstrong{\sphinxupquote{path}} (\sphinxstyleliteralemphasis{\sphinxupquote{str}}\sphinxstyleliteralemphasis{\sphinxupquote{ or }}\sphinxstyleliteralemphasis{\sphinxupquote{Path}}\sphinxstyleliteralemphasis{\sphinxupquote{, }}\sphinxstyleliteralemphasis{\sphinxupquote{optional}}) \textendash{} Path to the actual image file, defaults to None

\item {} 
\sphinxstyleliteralstrong{\sphinxupquote{bites}} (\sphinxstyleliteralemphasis{\sphinxupquote{str}}\sphinxstyleliteralemphasis{\sphinxupquote{ or }}\sphinxstyleliteralemphasis{\sphinxupquote{bytes}}\sphinxstyleliteralemphasis{\sphinxupquote{, }}\sphinxstyleliteralemphasis{\sphinxupquote{optional}}) \textendash{} Image encoded as base64, defaults to None

\item {} 
\sphinxstyleliteralstrong{\sphinxupquote{well\_number}} (\sphinxstyleliteralemphasis{\sphinxupquote{int}}\sphinxstyleliteralemphasis{\sphinxupquote{, }}\sphinxstyleliteralemphasis{\sphinxupquote{optional}}) \textendash{} Well number of the image, defaults to None

\item {} 
\sphinxstyleliteralstrong{\sphinxupquote{human\_class}} (\sphinxstyleliteralemphasis{\sphinxupquote{str}}\sphinxstyleliteralemphasis{\sphinxupquote{, }}\sphinxstyleliteralemphasis{\sphinxupquote{optional}}) \textendash{} Human classification of this image, defaults to None

\item {} 
\sphinxstyleliteralstrong{\sphinxupquote{machine\_class}} (\sphinxstyleliteralemphasis{\sphinxupquote{str}}\sphinxstyleliteralemphasis{\sphinxupquote{, }}\sphinxstyleliteralemphasis{\sphinxupquote{optional}}) \textendash{} MARCO classification of this image, defaults to None

\item {} 
\sphinxstyleliteralstrong{\sphinxupquote{prediction\_dict}} (\sphinxstyleliteralemphasis{\sphinxupquote{dict}}\sphinxstyleliteralemphasis{\sphinxupquote{, }}\sphinxstyleliteralemphasis{\sphinxupquote{optional}}) \textendash{} Dictionary containing MARCO model confidence for
all image classifications, defaults to None

\item {} 
\sphinxstyleliteralstrong{\sphinxupquote{plate\_id}} (\sphinxstyleliteralemphasis{\sphinxupquote{str}}\sphinxstyleliteralemphasis{\sphinxupquote{, }}\sphinxstyleliteralemphasis{\sphinxupquote{optional}}) \textendash{} HWI given unique ID for plate this image belongs to
, defaults to None

\item {} 
\sphinxstyleliteralstrong{\sphinxupquote{date}} (\sphinxstyleliteralemphasis{\sphinxupquote{Datetime}}\sphinxstyleliteralemphasis{\sphinxupquote{, }}\sphinxstyleliteralemphasis{\sphinxupquote{optional}}) \textendash{} Date this image was taken on, defaults to None

\item {} 
\sphinxstyleliteralstrong{\sphinxupquote{cocktail}} ({\hyperref[\detokenize{polo.crystallography:polo.crystallography.cocktail.Cocktail}]{\sphinxcrossref{\sphinxstyleliteralemphasis{\sphinxupquote{Cocktail}}}}}\sphinxstyleliteralemphasis{\sphinxupquote{, }}\sphinxstyleliteralemphasis{\sphinxupquote{optional}}) \textendash{} Cocktail assigned to the well this image is of, defaults to None

\item {} 
\sphinxstyleliteralstrong{\sphinxupquote{spectrum}} (\sphinxstyleliteralemphasis{\sphinxupquote{str}}\sphinxstyleliteralemphasis{\sphinxupquote{, }}\sphinxstyleliteralemphasis{\sphinxupquote{optional}}) \textendash{} Keyword describing the imaging tech used to take the image
, defaults to None

\item {} 
\sphinxstyleliteralstrong{\sphinxupquote{previous\_image}} ({\hyperref[\detokenize{polo.crystallography:polo.crystallography.image.Image}]{\sphinxcrossref{\sphinxstyleliteralemphasis{\sphinxupquote{Image}}}}}\sphinxstyleliteralemphasis{\sphinxupquote{, }}\sphinxstyleliteralemphasis{\sphinxupquote{optional}}) \textendash{} Image of the same well and sample but taken on a previous date
, defaults to None

\item {} 
\sphinxstyleliteralstrong{\sphinxupquote{next\_image}} ({\hyperref[\detokenize{polo.crystallography:polo.crystallography.image.Image}]{\sphinxcrossref{\sphinxstyleliteralemphasis{\sphinxupquote{Image}}}}}\sphinxstyleliteralemphasis{\sphinxupquote{, }}\sphinxstyleliteralemphasis{\sphinxupquote{optional}}) \textendash{} Image of the same well and sample but taken on a future 
date, defaults to None

\item {} 
\sphinxstyleliteralstrong{\sphinxupquote{alt\_image}} ({\hyperref[\detokenize{polo.crystallography:polo.crystallography.image.Image}]{\sphinxcrossref{\sphinxstyleliteralemphasis{\sphinxupquote{Image}}}}}\sphinxstyleliteralemphasis{\sphinxupquote{, }}\sphinxstyleliteralemphasis{\sphinxupquote{optional}}) \textendash{} Image of the same well and sample but taken with a 
different imaging tech, defaults to None

\end{itemize}

\end{description}\end{quote}
\index{bites() (polo.crystallography.image.Image property)@\spxentry{bites()}\spxextra{polo.crystallography.image.Image property}}

\begin{fulllineitems}
\phantomsection\label{\detokenize{polo.crystallography:polo.crystallography.image.Image.bites}}\pysigline{\sphinxbfcode{\sphinxupquote{property }}\sphinxbfcode{\sphinxupquote{bites}}}
\end{fulllineitems}

\index{classify\_image() (polo.crystallography.image.Image method)@\spxentry{classify\_image()}\spxextra{polo.crystallography.image.Image method}}

\begin{fulllineitems}
\phantomsection\label{\detokenize{polo.crystallography:polo.crystallography.image.Image.classify_image}}\pysiglinewithargsret{\sphinxbfcode{\sphinxupquote{classify\_image}}}{}{}
Classify the {\hyperref[\detokenize{polo.crystallography:polo.crystallography.image.Image}]{\sphinxcrossref{\sphinxcode{\sphinxupquote{Image}}}}}
using the MARCO CNN model. Sets the 
\sphinxcode{\sphinxupquote{machine class}} and 
\sphinxcode{\sphinxupquote{prediction\_dict}} 
attributes based on the model results.

\end{fulllineitems}

\index{clean\_base64\_string() (polo.crystallography.image.Image static method)@\spxentry{clean\_base64\_string()}\spxextra{polo.crystallography.image.Image static method}}

\begin{fulllineitems}
\phantomsection\label{\detokenize{polo.crystallography:polo.crystallography.image.Image.clean_base64_string}}\pysiglinewithargsret{\sphinxbfcode{\sphinxupquote{static }}\sphinxbfcode{\sphinxupquote{clean\_base64\_string}}}{\emph{\DUrole{n}{string}}}{}
Image instances may contain byte strings that store their actual
crystallization image encoded as base64. Previously, these byte strings
were written directly into the json file as strings causing the \sphinxtitleref{b}
byte string identifier to be written along with the actual base64 data.
This method removes those artifacts if they are present and returns a
clean byte string with only the actual base64 data.
\begin{quote}\begin{description}
\item[{Parameters}] \leavevmode
\sphinxstyleliteralstrong{\sphinxupquote{string}} (\sphinxstyleliteralemphasis{\sphinxupquote{str}}) \textendash{} A string to interrogate for base64 compliance

\item[{Returns}] \leavevmode
byte string with non\sphinxhyphen{}data artifacts removed

\item[{Return type}] \leavevmode
bytes

\end{description}\end{quote}

\end{fulllineitems}

\index{date() (polo.crystallography.image.Image property)@\spxentry{date()}\spxextra{polo.crystallography.image.Image property}}

\begin{fulllineitems}
\phantomsection\label{\detokenize{polo.crystallography:polo.crystallography.image.Image.date}}\pysigline{\sphinxbfcode{\sphinxupquote{property }}\sphinxbfcode{\sphinxupquote{date}}}
The date associated with this 
{\hyperref[\detokenize{polo.crystallography:polo.crystallography.image.Image}]{\sphinxcrossref{\sphinxcode{\sphinxupquote{Image}}}}}.
Presumably should be the date the image was taken.
\begin{quote}\begin{description}
\item[{Returns}] \leavevmode
Datetime object representation of
the {\hyperref[\detokenize{polo.crystallography:polo.crystallography.image.Image}]{\sphinxcrossref{\sphinxcode{\sphinxupquote{Image}}}}}’s
imaging date

\item[{Return type}] \leavevmode
datetime

\end{description}\end{quote}

\end{fulllineitems}

\index{delete\_all\_pixmap\_data() (polo.crystallography.image.Image method)@\spxentry{delete\_all\_pixmap\_data()}\spxextra{polo.crystallography.image.Image method}}

\begin{fulllineitems}
\phantomsection\label{\detokenize{polo.crystallography:polo.crystallography.image.Image.delete_all_pixmap_data}}\pysiglinewithargsret{\sphinxbfcode{\sphinxupquote{delete\_all\_pixmap\_data}}}{}{}
Deletes the pixmap data for the
{\hyperref[\detokenize{polo.crystallography:polo.crystallography.image.Image}]{\sphinxcrossref{\sphinxcode{\sphinxupquote{Image}}}}} instance this method is
called on and for any other 
\sphinxcode{\sphinxupquote{Image}} is linked to.
This includes images referenced by the 
\sphinxcode{\sphinxupquote{alt\_image}}
, \sphinxcode{\sphinxupquote{next\_image}} and
\sphinxcode{\sphinxupquote{previous\_image}}
attributes.

\end{fulllineitems}

\index{delete\_pixmap\_data() (polo.crystallography.image.Image method)@\spxentry{delete\_pixmap\_data()}\spxextra{polo.crystallography.image.Image method}}

\begin{fulllineitems}
\phantomsection\label{\detokenize{polo.crystallography:polo.crystallography.image.Image.delete_pixmap_data}}\pysiglinewithargsret{\sphinxbfcode{\sphinxupquote{delete\_pixmap\_data}}}{}{}
Replaces the {\hyperref[\detokenize{polo.crystallography:polo.crystallography.image.Image}]{\sphinxcrossref{\sphinxcode{\sphinxupquote{Image}}}}}’s
pixmap data with a null pixmap which
effectively deletes the existing pixmap data. Used to free up
memory after a pixmap is no longer needed.

\end{fulllineitems}

\index{encode\_base64() (polo.crystallography.image.Image method)@\spxentry{encode\_base64()}\spxextra{polo.crystallography.image.Image method}}

\begin{fulllineitems}
\phantomsection\label{\detokenize{polo.crystallography:polo.crystallography.image.Image.encode_base64}}\pysiglinewithargsret{\sphinxbfcode{\sphinxupquote{encode\_base64}}}{}{}
\end{fulllineitems}

\index{encode\_bytes() (polo.crystallography.image.Image method)@\spxentry{encode\_bytes()}\spxextra{polo.crystallography.image.Image method}}

\begin{fulllineitems}
\phantomsection\label{\detokenize{polo.crystallography:polo.crystallography.image.Image.encode_bytes}}\pysiglinewithargsret{\sphinxbfcode{\sphinxupquote{encode\_bytes}}}{}{}
If the {\hyperref[\detokenize{polo.crystallography:polo.crystallography.image.Image.path}]{\sphinxcrossref{\sphinxcode{\sphinxupquote{path}}}}}
attribute exists and is an image file then encodes
that file as a base64 string and returns the encoded
image data.
\begin{quote}\begin{description}
\item[{Returns}] \leavevmode
base64 encoded image

\item[{Return type}] \leavevmode
str

\end{description}\end{quote}

\end{fulllineitems}

\index{formated\_date() (polo.crystallography.image.Image property)@\spxentry{formated\_date()}\spxextra{polo.crystallography.image.Image property}}

\begin{fulllineitems}
\phantomsection\label{\detokenize{polo.crystallography:polo.crystallography.image.Image.formated_date}}\pysigline{\sphinxbfcode{\sphinxupquote{property }}\sphinxbfcode{\sphinxupquote{formated\_date}}}
Get the image’s \sphinxcode{\sphinxupquote{marco\_date}}
attribute formated in the month/date/year format. If the 
{\hyperref[\detokenize{polo.crystallography:polo.crystallography.image.Image}]{\sphinxcrossref{\sphinxcode{\sphinxupquote{Image}}}}} 
has no {\hyperref[\detokenize{polo.crystallography:polo.crystallography.image.Image.date}]{\sphinxcrossref{\sphinxcode{\sphinxupquote{date}}}}} returns
an empty string.
\begin{quote}\begin{description}
\item[{Returns}] \leavevmode
Date

\item[{Return type}] \leavevmode
str

\end{description}\end{quote}

\end{fulllineitems}

\index{get\_linked\_images\_by\_date() (polo.crystallography.image.Image method)@\spxentry{get\_linked\_images\_by\_date()}\spxextra{polo.crystallography.image.Image method}}

\begin{fulllineitems}
\phantomsection\label{\detokenize{polo.crystallography:polo.crystallography.image.Image.get_linked_images_by_date}}\pysiglinewithargsret{\sphinxbfcode{\sphinxupquote{get\_linked\_images\_by\_date}}}{}{}
Get all \sphinxcode{\sphinxupquote{Image}}
instance by date. Image linking by date is accomplished 
by creating a bi\sphinxhyphen{}directional linked list between 
{\hyperref[\detokenize{polo.crystallography:polo.crystallography.image.Image}]{\sphinxcrossref{\sphinxcode{\sphinxupquote{Image}}}}} instances, 
where each {\hyperref[\detokenize{polo.crystallography:polo.crystallography.image.Image}]{\sphinxcrossref{\sphinxcode{\sphinxupquote{Image}}}}} acts as a node
and the \sphinxcode{\sphinxupquote{next\_image}} and 
\sphinxcode{\sphinxupquote{previous\_images}} 
act as the forwards and backwards pointers respectively.
\begin{quote}\begin{description}
\item[{Returns}] \leavevmode
All \sphinxcode{\sphinxupquote{Image}}
by date

\item[{Return type}] \leavevmode
list

\end{description}\end{quote}

\end{fulllineitems}

\index{get\_linked\_images\_by\_spectrum() (polo.crystallography.image.Image method)@\spxentry{get\_linked\_images\_by\_spectrum()}\spxextra{polo.crystallography.image.Image method}}

\begin{fulllineitems}
\phantomsection\label{\detokenize{polo.crystallography:polo.crystallography.image.Image.get_linked_images_by_spectrum}}\pysiglinewithargsret{\sphinxbfcode{\sphinxupquote{get\_linked\_images\_by\_spectrum}}}{}{}
Get all \sphinxcode{\sphinxupquote{Image}} instance by
spectrum. Linking images by spectrum is accomplished by
creating a mono\sphinxhyphen{}directional circular linked list where
{\hyperref[\detokenize{polo.crystallography:polo.crystallography.image.Image}]{\sphinxcrossref{\sphinxcode{\sphinxupquote{Image}}}}} instances serve as nodes and their 
\sphinxcode{\sphinxupquote{alt\_image}} attribute
acts as the pointer to the next node.
\begin{quote}\begin{description}
\item[{Returns}] \leavevmode
List of all \sphinxtitleref{Images} linked to this
{\hyperref[\detokenize{polo.crystallography:polo.crystallography.image.Image}]{\sphinxcrossref{\sphinxcode{\sphinxupquote{Image}}}}} by spectrum

\item[{Return type}] \leavevmode
list

\end{description}\end{quote}

\end{fulllineitems}

\index{get\_tool\_tip() (polo.crystallography.image.Image method)@\spxentry{get\_tool\_tip()}\spxextra{polo.crystallography.image.Image method}}

\begin{fulllineitems}
\phantomsection\label{\detokenize{polo.crystallography:polo.crystallography.image.Image.get_tool_tip}}\pysiglinewithargsret{\sphinxbfcode{\sphinxupquote{get\_tool\_tip}}}{}{}
Create a string to use as the tooltip for this
{\hyperref[\detokenize{polo.crystallography:polo.crystallography.image.Image}]{\sphinxcrossref{\sphinxcode{\sphinxupquote{Image}}}}}.
\begin{quote}\begin{description}
\item[{Returns}] \leavevmode
Tooltip string

\item[{Return type}] \leavevmode
str

\end{description}\end{quote}

\end{fulllineitems}

\index{height() (polo.crystallography.image.Image method)@\spxentry{height()}\spxextra{polo.crystallography.image.Image method}}

\begin{fulllineitems}
\phantomsection\label{\detokenize{polo.crystallography:polo.crystallography.image.Image.height}}\pysiglinewithargsret{\sphinxbfcode{\sphinxupquote{height}}}{}{}
Get the height of the 
{\hyperref[\detokenize{polo.crystallography:polo.crystallography.image.Image}]{\sphinxcrossref{\sphinxcode{\sphinxupquote{Image}}}}}’s pixmap. 
The pixmap must be set for this function to 
return an actual size.
\begin{quote}\begin{description}
\item[{Returns}] \leavevmode
Height of the {\hyperref[\detokenize{polo.crystallography:polo.crystallography.image.Image}]{\sphinxcrossref{\sphinxcode{\sphinxupquote{Image}}}}}’s pixmap

\item[{Return type}] \leavevmode
int

\end{description}\end{quote}

\end{fulllineitems}

\index{human\_class() (polo.crystallography.image.Image property)@\spxentry{human\_class()}\spxextra{polo.crystallography.image.Image property}}

\begin{fulllineitems}
\phantomsection\label{\detokenize{polo.crystallography:polo.crystallography.image.Image.human_class}}\pysigline{\sphinxbfcode{\sphinxupquote{property }}\sphinxbfcode{\sphinxupquote{human\_class}}}
Return the {\hyperref[\detokenize{polo.crystallography:polo.crystallography.image.Image.human_class}]{\sphinxcrossref{\sphinxcode{\sphinxupquote{human\_class}}}}}
attribute which specifies the current human classification of the
{\hyperref[\detokenize{polo.crystallography:polo.crystallography.image.Image}]{\sphinxcrossref{\sphinxcode{\sphinxupquote{Image}}}}}.
\begin{quote}\begin{description}
\item[{Returns}] \leavevmode
Current human classification of the
{\hyperref[\detokenize{polo.crystallography:polo.crystallography.image.Image}]{\sphinxcrossref{\sphinxcode{\sphinxupquote{Image}}}}}

\item[{Return type}] \leavevmode
str

\end{description}\end{quote}

\end{fulllineitems}

\index{machine\_class() (polo.crystallography.image.Image property)@\spxentry{machine\_class()}\spxextra{polo.crystallography.image.Image property}}

\begin{fulllineitems}
\phantomsection\label{\detokenize{polo.crystallography:polo.crystallography.image.Image.machine_class}}\pysigline{\sphinxbfcode{\sphinxupquote{property }}\sphinxbfcode{\sphinxupquote{machine\_class}}}
MARCO classification of the {\hyperref[\detokenize{polo.crystallography:polo.crystallography.image.Image}]{\sphinxcrossref{\sphinxcode{\sphinxupquote{Image}}}}}.
\begin{quote}\begin{description}
\item[{Returns}] \leavevmode
Current MARCO classification of this image

\item[{Return type}] \leavevmode
str

\end{description}\end{quote}

\end{fulllineitems}

\index{no\_image() (polo.crystallography.image.Image class method)@\spxentry{no\_image()}\spxextra{polo.crystallography.image.Image class method}}

\begin{fulllineitems}
\phantomsection\label{\detokenize{polo.crystallography:polo.crystallography.image.Image.no_image}}\pysiglinewithargsret{\sphinxbfcode{\sphinxupquote{classmethod }}\sphinxbfcode{\sphinxupquote{no\_image}}}{}{}
Return an {\hyperref[\detokenize{polo.crystallography:polo.crystallography.image.Image}]{\sphinxcrossref{\sphinxcode{\sphinxupquote{Image}}}}} 
instance using the image data referenced by the
\sphinxcode{\sphinxupquote{polo.DEFAULT\_IMAGE\_PATH}} constant.
The default image is used to fill in for missing 
data and when filters cannot find any matching results.
\begin{quote}\begin{description}
\item[{Returns}] \leavevmode
Default {\hyperref[\detokenize{polo.crystallography:polo.crystallography.image.Image}]{\sphinxcrossref{\sphinxcode{\sphinxupquote{Image}}}}}

\item[{Return type}] \leavevmode
{\hyperref[\detokenize{polo.crystallography:polo.crystallography.image.Image}]{\sphinxcrossref{Image}}}

\end{description}\end{quote}

\end{fulllineitems}

\index{path() (polo.crystallography.image.Image property)@\spxentry{path()}\spxextra{polo.crystallography.image.Image property}}

\begin{fulllineitems}
\phantomsection\label{\detokenize{polo.crystallography:polo.crystallography.image.Image.path}}\pysigline{\sphinxbfcode{\sphinxupquote{property }}\sphinxbfcode{\sphinxupquote{path}}}
Filepath for the image. Note that if this path is loaded
from an xtal file, this path may not exists because the xtal
file may have been created on a different machine.
\begin{quote}\begin{description}
\item[{Returns}] \leavevmode
Path to image file

\item[{Return type}] \leavevmode
str

\end{description}\end{quote}

\end{fulllineitems}

\index{setPixmap() (polo.crystallography.image.Image method)@\spxentry{setPixmap()}\spxextra{polo.crystallography.image.Image method}}

\begin{fulllineitems}
\phantomsection\label{\detokenize{polo.crystallography:polo.crystallography.image.Image.setPixmap}}\pysiglinewithargsret{\sphinxbfcode{\sphinxupquote{setPixmap}}}{\emph{\DUrole{n}{scaling}\DUrole{o}{=}\DUrole{default_value}{None}}}{}
Loads the {\hyperref[\detokenize{polo.crystallography:polo.crystallography.image.Image}]{\sphinxcrossref{\sphinxcode{\sphinxupquote{Image}}}}}’s 
pixmap into memory which then allows for displaying
the {\hyperref[\detokenize{polo.crystallography:polo.crystallography.image.Image}]{\sphinxcrossref{\sphinxcode{\sphinxupquote{Image}}}}} to the user. 
\sphinxcode{\sphinxupquote{Image}} pixmap when
the {\hyperref[\detokenize{polo.crystallography:polo.crystallography.image.Image}]{\sphinxcrossref{\sphinxcode{\sphinxupquote{Image}}}}} actually
needs to be shown to the user as it is expensive
to hold in memory.
\begin{quote}\begin{description}
\item[{Parameters}] \leavevmode
\sphinxstyleliteralstrong{\sphinxupquote{scaling}} (\sphinxstyleliteralemphasis{\sphinxupquote{float}}\sphinxstyleliteralemphasis{\sphinxupquote{, }}\sphinxstyleliteralemphasis{\sphinxupquote{optional}}) \textendash{} Scaler for the pixmap; between 0 and 1, defaults to None

\end{description}\end{quote}

\end{fulllineitems}

\index{standard\_filter() (polo.crystallography.image.Image method)@\spxentry{standard\_filter()}\spxextra{polo.crystallography.image.Image method}}

\begin{fulllineitems}
\phantomsection\label{\detokenize{polo.crystallography:polo.crystallography.image.Image.standard_filter}}\pysiglinewithargsret{\sphinxbfcode{\sphinxupquote{standard\_filter}}}{\emph{\DUrole{n}{image\_types}}, \emph{\DUrole{n}{human}}, \emph{\DUrole{n}{marco}}, \emph{\DUrole{n}{favorite}}}{}
Method that determines if this 
{\hyperref[\detokenize{polo.crystallography:polo.crystallography.image.Image}]{\sphinxcrossref{\sphinxcode{\sphinxupquote{Image}}}}} should be
included in a set of filtered \sphinxcode{\sphinxupquote{Image}} meets the
filtering requirements specified by the method’s arguments,
otherwise returns False.
\begin{quote}\begin{description}
\item[{Parameters}] \leavevmode\begin{itemize}
\item {} 
\sphinxstyleliteralstrong{\sphinxupquote{image\_types}} (\sphinxstyleliteralemphasis{\sphinxupquote{list}}\sphinxstyleliteralemphasis{\sphinxupquote{ or }}\sphinxstyleliteralemphasis{\sphinxupquote{set}}) \textendash{} Collection of image classifications.
The {\hyperref[\detokenize{polo.crystallography:polo.crystallography.image.Image}]{\sphinxcrossref{\sphinxcode{\sphinxupquote{Image}}}}}’s
classification must in included in this collection 
for the method to return True.

\item {} 
\sphinxstyleliteralstrong{\sphinxupquote{human}} (\sphinxstyleliteralemphasis{\sphinxupquote{bool}}) \textendash{} If True, use the 
{\hyperref[\detokenize{polo.crystallography:polo.crystallography.image.Image}]{\sphinxcrossref{\sphinxcode{\sphinxupquote{Image}}}}}’s
human classification as the
overall image classification.

\item {} 
\sphinxstyleliteralstrong{\sphinxupquote{marco}} (\sphinxstyleliteralemphasis{\sphinxupquote{bool}}) \textendash{} If True, use the
{\hyperref[\detokenize{polo.crystallography:polo.crystallography.image.Image}]{\sphinxcrossref{\sphinxcode{\sphinxupquote{Image}}}}}’s
MARCO classification as the
overall image classification.

\end{itemize}

\item[{Returns}] \leavevmode
True if the
{\hyperref[\detokenize{polo.crystallography:polo.crystallography.image.Image}]{\sphinxcrossref{\sphinxcode{\sphinxupquote{Image}}}}}
meets the filter requirements, False otherwise

\item[{Return type}] \leavevmode
bool

\end{description}\end{quote}

\end{fulllineitems}

\index{to\_graphics\_scene() (polo.crystallography.image.Image class method)@\spxentry{to\_graphics\_scene()}\spxextra{polo.crystallography.image.Image class method}}

\begin{fulllineitems}
\phantomsection\label{\detokenize{polo.crystallography:polo.crystallography.image.Image.to_graphics_scene}}\pysiglinewithargsret{\sphinxbfcode{\sphinxupquote{classmethod }}\sphinxbfcode{\sphinxupquote{to\_graphics\_scene}}}{\emph{\DUrole{n}{image}}}{}
Convert an Image object to a \sphinxtitleref{QGraphicsScene} with
the Image added as a pixmap to the \sphinxtitleref{QGraphicsScene}.
\begin{quote}\begin{description}
\item[{Parameters}] \leavevmode
\sphinxstyleliteralstrong{\sphinxupquote{image}} ({\hyperref[\detokenize{polo.crystallography:polo.crystallography.image.Image}]{\sphinxcrossref{\sphinxstyleliteralemphasis{\sphinxupquote{Image}}}}}) \textendash{} Image instance

\item[{Returns}] \leavevmode
QGraphicsScene

\item[{Return type}] \leavevmode
QGraphicsScene

\end{description}\end{quote}

\end{fulllineitems}

\index{width() (polo.crystallography.image.Image method)@\spxentry{width()}\spxextra{polo.crystallography.image.Image method}}

\begin{fulllineitems}
\phantomsection\label{\detokenize{polo.crystallography:polo.crystallography.image.Image.width}}\pysiglinewithargsret{\sphinxbfcode{\sphinxupquote{width}}}{}{}
Get the height of the 
{\hyperref[\detokenize{polo.crystallography:polo.crystallography.image.Image}]{\sphinxcrossref{\sphinxcode{\sphinxupquote{Image}}}}}’s pixmap. 
The pixmap must be set for this function to return 
an actual size.
\begin{quote}\begin{description}
\item[{Returns}] \leavevmode
Width of the 
{\hyperref[\detokenize{polo.crystallography:polo.crystallography.image.Image}]{\sphinxcrossref{\sphinxcode{\sphinxupquote{Image}}}}}’s
pixmap

\item[{Return type}] \leavevmode
int

\end{description}\end{quote}

\end{fulllineitems}


\end{fulllineitems}



\subsubsection{polo.crystallography.run module}
\label{\detokenize{polo.crystallography:module-polo.crystallography.run}}\label{\detokenize{polo.crystallography:polo-crystallography-run-module}}\index{module@\spxentry{module}!polo.crystallography.run@\spxentry{polo.crystallography.run}}\index{polo.crystallography.run@\spxentry{polo.crystallography.run}!module@\spxentry{module}}\index{HWIRun (class in polo.crystallography.run)@\spxentry{HWIRun}\spxextra{class in polo.crystallography.run}}

\begin{fulllineitems}
\phantomsection\label{\detokenize{polo.crystallography:polo.crystallography.run.HWIRun}}\pysiglinewithargsret{\sphinxbfcode{\sphinxupquote{class }}\sphinxcode{\sphinxupquote{polo.crystallography.run.}}\sphinxbfcode{\sphinxupquote{HWIRun}}}{\emph{\DUrole{n}{cocktail\_menu}}, \emph{\DUrole{n}{plate\_id}\DUrole{o}{=}\DUrole{default_value}{None}}, \emph{\DUrole{n}{num\_wells}\DUrole{o}{=}\DUrole{default_value}{1536}}, \emph{\DUrole{n}{alt\_spectrum}\DUrole{o}{=}\DUrole{default_value}{None}}, \emph{\DUrole{n}{next\_run}\DUrole{o}{=}\DUrole{default_value}{None}}, \emph{\DUrole{n}{previous\_run}\DUrole{o}{=}\DUrole{default_value}{None}}, \emph{\DUrole{o}{**}\DUrole{n}{kwargs}}}{}
Bases: {\hyperref[\detokenize{polo.crystallography:polo.crystallography.run.Run}]{\sphinxcrossref{\sphinxcode{\sphinxupquote{polo.crystallography.run.Run}}}}}
\index{AllOWED\_PLOTS (polo.crystallography.run.HWIRun attribute)@\spxentry{AllOWED\_PLOTS}\spxextra{polo.crystallography.run.HWIRun attribute}}

\begin{fulllineitems}
\phantomsection\label{\detokenize{polo.crystallography:polo.crystallography.run.HWIRun.AllOWED_PLOTS}}\pysigline{\sphinxbfcode{\sphinxupquote{AllOWED\_PLOTS}}\sphinxbfcode{\sphinxupquote{ = {[}\textquotesingle{}Classification Counts\textquotesingle{}, \textquotesingle{}MARCO Accuracy\textquotesingle{}, \textquotesingle{}Classification Progress\textquotesingle{}, \textquotesingle{}Plate Heatmaps\textquotesingle{}{]}}}}
\end{fulllineitems}

\index{add\_images\_from\_dir() (polo.crystallography.run.HWIRun method)@\spxentry{add\_images\_from\_dir()}\spxextra{polo.crystallography.run.HWIRun method}}

\begin{fulllineitems}
\phantomsection\label{\detokenize{polo.crystallography:polo.crystallography.run.HWIRun.add_images_from_dir}}\pysiglinewithargsret{\sphinxbfcode{\sphinxupquote{add\_images\_from\_dir}}}{}{}
Populates the \sphinxcode{\sphinxupquote{images}} 
attribute with a list of \sphinxtitleref{Images} instances
read from the \sphinxcode{\sphinxupquote{image\_dir}}
attribute filepath. Currently is 
dependent on having a cocktail Menu available.

\end{fulllineitems}

\index{get\_linked\_alt\_runs() (polo.crystallography.run.HWIRun method)@\spxentry{get\_linked\_alt\_runs()}\spxextra{polo.crystallography.run.HWIRun method}}

\begin{fulllineitems}
\phantomsection\label{\detokenize{polo.crystallography:polo.crystallography.run.HWIRun.get_linked_alt_runs}}\pysiglinewithargsret{\sphinxbfcode{\sphinxupquote{get\_linked\_alt\_runs}}}{}{}
Return all \sphinxcode{\sphinxupquote{HWIRun}} is linked to by spectrum. See
\sphinxcode{\sphinxupquote{link\_to\_alt\_spectrum()}}.
\begin{quote}\begin{description}
\item[{Returns}] \leavevmode
List of runs linked to this run by spectrum

\item[{Return type}] \leavevmode
list

\end{description}\end{quote}

\end{fulllineitems}

\index{get\_tooltip() (polo.crystallography.run.HWIRun method)@\spxentry{get\_tooltip()}\spxextra{polo.crystallography.run.HWIRun method}}

\begin{fulllineitems}
\phantomsection\label{\detokenize{polo.crystallography:polo.crystallography.run.HWIRun.get_tooltip}}\pysiglinewithargsret{\sphinxbfcode{\sphinxupquote{get\_tooltip}}}{}{}
The same as \sphinxcode{\sphinxupquote{get\_tooltip()}}.

\end{fulllineitems}

\index{insert\_into\_alt\_spec\_chain() (polo.crystallography.run.HWIRun method)@\spxentry{insert\_into\_alt\_spec\_chain()}\spxextra{polo.crystallography.run.HWIRun method}}

\begin{fulllineitems}
\phantomsection\label{\detokenize{polo.crystallography:polo.crystallography.run.HWIRun.insert_into_alt_spec_chain}}\pysiglinewithargsret{\sphinxbfcode{\sphinxupquote{insert\_into\_alt\_spec\_chain}}}{}{}
When runs are first loaded into Polo they are automatically linked together.
Normally, \sphinxtitleref{HWIRuns} that are linked by date should contain only 
visible spectrum images and \sphinxcode{\sphinxupquote{alt\_spectrum}} attribute.
This means that one could navigate from a visible
spectrum {\hyperref[\detokenize{polo.crystallography:polo.crystallography.run.HWIRun}]{\sphinxcrossref{\sphinxcode{\sphinxupquote{HWIRun}}}}} to all alt spectrum 
\sphinxcode{\sphinxupquote{link\_to\_alt\_spectrum}}.

\end{fulllineitems}

\index{link\_to\_alt\_spectrum() (polo.crystallography.run.HWIRun method)@\spxentry{link\_to\_alt\_spectrum()}\spxextra{polo.crystallography.run.HWIRun method}}

\begin{fulllineitems}
\phantomsection\label{\detokenize{polo.crystallography:polo.crystallography.run.HWIRun.link_to_alt_spectrum}}\pysiglinewithargsret{\sphinxbfcode{\sphinxupquote{link\_to\_alt\_spectrum}}}{\emph{\DUrole{n}{other\_run}}}{}
Similar to \sphinxcode{\sphinxupquote{link\_to\_next\_date()}}
except instead of creating a linked list through the 
\sphinxcode{\sphinxupquote{next\_run}} and
\sphinxcode{\sphinxupquote{previous\_run}}
attributes this method does so through the 
\sphinxcode{\sphinxupquote{alt\_spectrum}}
attribute. The linked list created is mono\sphinxhyphen{}directional so if a
series of {\hyperref[\detokenize{polo.crystallography:polo.crystallography.run.HWIRun}]{\sphinxcrossref{\sphinxcode{\sphinxupquote{HWIRun}}}}}’s 
are being linked the last run should be linked to the
first run to circularize the linked list.
\begin{quote}\begin{description}
\item[{Parameters}] \leavevmode
\sphinxstyleliteralstrong{\sphinxupquote{other\_run}} ({\hyperref[\detokenize{polo.crystallography:polo.crystallography.run.HWIRun}]{\sphinxcrossref{\sphinxstyleliteralemphasis{\sphinxupquote{HWIRun}}}}}) \textendash{} {\hyperref[\detokenize{polo.crystallography:polo.crystallography.run.HWIRun}]{\sphinxcrossref{\sphinxcode{\sphinxupquote{HWIRun}}}}} to 
link to this {\hyperref[\detokenize{polo.crystallography:polo.crystallography.run.HWIRun}]{\sphinxcrossref{\sphinxcode{\sphinxupquote{HWIRun}}}}} by spectrum

\end{description}\end{quote}

\end{fulllineitems}

\index{link\_to\_next\_date() (polo.crystallography.run.HWIRun method)@\spxentry{link\_to\_next\_date()}\spxextra{polo.crystallography.run.HWIRun method}}

\begin{fulllineitems}
\phantomsection\label{\detokenize{polo.crystallography:polo.crystallography.run.HWIRun.link_to_next_date}}\pysiglinewithargsret{\sphinxbfcode{\sphinxupquote{link\_to\_next\_date}}}{\emph{\DUrole{n}{other\_run}}}{}
Link this {\hyperref[\detokenize{polo.crystallography:polo.crystallography.run.HWIRun}]{\sphinxcrossref{\sphinxcode{\sphinxupquote{HWIRun}}}}} to another 
{\hyperref[\detokenize{polo.crystallography:polo.crystallography.run.HWIRun}]{\sphinxcrossref{\sphinxcode{\sphinxupquote{HWIRun}}}}} instance that is of the same
sample but photographed at a later date. This creates a
bi\sphinxhyphen{}directional linked list structure between the two 
\sphinxcode{\sphinxupquote{HWIRun}} instance will point to the 
\sphinxtitleref{other\_run} through the 
\sphinxcode{\sphinxupquote{next\_run}} attribute and
\sphinxtitleref{other\_run} will point back to this {\hyperref[\detokenize{polo.crystallography:polo.crystallography.run.HWIRun}]{\sphinxcrossref{\sphinxcode{\sphinxupquote{HWIRun}}}}} through
its \sphinxcode{\sphinxupquote{previous\_run}}
attribute. This method does not attempt to recognize
which run was imaged first so this should be determined before calling,
likely by sorting a list of \sphinxtitleref{HWIRun instances by their 
:attr:\textasciigrave{}\textasciitilde{}polo.crystallography.run.Run.date} attribute.

Example:

\begin{sphinxVerbatim}[commandchars=\\\{\}]
\PYG{c+c1}{\PYGZsh{} Starting with a collection of Run objects in a list}
\PYG{n}{runs} \PYG{o}{=} \PYG{p}{[}\PYG{n}{run\PYGZus{}b}\PYG{p}{,} \PYG{n}{run\PYGZus{}a}\PYG{p}{,} \PYG{n}{run\PYGZus{}d}\PYG{p}{,} \PYG{n}{run\PYGZus{}c}\PYG{p}{]}
\PYG{c+c1}{\PYGZsh{} sort them by date}
\PYG{n}{runs} \PYG{o}{=} \PYG{n+nb}{sorted}\PYG{p}{(}\PYG{n}{runs}\PYG{p}{,} \PYG{k}{lambda} \PYG{n}{r}\PYG{p}{:} \PYG{n}{r}\PYG{o}{.}\PYG{n}{date}\PYG{p}{)}
\PYG{c+c1}{\PYGZsh{} link them together by date}
\PYG{p}{[}\PYG{n}{r}\PYG{p}{[}\PYG{n}{i}\PYG{p}{]}\PYG{o}{.}\PYG{n}{link\PYGZus{}to\PYGZus{}next\PYGZus{}date}\PYG{p}{(}\PYG{n}{r}\PYG{p}{[}\PYG{n}{i}\PYG{o}{+}\PYG{l+m+mi}{1}\PYG{p}{]}\PYG{p}{)} \PYG{k}{for} \PYG{n}{i} \PYG{o+ow}{in} \PYG{n+nb}{range}\PYG{p}{(}\PYG{n+nb}{len}\PYG{p}{(}\PYG{n}{runs}\PYG{p}{)}\PYG{o}{\PYGZhy{}}\PYG{l+m+mi}{2}\PYG{p}{)}\PYG{p}{]}
\end{sphinxVerbatim}

This would create a linked list with a structure link the representation
below.

\begin{sphinxVerbatim}[commandchars=\\\{\}]
run\PYGZus{}a \PYGZlt{}\PYGZhy{}\PYGZgt{} run\PYGZus{}b \PYGZlt{}\PYGZhy{}\PYGZgt{} run\PYGZus{}c \PYGZlt{}\PYGZhy{}\PYGZgt{} run\PYGZus{}d
\end{sphinxVerbatim}
\begin{quote}\begin{description}
\item[{Parameters}] \leavevmode
\sphinxstyleliteralstrong{\sphinxupquote{other\_run}} ({\hyperref[\detokenize{polo.crystallography:polo.crystallography.run.HWIRun}]{\sphinxcrossref{\sphinxstyleliteralemphasis{\sphinxupquote{HWIRun}}}}}) \textendash{} {\hyperref[\detokenize{polo.crystallography:polo.crystallography.run.HWIRun}]{\sphinxcrossref{\sphinxcode{\sphinxupquote{HWIRun}}}}}
instance representing the next imaging run

\end{description}\end{quote}

\end{fulllineitems}


\end{fulllineitems}

\index{Run (class in polo.crystallography.run)@\spxentry{Run}\spxextra{class in polo.crystallography.run}}

\begin{fulllineitems}
\phantomsection\label{\detokenize{polo.crystallography:polo.crystallography.run.Run}}\pysiglinewithargsret{\sphinxbfcode{\sphinxupquote{class }}\sphinxcode{\sphinxupquote{polo.crystallography.run.}}\sphinxbfcode{\sphinxupquote{Run}}}{\emph{\DUrole{n}{image\_dir}}, \emph{\DUrole{n}{run\_name}}, \emph{\DUrole{n}{image\_spectrum}\DUrole{o}{=}\DUrole{default_value}{None}}, \emph{\DUrole{n}{date}\DUrole{o}{=}\DUrole{default_value}{None}}, \emph{\DUrole{n}{images}\DUrole{o}{=}\DUrole{default_value}{{[}{]}}}, \emph{\DUrole{o}{**}\DUrole{n}{kwargs}}}{}
Bases: \sphinxcode{\sphinxupquote{object}}
\index{AllOWED\_PLOTS (polo.crystallography.run.Run attribute)@\spxentry{AllOWED\_PLOTS}\spxextra{polo.crystallography.run.Run attribute}}

\begin{fulllineitems}
\phantomsection\label{\detokenize{polo.crystallography:polo.crystallography.run.Run.AllOWED_PLOTS}}\pysigline{\sphinxbfcode{\sphinxupquote{AllOWED\_PLOTS}}\sphinxbfcode{\sphinxupquote{ = {[}\textquotesingle{}Classification Counts\textquotesingle{}, \textquotesingle{}MARCO Accuracy\textquotesingle{}, \textquotesingle{}Classification Progress\textquotesingle{}{]}}}}
\end{fulllineitems}

\index{add\_images\_from\_dir() (polo.crystallography.run.Run method)@\spxentry{add\_images\_from\_dir()}\spxextra{polo.crystallography.run.Run method}}

\begin{fulllineitems}
\phantomsection\label{\detokenize{polo.crystallography:polo.crystallography.run.Run.add_images_from_dir}}\pysiglinewithargsret{\sphinxbfcode{\sphinxupquote{add\_images\_from\_dir}}}{}{}
Adds the contents of a directory to 
\sphinxcode{\sphinxupquote{images}}
attribute.

\end{fulllineitems}

\index{encode\_images\_to\_base64() (polo.crystallography.run.Run method)@\spxentry{encode\_images\_to\_base64()}\spxextra{polo.crystallography.run.Run method}}

\begin{fulllineitems}
\phantomsection\label{\detokenize{polo.crystallography:polo.crystallography.run.Run.encode_images_to_base64}}\pysiglinewithargsret{\sphinxbfcode{\sphinxupquote{encode\_images\_to\_base64}}}{}{}
Helper method that encodes all images in the
\sphinxtitleref{class\textasciigrave{}\textasciitilde{}polo.crystallography.run.Run} to base64.

\end{fulllineitems}

\index{get\_current\_hits() (polo.crystallography.run.Run method)@\spxentry{get\_current\_hits()}\spxextra{polo.crystallography.run.Run method}}

\begin{fulllineitems}
\phantomsection\label{\detokenize{polo.crystallography:polo.crystallography.run.Run.get_current_hits}}\pysiglinewithargsret{\sphinxbfcode{\sphinxupquote{get\_current\_hits}}}{}{}
\end{fulllineitems}

\index{get\_images\_by\_classification() (polo.crystallography.run.Run method)@\spxentry{get\_images\_by\_classification()}\spxextra{polo.crystallography.run.Run method}}

\begin{fulllineitems}
\phantomsection\label{\detokenize{polo.crystallography:polo.crystallography.run.Run.get_images_by_classification}}\pysiglinewithargsret{\sphinxbfcode{\sphinxupquote{get\_images\_by\_classification}}}{\emph{\DUrole{n}{human}\DUrole{o}{=}\DUrole{default_value}{True}}}{}
Create a dictionary of image classifications. Keys are
each type of classification and values are lists of
{\hyperref[\detokenize{polo.crystallography:polo.crystallography.image.Image}]{\sphinxcrossref{\sphinxcode{\sphinxupquote{Image}}}}} s 
with classification equal to the key value. The \sphinxtitleref{human}
boolean determines what classifier should be used to
determine the image type. Human = True sets the human
as the classifier and False sets MARCO as the classifier.

\end{fulllineitems}

\index{get\_tooltip() (polo.crystallography.run.Run method)@\spxentry{get\_tooltip()}\spxextra{polo.crystallography.run.Run method}}

\begin{fulllineitems}
\phantomsection\label{\detokenize{polo.crystallography:polo.crystallography.run.Run.get_tooltip}}\pysiglinewithargsret{\sphinxbfcode{\sphinxupquote{get\_tooltip}}}{}{}
\end{fulllineitems}

\index{image\_filter\_query() (polo.crystallography.run.Run method)@\spxentry{image\_filter\_query()}\spxextra{polo.crystallography.run.Run method}}

\begin{fulllineitems}
\phantomsection\label{\detokenize{polo.crystallography:polo.crystallography.run.Run.image_filter_query}}\pysiglinewithargsret{\sphinxbfcode{\sphinxupquote{image\_filter\_query}}}{\emph{\DUrole{n}{image\_types}}, \emph{\DUrole{n}{human}}, \emph{\DUrole{n}{marco}}, \emph{\DUrole{n}{favorite}}}{}
General use method for returning :class:{\color{red}\bfseries{}\textasciigrave{}}\textasciitilde{}polo.crystallography.image.Image\textasciigrave{}s
based on a set of filters. Used whereever a user is allowed to narrow the set
of :class:{\color{red}\bfseries{}\textasciigrave{}}\textasciitilde{}polo.crystallography.image.Image\textasciigrave{}s available for view.
\begin{quote}\begin{description}
\item[{Parameters}] \leavevmode\begin{itemize}
\item {} 
\sphinxstyleliteralstrong{\sphinxupquote{image\_types}} (\sphinxstyleliteralemphasis{\sphinxupquote{list}}\sphinxstyleliteralemphasis{\sphinxupquote{ or }}\sphinxstyleliteralemphasis{\sphinxupquote{set}}) \textendash{} Returned images must have a
classification that is included in 
this variable

\item {} 
\sphinxstyleliteralstrong{\sphinxupquote{human}} (\sphinxstyleliteralemphasis{\sphinxupquote{bool}}) \textendash{} Qualify the classification type
with a human classifier.

\item {} 
\sphinxstyleliteralstrong{\sphinxupquote{marco}} (\sphinxstyleliteralemphasis{\sphinxupquote{bool}}) \textendash{} Qualify the classification type with
a MARCO classifier.

\item {} 
\sphinxstyleliteralstrong{\sphinxupquote{favorite}} (\sphinxstyleliteralemphasis{\sphinxupquote{bool}}) \textendash{} Returned images must be marked as 
\sphinxtitleref{favorite} if set to True

\end{itemize}

\end{description}\end{quote}

\end{fulllineitems}

\index{unload\_all\_pixmaps() (polo.crystallography.run.Run method)@\spxentry{unload\_all\_pixmaps()}\spxextra{polo.crystallography.run.Run method}}

\begin{fulllineitems}
\phantomsection\label{\detokenize{polo.crystallography:polo.crystallography.run.Run.unload_all_pixmaps}}\pysiglinewithargsret{\sphinxbfcode{\sphinxupquote{unload\_all\_pixmaps}}}{\emph{\DUrole{n}{start}\DUrole{o}{=}\DUrole{default_value}{None}}, \emph{\DUrole{n}{end}\DUrole{o}{=}\DUrole{default_value}{None}}, \emph{\DUrole{n}{a}\DUrole{o}{=}\DUrole{default_value}{False}}}{}
Delete the pixmap data of all 
{\hyperref[\detokenize{polo.crystallography:polo.crystallography.image.Image}]{\sphinxcrossref{\sphinxcode{\sphinxupquote{Image}}}}} instances referenced by the
\sphinxcode{\sphinxupquote{images}} attribute. 
This method should be used to free up memory after the 
:class:{\color{red}\bfseries{}\textasciigrave{}}\textasciitilde{}polo.crystallography.run.Run\textasciigrave{}is no longer being
viewed by the user.
\begin{quote}\begin{description}
\item[{Parameters}] \leavevmode\begin{itemize}
\item {} 
\sphinxstyleliteralstrong{\sphinxupquote{start}} (\sphinxstyleliteralemphasis{\sphinxupquote{int}}\sphinxstyleliteralemphasis{\sphinxupquote{, }}\sphinxstyleliteralemphasis{\sphinxupquote{optional}}) \textendash{} Start index for range of images to unload, defaults to None

\item {} 
\sphinxstyleliteralstrong{\sphinxupquote{end}} (\sphinxstyleliteralemphasis{\sphinxupquote{int}}\sphinxstyleliteralemphasis{\sphinxupquote{, }}\sphinxstyleliteralemphasis{\sphinxupquote{optional}}) \textendash{} End index for range of images to unload, defaults to None

\item {} 
\sphinxstyleliteralstrong{\sphinxupquote{a}} \textendash{} Flag to unload pixmap data for all \sphinxtitleref{Images}
this {\hyperref[\detokenize{polo.crystallography:polo.crystallography.image.Image}]{\sphinxcrossref{\sphinxcode{\sphinxupquote{Image}}}}} is linked
to

\end{itemize}

\end{description}\end{quote}

\end{fulllineitems}


\end{fulllineitems}



\subsubsection{Module contents}
\label{\detokenize{polo.crystallography:module-polo.crystallography}}\label{\detokenize{polo.crystallography:module-contents}}\index{module@\spxentry{module}!polo.crystallography@\spxentry{polo.crystallography}}\index{polo.crystallography@\spxentry{polo.crystallography}!module@\spxentry{module}}

\subsection{polo.marco package}
\label{\detokenize{polo.marco:polo-marco-package}}\label{\detokenize{polo.marco::doc}}

\subsubsection{Submodules}
\label{\detokenize{polo.marco:submodules}}

\subsubsection{polo.marco.run\_marco module}
\label{\detokenize{polo.marco:module-polo.marco.run_marco}}\label{\detokenize{polo.marco:polo-marco-run-marco-module}}\index{module@\spxentry{module}!polo.marco.run\_marco@\spxentry{polo.marco.run\_marco}}\index{polo.marco.run\_marco@\spxentry{polo.marco.run\_marco}!module@\spxentry{module}}\index{classify\_image() (in module polo.marco.run\_marco)@\spxentry{classify\_image()}\spxextra{in module polo.marco.run\_marco}}

\begin{fulllineitems}
\phantomsection\label{\detokenize{polo.marco:polo.marco.run_marco.classify_image}}\pysiglinewithargsret{\sphinxcode{\sphinxupquote{polo.marco.run\_marco.}}\sphinxbfcode{\sphinxupquote{classify\_image}}}{\emph{\DUrole{n}{tf\_predictor}}, \emph{\DUrole{n}{image\_path}}}{}
Given a tensorflow predictor (the MARCO model) and the path to an image, 
runs the model on that image. Returns a tuple where the first item is the
classification with greatest confidence and the second is a dictionary where
keys are image classification types and values are model confidence for that
classification. The image classifications of this dictionary are used
universally throughout the program and are accessible through the
\sphinxcode{\sphinxupquote{IMAGE\_CLASSIFICATIONS}} constant.
\begin{quote}\begin{description}
\item[{Parameters}] \leavevmode\begin{itemize}
\item {} 
\sphinxstyleliteralstrong{\sphinxupquote{tf\_predictor}} (\sphinxstyleliteralemphasis{\sphinxupquote{tensorflow model}}) \textendash{} Loaded MARCO model

\item {} 
\sphinxstyleliteralstrong{\sphinxupquote{image\_path}} (\sphinxstyleliteralemphasis{\sphinxupquote{str}}) \textendash{} Path to the image to be classified by the model

\end{itemize}

\item[{Returns}] \leavevmode
tuple

\item[{Return type}] \leavevmode
tuple

\end{description}\end{quote}

\end{fulllineitems}



\subsubsection{Module contents}
\label{\detokenize{polo.marco:module-polo.marco}}\label{\detokenize{polo.marco:module-contents}}\index{module@\spxentry{module}!polo.marco@\spxentry{polo.marco}}\index{polo.marco@\spxentry{polo.marco}!module@\spxentry{module}}

\subsection{polo.threads package}
\label{\detokenize{polo.threads:polo-threads-package}}\label{\detokenize{polo.threads::doc}}

\subsubsection{Submodules}
\label{\detokenize{polo.threads:submodules}}

\subsubsection{polo.threads.thread module}
\label{\detokenize{polo.threads:module-polo.threads.thread}}\label{\detokenize{polo.threads:polo-threads-thread-module}}\index{module@\spxentry{module}!polo.threads.thread@\spxentry{polo.threads.thread}}\index{polo.threads.thread@\spxentry{polo.threads.thread}!module@\spxentry{module}}\index{ClassificationThread (class in polo.threads.thread)@\spxentry{ClassificationThread}\spxextra{class in polo.threads.thread}}

\begin{fulllineitems}
\phantomsection\label{\detokenize{polo.threads:polo.threads.thread.ClassificationThread}}\pysiglinewithargsret{\sphinxbfcode{\sphinxupquote{class }}\sphinxcode{\sphinxupquote{polo.threads.thread.}}\sphinxbfcode{\sphinxupquote{ClassificationThread}}}{\emph{\DUrole{n}{run\_object}}}{}
Bases: {\hyperref[\detokenize{polo.threads:polo.threads.thread.thread}]{\sphinxcrossref{\sphinxcode{\sphinxupquote{polo.threads.thread.thread}}}}}

Thread that is specifically for classifying images using the MARCO
model. This is a very CPU intensive process so it cannot be run on
the GUI thread.
\begin{quote}\begin{description}
\item[{Parameters}] \leavevmode
\sphinxstyleliteralstrong{\sphinxupquote{run\_object}} ({\hyperref[\detokenize{polo.crystallography:polo.crystallography.run.Run}]{\sphinxcrossref{\sphinxstyleliteralemphasis{\sphinxupquote{Run}}}}}\sphinxstyleliteralemphasis{\sphinxupquote{ or }}{\hyperref[\detokenize{polo.crystallography:polo.crystallography.run.HWIRun}]{\sphinxcrossref{\sphinxstyleliteralemphasis{\sphinxupquote{HWIRun}}}}}) \textendash{} Run who’s images are to be classified

\end{description}\end{quote}
\index{change\_value (polo.threads.thread.ClassificationThread attribute)@\spxentry{change\_value}\spxextra{polo.threads.thread.ClassificationThread attribute}}

\begin{fulllineitems}
\phantomsection\label{\detokenize{polo.threads:polo.threads.thread.ClassificationThread.change_value}}\pysigline{\sphinxbfcode{\sphinxupquote{change\_value}}}
\end{fulllineitems}

\index{estimated\_time (polo.threads.thread.ClassificationThread attribute)@\spxentry{estimated\_time}\spxextra{polo.threads.thread.ClassificationThread attribute}}

\begin{fulllineitems}
\phantomsection\label{\detokenize{polo.threads:polo.threads.thread.ClassificationThread.estimated_time}}\pysigline{\sphinxbfcode{\sphinxupquote{estimated\_time}}}
\end{fulllineitems}

\index{run() (polo.threads.thread.ClassificationThread method)@\spxentry{run()}\spxextra{polo.threads.thread.ClassificationThread method}}

\begin{fulllineitems}
\phantomsection\label{\detokenize{polo.threads:polo.threads.thread.ClassificationThread.run}}\pysiglinewithargsret{\sphinxbfcode{\sphinxupquote{run}}}{}{}
Method that actually does the classification work. Emits the the
{\hyperref[\detokenize{polo.threads:polo.threads.thread.ClassificationThread.change_value}]{\sphinxcrossref{\sphinxcode{\sphinxupquote{change\_value}}}}} signal everytime an image is classified. This is primary
to update the progress bar widget in the \sphinxtitleref{RunOrganizer} widget to
notify the user how many images have been classified. Additionally,
every five images classified the {\hyperref[\detokenize{polo.threads:polo.threads.thread.ClassificationThread.estimated_time}]{\sphinxcrossref{\sphinxcode{\sphinxupquote{estimated\_time}}}}} signal is emitted
which includes a tuple that contains as the first item the time in
seconds it took to classify the last five images and the number
of images that remain to be classified as the second item. This allows
for making an estimate on about how much time remains in until the
thread finishes.

\end{fulllineitems}


\end{fulllineitems}

\index{FTPDownloadThread (class in polo.threads.thread)@\spxentry{FTPDownloadThread}\spxextra{class in polo.threads.thread}}

\begin{fulllineitems}
\phantomsection\label{\detokenize{polo.threads:polo.threads.thread.FTPDownloadThread}}\pysiglinewithargsret{\sphinxbfcode{\sphinxupquote{class }}\sphinxcode{\sphinxupquote{polo.threads.thread.}}\sphinxbfcode{\sphinxupquote{FTPDownloadThread}}}{\emph{\DUrole{n}{ftp\_connection}}, \emph{\DUrole{n}{file\_paths}}, \emph{\DUrole{n}{save\_dir\_path}}}{}
Bases: {\hyperref[\detokenize{polo.threads:polo.threads.thread.thread}]{\sphinxcrossref{\sphinxcode{\sphinxupquote{polo.threads.thread.thread}}}}}

Thread specific for downloading files from a remote FTP server.
\begin{quote}\begin{description}
\item[{Parameters}] \leavevmode\begin{itemize}
\item {} 
\sphinxstyleliteralstrong{\sphinxupquote{ftp\_connection}} (\sphinxcode{\sphinxupquote{FTP}}) \textendash{} FTP connection object to download files from

\item {} 
\sphinxstyleliteralstrong{\sphinxupquote{file\_paths}} (\sphinxstyleliteralemphasis{\sphinxupquote{list}}) \textendash{} List absolute filepaths on the FTP server to download

\item {} 
\sphinxstyleliteralstrong{\sphinxupquote{save\_dir\_path}} (\sphinxstyleliteralemphasis{\sphinxupquote{str}}\sphinxstyleliteralemphasis{\sphinxupquote{ or }}\sphinxstyleliteralemphasis{\sphinxupquote{Path}}) \textendash{} Path on the local machine to store all downloaded files in

\end{itemize}

\end{description}\end{quote}
\index{download\_path (polo.threads.thread.FTPDownloadThread attribute)@\spxentry{download\_path}\spxextra{polo.threads.thread.FTPDownloadThread attribute}}

\begin{fulllineitems}
\phantomsection\label{\detokenize{polo.threads:polo.threads.thread.FTPDownloadThread.download_path}}\pysigline{\sphinxbfcode{\sphinxupquote{download\_path}}}
\end{fulllineitems}

\index{file\_downloaded (polo.threads.thread.FTPDownloadThread attribute)@\spxentry{file\_downloaded}\spxextra{polo.threads.thread.FTPDownloadThread attribute}}

\begin{fulllineitems}
\phantomsection\label{\detokenize{polo.threads:polo.threads.thread.FTPDownloadThread.file_downloaded}}\pysigline{\sphinxbfcode{\sphinxupquote{file\_downloaded}}}
\end{fulllineitems}

\index{run() (polo.threads.thread.FTPDownloadThread method)@\spxentry{run()}\spxextra{polo.threads.thread.FTPDownloadThread method}}

\begin{fulllineitems}
\phantomsection\label{\detokenize{polo.threads:polo.threads.thread.FTPDownloadThread.run}}\pysiglinewithargsret{\sphinxbfcode{\sphinxupquote{run}}}{\emph{\DUrole{n}{self}}}{}
\end{fulllineitems}


\end{fulllineitems}

\index{QuickThread (class in polo.threads.thread)@\spxentry{QuickThread}\spxextra{class in polo.threads.thread}}

\begin{fulllineitems}
\phantomsection\label{\detokenize{polo.threads:polo.threads.thread.QuickThread}}\pysiglinewithargsret{\sphinxbfcode{\sphinxupquote{class }}\sphinxcode{\sphinxupquote{polo.threads.thread.}}\sphinxbfcode{\sphinxupquote{QuickThread}}}{\emph{\DUrole{n}{job\_func}}, \emph{\DUrole{n}{parent}\DUrole{o}{=}\DUrole{default_value}{None}}, \emph{\DUrole{o}{**}\DUrole{n}{kwargs}}}{}
Bases: {\hyperref[\detokenize{polo.threads:polo.threads.thread.thread}]{\sphinxcrossref{\sphinxcode{\sphinxupquote{polo.threads.thread.thread}}}}}

QuickThreads are very similar
to thread objects except instead of you writing code that would be
executed by the \sphinxtitleref{run} method directly, the function that the \sphinxtitleref{QuickThread}
will execute is passed as an argument to the \sphinxtitleref{\_\_init\_\_}. Any arguments
that the passed function requires are passed as key word arguments. Once
the thread finished any values returned by the passed function are stored
in the \sphinxtitleref{QuickThread}’s \sphinxcode{\sphinxupquote{polo.threads.thread.QuickThread.results}}
attribute.

\begin{sphinxVerbatim}[commandchars=\\\{\}]
\PYG{n}{my\PYGZus{}func} \PYG{o}{=} \PYG{k}{lambda} \PYG{n}{x}\PYG{p}{,} \PYG{n}{y}\PYG{p}{:} \PYG{n}{x} \PYG{o}{+} \PYG{n}{y}
\PYG{n}{x}\PYG{p}{,} \PYG{n}{y} \PYG{o}{=} \PYG{l+m+mi}{40}\PYG{p}{,} \PYG{l+m+mi}{60}
\PYG{n}{my\PYGZus{}thread} \PYG{o}{=} \PYG{n}{QuickThread}\PYG{p}{(}\PYG{n}{job\PYGZus{}func}\PYG{o}{=}\PYG{n}{my\PYGZus{}func}\PYG{p}{,} \PYG{n}{x}\PYG{o}{=}\PYG{n}{x}\PYG{p}{,} \PYG{n}{y}\PYG{o}{=}\PYG{n}{y}\PYG{p}{)}
\PYG{c+c1}{\PYGZsh{} set up the thread with my\PYGZus{}func and the args we want to pass}
\PYG{n}{my\PYGZus{}thread}\PYG{o}{.}\PYG{n}{start}\PYG{p}{(}\PYG{p}{)}
\PYG{c+c1}{\PYGZsh{} my\PYGZus{}thread.result will = 100 (x + y)}
\end{sphinxVerbatim}
\begin{quote}\begin{description}
\item[{Parameters}] \leavevmode
\sphinxstyleliteralstrong{\sphinxupquote{job\_func}} (\sphinxstyleliteralemphasis{\sphinxupquote{func}}) \textendash{} Function to execute on the thread

\end{description}\end{quote}
\index{run() (polo.threads.thread.QuickThread method)@\spxentry{run()}\spxextra{polo.threads.thread.QuickThread method}}

\begin{fulllineitems}
\phantomsection\label{\detokenize{polo.threads:polo.threads.thread.QuickThread.run}}\pysiglinewithargsret{\sphinxbfcode{\sphinxupquote{run}}}{\emph{\DUrole{n}{self}}}{}
\end{fulllineitems}


\end{fulllineitems}

\index{thread (class in polo.threads.thread)@\spxentry{thread}\spxextra{class in polo.threads.thread}}

\begin{fulllineitems}
\phantomsection\label{\detokenize{polo.threads:polo.threads.thread.thread}}\pysiglinewithargsret{\sphinxbfcode{\sphinxupquote{class }}\sphinxcode{\sphinxupquote{polo.threads.thread.}}\sphinxbfcode{\sphinxupquote{thread}}}{\emph{\DUrole{n}{parent}\DUrole{o}{=}\DUrole{default_value}{None}}}{}
Bases: \sphinxcode{\sphinxupquote{PyQt5.QtCore.QThread}}

Very basic wrapper class around \sphinxcode{\sphinxupquote{QThread}} class. Should be
inherited by a more specific class and then the \sphinxtitleref{run} method
can be overwritten to provide functionality. Whatever code is in the
{\hyperref[\detokenize{polo.threads:polo.threads.thread.thread.run}]{\sphinxcrossref{\sphinxcode{\sphinxupquote{run()}}}}} method will be executed when
\sphinxcode{\sphinxupquote{start()}} is called. The
{\hyperref[\detokenize{polo.threads:polo.threads.thread.thread.run}]{\sphinxcrossref{\sphinxcode{\sphinxupquote{run()}}}}} method should not be called
explicitly.
\begin{quote}\begin{description}
\item[{Parameters}] \leavevmode
\sphinxstyleliteralstrong{\sphinxupquote{parent}} (\sphinxstyleliteralemphasis{\sphinxupquote{QWidget}}\sphinxstyleliteralemphasis{\sphinxupquote{, }}\sphinxstyleliteralemphasis{\sphinxupquote{optional}}) \textendash{} parent widget, defaults to None

\end{description}\end{quote}
\index{run() (polo.threads.thread.thread method)@\spxentry{run()}\spxextra{polo.threads.thread.thread method}}

\begin{fulllineitems}
\phantomsection\label{\detokenize{polo.threads:polo.threads.thread.thread.run}}\pysiglinewithargsret{\sphinxbfcode{\sphinxupquote{run}}}{\emph{\DUrole{n}{self}}}{}
\end{fulllineitems}


\end{fulllineitems}



\subsubsection{Module contents}
\label{\detokenize{polo.threads:module-polo.threads}}\label{\detokenize{polo.threads:module-contents}}\index{module@\spxentry{module}!polo.threads@\spxentry{polo.threads}}\index{polo.threads@\spxentry{polo.threads}!module@\spxentry{module}}

\subsection{polo.utils package}
\label{\detokenize{polo.utils:polo-utils-package}}\label{\detokenize{polo.utils::doc}}

\subsubsection{Submodules}
\label{\detokenize{polo.utils:submodules}}

\subsubsection{polo.utils.dialog\_utils module}
\label{\detokenize{polo.utils:module-polo.utils.dialog_utils}}\label{\detokenize{polo.utils:polo-utils-dialog-utils-module}}\index{module@\spxentry{module}!polo.utils.dialog\_utils@\spxentry{polo.utils.dialog\_utils}}\index{polo.utils.dialog\_utils@\spxentry{polo.utils.dialog\_utils}!module@\spxentry{module}}\index{make\_message\_box() (in module polo.utils.dialog\_utils)@\spxentry{make\_message\_box()}\spxextra{in module polo.utils.dialog\_utils}}

\begin{fulllineitems}
\phantomsection\label{\detokenize{polo.utils:polo.utils.dialog_utils.make_message_box}}\pysiglinewithargsret{\sphinxcode{\sphinxupquote{polo.utils.dialog\_utils.}}\sphinxbfcode{\sphinxupquote{make\_message\_box}}}{\emph{\DUrole{n}{message}}, \emph{\DUrole{n}{parent}\DUrole{o}{=}\DUrole{default_value}{None}}, \emph{\DUrole{n}{icon}\DUrole{o}{=}\DUrole{default_value}{1}}, \emph{\DUrole{n}{buttons}\DUrole{o}{=}\DUrole{default_value}{1024}}, \emph{\DUrole{n}{connected\_function}\DUrole{o}{=}\DUrole{default_value}{None}}}{}
General helper function to create popup message box dialogs to convey
situational information to the user.
\begin{quote}\begin{description}
\item[{Parameters}] \leavevmode\begin{itemize}
\item {} 
\sphinxstyleliteralstrong{\sphinxupquote{message}} (\sphinxstyleliteralemphasis{\sphinxupquote{str}}) \textendash{} The message to display to the user.

\item {} 
\sphinxstyleliteralstrong{\sphinxupquote{parent}} (\sphinxstyleliteralemphasis{\sphinxupquote{QDialog}}\sphinxstyleliteralemphasis{\sphinxupquote{, }}\sphinxstyleliteralemphasis{\sphinxupquote{optional}}) \textendash{} Parent dialog, defaults to None

\item {} 
\sphinxstyleliteralstrong{\sphinxupquote{icon}} (\sphinxstyleliteralemphasis{\sphinxupquote{int}}\sphinxstyleliteralemphasis{\sphinxupquote{, }}\sphinxstyleliteralemphasis{\sphinxupquote{optional}}) \textendash{} QMessageBox icon to display along with the message,
defaults to QtWidgets.QMessageBox.Information

\item {} 
\sphinxstyleliteralstrong{\sphinxupquote{buttons}} (\sphinxstyleliteralemphasis{\sphinxupquote{set}}\sphinxstyleliteralemphasis{\sphinxupquote{, }}\sphinxstyleliteralemphasis{\sphinxupquote{optional}}) \textendash{} Buttons to include in the message box,
defaults to QtWidgets.QMessageBox.Ok

\item {} 
\sphinxstyleliteralstrong{\sphinxupquote{connected\_function}} (\sphinxstyleliteralemphasis{\sphinxupquote{func}}\sphinxstyleliteralemphasis{\sphinxupquote{, }}\sphinxstyleliteralemphasis{\sphinxupquote{optional}}) \textendash{} Function to connect to
buttonClicked event, defaults to None

\end{itemize}

\item[{Returns}] \leavevmode
The message box.

\item[{Return type}] \leavevmode
QMessageBox

\end{description}\end{quote}

\end{fulllineitems}



\subsubsection{polo.utils.exceptions module}
\label{\detokenize{polo.utils:module-polo.utils.exceptions}}\label{\detokenize{polo.utils:polo-utils-exceptions-module}}\index{module@\spxentry{module}!polo.utils.exceptions@\spxentry{polo.utils.exceptions}}\index{polo.utils.exceptions@\spxentry{polo.utils.exceptions}!module@\spxentry{module}}\index{EmptyDirectoryError@\spxentry{EmptyDirectoryError}}

\begin{fulllineitems}
\phantomsection\label{\detokenize{polo.utils:polo.utils.exceptions.EmptyDirectoryError}}\pysigline{\sphinxbfcode{\sphinxupquote{exception }}\sphinxcode{\sphinxupquote{polo.utils.exceptions.}}\sphinxbfcode{\sphinxupquote{EmptyDirectoryError}}}
Bases: \sphinxcode{\sphinxupquote{Exception}}

Raised when attempting to load images from an empty directory

\end{fulllineitems}

\index{EmptyRunNameError@\spxentry{EmptyRunNameError}}

\begin{fulllineitems}
\phantomsection\label{\detokenize{polo.utils:polo.utils.exceptions.EmptyRunNameError}}\pysigline{\sphinxbfcode{\sphinxupquote{exception }}\sphinxcode{\sphinxupquote{polo.utils.exceptions.}}\sphinxbfcode{\sphinxupquote{EmptyRunNameError}}}
Bases: \sphinxcode{\sphinxupquote{Exception}}

Raised when reading in an HWI directory but it does not contain number
of images corresponding to number of wells.

\end{fulllineitems}

\index{ForbiddenImageTypeError@\spxentry{ForbiddenImageTypeError}}

\begin{fulllineitems}
\phantomsection\label{\detokenize{polo.utils:polo.utils.exceptions.ForbiddenImageTypeError}}\pysigline{\sphinxbfcode{\sphinxupquote{exception }}\sphinxcode{\sphinxupquote{polo.utils.exceptions.}}\sphinxbfcode{\sphinxupquote{ForbiddenImageTypeError}}}
Bases: \sphinxcode{\sphinxupquote{Exception}}

Raised when user attempts to load in an image that is not in the allowed
image types.

\end{fulllineitems}

\index{IncompletePlateError@\spxentry{IncompletePlateError}}

\begin{fulllineitems}
\phantomsection\label{\detokenize{polo.utils:polo.utils.exceptions.IncompletePlateError}}\pysigline{\sphinxbfcode{\sphinxupquote{exception }}\sphinxcode{\sphinxupquote{polo.utils.exceptions.}}\sphinxbfcode{\sphinxupquote{IncompletePlateError}}}
Bases: \sphinxcode{\sphinxupquote{Exception}}

Raised when reading in an HWI directory but it does not contain number
of images corresponding to number of wells.

\end{fulllineitems}

\index{InvalidCocktailFile@\spxentry{InvalidCocktailFile}}

\begin{fulllineitems}
\phantomsection\label{\detokenize{polo.utils:polo.utils.exceptions.InvalidCocktailFile}}\pysigline{\sphinxbfcode{\sphinxupquote{exception }}\sphinxcode{\sphinxupquote{polo.utils.exceptions.}}\sphinxbfcode{\sphinxupquote{InvalidCocktailFile}}}
Bases: \sphinxcode{\sphinxupquote{Exception}}

Raised when user attempts to load in a file containing well cocktail
information that does not confrom to existing formating standards.

\end{fulllineitems}

\index{NotASolutionError@\spxentry{NotASolutionError}}

\begin{fulllineitems}
\phantomsection\label{\detokenize{polo.utils:polo.utils.exceptions.NotASolutionError}}\pysigline{\sphinxbfcode{\sphinxupquote{exception }}\sphinxcode{\sphinxupquote{polo.utils.exceptions.}}\sphinxbfcode{\sphinxupquote{NotASolutionError}}}
Bases: \sphinxcode{\sphinxupquote{Exception}}

Raised when user attempts to load in an image that is not in the allowed
image types.

\end{fulllineitems}

\index{NotHWIDirectoryError@\spxentry{NotHWIDirectoryError}}

\begin{fulllineitems}
\phantomsection\label{\detokenize{polo.utils:polo.utils.exceptions.NotHWIDirectoryError}}\pysigline{\sphinxbfcode{\sphinxupquote{exception }}\sphinxcode{\sphinxupquote{polo.utils.exceptions.}}\sphinxbfcode{\sphinxupquote{NotHWIDirectoryError}}}
Bases: \sphinxcode{\sphinxupquote{Exception}}

Raised when user attempts to read in a directory as HWI but it does
not look like one.

TODO: Add utils function to determine when to raise this exception.

\end{fulllineitems}



\subsubsection{polo.utils.ftp\_utils module}
\label{\detokenize{polo.utils:module-polo.utils.ftp_utils}}\label{\detokenize{polo.utils:polo-utils-ftp-utils-module}}\index{module@\spxentry{module}!polo.utils.ftp\_utils@\spxentry{polo.utils.ftp\_utils}}\index{polo.utils.ftp\_utils@\spxentry{polo.utils.ftp\_utils}!module@\spxentry{module}}\index{catch\_ftp\_errors() (in module polo.utils.ftp\_utils)@\spxentry{catch\_ftp\_errors()}\spxextra{in module polo.utils.ftp\_utils}}

\begin{fulllineitems}
\phantomsection\label{\detokenize{polo.utils:polo.utils.ftp_utils.catch_ftp_errors}}\pysiglinewithargsret{\sphinxcode{\sphinxupquote{polo.utils.ftp\_utils.}}\sphinxbfcode{\sphinxupquote{catch\_ftp\_errors}}}{\emph{\DUrole{n}{funct}}}{}
General decorator function for catching any errors thrown by other
ftp\_utils functions.

\end{fulllineitems}

\index{logon() (in module polo.utils.ftp\_utils)@\spxentry{logon()}\spxextra{in module polo.utils.ftp\_utils}}

\begin{fulllineitems}
\phantomsection\label{\detokenize{polo.utils:polo.utils.ftp_utils.logon}}\pysiglinewithargsret{\sphinxcode{\sphinxupquote{polo.utils.ftp\_utils.}}\sphinxbfcode{\sphinxupquote{logon}}}{\emph{\DUrole{o}{*}\DUrole{n}{args}}, \emph{\DUrole{o}{**}\DUrole{n}{kwargs}}}{}
\end{fulllineitems}



\subsubsection{polo.utils.io\_utils module}
\label{\detokenize{polo.utils:module-polo.utils.io_utils}}\label{\detokenize{polo.utils:polo-utils-io-utils-module}}\index{module@\spxentry{module}!polo.utils.io\_utils@\spxentry{polo.utils.io\_utils}}\index{polo.utils.io\_utils@\spxentry{polo.utils.io\_utils}!module@\spxentry{module}}\index{BarTender (class in polo.utils.io\_utils)@\spxentry{BarTender}\spxextra{class in polo.utils.io\_utils}}

\begin{fulllineitems}
\phantomsection\label{\detokenize{polo.utils:polo.utils.io_utils.BarTender}}\pysiglinewithargsret{\sphinxbfcode{\sphinxupquote{class }}\sphinxcode{\sphinxupquote{polo.utils.io\_utils.}}\sphinxbfcode{\sphinxupquote{BarTender}}}{\emph{\DUrole{n}{cocktail\_dir}}, \emph{\DUrole{n}{cocktail\_meta}}}{}
Bases: \sphinxcode{\sphinxupquote{object}}

Class for organizing and accessing 
{\hyperref[\detokenize{polo.utils:polo.utils.io_utils.Menu}]{\sphinxcrossref{\sphinxcode{\sphinxupquote{Menu}}}}} data.
\begin{quote}\begin{description}
\item[{Parameters}] \leavevmode\begin{itemize}
\item {} 
\sphinxstyleliteralstrong{\sphinxupquote{cocktail\_dir}} (\sphinxstyleliteralemphasis{\sphinxupquote{str}}\sphinxstyleliteralemphasis{\sphinxupquote{ or }}\sphinxstyleliteralemphasis{\sphinxupquote{Path}}) \textendash{} Directory containing cocktail menu csv filepaths

\item {} 
\sphinxstyleliteralstrong{\sphinxupquote{cocktail\_meta}} (\sphinxstyleliteralemphasis{\sphinxupquote{Path}}\sphinxstyleliteralemphasis{\sphinxupquote{ or }}\sphinxstyleliteralemphasis{\sphinxupquote{str}}) \textendash{} Path to cocktail metadata file which describes the contents of
each cocktail menu csv file

\end{itemize}

\end{description}\end{quote}

Cocktail metadata file should be a csv file with the following
headers ordered from top to bottom. Each header name is followed by a
description.

\begin{sphinxVerbatim}[commandchars=\\\{\}]
File Name: Name of cocktail menu file
Dates Used: Range of dates the cocktail menu was used (m/d/y\PYGZhy{}m/d/y)
Plate Number
Screen Type: \PYGZsq{}m\PYGZsq{} for membrane screens, \PYGZsq{}s\PYGZsq{} for soluble screens
\end{sphinxVerbatim}
\index{add\_menus\_from\_metadata() (polo.utils.io\_utils.BarTender method)@\spxentry{add\_menus\_from\_metadata()}\spxextra{polo.utils.io\_utils.BarTender method}}

\begin{fulllineitems}
\phantomsection\label{\detokenize{polo.utils:polo.utils.io_utils.BarTender.add_menus_from_metadata}}\pysiglinewithargsret{\sphinxbfcode{\sphinxupquote{add\_menus\_from\_metadata}}}{}{}
Adds {\hyperref[\detokenize{polo.utils:polo.utils.io_utils.Menu}]{\sphinxcrossref{\sphinxcode{\sphinxupquote{Menu}}}}} objects to the :attr:polo.utils.io\_utils.BarTender.menus\textasciigrave{}
attribute by reading the cocktail csv files included
in the \sphinxcode{\sphinxupquote{COCKTAIL\_DATA\_PATH}} directory.

\end{fulllineitems}

\index{date\_range\_parser() (polo.utils.io\_utils.BarTender static method)@\spxentry{date\_range\_parser()}\spxextra{polo.utils.io\_utils.BarTender static method}}

\begin{fulllineitems}
\phantomsection\label{\detokenize{polo.utils:polo.utils.io_utils.BarTender.date_range_parser}}\pysiglinewithargsret{\sphinxbfcode{\sphinxupquote{static }}\sphinxbfcode{\sphinxupquote{date\_range\_parser}}}{\emph{\DUrole{n}{date\_range\_string}}}{}
Utility function for converting the date ranges in the cocktail
metadata csv file to datetime objects using the 
{\hyperref[\detokenize{polo.utils:polo.utils.io_utils.BarTender.datetime_converter}]{\sphinxcrossref{\sphinxcode{\sphinxupquote{datetime\_converter()}}}}}
method.

Date ranges should have the format

\begin{sphinxVerbatim}[commandchars=\\\{\}]
\PYGZsq{}start date \PYGZhy{} end date\PYGZsq{}

If the date range is for the most recent cocktail menu then there
will not be an end date and the format will be

\PYGZsq{}start date \PYGZhy{} \PYGZsq{}
\end{sphinxVerbatim}
\begin{quote}\begin{description}
\item[{Parameters}] \leavevmode
\sphinxstyleliteralstrong{\sphinxupquote{date\_range\_string}} (\sphinxstyleliteralemphasis{\sphinxupquote{str}}) \textendash{} string to pull dates out of

\item[{Returns}] \leavevmode
tuple of datetime objects, start date and end date

\item[{Return type}] \leavevmode
tuple

\end{description}\end{quote}

\end{fulllineitems}

\index{datetime\_converter() (polo.utils.io\_utils.BarTender static method)@\spxentry{datetime\_converter()}\spxextra{polo.utils.io\_utils.BarTender static method}}

\begin{fulllineitems}
\phantomsection\label{\detokenize{polo.utils:polo.utils.io_utils.BarTender.datetime_converter}}\pysiglinewithargsret{\sphinxbfcode{\sphinxupquote{static }}\sphinxbfcode{\sphinxupquote{datetime\_converter}}}{\emph{\DUrole{n}{date\_string}}}{}
General utility function for converting strings to datetime objects.
Attempts to convert the string by trying a couple of datetime
formats that are common in cocktail menu files and other
locations in the HWI file universe Polo runs across.
\begin{quote}\begin{description}
\item[{Parameters}] \leavevmode
\sphinxstyleliteralstrong{\sphinxupquote{date\_string}} (\sphinxstyleliteralemphasis{\sphinxupquote{str}}) \textendash{} string to convert to datetime

\item[{Returns}] \leavevmode
datetime object

\item[{Return type}] \leavevmode
datetime

\end{description}\end{quote}

\end{fulllineitems}

\index{get\_menu\_by\_basename() (polo.utils.io\_utils.BarTender method)@\spxentry{get\_menu\_by\_basename()}\spxextra{polo.utils.io\_utils.BarTender method}}

\begin{fulllineitems}
\phantomsection\label{\detokenize{polo.utils:polo.utils.io_utils.BarTender.get_menu_by_basename}}\pysiglinewithargsret{\sphinxbfcode{\sphinxupquote{get\_menu\_by\_basename}}}{\emph{\DUrole{n}{basename}}}{}
Uses the basename of a {\hyperref[\detokenize{polo.utils:polo.utils.io_utils.Menu}]{\sphinxcrossref{\sphinxcode{\sphinxupquote{Menu}}}}} file path to return a {\hyperref[\detokenize{polo.utils:polo.utils.io_utils.Menu}]{\sphinxcrossref{\sphinxcode{\sphinxupquote{Menu}}}}} object.
Useful for retrieving menus based on the text of comboBoxes since
when menus are displayed to the user only the basename is used.
\begin{quote}\begin{description}
\item[{Parameters}] \leavevmode
\sphinxstyleliteralstrong{\sphinxupquote{basename}} (\sphinxstyleliteralemphasis{\sphinxupquote{str}}) \textendash{} Basename of a {\hyperref[\detokenize{polo.utils:polo.utils.io_utils.Menu}]{\sphinxcrossref{\sphinxcode{\sphinxupquote{Menu}}}}} file path

\item[{Returns}] \leavevmode
Menu instance who’s basename matches the \sphinxtitleref{basename} argument,
returns None is no menu is found

\item[{Return type}] \leavevmode
{\hyperref[\detokenize{polo.utils:polo.utils.io_utils.Menu}]{\sphinxcrossref{Menu}}} or None

\end{description}\end{quote}

\end{fulllineitems}

\index{get\_menu\_by\_date() (polo.utils.io\_utils.BarTender method)@\spxentry{get\_menu\_by\_date()}\spxextra{polo.utils.io\_utils.BarTender method}}

\begin{fulllineitems}
\phantomsection\label{\detokenize{polo.utils:polo.utils.io_utils.BarTender.get_menu_by_date}}\pysiglinewithargsret{\sphinxbfcode{\sphinxupquote{get\_menu\_by\_date}}}{\emph{\DUrole{n}{date}}, \emph{\DUrole{n}{type\_}\DUrole{o}{=}\DUrole{default_value}{\textquotesingle{}s\textquotesingle{}}}}{}
Get a {\hyperref[\detokenize{polo.utils:polo.utils.io_utils.Menu}]{\sphinxcrossref{\sphinxcode{\sphinxupquote{Menu}}}}} instance who’s usage dates include the
date passed through the \sphinxtitleref{date} argument and
matches the screen type passed through the \sphinxtitleref{type\_} argument.

Screen types can either be ‘s’ for ‘soluble’ screens or ‘m’ for
membrane screens.
\begin{quote}\begin{description}
\item[{Parameters}] \leavevmode\begin{itemize}
\item {} 
\sphinxstyleliteralstrong{\sphinxupquote{date}} (\sphinxstyleliteralemphasis{\sphinxupquote{datetime}}) \textendash{} Date to search menus with

\item {} 
\sphinxstyleliteralstrong{\sphinxupquote{type}} (\sphinxstyleliteralemphasis{\sphinxupquote{str}}) \textendash{} Type of screen to return (soluble or membrane)

\end{itemize}

\item[{Returns}] \leavevmode
menu matching the given date and type

\item[{Return type}] \leavevmode
{\hyperref[\detokenize{polo.utils:polo.utils.io_utils.Menu}]{\sphinxcrossref{Menu}}}

\end{description}\end{quote}

\end{fulllineitems}

\index{get\_menu\_by\_path() (polo.utils.io\_utils.BarTender method)@\spxentry{get\_menu\_by\_path()}\spxextra{polo.utils.io\_utils.BarTender method}}

\begin{fulllineitems}
\phantomsection\label{\detokenize{polo.utils:polo.utils.io_utils.BarTender.get_menu_by_path}}\pysiglinewithargsret{\sphinxbfcode{\sphinxupquote{get\_menu\_by\_path}}}{\emph{\DUrole{n}{path}}}{}
Returns a {\hyperref[\detokenize{polo.utils:polo.utils.io_utils.Menu}]{\sphinxcrossref{\sphinxcode{\sphinxupquote{Menu}}}}} instance by its file path, which is
used as the key for accessing the menus attribute normally.
\begin{quote}\begin{description}
\item[{Parameters}] \leavevmode
\sphinxstyleliteralstrong{\sphinxupquote{path}} (\sphinxstyleliteralemphasis{\sphinxupquote{str}}) \textendash{} file path of a menu csv file

\item[{Returns}] \leavevmode
Menu instance that is mapped to given path

\item[{Return type}] \leavevmode
{\hyperref[\detokenize{polo.utils:polo.utils.io_utils.Menu}]{\sphinxcrossref{Menu}}}

\end{description}\end{quote}

\end{fulllineitems}

\index{get\_menus\_by\_type() (polo.utils.io\_utils.BarTender method)@\spxentry{get\_menus\_by\_type()}\spxextra{polo.utils.io\_utils.BarTender method}}

\begin{fulllineitems}
\phantomsection\label{\detokenize{polo.utils:polo.utils.io_utils.BarTender.get_menus_by_type}}\pysiglinewithargsret{\sphinxbfcode{\sphinxupquote{get\_menus\_by\_type}}}{\emph{\DUrole{n}{type\_}\DUrole{o}{=}\DUrole{default_value}{\textquotesingle{}s\textquotesingle{}}}}{}
Returns all {\hyperref[\detokenize{polo.utils:polo.utils.io_utils.Menu}]{\sphinxcrossref{\sphinxcode{\sphinxupquote{Menu}}}}} instances of a given screen type.

‘s’ for soluble screens and ‘m’ for membrane screens. No other
characters should be passed to \sphinxtitleref{type\_}.
\begin{quote}\begin{description}
\item[{Parameters}] \leavevmode
\sphinxstyleliteralstrong{\sphinxupquote{type}} (\sphinxstyleliteralemphasis{\sphinxupquote{str}}\sphinxstyleliteralemphasis{\sphinxupquote{ (}}\sphinxstyleliteralemphasis{\sphinxupquote{max length 1}}\sphinxstyleliteralemphasis{\sphinxupquote{)}}) \textendash{} Key for type of screen to return

\item[{Returns}] \leavevmode
list of menus of that screen type

\item[{Return type}] \leavevmode
list

\end{description}\end{quote}

\end{fulllineitems}


\end{fulllineitems}

\index{CocktailMenuReader (class in polo.utils.io\_utils)@\spxentry{CocktailMenuReader}\spxextra{class in polo.utils.io\_utils}}

\begin{fulllineitems}
\phantomsection\label{\detokenize{polo.utils:polo.utils.io_utils.CocktailMenuReader}}\pysiglinewithargsret{\sphinxbfcode{\sphinxupquote{class }}\sphinxcode{\sphinxupquote{polo.utils.io\_utils.}}\sphinxbfcode{\sphinxupquote{CocktailMenuReader}}}{\emph{\DUrole{n}{menu\_file\_path}}, \emph{\DUrole{n}{delim}\DUrole{o}{=}\DUrole{default_value}{\textquotesingle{},\textquotesingle{}}}}{}
Bases: \sphinxcode{\sphinxupquote{object}}

CocktailMenuReader instances should be used to read a csv file containing
a collection of cocktail screens. The csv file should contain cocktail
related formulations and assign each cocktail to a specific well in the
screening plate. CocktailMenuReader is essentially a wrapper around 
the \sphinxcode{\sphinxupquote{csv.DictReader}} class. However it returns a \sphinxcode{\sphinxupquote{Cocktail}} instance 
instead of returning a dictionary via when it’s \_\_iter\_\_ method is called.

\begin{sphinxVerbatim}[commandchars=\\\{\}]
\PYG{n}{cocktail\PYGZus{}menu} \PYG{o}{=} \PYG{l+s+s1}{\PYGZsq{}}\PYG{l+s+s1}{path/to/my/csv}\PYG{l+s+s1}{\PYGZsq{}}
\PYG{n}{my\PYGZus{}reader} \PYG{o}{=} \PYG{n}{CocktailMenuReader}\PYG{p}{(}\PYG{n}{cocktail\PYGZus{}menu}\PYG{p}{)}
\PYG{n}{csv\PYGZus{}cocktails} \PYG{o}{=} \PYG{n}{my\PYGZus{}reader}\PYG{o}{.}\PYG{n}{read\PYGZus{}menu\PYGZus{}file}\PYG{p}{(}\PYG{p}{)}
\PYG{c+c1}{\PYGZsh{} csv\PYGZus{}cocktails now holds list of Cocktail objects read from}
\PYG{c+c1}{\PYGZsh{} cocktail\PYGZus{}menu}
\end{sphinxVerbatim}
\begin{quote}\begin{description}
\item[{Parameters}] \leavevmode\begin{itemize}
\item {} 
\sphinxstyleliteralstrong{\sphinxupquote{menu\_file}} (\sphinxstyleliteralemphasis{\sphinxupquote{str}}\sphinxstyleliteralemphasis{\sphinxupquote{ or }}\sphinxstyleliteralemphasis{\sphinxupquote{Path}}) \textendash{} Path to cocktail menu file to read. Should be csv
formated

\item {} 
\sphinxstyleliteralstrong{\sphinxupquote{delim}} (\sphinxstyleliteralemphasis{\sphinxupquote{str}}\sphinxstyleliteralemphasis{\sphinxupquote{, }}\sphinxstyleliteralemphasis{\sphinxupquote{optional}}) \textendash{} Seperator for menu\_file; really should not need to 
be changed, defaults to ‘,’

\end{itemize}

\end{description}\end{quote}
\index{cocktail\_map (polo.utils.io\_utils.CocktailMenuReader attribute)@\spxentry{cocktail\_map}\spxextra{polo.utils.io\_utils.CocktailMenuReader attribute}}

\begin{fulllineitems}
\phantomsection\label{\detokenize{polo.utils:polo.utils.io_utils.CocktailMenuReader.cocktail_map}}\pysigline{\sphinxbfcode{\sphinxupquote{cocktail\_map}}\sphinxbfcode{\sphinxupquote{ = \{0: \textquotesingle{}well\_assignment\textquotesingle{}, 1: \textquotesingle{}number\textquotesingle{}, 2: \textquotesingle{}commercial\_code\textquotesingle{}, 8: \textquotesingle{}pH\textquotesingle{}\}}}}
\end{fulllineitems}

\index{formula\_pos (polo.utils.io\_utils.CocktailMenuReader attribute)@\spxentry{formula\_pos}\spxextra{polo.utils.io\_utils.CocktailMenuReader attribute}}

\begin{fulllineitems}
\phantomsection\label{\detokenize{polo.utils:polo.utils.io_utils.CocktailMenuReader.formula_pos}}\pysigline{\sphinxbfcode{\sphinxupquote{formula\_pos}}\sphinxbfcode{\sphinxupquote{ = 4}}}
\end{fulllineitems}

\index{read\_menu\_file() (polo.utils.io\_utils.CocktailMenuReader method)@\spxentry{read\_menu\_file()}\spxextra{polo.utils.io\_utils.CocktailMenuReader method}}

\begin{fulllineitems}
\phantomsection\label{\detokenize{polo.utils:polo.utils.io_utils.CocktailMenuReader.read_menu_file}}\pysiglinewithargsret{\sphinxbfcode{\sphinxupquote{read\_menu\_file}}}{}{}
Read the contents of cocktail menu csv file. The {\hyperref[\detokenize{polo.utils:polo.utils.io_utils.Menu}]{\sphinxcrossref{\sphinxcode{\sphinxupquote{Menu}}}}} file path
is read from the \sphinxcode{\sphinxupquote{Menu.menu\_file\_path}} attribute. The first \sphinxstylestrong{two} lines
of all the cocktail menu files included in Polo are header lines and
so the reader will skip the first two lines before actually reading
in any data. Each row is converted to a
{\hyperref[\detokenize{polo.crystallography:polo.crystallography.cocktail.Cocktail}]{\sphinxcrossref{\sphinxcode{\sphinxupquote{Cocktail}}}}} object and then
added to a dictionary based on the  {\hyperref[\detokenize{polo.crystallography:polo.crystallography.cocktail.Cocktail}]{\sphinxcrossref{\sphinxcode{\sphinxupquote{Cocktail}}}}}
instance’s well assignment.
\begin{quote}\begin{description}
\item[{Returns}] \leavevmode
Dictionary of Cocktail instances. Key value is the Cocktail’s
well assignment in the screening plate (base 1).

\item[{Return type}] \leavevmode
dict

\end{description}\end{quote}

\end{fulllineitems}

\index{set\_cocktail\_map() (polo.utils.io\_utils.CocktailMenuReader class method)@\spxentry{set\_cocktail\_map()}\spxextra{polo.utils.io\_utils.CocktailMenuReader class method}}

\begin{fulllineitems}
\phantomsection\label{\detokenize{polo.utils:polo.utils.io_utils.CocktailMenuReader.set_cocktail_map}}\pysiglinewithargsret{\sphinxbfcode{\sphinxupquote{classmethod }}\sphinxbfcode{\sphinxupquote{set\_cocktail\_map}}}{\emph{\DUrole{n}{map}}}{}
Classmethod to edit the
{\hyperref[\detokenize{polo.utils:polo.utils.io_utils.CocktailMenuReader.cocktail_map}]{\sphinxcrossref{\sphinxcode{\sphinxupquote{cocktail\_map}}}}}.
The {\hyperref[\detokenize{polo.utils:polo.utils.io_utils.CocktailMenuReader.cocktail_map}]{\sphinxcrossref{\sphinxcode{\sphinxupquote{cocktail\_map}}}}}
describes where the \sphinxcode{\sphinxupquote{Cocktail}} level information 
is stored in a given cocktail row in the csv file. 
It is a dictionary that maps specific indices in a row to
\sphinxcode{\sphinxupquote{Cocktail}} attributes.

The default cocktail\_map dictionary is below.

\begin{sphinxVerbatim}[commandchars=\\\{\}]
\PYG{g+gp}{\PYGZgt{}\PYGZgt{}\PYGZgt{} }\PYG{n}{cocktail\PYGZus{}map} \PYG{o}{=} \PYG{p}{\PYGZob{}}
\PYG{g+go}{0: \PYGZsq{}well\PYGZus{}assignment\PYGZsq{},}
\PYG{g+go}{1: \PYGZsq{}number\PYGZsq{},}
\PYG{g+go}{8: \PYGZsq{}pH\PYGZsq{},}
\PYG{g+go}{2: \PYGZsq{}commercial\PYGZus{}code\PYGZsq{}}
\PYG{g+go}{\PYGZcb{}}
\end{sphinxVerbatim}

This tells instances of CocktailMenuReader to look at index 0 of a row
for the well\_assignment attribute of the Cocktail class, index 1 for
the number attribute of the Cocktail class, etc.
\begin{quote}\begin{description}
\item[{Parameters}] \leavevmode
\sphinxstyleliteralstrong{\sphinxupquote{map}} (\sphinxstyleliteralemphasis{\sphinxupquote{dict}}) \textendash{} Dictionary mapping csv row indicies to Cocktail object
attributes

\end{description}\end{quote}

\end{fulllineitems}

\index{set\_formula\_pos() (polo.utils.io\_utils.CocktailMenuReader class method)@\spxentry{set\_formula\_pos()}\spxextra{polo.utils.io\_utils.CocktailMenuReader class method}}

\begin{fulllineitems}
\phantomsection\label{\detokenize{polo.utils:polo.utils.io_utils.CocktailMenuReader.set_formula_pos}}\pysiglinewithargsret{\sphinxbfcode{\sphinxupquote{classmethod }}\sphinxbfcode{\sphinxupquote{set\_formula\_pos}}}{\emph{\DUrole{n}{pos}}}{}
Classmethod to change the {\hyperref[\detokenize{polo.utils:polo.utils.io_utils.CocktailMenuReader.formula_pos}]{\sphinxcrossref{\sphinxcode{\sphinxupquote{CocktailMenuReader.formula\_pos}}}}}
attribute. The {\hyperref[\detokenize{polo.utils:polo.utils.io_utils.CocktailMenuReader.formula_pos}]{\sphinxcrossref{\sphinxcode{\sphinxupquote{formula\_pos}}}}}
describes the location (base 0) of the chemical formula in a row of
a cocktail menu file csv. For some reason, HWI cocktail menu files
will only have one chemical formula per row (cocktail) no matter
the number of reagents that composite that cocktail. This is why
its location is represented using an int instead of a dict.

Generally, {\hyperref[\detokenize{polo.utils:polo.utils.io_utils.CocktailMenuReader.formula_pos}]{\sphinxcrossref{\sphinxcode{\sphinxupquote{CocktailMenuReader.formula\_pos}}}}}
should not be changed without a very good
reason as the position of the chemical formula is consistent across
all HWI cocktail menu files.
\begin{quote}\begin{description}
\item[{Parameters}] \leavevmode
\sphinxstyleliteralstrong{\sphinxupquote{pos}} (\sphinxstyleliteralemphasis{\sphinxupquote{int}}) \textendash{} Index where chemical formula can be found

\end{description}\end{quote}

\end{fulllineitems}


\end{fulllineitems}

\index{HtmlWriter (class in polo.utils.io\_utils)@\spxentry{HtmlWriter}\spxextra{class in polo.utils.io\_utils}}

\begin{fulllineitems}
\phantomsection\label{\detokenize{polo.utils:polo.utils.io_utils.HtmlWriter}}\pysiglinewithargsret{\sphinxbfcode{\sphinxupquote{class }}\sphinxcode{\sphinxupquote{polo.utils.io\_utils.}}\sphinxbfcode{\sphinxupquote{HtmlWriter}}}{\emph{\DUrole{n}{run}}, \emph{\DUrole{o}{**}\DUrole{n}{kwargs}}}{}
Bases: {\hyperref[\detokenize{polo.utils:polo.utils.io_utils.RunSerializer}]{\sphinxcrossref{\sphinxcode{\sphinxupquote{polo.utils.io\_utils.RunSerializer}}}}}
\index{make\_template() (polo.utils.io\_utils.HtmlWriter static method)@\spxentry{make\_template()}\spxextra{polo.utils.io\_utils.HtmlWriter static method}}

\begin{fulllineitems}
\phantomsection\label{\detokenize{polo.utils:polo.utils.io_utils.HtmlWriter.make_template}}\pysiglinewithargsret{\sphinxbfcode{\sphinxupquote{static }}\sphinxbfcode{\sphinxupquote{make\_template}}}{\emph{\DUrole{n}{template\_path}}}{}
Given a path to an html file to serve as a jinja2 template, read the
file and create a new template object.
\begin{quote}\begin{description}
\item[{Parameters}] \leavevmode
\sphinxstyleliteralstrong{\sphinxupquote{template\_path}} \textendash{} Path to the jinja2 template file.

\end{description}\end{quote}

\end{fulllineitems}

\index{write\_complete\_run() (polo.utils.io\_utils.HtmlWriter method)@\spxentry{write\_complete\_run()}\spxextra{polo.utils.io\_utils.HtmlWriter method}}

\begin{fulllineitems}
\phantomsection\label{\detokenize{polo.utils:polo.utils.io_utils.HtmlWriter.write_complete_run}}\pysiglinewithargsret{\sphinxbfcode{\sphinxupquote{write\_complete\_run}}}{\emph{\DUrole{n}{output\_path}}, \emph{\DUrole{n}{encode\_images}\DUrole{o}{=}\DUrole{default_value}{True}}}{}
Create an HTML report from a \sphinxcode{\sphinxupquote{Run}} or \sphinxcode{\sphinxupquote{HWIRun}}
instance.
\begin{quote}\begin{description}
\item[{Parameters}] \leavevmode\begin{itemize}
\item {} 
\sphinxstyleliteralstrong{\sphinxupquote{output\_path}} (\sphinxstyleliteralemphasis{\sphinxupquote{str}}\sphinxstyleliteralemphasis{\sphinxupquote{ or }}\sphinxstyleliteralemphasis{\sphinxupquote{Path}}) \textendash{} Path to write html file to.

\item {} 
\sphinxstyleliteralstrong{\sphinxupquote{encode\_images}} (\sphinxstyleliteralemphasis{\sphinxupquote{bool}}\sphinxstyleliteralemphasis{\sphinxupquote{, }}\sphinxstyleliteralemphasis{\sphinxupquote{optional}}) \textendash{} Write images as base64 directly to the html file,
defaults to True. Greatly increases the file size
but means that report will still contain images
even if the originals are deleted or removed.

\end{itemize}

\item[{Returns}] \leavevmode
Path to html report if write succeeds, Exception otherwise.

\item[{Return type}] \leavevmode
str or Exception

\end{description}\end{quote}

\end{fulllineitems}

\index{write\_grid\_screen() (polo.utils.io\_utils.HtmlWriter method)@\spxentry{write\_grid\_screen()}\spxextra{polo.utils.io\_utils.HtmlWriter method}}

\begin{fulllineitems}
\phantomsection\label{\detokenize{polo.utils:polo.utils.io_utils.HtmlWriter.write_grid_screen}}\pysiglinewithargsret{\sphinxbfcode{\sphinxupquote{write\_grid\_screen}}}{\emph{\DUrole{n}{output\_path}}, \emph{\DUrole{n}{plate\_list}}, \emph{\DUrole{n}{well\_number}}, \emph{\DUrole{n}{x\_reagent}}, \emph{\DUrole{n}{y\_reagent}}, \emph{\DUrole{n}{well\_volume}}, \emph{\DUrole{n}{run\_name}\DUrole{o}{=}\DUrole{default_value}{None}}}{}
Write the contents of optimization grid screen to an html file.
\begin{quote}\begin{description}
\item[{Parameters}] \leavevmode\begin{itemize}
\item {} 
\sphinxstyleliteralstrong{\sphinxupquote{output\_path}} (\sphinxstyleliteralemphasis{\sphinxupquote{str}}) \textendash{} Path to html file

\item {} 
\sphinxstyleliteralstrong{\sphinxupquote{plate\_list}} (\sphinxstyleliteralemphasis{\sphinxupquote{list}}) \textendash{} list containing grid screen data

\item {} 
\sphinxstyleliteralstrong{\sphinxupquote{well\_number}} (\sphinxstyleliteralemphasis{\sphinxupquote{int}}\sphinxstyleliteralemphasis{\sphinxupquote{ or }}\sphinxstyleliteralemphasis{\sphinxupquote{str}}) \textendash{} well number of hit screen is created from

\item {} 
\sphinxstyleliteralstrong{\sphinxupquote{x\_reagent}} ({\hyperref[\detokenize{polo.crystallography:polo.crystallography.cocktail.Reagent}]{\sphinxcrossref{\sphinxstyleliteralemphasis{\sphinxupquote{Reagent}}}}}) \textendash{} reagent varied in x direction

\item {} 
\sphinxstyleliteralstrong{\sphinxupquote{y\_reagent}} ({\hyperref[\detokenize{polo.crystallography:polo.crystallography.cocktail.Reagent}]{\sphinxcrossref{\sphinxstyleliteralemphasis{\sphinxupquote{Reagent}}}}}) \textendash{} reagent varied in y direction

\item {} 
\sphinxstyleliteralstrong{\sphinxupquote{well\_volume}} (\sphinxstyleliteralemphasis{\sphinxupquote{int}}\sphinxstyleliteralemphasis{\sphinxupquote{ or }}\sphinxstyleliteralemphasis{\sphinxupquote{str}}) \textendash{} Volume of well used in screen

\item {} 
\sphinxstyleliteralstrong{\sphinxupquote{run\_name}} (\sphinxstyleliteralemphasis{\sphinxupquote{str}}\sphinxstyleliteralemphasis{\sphinxupquote{, }}\sphinxstyleliteralemphasis{\sphinxupquote{optional}}) \textendash{} name of run, defaults to None

\end{itemize}

\end{description}\end{quote}

\end{fulllineitems}


\end{fulllineitems}

\index{JsonWriter (class in polo.utils.io\_utils)@\spxentry{JsonWriter}\spxextra{class in polo.utils.io\_utils}}

\begin{fulllineitems}
\phantomsection\label{\detokenize{polo.utils:polo.utils.io_utils.JsonWriter}}\pysiglinewithargsret{\sphinxbfcode{\sphinxupquote{class }}\sphinxcode{\sphinxupquote{polo.utils.io\_utils.}}\sphinxbfcode{\sphinxupquote{JsonWriter}}}{\emph{\DUrole{n}{run}}, \emph{\DUrole{n}{output\_path}}}{}
Bases: {\hyperref[\detokenize{polo.utils:polo.utils.io_utils.RunSerializer}]{\sphinxcrossref{\sphinxcode{\sphinxupquote{polo.utils.io\_utils.RunSerializer}}}}}

Small class that can be used to serialize a run to a
json formated file.
\begin{quote}\begin{description}
\item[{Parameters}] \leavevmode\begin{itemize}
\item {} 
\sphinxstyleliteralstrong{\sphinxupquote{run}} ({\hyperref[\detokenize{polo.crystallography:polo.crystallography.run.Run}]{\sphinxcrossref{\sphinxstyleliteralemphasis{\sphinxupquote{Run}}}}}\sphinxstyleliteralemphasis{\sphinxupquote{ or }}{\hyperref[\detokenize{polo.crystallography:polo.crystallography.run.HWIRun}]{\sphinxcrossref{\sphinxstyleliteralemphasis{\sphinxupquote{HWIRun}}}}}) \textendash{} Run to write as a json file

\item {} 
\sphinxstyleliteralstrong{\sphinxupquote{output\_path}} (\sphinxstyleliteralemphasis{\sphinxupquote{str}}\sphinxstyleliteralemphasis{\sphinxupquote{ or }}\sphinxstyleliteralemphasis{\sphinxupquote{Path}}) \textendash{} Path to write json file to

\end{itemize}

\end{description}\end{quote}
\index{json\_encoder() (polo.utils.io\_utils.JsonWriter static method)@\spxentry{json\_encoder()}\spxextra{polo.utils.io\_utils.JsonWriter static method}}

\begin{fulllineitems}
\phantomsection\label{\detokenize{polo.utils:polo.utils.io_utils.JsonWriter.json_encoder}}\pysiglinewithargsret{\sphinxbfcode{\sphinxupquote{static }}\sphinxbfcode{\sphinxupquote{json\_encoder}}}{\emph{\DUrole{n}{obj}}}{}
General purpose json encoder for encoding python objects. Very
similar to the encoder function 
{\hyperref[\detokenize{polo.utils:polo.utils.io_utils.XtalWriter.json_encoder}]{\sphinxcrossref{\sphinxcode{\sphinxupquote{json\_encoder()}}}}} except does not
include class and module information in the returned dictionary. If
the object cannot be converted to a dictionary it is returned as a
string.
\begin{quote}\begin{description}
\item[{Parameters}] \leavevmode
\sphinxstyleliteralstrong{\sphinxupquote{obj}} (\sphinxstyleliteralemphasis{\sphinxupquote{obj}}) \textendash{} Object to convert to dictionary

\item[{Returns}] \leavevmode
dict or str

\item[{Return type}] \leavevmode
dict or str

\end{description}\end{quote}

\end{fulllineitems}

\index{write\_json() (polo.utils.io\_utils.JsonWriter method)@\spxentry{write\_json()}\spxextra{polo.utils.io\_utils.JsonWriter method}}

\begin{fulllineitems}
\phantomsection\label{\detokenize{polo.utils:polo.utils.io_utils.JsonWriter.write_json}}\pysiglinewithargsret{\sphinxbfcode{\sphinxupquote{write\_json}}}{}{}
Write the \sphinxcode{\sphinxupquote{Run}} instance referenced by the \sphinxcode{\sphinxupquote{run}}
attribute to a json file at
the location specified by the
\sphinxcode{\sphinxupquote{output\_path}} attribute.
If the file is written successfully returns True
otherwise returns an Exception.
\begin{quote}\begin{description}
\item[{Returns}] \leavevmode
True or Exception

\item[{Return type}] \leavevmode
bool, Exception

\end{description}\end{quote}

\end{fulllineitems}


\end{fulllineitems}

\index{Menu (class in polo.utils.io\_utils)@\spxentry{Menu}\spxextra{class in polo.utils.io\_utils}}

\begin{fulllineitems}
\phantomsection\label{\detokenize{polo.utils:polo.utils.io_utils.Menu}}\pysiglinewithargsret{\sphinxbfcode{\sphinxupquote{class }}\sphinxcode{\sphinxupquote{polo.utils.io\_utils.}}\sphinxbfcode{\sphinxupquote{Menu}}}{\emph{\DUrole{n}{path}}, \emph{\DUrole{n}{start\_date}}, \emph{\DUrole{n}{end\_date}}, \emph{\DUrole{n}{type\_}}, \emph{\DUrole{n}{cocktails}\DUrole{o}{=}\DUrole{default_value}{\{\}}}}{}
Bases: \sphinxcode{\sphinxupquote{object}}
\index{cocktails() (polo.utils.io\_utils.Menu property)@\spxentry{cocktails()}\spxextra{polo.utils.io\_utils.Menu property}}

\begin{fulllineitems}
\phantomsection\label{\detokenize{polo.utils:polo.utils.io_utils.Menu.cocktails}}\pysigline{\sphinxbfcode{\sphinxupquote{property }}\sphinxbfcode{\sphinxupquote{cocktails}}}
\end{fulllineitems}


\end{fulllineitems}

\index{MsoReader (class in polo.utils.io\_utils)@\spxentry{MsoReader}\spxextra{class in polo.utils.io\_utils}}

\begin{fulllineitems}
\phantomsection\label{\detokenize{polo.utils:polo.utils.io_utils.MsoReader}}\pysiglinewithargsret{\sphinxbfcode{\sphinxupquote{class }}\sphinxcode{\sphinxupquote{polo.utils.io\_utils.}}\sphinxbfcode{\sphinxupquote{MsoReader}}}{\emph{\DUrole{n}{mso\_path}}}{}
Bases: \sphinxcode{\sphinxupquote{object}}

The MsoReader class is used to parse the content of mso formated
files and apply the image classifications stored in these files to
runs in Polo.
\index{classify\_images\_from\_mso\_file() (polo.utils.io\_utils.MsoReader method)@\spxentry{classify\_images\_from\_mso\_file()}\spxextra{polo.utils.io\_utils.MsoReader method}}

\begin{fulllineitems}
\phantomsection\label{\detokenize{polo.utils:polo.utils.io_utils.MsoReader.classify_images_from_mso_file}}\pysiglinewithargsret{\sphinxbfcode{\sphinxupquote{classify\_images\_from\_mso\_file}}}{\emph{\DUrole{n}{images}}}{}
Applies the image classifications stored in an mso file to a
collection of {\hyperref[\detokenize{polo.crystallography:polo.crystallography.image.Image}]{\sphinxcrossref{\sphinxcode{\sphinxupquote{Image}}}}} objects.
Allows for some degree of compatability
with MarcoscopeJ and for users who have stored their image classifications
in the mso format. Additionally, human classification backups are
saved as mso files when Polo is closed as they take up much less
space than xtal files.
\begin{quote}\begin{description}
\item[{Parameters}] \leavevmode
\sphinxstyleliteralstrong{\sphinxupquote{images}} (\sphinxstyleliteralemphasis{\sphinxupquote{list}}) \textendash{} List of images to apply classifications to

\item[{Returns}] \leavevmode
List of images with mso classifications applied,
or Exception if this fails

\item[{Return type}] \leavevmode
list or Exception

\end{description}\end{quote}

\end{fulllineitems}

\index{read\_mso\_classification() (polo.utils.io\_utils.MsoReader static method)@\spxentry{read\_mso\_classification()}\spxextra{polo.utils.io\_utils.MsoReader static method}}

\begin{fulllineitems}
\phantomsection\label{\detokenize{polo.utils:polo.utils.io_utils.MsoReader.read_mso_classification}}\pysiglinewithargsret{\sphinxbfcode{\sphinxupquote{static }}\sphinxbfcode{\sphinxupquote{read\_mso\_classification}}}{\emph{\DUrole{n}{mso\_classification}}}{}
Helper method to read and convert image classifications in a mso
file to MARCO image classifications. The exact conversion scheme is
layed out in the \sphinxcode{\sphinxupquote{REV\_MSO\_DICT}} constant.
Additionally, MarcoscopeJ will allow
images to have multiple classifications by assigning multiple
image codes seperated by “\sphinxhyphen{}“. If this is the case Polo takes the
classification corresponding to the mso code with the highest value.
\begin{quote}\begin{description}
\item[{Parameters}] \leavevmode
\sphinxstyleliteralstrong{\sphinxupquote{mso\_classification}} (\sphinxstyleliteralemphasis{\sphinxupquote{str}}) \textendash{} Mso image classification code

\item[{Returns}] \leavevmode
MARCO classification if code can be decoded, None otherwise

\item[{Return type}] \leavevmode
str or None

\end{description}\end{quote}

\end{fulllineitems}


\end{fulllineitems}

\index{MsoWriter (class in polo.utils.io\_utils)@\spxentry{MsoWriter}\spxextra{class in polo.utils.io\_utils}}

\begin{fulllineitems}
\phantomsection\label{\detokenize{polo.utils:polo.utils.io_utils.MsoWriter}}\pysiglinewithargsret{\sphinxbfcode{\sphinxupquote{class }}\sphinxcode{\sphinxupquote{polo.utils.io\_utils.}}\sphinxbfcode{\sphinxupquote{MsoWriter}}}{\emph{\DUrole{n}{run}}, \emph{\DUrole{n}{output\_path}}}{}
Bases: {\hyperref[\detokenize{polo.utils:polo.utils.io_utils.RunSerializer}]{\sphinxcrossref{\sphinxcode{\sphinxupquote{polo.utils.io\_utils.RunSerializer}}}}}
\index{first\_line() (polo.utils.io\_utils.MsoWriter property)@\spxentry{first\_line()}\spxextra{polo.utils.io\_utils.MsoWriter property}}

\begin{fulllineitems}
\phantomsection\label{\detokenize{polo.utils:polo.utils.io_utils.MsoWriter.first_line}}\pysigline{\sphinxbfcode{\sphinxupquote{property }}\sphinxbfcode{\sphinxupquote{first\_line}}}
Create the first line of the mso file.
\begin{quote}\begin{description}
\item[{Returns}] \leavevmode
List to write as the first line of the mso file.

\item[{Return type}] \leavevmode
list

\end{description}\end{quote}

\end{fulllineitems}

\index{get\_cocktail\_csv\_data() (polo.utils.io\_utils.MsoWriter method)@\spxentry{get\_cocktail\_csv\_data()}\spxextra{polo.utils.io\_utils.MsoWriter method}}

\begin{fulllineitems}
\phantomsection\label{\detokenize{polo.utils:polo.utils.io_utils.MsoWriter.get_cocktail_csv_data}}\pysiglinewithargsret{\sphinxbfcode{\sphinxupquote{get\_cocktail\_csv\_data}}}{}{}
Reads and returns the cocktail csv data assigned
to the  \sphinxcode{\sphinxupquote{cocktail\_menu}}
attribute of the MsoWriter’s
\sphinxcode{\sphinxupquote{run}} attribute.
\begin{quote}\begin{description}
\item[{Returns}] \leavevmode
List of lists containing cocktail csv data.

\item[{Return type}] \leavevmode
list

\end{description}\end{quote}

\end{fulllineitems}

\index{mso\_version (polo.utils.io\_utils.MsoWriter attribute)@\spxentry{mso\_version}\spxextra{polo.utils.io\_utils.MsoWriter attribute}}

\begin{fulllineitems}
\phantomsection\label{\detokenize{polo.utils:polo.utils.io_utils.MsoWriter.mso_version}}\pysigline{\sphinxbfcode{\sphinxupquote{mso\_version}}\sphinxbfcode{\sphinxupquote{ = \textquotesingle{}msoversion2\textquotesingle{}}}}
\end{fulllineitems}

\index{row\_formater() (polo.utils.io\_utils.MsoWriter static method)@\spxentry{row\_formater()}\spxextra{polo.utils.io\_utils.MsoWriter static method}}

\begin{fulllineitems}
\phantomsection\label{\detokenize{polo.utils:polo.utils.io_utils.MsoWriter.row_formater}}\pysiglinewithargsret{\sphinxbfcode{\sphinxupquote{static }}\sphinxbfcode{\sphinxupquote{row\_formater}}}{\emph{\DUrole{n}{cocktail\_row}}}{}
Format a cocktail row as read from a cocktail csv
file to an mso file row. Main change is appending empty
strings to the cocktail row so list ends up always having
a length of 17. This is important because the image
classification code always occurs at the 18th item in
an mso file row.
\begin{quote}\begin{description}
\item[{Parameters}] \leavevmode
\sphinxstyleliteralstrong{\sphinxupquote{cocktail\_row}} (\sphinxstyleliteralemphasis{\sphinxupquote{list}}) \textendash{} Cocktail row as read from cocktail csv file.

\item[{Returns}] \leavevmode
Cocktail row reformated for mso writing.

\item[{Return type}] \leavevmode
list

\end{description}\end{quote}

\end{fulllineitems}

\index{write\_mso\_file() (polo.utils.io\_utils.MsoWriter method)@\spxentry{write\_mso\_file()}\spxextra{polo.utils.io\_utils.MsoWriter method}}

\begin{fulllineitems}
\phantomsection\label{\detokenize{polo.utils:polo.utils.io_utils.MsoWriter.write_mso_file}}\pysiglinewithargsret{\sphinxbfcode{\sphinxupquote{write\_mso\_file}}}{\emph{\DUrole{n}{use\_marco\_classifications}\DUrole{o}{=}\DUrole{default_value}{False}}}{}
Writes an mso formated file for use with MarcoScopeJ based on
the images and classifications of the \sphinxcode{\sphinxupquote{Run}} instance
referenced by the MsoWriter’s \sphinxcode{\sphinxupquote{run}}
attribute.
\begin{quote}\begin{description}
\item[{Parameters}] \leavevmode
\sphinxstyleliteralstrong{\sphinxupquote{use\_marco\_classifications}} (\sphinxstyleliteralemphasis{\sphinxupquote{bool}}\sphinxstyleliteralemphasis{\sphinxupquote{, }}\sphinxstyleliteralemphasis{\sphinxupquote{optional}}) \textendash{} Include the MARCO classification
in the mso file instead of human
classifications, defaults to False

\item[{Returns}] \leavevmode
True if file is written successfully, False otherwise.

\item[{Return type}] \leavevmode
bool

\end{description}\end{quote}

\end{fulllineitems}


\end{fulllineitems}

\index{PptxWriter (class in polo.utils.io\_utils)@\spxentry{PptxWriter}\spxextra{class in polo.utils.io\_utils}}

\begin{fulllineitems}
\phantomsection\label{\detokenize{polo.utils:polo.utils.io_utils.PptxWriter}}\pysiglinewithargsret{\sphinxbfcode{\sphinxupquote{class }}\sphinxcode{\sphinxupquote{polo.utils.io\_utils.}}\sphinxbfcode{\sphinxupquote{PptxWriter}}}{\emph{\DUrole{n}{output\_path}}, \emph{\DUrole{n}{image\_types}\DUrole{o}{=}\DUrole{default_value}{None}}, \emph{\DUrole{n}{human}\DUrole{o}{=}\DUrole{default_value}{False}}, \emph{\DUrole{n}{marco}\DUrole{o}{=}\DUrole{default_value}{False}}, \emph{\DUrole{n}{favorite}\DUrole{o}{=}\DUrole{default_value}{False}}}{}
Bases: \sphinxcode{\sphinxupquote{object}}

Use for creating pptx presentation slides from \sphinxcode{\sphinxupquote{Run}}
or \sphinxcode{\sphinxupquote{HWIRun}} instances.
\begin{quote}\begin{description}
\item[{Parameters}] \leavevmode\begin{itemize}
\item {} 
\sphinxstyleliteralstrong{\sphinxupquote{output\_path}} (\sphinxstyleliteralemphasis{\sphinxupquote{str}}\sphinxstyleliteralemphasis{\sphinxupquote{ or }}\sphinxstyleliteralemphasis{\sphinxupquote{Path}}) \textendash{} Path to write pptx file to.

\item {} 
\sphinxstyleliteralstrong{\sphinxupquote{included\_attributes}} (\sphinxstyleliteralemphasis{\sphinxupquote{dict}}\sphinxstyleliteralemphasis{\sphinxupquote{, }}\sphinxstyleliteralemphasis{\sphinxupquote{optional}}) \textendash{} {[}description{]}, defaults to \{\}

\item {} 
\sphinxstyleliteralstrong{\sphinxupquote{image\_types}} (\sphinxstyleliteralemphasis{\sphinxupquote{set}}\sphinxstyleliteralemphasis{\sphinxupquote{ or }}\sphinxstyleliteralemphasis{\sphinxupquote{list}}\sphinxstyleliteralemphasis{\sphinxupquote{, }}\sphinxstyleliteralemphasis{\sphinxupquote{optional}}) \textendash{} Images included in the presentation
must have a classification in this set, defaults to None

\item {} 
\sphinxstyleliteralstrong{\sphinxupquote{human}} (\sphinxstyleliteralemphasis{\sphinxupquote{bool}}\sphinxstyleliteralemphasis{\sphinxupquote{, }}\sphinxstyleliteralemphasis{\sphinxupquote{optional}}) \textendash{} Use human classification as the image classification, defaults to False

\item {} 
\sphinxstyleliteralstrong{\sphinxupquote{marco}} (\sphinxstyleliteralemphasis{\sphinxupquote{bool}}\sphinxstyleliteralemphasis{\sphinxupquote{, }}\sphinxstyleliteralemphasis{\sphinxupquote{optional}}) \textendash{} Use the MARCO classification as the image classification, defaults to False

\item {} 
\sphinxstyleliteralstrong{\sphinxupquote{favorite}} (\sphinxstyleliteralemphasis{\sphinxupquote{bool}}\sphinxstyleliteralemphasis{\sphinxupquote{, }}\sphinxstyleliteralemphasis{\sphinxupquote{optional}}) \textendash{} Only include images marked as favorite, defaults to False

\end{itemize}

\end{description}\end{quote}
\index{add\_classification\_slide() (polo.utils.io\_utils.PptxWriter method)@\spxentry{add\_classification\_slide()}\spxextra{polo.utils.io\_utils.PptxWriter method}}

\begin{fulllineitems}
\phantomsection\label{\detokenize{polo.utils:polo.utils.io_utils.PptxWriter.add_classification_slide}}\pysiglinewithargsret{\sphinxbfcode{\sphinxupquote{add\_classification\_slide}}}{\emph{\DUrole{n}{well\_number}}, \emph{\DUrole{n}{rep\_image}}}{}
Add a slide containing details about an images MARCO
and human classification in a table.
\begin{quote}\begin{description}
\item[{Parameters}] \leavevmode\begin{itemize}
\item {} 
\sphinxstyleliteralstrong{\sphinxupquote{well\_number}} (\sphinxstyleliteralemphasis{\sphinxupquote{int}}) \textendash{} Well number (index) of image to use in
the title of the slide

\item {} 
\sphinxstyleliteralstrong{\sphinxupquote{rep\_image}} ({\hyperref[\detokenize{polo.crystallography:polo.crystallography.image.Image}]{\sphinxcrossref{\sphinxstyleliteralemphasis{\sphinxupquote{Image}}}}}) \textendash{} Image object to make slide from

\end{itemize}

\end{description}\end{quote}

\end{fulllineitems}

\index{add\_cocktail\_slide() (polo.utils.io\_utils.PptxWriter method)@\spxentry{add\_cocktail\_slide()}\spxextra{polo.utils.io\_utils.PptxWriter method}}

\begin{fulllineitems}
\phantomsection\label{\detokenize{polo.utils:polo.utils.io_utils.PptxWriter.add_cocktail_slide}}\pysiglinewithargsret{\sphinxbfcode{\sphinxupquote{add\_cocktail\_slide}}}{\emph{\DUrole{n}{well}}, \emph{\DUrole{n}{cocktail}}}{}
Add slide with details on \sphinxcode{\sphinxupquote{Cocktail}} information.
\begin{quote}\begin{description}
\item[{Parameters}] \leavevmode\begin{itemize}
\item {} 
\sphinxstyleliteralstrong{\sphinxupquote{well}} (\sphinxstyleliteralemphasis{\sphinxupquote{int}}) \textendash{} Well number to use in slide title

\item {} 
\sphinxstyleliteralstrong{\sphinxupquote{cocktail}} ({\hyperref[\detokenize{polo.crystallography:polo.crystallography.cocktail.Cocktail}]{\sphinxcrossref{\sphinxstyleliteralemphasis{\sphinxupquote{Cocktail}}}}}) \textendash{} Cocktail to write as a slide

\end{itemize}

\end{description}\end{quote}

\end{fulllineitems}

\index{add\_image\_to\_slide() (polo.utils.io\_utils.PptxWriter method)@\spxentry{add\_image\_to\_slide()}\spxextra{polo.utils.io\_utils.PptxWriter method}}

\begin{fulllineitems}
\phantomsection\label{\detokenize{polo.utils:polo.utils.io_utils.PptxWriter.add_image_to_slide}}\pysiglinewithargsret{\sphinxbfcode{\sphinxupquote{add\_image\_to\_slide}}}{\emph{\DUrole{n}{image}}, \emph{\DUrole{n}{slide}}, \emph{\DUrole{n}{left}}, \emph{\DUrole{n}{top}}, \emph{\DUrole{n}{height}}}{}
Helper method for adding images to a slide. If the \sphinxcode{\sphinxupquote{Image}}
does not have a file written on the local machine as can be
the case with saved runs who’s image data only exists in
their xtal files this method will write a temporary image
file to the Polo \sphinxcode{\sphinxupquote{TEMP\_DIR}} which then should be deleted after
the presentaton file is written.
\begin{quote}\begin{description}
\item[{Parameters}] \leavevmode\begin{itemize}
\item {} 
\sphinxstyleliteralstrong{\sphinxupquote{image}} ({\hyperref[\detokenize{polo.crystallography:polo.crystallography.image.Image}]{\sphinxcrossref{\sphinxstyleliteralemphasis{\sphinxupquote{Image}}}}}) \textendash{} Image to add to the slide

\item {} 
\sphinxstyleliteralstrong{\sphinxupquote{slide}} (\sphinxstyleliteralemphasis{\sphinxupquote{slide}}) \textendash{} Slide to add the image to

\item {} 
\sphinxstyleliteralstrong{\sphinxupquote{left}} (\sphinxstyleliteralemphasis{\sphinxupquote{float}}) \textendash{} Left cordinate location of the image in inches

\item {} 
\sphinxstyleliteralstrong{\sphinxupquote{top}} (\sphinxstyleliteralemphasis{\sphinxupquote{float}}) \textendash{} Top cordinate location of the image in inches

\item {} 
\sphinxstyleliteralstrong{\sphinxupquote{height}} (\sphinxstyleliteralemphasis{\sphinxupquote{float}}) \textendash{} Height of the image in inches

\end{itemize}

\item[{Returns}] \leavevmode
{[}{]}

\item[{Return type}] \leavevmode
{[}type{]}

\end{description}\end{quote}

\end{fulllineitems}

\index{add\_multi\_image\_slide() (polo.utils.io\_utils.PptxWriter method)@\spxentry{add\_multi\_image\_slide()}\spxextra{polo.utils.io\_utils.PptxWriter method}}

\begin{fulllineitems}
\phantomsection\label{\detokenize{polo.utils:polo.utils.io_utils.PptxWriter.add_multi_image_slide}}\pysiglinewithargsret{\sphinxbfcode{\sphinxupquote{add\_multi\_image\_slide}}}{\emph{\DUrole{n}{slide}}, \emph{\DUrole{n}{images}}, \emph{\DUrole{n}{labeler}}}{}
General helper method for adding a slide that will have multiple
images.
\begin{quote}\begin{description}
\item[{Parameters}] \leavevmode\begin{itemize}
\item {} 
\sphinxstyleliteralstrong{\sphinxupquote{slide}} (\sphinxstyleliteralemphasis{\sphinxupquote{slide}}) \textendash{} Slide to add the images to

\item {} 
\sphinxstyleliteralstrong{\sphinxupquote{images}} (\sphinxstyleliteralemphasis{\sphinxupquote{list}}) \textendash{} Images to add to the slide

\item {} 
\sphinxstyleliteralstrong{\sphinxupquote{labeler}} (\sphinxstyleliteralemphasis{\sphinxupquote{func}}) \textendash{} Function to use to label the individual images

\end{itemize}

\item[{Returns}] \leavevmode
slide with images added

\item[{Return type}] \leavevmode
slide

\end{description}\end{quote}

\end{fulllineitems}

\index{add\_multi\_spectrum\_slide() (polo.utils.io\_utils.PptxWriter method)@\spxentry{add\_multi\_spectrum\_slide()}\spxextra{polo.utils.io\_utils.PptxWriter method}}

\begin{fulllineitems}
\phantomsection\label{\detokenize{polo.utils:polo.utils.io_utils.PptxWriter.add_multi_spectrum_slide}}\pysiglinewithargsret{\sphinxbfcode{\sphinxupquote{add\_multi\_spectrum\_slide}}}{\emph{\DUrole{n}{images}}, \emph{\DUrole{n}{well\_number}}}{}
Create a slide to show a all spectrums a well has been
imaged in.
\begin{quote}\begin{description}
\item[{Parameters}] \leavevmode\begin{itemize}
\item {} 
\sphinxstyleliteralstrong{\sphinxupquote{images}} (\sphinxstyleliteralemphasis{\sphinxupquote{list}}) \textendash{} Images to include on the slide

\item {} 
\sphinxstyleliteralstrong{\sphinxupquote{well\_number}} (\sphinxstyleliteralemphasis{\sphinxupquote{int}}) \textendash{} Well number to use in the slide title

\end{itemize}

\item[{Returns}] \leavevmode
New slide

\item[{Return type}] \leavevmode
slide

\end{description}\end{quote}

\end{fulllineitems}

\index{add\_new\_slide() (polo.utils.io\_utils.PptxWriter method)@\spxentry{add\_new\_slide()}\spxextra{polo.utils.io\_utils.PptxWriter method}}

\begin{fulllineitems}
\phantomsection\label{\detokenize{polo.utils:polo.utils.io_utils.PptxWriter.add_new_slide}}\pysiglinewithargsret{\sphinxbfcode{\sphinxupquote{add\_new\_slide}}}{\emph{\DUrole{n}{template}\DUrole{o}{=}\DUrole{default_value}{5}}}{}
\end{fulllineitems}

\index{add\_single\_image\_slide() (polo.utils.io\_utils.PptxWriter method)@\spxentry{add\_single\_image\_slide()}\spxextra{polo.utils.io\_utils.PptxWriter method}}

\begin{fulllineitems}
\phantomsection\label{\detokenize{polo.utils:polo.utils.io_utils.PptxWriter.add_single_image_slide}}\pysiglinewithargsret{\sphinxbfcode{\sphinxupquote{add\_single\_image\_slide}}}{\emph{\DUrole{n}{image}}, \emph{\DUrole{n}{title}}, \emph{\DUrole{n}{metadata}\DUrole{o}{=}\DUrole{default_value}{None}}, \emph{\DUrole{n}{img\_scaler}\DUrole{o}{=}\DUrole{default_value}{0.65}}}{}
General helper method for adding a slide with a single image to a
presentation.
\begin{quote}\begin{description}
\item[{Parameters}] \leavevmode\begin{itemize}
\item {} 
\sphinxstyleliteralstrong{\sphinxupquote{image}} ({\hyperref[\detokenize{polo.crystallography:polo.crystallography.image.Image}]{\sphinxcrossref{\sphinxstyleliteralemphasis{\sphinxupquote{Image}}}}}) \textendash{} Image to add to the slide

\item {} 
\sphinxstyleliteralstrong{\sphinxupquote{title}} (\sphinxstyleliteralemphasis{\sphinxupquote{str}}) \textendash{} Title to use for the slide

\item {} 
\sphinxstyleliteralstrong{\sphinxupquote{metadata}} (\sphinxstyleliteralemphasis{\sphinxupquote{str}}\sphinxstyleliteralemphasis{\sphinxupquote{, }}\sphinxstyleliteralemphasis{\sphinxupquote{optional}}) \textendash{} Additional information to write to the slide, defaults to None

\item {} 
\sphinxstyleliteralstrong{\sphinxupquote{img\_scaler}} (\sphinxstyleliteralemphasis{\sphinxupquote{float}}\sphinxstyleliteralemphasis{\sphinxupquote{, }}\sphinxstyleliteralemphasis{\sphinxupquote{optional}}) \textendash{} Scaler to apply to size of the image, defaults to 0.65
,should be between 0 and 1. 1 is full sized image.

\end{itemize}

\item[{Returns}] \leavevmode
The new slide with Image added

\item[{Return type}] \leavevmode
slide

\end{description}\end{quote}

\end{fulllineitems}

\index{add\_table\_to\_slide() (polo.utils.io\_utils.PptxWriter method)@\spxentry{add\_table\_to\_slide()}\spxextra{polo.utils.io\_utils.PptxWriter method}}

\begin{fulllineitems}
\phantomsection\label{\detokenize{polo.utils:polo.utils.io_utils.PptxWriter.add_table_to_slide}}\pysiglinewithargsret{\sphinxbfcode{\sphinxupquote{add\_table\_to\_slide}}}{\emph{\DUrole{n}{slide}}, \emph{\DUrole{n}{data}}, \emph{\DUrole{n}{left}}, \emph{\DUrole{n}{top}}}{}
General helper method for adding a table to a slide.
\begin{quote}\begin{description}
\item[{Parameters}] \leavevmode\begin{itemize}
\item {} 
\sphinxstyleliteralstrong{\sphinxupquote{slide}} (\sphinxstyleliteralemphasis{\sphinxupquote{slide}}) \textendash{} Slide to add the table to

\item {} 
\sphinxstyleliteralstrong{\sphinxupquote{data}} (\sphinxstyleliteralemphasis{\sphinxupquote{list}}) \textendash{} List of lists that has the data to write to the table

\item {} 
\sphinxstyleliteralstrong{\sphinxupquote{left}} (\sphinxstyleliteralemphasis{\sphinxupquote{float}}) \textendash{} Left offset in inches to place to table

\item {} 
\sphinxstyleliteralstrong{\sphinxupquote{top}} (\sphinxstyleliteralemphasis{\sphinxupquote{float}}) \textendash{} Top cordinate for placing the table

\end{itemize}

\item[{Returns}] \leavevmode
Slide with table added

\item[{Return type}] \leavevmode
slide

\end{description}\end{quote}

\end{fulllineitems}

\index{add\_text\_to\_slide() (polo.utils.io\_utils.PptxWriter method)@\spxentry{add\_text\_to\_slide()}\spxextra{polo.utils.io\_utils.PptxWriter method}}

\begin{fulllineitems}
\phantomsection\label{\detokenize{polo.utils:polo.utils.io_utils.PptxWriter.add_text_to_slide}}\pysiglinewithargsret{\sphinxbfcode{\sphinxupquote{add\_text\_to\_slide}}}{\emph{\DUrole{n}{slide}}, \emph{\DUrole{n}{text}}, \emph{\DUrole{n}{left}}, \emph{\DUrole{n}{top}}, \emph{\DUrole{n}{width}}, \emph{\DUrole{n}{height}}, \emph{\DUrole{n}{rotation}\DUrole{o}{=}\DUrole{default_value}{0}}, \emph{\DUrole{n}{font\_size}\DUrole{o}{=}\DUrole{default_value}{14}}}{}
Helper method to add text to a slide
\begin{quote}\begin{description}
\item[{Parameters}] \leavevmode\begin{itemize}
\item {} 
\sphinxstyleliteralstrong{\sphinxupquote{slide}} (\sphinxstyleliteralemphasis{\sphinxupquote{slide}}) \textendash{} Slide to add text to

\item {} 
\sphinxstyleliteralstrong{\sphinxupquote{text}} (\sphinxstyleliteralemphasis{\sphinxupquote{str}}) \textendash{} Text to add to the slide

\item {} 
\sphinxstyleliteralstrong{\sphinxupquote{left}} (\sphinxstyleliteralemphasis{\sphinxupquote{float}}) \textendash{} Left cordinate location of the text in inches

\item {} 
\sphinxstyleliteralstrong{\sphinxupquote{top}} (\sphinxstyleliteralemphasis{\sphinxupquote{float}}) \textendash{} Top cordinate location of the text in inches

\item {} 
\sphinxstyleliteralstrong{\sphinxupquote{width}} (\sphinxstyleliteralemphasis{\sphinxupquote{float}}) \textendash{} Width of the text in inches

\item {} 
\sphinxstyleliteralstrong{\sphinxupquote{height}} (\sphinxstyleliteralemphasis{\sphinxupquote{float}}) \textendash{} Height of the text in inches

\item {} 
\sphinxstyleliteralstrong{\sphinxupquote{rotation}} (\sphinxstyleliteralemphasis{\sphinxupquote{int}}\sphinxstyleliteralemphasis{\sphinxupquote{, }}\sphinxstyleliteralemphasis{\sphinxupquote{optional}}) \textendash{} Rotation to apply to the text in degrees, defaults to 0

\item {} 
\sphinxstyleliteralstrong{\sphinxupquote{font\_size}} (\sphinxstyleliteralemphasis{\sphinxupquote{int}}\sphinxstyleliteralemphasis{\sphinxupquote{, }}\sphinxstyleliteralemphasis{\sphinxupquote{optional}}) \textendash{} Font size of text, defaults to 14

\end{itemize}

\item[{Returns}] \leavevmode
Slide with text added

\item[{Return type}] \leavevmode
slide

\end{description}\end{quote}

\end{fulllineitems}

\index{add\_timeline\_slide() (polo.utils.io\_utils.PptxWriter method)@\spxentry{add\_timeline\_slide()}\spxextra{polo.utils.io\_utils.PptxWriter method}}

\begin{fulllineitems}
\phantomsection\label{\detokenize{polo.utils:polo.utils.io_utils.PptxWriter.add_timeline_slide}}\pysiglinewithargsret{\sphinxbfcode{\sphinxupquote{add\_timeline\_slide}}}{\emph{\DUrole{n}{images}}, \emph{\DUrole{n}{well\_number}}}{}
Create a timeline (time resolved) slide that
show the progression of a sample in a particular
well.
\begin{quote}\begin{description}
\item[{Parameters}] \leavevmode\begin{itemize}
\item {} 
\sphinxstyleliteralstrong{\sphinxupquote{images}} (\sphinxstyleliteralemphasis{\sphinxupquote{list}}) \textendash{} List of images to include in the slide

\item {} 
\sphinxstyleliteralstrong{\sphinxupquote{well\_number}} (\sphinxstyleliteralemphasis{\sphinxupquote{int}}) \textendash{} Well number to use in the title of the slide

\end{itemize}

\item[{Returns}] \leavevmode
New slide

\item[{Return type}] \leavevmode
slide

\end{description}\end{quote}

\end{fulllineitems}

\index{delete\_presentation() (polo.utils.io\_utils.PptxWriter method)@\spxentry{delete\_presentation()}\spxextra{polo.utils.io\_utils.PptxWriter method}}

\begin{fulllineitems}
\phantomsection\label{\detokenize{polo.utils:polo.utils.io_utils.PptxWriter.delete_presentation}}\pysiglinewithargsret{\sphinxbfcode{\sphinxupquote{delete\_presentation}}}{}{}
\end{fulllineitems}

\index{delete\_temp\_images() (polo.utils.io\_utils.PptxWriter method)@\spxentry{delete\_temp\_images()}\spxextra{polo.utils.io\_utils.PptxWriter method}}

\begin{fulllineitems}
\phantomsection\label{\detokenize{polo.utils:polo.utils.io_utils.PptxWriter.delete_temp_images}}\pysiglinewithargsret{\sphinxbfcode{\sphinxupquote{delete\_temp\_images}}}{}{}
Delete an temporary images used to create the pptx presentation.
\begin{quote}\begin{description}
\item[{Returns}] \leavevmode
True, if images are removed successfully, Exception otherwise.

\item[{Return type}] \leavevmode
bool or Exception

\end{description}\end{quote}

\end{fulllineitems}

\index{make\_single\_run\_presentation() (polo.utils.io\_utils.PptxWriter method)@\spxentry{make\_single\_run\_presentation()}\spxextra{polo.utils.io\_utils.PptxWriter method}}

\begin{fulllineitems}
\phantomsection\label{\detokenize{polo.utils:polo.utils.io_utils.PptxWriter.make_single_run_presentation}}\pysiglinewithargsret{\sphinxbfcode{\sphinxupquote{make\_single\_run\_presentation}}}{\emph{\DUrole{n}{run}}, \emph{\DUrole{n}{title}}, \emph{\DUrole{n}{subtitle}\DUrole{o}{=}\DUrole{default_value}{None}}, \emph{\DUrole{n}{cocktail\_data}\DUrole{o}{=}\DUrole{default_value}{True}}, \emph{\DUrole{n}{all\_specs}\DUrole{o}{=}\DUrole{default_value}{False}}, \emph{\DUrole{n}{all\_dates}\DUrole{o}{=}\DUrole{default_value}{False}}}{}
\end{fulllineitems}

\index{sort\_runs\_by\_spectrum() (polo.utils.io\_utils.PptxWriter method)@\spxentry{sort\_runs\_by\_spectrum()}\spxextra{polo.utils.io\_utils.PptxWriter method}}

\begin{fulllineitems}
\phantomsection\label{\detokenize{polo.utils:polo.utils.io_utils.PptxWriter.sort_runs_by_spectrum}}\pysiglinewithargsret{\sphinxbfcode{\sphinxupquote{sort\_runs\_by\_spectrum}}}{\emph{\DUrole{n}{runs}}}{}
Divids runs into two lists, one containing visible spectrum
runs and another containing all non\sphinxhyphen{}visible runs.
\begin{quote}\begin{description}
\item[{Parameters}] \leavevmode
\sphinxstyleliteralstrong{\sphinxupquote{runs}} (\sphinxstyleliteralemphasis{\sphinxupquote{list}}) \textendash{} List or runs

\item[{Returns}] \leavevmode
tuple, first item is visible runs second is non\sphinxhyphen{}visible runs

\item[{Return type}] \leavevmode
tuple

\end{description}\end{quote}

\end{fulllineitems}


\end{fulllineitems}

\index{RunCsvWriter (class in polo.utils.io\_utils)@\spxentry{RunCsvWriter}\spxextra{class in polo.utils.io\_utils}}

\begin{fulllineitems}
\phantomsection\label{\detokenize{polo.utils:polo.utils.io_utils.RunCsvWriter}}\pysiglinewithargsret{\sphinxbfcode{\sphinxupquote{class }}\sphinxcode{\sphinxupquote{polo.utils.io\_utils.}}\sphinxbfcode{\sphinxupquote{RunCsvWriter}}}{\emph{\DUrole{n}{run}}, \emph{\DUrole{n}{output\_path}\DUrole{o}{=}\DUrole{default_value}{None}}, \emph{\DUrole{o}{**}\DUrole{n}{kwargs}}}{}
Bases: {\hyperref[\detokenize{polo.utils:polo.utils.io_utils.RunSerializer}]{\sphinxcrossref{\sphinxcode{\sphinxupquote{polo.utils.io\_utils.RunSerializer}}}}}
\index{fieldnames() (polo.utils.io\_utils.RunCsvWriter property)@\spxentry{fieldnames()}\spxextra{polo.utils.io\_utils.RunCsvWriter property}}

\begin{fulllineitems}
\phantomsection\label{\detokenize{polo.utils:polo.utils.io_utils.RunCsvWriter.fieldnames}}\pysigline{\sphinxbfcode{\sphinxupquote{property }}\sphinxbfcode{\sphinxupquote{fieldnames}}}
Get the current fieldnames based on the data stored in the
\sphinxcode{\sphinxupquote{run}} attribute.
Currently is somewhat expensive to call since it
requires parsing all records in 
\sphinxcode{\sphinxupquote{run}}
in order to determine all the
fieldnames that should be included in order to definitely avoid
keyerrors later down the line.
\begin{quote}\begin{description}
\item[{Returns}] \leavevmode
List of fieldnames (headers) for the csv data

\item[{Return type}] \leavevmode
list

\end{description}\end{quote}

\end{fulllineitems}

\index{get\_csv\_data() (polo.utils.io\_utils.RunCsvWriter method)@\spxentry{get\_csv\_data()}\spxextra{polo.utils.io\_utils.RunCsvWriter method}}

\begin{fulllineitems}
\phantomsection\label{\detokenize{polo.utils:polo.utils.io_utils.RunCsvWriter.get_csv_data}}\pysiglinewithargsret{\sphinxbfcode{\sphinxupquote{get\_csv\_data}}}{}{}
Convert the  \sphinxcode{\sphinxupquote{run}}
attribute to csv style data. Returns a tuple of
headers and a list of dictionaries with each dictionary representing
one row of csv data.
\begin{quote}\begin{description}
\item[{Returns}] \leavevmode
Tuple, list of headers and list of dicts

\item[{Return type}] \leavevmode
tuple

\end{description}\end{quote}

\end{fulllineitems}

\index{image\_to\_row() (polo.utils.io\_utils.RunCsvWriter class method)@\spxentry{image\_to\_row()}\spxextra{polo.utils.io\_utils.RunCsvWriter class method}}

\begin{fulllineitems}
\phantomsection\label{\detokenize{polo.utils:polo.utils.io_utils.RunCsvWriter.image_to_row}}\pysiglinewithargsret{\sphinxbfcode{\sphinxupquote{classmethod }}\sphinxbfcode{\sphinxupquote{image\_to\_row}}}{\emph{\DUrole{n}{image}}}{}
Given an \sphinxcode{\sphinxupquote{Image}} object, convert it into a list that could be
easily written to a csv file.
\begin{quote}\begin{description}
\item[{Parameters}] \leavevmode
\sphinxstyleliteralstrong{\sphinxupquote{image}} ({\hyperref[\detokenize{polo.crystallography:polo.crystallography.image.Image}]{\sphinxcrossref{\sphinxstyleliteralemphasis{\sphinxupquote{Image}}}}}) \textendash{} \sphinxcode{\sphinxupquote{Image}} object to convert to list

\item[{Returns}] \leavevmode
List

\item[{Return type}] \leavevmode
list

\end{description}\end{quote}

\end{fulllineitems}

\index{output\_path() (polo.utils.io\_utils.RunCsvWriter property)@\spxentry{output\_path()}\spxextra{polo.utils.io\_utils.RunCsvWriter property}}

\begin{fulllineitems}
\phantomsection\label{\detokenize{polo.utils:polo.utils.io_utils.RunCsvWriter.output_path}}\pysigline{\sphinxbfcode{\sphinxupquote{property }}\sphinxbfcode{\sphinxupquote{output\_path}}}
Get the hidden attribute \sphinxtitleref{\_output\_path}.
\begin{quote}\begin{description}
\item[{Returns}] \leavevmode
Output path

\item[{Return type}] \leavevmode
str

\end{description}\end{quote}

\end{fulllineitems}

\index{write\_csv() (polo.utils.io\_utils.RunCsvWriter method)@\spxentry{write\_csv()}\spxextra{polo.utils.io\_utils.RunCsvWriter method}}

\begin{fulllineitems}
\phantomsection\label{\detokenize{polo.utils:polo.utils.io_utils.RunCsvWriter.write_csv}}\pysiglinewithargsret{\sphinxbfcode{\sphinxupquote{write\_csv}}}{}{}
Write the \sphinxcode{\sphinxupquote{Run}} object referenced by the 
\sphinxcode{\sphinxupquote{run}} 
attribute as a csv file to the location specified
by the  {\hyperref[\detokenize{polo.utils:polo.utils.io_utils.RunCsvWriter.output_path}]{\sphinxcrossref{\sphinxcode{\sphinxupquote{output\_path}}}}}
attribute.
\begin{quote}\begin{description}
\item[{Returns}] \leavevmode
True, if csv file content was written successfully,
return error thrown otherwise.

\item[{Return type}] \leavevmode
Bool or Exception

\end{description}\end{quote}

\end{fulllineitems}


\end{fulllineitems}

\index{RunDeserializer (class in polo.utils.io\_utils)@\spxentry{RunDeserializer}\spxextra{class in polo.utils.io\_utils}}

\begin{fulllineitems}
\phantomsection\label{\detokenize{polo.utils:polo.utils.io_utils.RunDeserializer}}\pysiglinewithargsret{\sphinxbfcode{\sphinxupquote{class }}\sphinxcode{\sphinxupquote{polo.utils.io\_utils.}}\sphinxbfcode{\sphinxupquote{RunDeserializer}}}{\emph{\DUrole{n}{xtal\_path}}}{}
Bases: \sphinxcode{\sphinxupquote{object}}
\index{clean\_base64\_string() (polo.utils.io\_utils.RunDeserializer static method)@\spxentry{clean\_base64\_string()}\spxextra{polo.utils.io\_utils.RunDeserializer static method}}

\begin{fulllineitems}
\phantomsection\label{\detokenize{polo.utils:polo.utils.io_utils.RunDeserializer.clean_base64_string}}\pysiglinewithargsret{\sphinxbfcode{\sphinxupquote{static }}\sphinxbfcode{\sphinxupquote{clean\_base64\_string}}}{\emph{string}, \emph{out\_fmt=\textless{}class \textquotesingle{}bytes\textquotesingle{}\textgreater{}}}{}
Image instances may contain byte strings that store their actual
crystallization image encoded as base64. Previously, these byte strings
were written directly into the json file as strings causing the b’
byte string identifier to be written along with the actual base64 data.
This method removes those artifacts if they are present and returns a
clean byte string with only the actual base64 data.
\begin{quote}\begin{description}
\item[{Parameters}] \leavevmode
\sphinxstyleliteralstrong{\sphinxupquote{string}} (\sphinxstyleliteralemphasis{\sphinxupquote{str}}) \textendash{} a string to interrogate

\item[{Returns}] \leavevmode
byte string with non\sphinxhyphen{}data artifacts removed

\item[{Return type}] \leavevmode
bytes

\end{description}\end{quote}

\end{fulllineitems}

\index{dict\_to\_obj() (polo.utils.io\_utils.RunDeserializer static method)@\spxentry{dict\_to\_obj()}\spxextra{polo.utils.io\_utils.RunDeserializer static method}}

\begin{fulllineitems}
\phantomsection\label{\detokenize{polo.utils:polo.utils.io_utils.RunDeserializer.dict_to_obj}}\pysiglinewithargsret{\sphinxbfcode{\sphinxupquote{static }}\sphinxbfcode{\sphinxupquote{dict\_to\_obj}}}{\emph{\DUrole{n}{d}}}{}
Opposite of the obj\_to\_dict method in XtalWriter class, this method
takes a dictionary instance that has been previously serialized and
attempts to convert it back into an object instance. Used as the
\sphinxtitleref{object\_hook} argument when calling \sphinxtitleref{json.loads} to read xtal files.
\begin{quote}\begin{description}
\item[{Parameters}] \leavevmode
\sphinxstyleliteralstrong{\sphinxupquote{d}} (\sphinxstyleliteralemphasis{\sphinxupquote{dict}}) \textendash{} dictionary to convert back to object

\item[{Returns}] \leavevmode
an object

\item[{Return type}] \leavevmode
object

\end{description}\end{quote}

\end{fulllineitems}

\index{make\_read\_xtal\_thread() (polo.utils.io\_utils.RunDeserializer method)@\spxentry{make\_read\_xtal\_thread()}\spxextra{polo.utils.io\_utils.RunDeserializer method}}

\begin{fulllineitems}
\phantomsection\label{\detokenize{polo.utils:polo.utils.io_utils.RunDeserializer.make_read_xtal_thread}}\pysiglinewithargsret{\sphinxbfcode{\sphinxupquote{make\_read\_xtal\_thread}}}{}{}
\end{fulllineitems}

\index{xtal\_header\_reader() (polo.utils.io\_utils.RunDeserializer method)@\spxentry{xtal\_header\_reader()}\spxextra{polo.utils.io\_utils.RunDeserializer method}}

\begin{fulllineitems}
\phantomsection\label{\detokenize{polo.utils:polo.utils.io_utils.RunDeserializer.xtal_header_reader}}\pysiglinewithargsret{\sphinxbfcode{\sphinxupquote{xtal\_header\_reader}}}{\emph{\DUrole{n}{xtal\_file\_io}}}{}
Reads the header section of an open xtal file. Should always be
called before reading the json content of an xtal file. Note than
xtal files must always have a line of equal signs before the json
content even if there is no header content otherwise this method will
read one line into the json content causing the json reader to
throw an error.
\begin{quote}\begin{description}
\item[{Parameters}] \leavevmode
\sphinxstyleliteralstrong{\sphinxupquote{xtal\_file\_io}} (\sphinxstyleliteralemphasis{\sphinxupquote{TextIoWrapper}}) \textendash{} xtal file currently being read

\item[{Returns}] \leavevmode
xtal header contents

\item[{Return type}] \leavevmode
list

\end{description}\end{quote}

\end{fulllineitems}

\index{xtal\_to\_run() (polo.utils.io\_utils.RunDeserializer method)@\spxentry{xtal\_to\_run()}\spxextra{polo.utils.io\_utils.RunDeserializer method}}

\begin{fulllineitems}
\phantomsection\label{\detokenize{polo.utils:polo.utils.io_utils.RunDeserializer.xtal_to_run}}\pysiglinewithargsret{\sphinxbfcode{\sphinxupquote{xtal\_to\_run}}}{\emph{\DUrole{o}{**}\DUrole{n}{kwargs}}}{}
Attempt to convert the file specified by the path stored in the
\sphinxcode{\sphinxupquote{xtal\_path}}
attribute to a \sphinxcode{\sphinxupquote{Run}} object.
\begin{quote}\begin{description}
\item[{Returns}] \leavevmode
Run object encoded by an xtal file

\item[{Return type}] \leavevmode
{\hyperref[\detokenize{polo.crystallography:polo.crystallography.run.Run}]{\sphinxcrossref{Run}}}

\end{description}\end{quote}

\end{fulllineitems}

\index{xtal\_to\_run\_on\_thread() (polo.utils.io\_utils.RunDeserializer method)@\spxentry{xtal\_to\_run\_on\_thread()}\spxextra{polo.utils.io\_utils.RunDeserializer method}}

\begin{fulllineitems}
\phantomsection\label{\detokenize{polo.utils:polo.utils.io_utils.RunDeserializer.xtal_to_run_on_thread}}\pysiglinewithargsret{\sphinxbfcode{\sphinxupquote{xtal\_to\_run\_on\_thread}}}{}{}
Wrapper method around \sphinxtitleref{xtal\_to\_run} method. Does the exact same thing
except creates a \sphinxtitleref{QuickThread} instance and runs \sphinxtitleref{xtal\_to\_run} on the
thread. When finished adds the newly created run to the main window’s
loaded\_run dictionary to signify that the run has been loaded and is
ready for further operations.

\end{fulllineitems}


\end{fulllineitems}

\index{RunImporter (class in polo.utils.io\_utils)@\spxentry{RunImporter}\spxextra{class in polo.utils.io\_utils}}

\begin{fulllineitems}
\phantomsection\label{\detokenize{polo.utils:polo.utils.io_utils.RunImporter}}\pysigline{\sphinxbfcode{\sphinxupquote{class }}\sphinxcode{\sphinxupquote{polo.utils.io\_utils.}}\sphinxbfcode{\sphinxupquote{RunImporter}}}
Bases: \sphinxcode{\sphinxupquote{object}}

Class to hold general use methods for importing runs into Polo.
\index{crack\_open\_a\_rar\_one() (polo.utils.io\_utils.RunImporter static method)@\spxentry{crack\_open\_a\_rar\_one()}\spxextra{polo.utils.io\_utils.RunImporter static method}}

\begin{fulllineitems}
\phantomsection\label{\detokenize{polo.utils:polo.utils.io_utils.RunImporter.crack_open_a_rar_one}}\pysiglinewithargsret{\sphinxbfcode{\sphinxupquote{static }}\sphinxbfcode{\sphinxupquote{crack\_open\_a\_rar\_one}}}{\emph{\DUrole{n}{rar\_path}}}{}
Method to open a compressed rar archive.
\begin{quote}\begin{description}
\item[{Parameters}] \leavevmode
\sphinxstyleliteralstrong{\sphinxupquote{rar\_path}} (\sphinxstyleliteralemphasis{\sphinxupquote{str}}\sphinxstyleliteralemphasis{\sphinxupquote{ or }}\sphinxstyleliteralemphasis{\sphinxupquote{Path}}) \textendash{} Path to rar archive.

\item[{Returns}] \leavevmode
Path to uncompressed archive if successful

\item[{Return type}] \leavevmode
Path

\end{description}\end{quote}

\end{fulllineitems}

\index{directory\_validator() (polo.utils.io\_utils.RunImporter static method)@\spxentry{directory\_validator()}\spxextra{polo.utils.io\_utils.RunImporter static method}}

\begin{fulllineitems}
\phantomsection\label{\detokenize{polo.utils:polo.utils.io_utils.RunImporter.directory_validator}}\pysiglinewithargsret{\sphinxbfcode{\sphinxupquote{static }}\sphinxbfcode{\sphinxupquote{directory\_validator}}}{\emph{\DUrole{n}{dir\_path}}}{}
Check if a directory should proceed further down the import
pipeline. Includes checks to make sure the directory exists,
is a directory and that the directory contains images of
filetypes that can be imported.
\begin{quote}\begin{description}
\item[{Parameters}] \leavevmode
\sphinxstyleliteralstrong{\sphinxupquote{dir\_path}} (\sphinxstyleliteralemphasis{\sphinxupquote{str}}\sphinxstyleliteralemphasis{\sphinxupquote{ or }}\sphinxstyleliteralemphasis{\sphinxupquote{Path}}) \textendash{} Path to a directory

\item[{Returns}] \leavevmode
True, if directory can be imported, an Exception otherwise

\item[{Return type}] \leavevmode
bool or Exception

\end{description}\end{quote}

\end{fulllineitems}

\index{import\_from\_xtal\_thread() (polo.utils.io\_utils.RunImporter static method)@\spxentry{import\_from\_xtal\_thread()}\spxextra{polo.utils.io\_utils.RunImporter static method}}

\begin{fulllineitems}
\phantomsection\label{\detokenize{polo.utils:polo.utils.io_utils.RunImporter.import_from_xtal_thread}}\pysiglinewithargsret{\sphinxbfcode{\sphinxupquote{static }}\sphinxbfcode{\sphinxupquote{import\_from\_xtal\_thread}}}{\emph{\DUrole{n}{xtal\_path}}}{}
Given the path to an xtal file returns a \sphinxcode{\sphinxupquote{QuickThread}}
which can be run to load the run serialized in the
xtal file.
\begin{quote}\begin{description}
\item[{Parameters}] \leavevmode
\sphinxstyleliteralstrong{\sphinxupquote{xtal\_path}} (\sphinxstyleliteralemphasis{\sphinxupquote{str}}) \textendash{} File path to xtal file.

\item[{Returns}] \leavevmode
QuickThread for deserializing the given xtal file.

\item[{Return type}] \leavevmode
{\hyperref[\detokenize{polo.threads:polo.threads.thread.QuickThread}]{\sphinxcrossref{QuickThread}}}

\end{description}\end{quote}

\end{fulllineitems}

\index{import\_general\_run() (polo.utils.io\_utils.RunImporter static method)@\spxentry{import\_general\_run()}\spxextra{polo.utils.io\_utils.RunImporter static method}}

\begin{fulllineitems}
\phantomsection\label{\detokenize{polo.utils:polo.utils.io_utils.RunImporter.import_general_run}}\pysiglinewithargsret{\sphinxbfcode{\sphinxupquote{static }}\sphinxbfcode{\sphinxupquote{import\_general\_run}}}{\emph{\DUrole{n}{data\_dir}}, \emph{\DUrole{o}{**}\DUrole{n}{kwargs}}}{}
Attempt to import a \sphinxcode{\sphinxupquote{Run}} from a directory of images.
\begin{quote}\begin{description}
\item[{Parameters}] \leavevmode
\sphinxstyleliteralstrong{\sphinxupquote{data\_dir}} (\sphinxstyleliteralemphasis{\sphinxupquote{{[}}}\sphinxstyleliteralemphasis{\sphinxupquote{type}}\sphinxstyleliteralemphasis{\sphinxupquote{{]}}}) \textendash{} {[}description{]}

\item[{Returns}] \leavevmode
{[}description{]}

\item[{Return type}] \leavevmode
{[}type{]}

\end{description}\end{quote}

\end{fulllineitems}

\index{import\_hwi\_run() (polo.utils.io\_utils.RunImporter static method)@\spxentry{import\_hwi\_run()}\spxextra{polo.utils.io\_utils.RunImporter static method}}

\begin{fulllineitems}
\phantomsection\label{\detokenize{polo.utils:polo.utils.io_utils.RunImporter.import_hwi_run}}\pysiglinewithargsret{\sphinxbfcode{\sphinxupquote{static }}\sphinxbfcode{\sphinxupquote{import\_hwi\_run}}}{\emph{\DUrole{n}{data\_dir}}, \emph{\DUrole{o}{**}\DUrole{n}{kwargs}}}{}
Attempt to create a \sphinxcode{\sphinxupquote{HWIRun}} from a directory of
images.
\begin{quote}\begin{description}
\item[{Parameters}] \leavevmode
\sphinxstyleliteralstrong{\sphinxupquote{data\_dir}} (\sphinxstyleliteralemphasis{\sphinxupquote{str}}\sphinxstyleliteralemphasis{\sphinxupquote{ or }}\sphinxstyleliteralemphasis{\sphinxupquote{Path}}) \textendash{} Directory to import from.

\item[{Returns}] \leavevmode
HWIRun if import is successful, False otherwise

\item[{Return type}] \leavevmode
{\hyperref[\detokenize{polo.crystallography:polo.crystallography.run.HWIRun}]{\sphinxcrossref{HWIRun}}}, bool

\end{description}\end{quote}

\end{fulllineitems}

\index{import\_run\_from\_directory() (polo.utils.io\_utils.RunImporter static method)@\spxentry{import\_run\_from\_directory()}\spxextra{polo.utils.io\_utils.RunImporter static method}}

\begin{fulllineitems}
\phantomsection\label{\detokenize{polo.utils:polo.utils.io_utils.RunImporter.import_run_from_directory}}\pysiglinewithargsret{\sphinxbfcode{\sphinxupquote{static }}\sphinxbfcode{\sphinxupquote{import\_run\_from\_directory}}}{\emph{\DUrole{n}{data\_dir}}, \emph{\DUrole{o}{**}\DUrole{n}{kwargs}}}{}
Imports a run from a local directory. First attempts to import
the run as an \sphinxcode{\sphinxupquote{HWIRun}} and if this fails attempts an import as
a general \sphinxcode{\sphinxupquote{Run}} object.
\begin{quote}\begin{description}
\item[{Parameters}] \leavevmode
\sphinxstyleliteralstrong{\sphinxupquote{data\_dir}} (\sphinxstyleliteralemphasis{\sphinxupquote{str}}\sphinxstyleliteralemphasis{\sphinxupquote{ or }}\sphinxstyleliteralemphasis{\sphinxupquote{Path}}) \textendash{} Directory to build the Run from

\item[{Returns}] \leavevmode
Run, HWIRun or False depending on directory content
and if import succeeds.

\item[{Return type}] \leavevmode
{\hyperref[\detokenize{polo.crystallography:polo.crystallography.run.Run}]{\sphinxcrossref{Run}}}, {\hyperref[\detokenize{polo.crystallography:polo.crystallography.run.HWIRun}]{\sphinxcrossref{HWIRun}}}, bool

\end{description}\end{quote}

\end{fulllineitems}

\index{make\_xtal\_file\_dialog() (polo.utils.io\_utils.RunImporter static method)@\spxentry{make\_xtal\_file\_dialog()}\spxextra{polo.utils.io\_utils.RunImporter static method}}

\begin{fulllineitems}
\phantomsection\label{\detokenize{polo.utils:polo.utils.io_utils.RunImporter.make_xtal_file_dialog}}\pysiglinewithargsret{\sphinxbfcode{\sphinxupquote{static }}\sphinxbfcode{\sphinxupquote{make\_xtal\_file\_dialog}}}{\emph{\DUrole{n}{parent}\DUrole{o}{=}\DUrole{default_value}{None}}}{}
Create a file dialog specifically for browsing for
xtal files.
\begin{quote}\begin{description}
\item[{Parameters}] \leavevmode
\sphinxstyleliteralstrong{\sphinxupquote{parent}} (\sphinxstyleliteralemphasis{\sphinxupquote{QDialog}}\sphinxstyleliteralemphasis{\sphinxupquote{, }}\sphinxstyleliteralemphasis{\sphinxupquote{optional}}) \textendash{} Parent for the file dialog, defaults to None

\item[{Returns}] \leavevmode
QFileDialog

\item[{Return type}] \leavevmode
QFileDialog

\end{description}\end{quote}

\end{fulllineitems}

\index{parse\_hwi\_dir\_metadata() (polo.utils.io\_utils.RunImporter static method)@\spxentry{parse\_hwi\_dir\_metadata()}\spxextra{polo.utils.io\_utils.RunImporter static method}}

\begin{fulllineitems}
\phantomsection\label{\detokenize{polo.utils:polo.utils.io_utils.RunImporter.parse_hwi_dir_metadata}}\pysiglinewithargsret{\sphinxbfcode{\sphinxupquote{static }}\sphinxbfcode{\sphinxupquote{parse\_hwi\_dir\_metadata}}}{\emph{\DUrole{n}{dir\_name}}}{}
Parse the directory name of an \sphinxcode{\sphinxupquote{HWIRun}} directory to pull out
metadata. If the directory name conforms to HWI naming conventions
the function should be able to return the image spectrum, the plate id,
the date and the run name. If the directory does not conform to HWI
naming standards this will cause an exception which is caught and the
function will return None.

If the parse is successful will return a dictionary with the following
format.
\begin{quote}

\begin{sphinxVerbatim}[commandchars=\\\{\}]
\PYGZob{}\PYGZsq{}image\PYGZus{}spectrum\PYGZsq{}: String describing the image spectrum,
\PYGZsq{}plate\PYGZus{}id\PYGZsq{}: String representing the plate id,
\PYGZsq{}date\PYGZsq{}: Datetime instance of imaging date,
\PYGZsq{}run\PYGZus{}name\PYGZsq{}: String holding the run name
\PYGZcb{}
\end{sphinxVerbatim}
\end{quote}
\begin{quote}\begin{description}
\item[{Parameters}] \leavevmode
\sphinxstyleliteralstrong{\sphinxupquote{dir\_name}} (\sphinxstyleliteralemphasis{\sphinxupquote{str}}\sphinxstyleliteralemphasis{\sphinxupquote{ or }}\sphinxstyleliteralemphasis{\sphinxupquote{Path}}) \textendash{} Name of directory to check for metadata

\item[{Returns}] \leavevmode
Dictionary of extracted metadata or None if parse fails

\item[{Return type}] \leavevmode
dict or None

\end{description}\end{quote}

\end{fulllineitems}

\index{unpack\_rar\_archive\_thread() (polo.utils.io\_utils.RunImporter static method)@\spxentry{unpack\_rar\_archive\_thread()}\spxextra{polo.utils.io\_utils.RunImporter static method}}

\begin{fulllineitems}
\phantomsection\label{\detokenize{polo.utils:polo.utils.io_utils.RunImporter.unpack_rar_archive_thread}}\pysiglinewithargsret{\sphinxbfcode{\sphinxupquote{static }}\sphinxbfcode{\sphinxupquote{unpack\_rar\_archive\_thread}}}{\emph{\DUrole{n}{archive\_path}}}{}
Create a \sphinxcode{\sphinxupquote{QuickThread}} that is setup to de\sphinxhyphen{}compress
a rar archive file.
\begin{quote}\begin{description}
\item[{Parameters}] \leavevmode
\sphinxstyleliteralstrong{\sphinxupquote{archive\_path}} (\sphinxstyleliteralemphasis{\sphinxupquote{str}}\sphinxstyleliteralemphasis{\sphinxupquote{ or }}\sphinxstyleliteralemphasis{\sphinxupquote{Path}}) \textendash{} Path to rar archive.

\item[{Returns}] \leavevmode
QuickThread that can be run to de\sphinxhyphen{}compress the given archive path.

\item[{Return type}] \leavevmode
{\hyperref[\detokenize{polo.threads:polo.threads.thread.QuickThread}]{\sphinxcrossref{QuickThread}}}

\end{description}\end{quote}

\end{fulllineitems}


\end{fulllineitems}

\index{RunLinker (class in polo.utils.io\_utils)@\spxentry{RunLinker}\spxextra{class in polo.utils.io\_utils}}

\begin{fulllineitems}
\phantomsection\label{\detokenize{polo.utils:polo.utils.io_utils.RunLinker}}\pysigline{\sphinxbfcode{\sphinxupquote{class }}\sphinxcode{\sphinxupquote{polo.utils.io\_utils.}}\sphinxbfcode{\sphinxupquote{RunLinker}}}
Bases: \sphinxcode{\sphinxupquote{object}}

Class to hold methods relating to linking runs either by
date or by spectrum.
\index{link\_runs\_by\_date() (polo.utils.io\_utils.RunLinker static method)@\spxentry{link\_runs\_by\_date()}\spxextra{polo.utils.io\_utils.RunLinker static method}}

\begin{fulllineitems}
\phantomsection\label{\detokenize{polo.utils:polo.utils.io_utils.RunLinker.link_runs_by_date}}\pysiglinewithargsret{\sphinxbfcode{\sphinxupquote{static }}\sphinxbfcode{\sphinxupquote{link\_runs\_by\_date}}}{\emph{\DUrole{n}{runs}}}{}
\end{fulllineitems}

\index{link\_runs\_by\_spectrum() (polo.utils.io\_utils.RunLinker static method)@\spxentry{link\_runs\_by\_spectrum()}\spxextra{polo.utils.io\_utils.RunLinker static method}}

\begin{fulllineitems}
\phantomsection\label{\detokenize{polo.utils:polo.utils.io_utils.RunLinker.link_runs_by_spectrum}}\pysiglinewithargsret{\sphinxbfcode{\sphinxupquote{static }}\sphinxbfcode{\sphinxupquote{link\_runs\_by\_spectrum}}}{\emph{\DUrole{n}{runs}}}{}
Link a collection of {\hyperref[\detokenize{polo.crystallography:polo.crystallography.run.HWIRun}]{\sphinxcrossref{\sphinxcode{\sphinxupquote{HWIRun}}}}} instances
together by spectrum. All non\sphinxhyphen{}visible
{\hyperref[\detokenize{polo.crystallography:polo.crystallography.run.HWIRun}]{\sphinxcrossref{\sphinxcode{\sphinxupquote{HWIRun}}}}} instances
are linked together in a monodirectional circular linked list.
Each visible {\hyperref[\detokenize{polo.crystallography:polo.crystallography.run.HWIRun}]{\sphinxcrossref{\sphinxcode{\sphinxupquote{HWIRun}}}}} instance
will then point to the same non\sphinxhyphen{}visible run through
their \sphinxcode{\sphinxupquote{alt\_spectrum}}
attribute as a way to access the non\sphinxhyphen{}visible
linked list.
\begin{quote}\begin{description}
\item[{Parameters}] \leavevmode
\sphinxstyleliteralstrong{\sphinxupquote{runs}} (\sphinxstyleliteralemphasis{\sphinxupquote{list}}) \textendash{} List of runs to link together

\item[{Returns}] \leavevmode
List of runs linked by spectrum

\item[{Return type}] \leavevmode
list

\end{description}\end{quote}

\end{fulllineitems}

\index{the\_big\_link() (polo.utils.io\_utils.RunLinker static method)@\spxentry{the\_big\_link()}\spxextra{polo.utils.io\_utils.RunLinker static method}}

\begin{fulllineitems}
\phantomsection\label{\detokenize{polo.utils:polo.utils.io_utils.RunLinker.the_big_link}}\pysiglinewithargsret{\sphinxbfcode{\sphinxupquote{static }}\sphinxbfcode{\sphinxupquote{the\_big\_link}}}{\emph{\DUrole{n}{runs}}}{}
Wrapper method to do all the linking required for a collection of
runs. First calls {\hyperref[\detokenize{polo.utils:polo.utils.io_utils.RunLinker.unlink_runs_completely}]{\sphinxcrossref{\sphinxcode{\sphinxupquote{unlink\_runs\_completely()}}}}}
to separate any existing links so things do not get tangled. Then 
calls {\hyperref[\detokenize{polo.utils:polo.utils.io_utils.RunLinker.link_runs_by_date}]{\sphinxcrossref{\sphinxcode{\sphinxupquote{link\_runs\_by\_date()}}}}} and
{\hyperref[\detokenize{polo.utils:polo.utils.io_utils.RunLinker.link_runs_by_spectrum}]{\sphinxcrossref{\sphinxcode{\sphinxupquote{link\_runs\_by\_spectrum()}}}}}.
\begin{quote}\begin{description}
\item[{Parameters}] \leavevmode
\sphinxstyleliteralstrong{\sphinxupquote{runs}} (\sphinxstyleliteralemphasis{\sphinxupquote{list}}) \textendash{} List of runs to link

\item[{Returns}] \leavevmode
List of runs with links made

\item[{Return type}] \leavevmode
list

\end{description}\end{quote}

\end{fulllineitems}

\index{unlink\_runs\_completely() (polo.utils.io\_utils.RunLinker static method)@\spxentry{unlink\_runs\_completely()}\spxextra{polo.utils.io\_utils.RunLinker static method}}

\begin{fulllineitems}
\phantomsection\label{\detokenize{polo.utils:polo.utils.io_utils.RunLinker.unlink_runs_completely}}\pysiglinewithargsret{\sphinxbfcode{\sphinxupquote{static }}\sphinxbfcode{\sphinxupquote{unlink\_runs\_completely}}}{\emph{\DUrole{n}{runs}}}{}
Cuts all links between the {\hyperref[\detokenize{polo.crystallography:polo.crystallography.run.HWIRun}]{\sphinxcrossref{\sphinxcode{\sphinxupquote{HWIRun}}}}} instances
passed through the \sphinxtitleref{runs} argument and the
{\hyperref[\detokenize{polo.crystallography:polo.crystallography.image.Image}]{\sphinxcrossref{\sphinxcode{\sphinxupquote{Image}}}}} instances
in those runs.
\begin{quote}\begin{description}
\item[{Parameters}] \leavevmode
\sphinxstyleliteralstrong{\sphinxupquote{runs}} (\sphinxstyleliteralemphasis{\sphinxupquote{list}}) \textendash{} List of runs

\item[{Returns}] \leavevmode
List of runs without any links

\item[{Return type}] \leavevmode
list

\end{description}\end{quote}

\end{fulllineitems}


\end{fulllineitems}

\index{RunSerializer (class in polo.utils.io\_utils)@\spxentry{RunSerializer}\spxextra{class in polo.utils.io\_utils}}

\begin{fulllineitems}
\phantomsection\label{\detokenize{polo.utils:polo.utils.io_utils.RunSerializer}}\pysiglinewithargsret{\sphinxbfcode{\sphinxupquote{class }}\sphinxcode{\sphinxupquote{polo.utils.io\_utils.}}\sphinxbfcode{\sphinxupquote{RunSerializer}}}{\emph{\DUrole{n}{run}}}{}
Bases: \sphinxcode{\sphinxupquote{object}}
\index{make\_thread() (polo.utils.io\_utils.RunSerializer class method)@\spxentry{make\_thread()}\spxextra{polo.utils.io\_utils.RunSerializer class method}}

\begin{fulllineitems}
\phantomsection\label{\detokenize{polo.utils:polo.utils.io_utils.RunSerializer.make_thread}}\pysiglinewithargsret{\sphinxbfcode{\sphinxupquote{classmethod }}\sphinxbfcode{\sphinxupquote{make\_thread}}}{\emph{\DUrole{n}{job\_function}}, \emph{\DUrole{o}{**}\DUrole{n}{kwargs}}}{}
Creates a new \sphinxcode{\sphinxupquote{QuickThread}} object. The job function is the
function the thread will execute and and arguments that the job
function requires should be passed has keyword arguments. These are
stored as a dictionary in the new thread object until the thread is
activated and they are passed as arguments.

\end{fulllineitems}

\index{path\_suffix\_checker() (polo.utils.io\_utils.RunSerializer static method)@\spxentry{path\_suffix\_checker()}\spxextra{polo.utils.io\_utils.RunSerializer static method}}

\begin{fulllineitems}
\phantomsection\label{\detokenize{polo.utils:polo.utils.io_utils.RunSerializer.path_suffix_checker}}\pysiglinewithargsret{\sphinxbfcode{\sphinxupquote{static }}\sphinxbfcode{\sphinxupquote{path\_suffix\_checker}}}{\emph{\DUrole{n}{path}}, \emph{\DUrole{n}{desired\_suffix}}}{}
Check is a file path has a desired suffix, if not then replace the
current suffix with the desired suffix. Useful for checking filenames
that are taken from user input.
\begin{quote}\begin{description}
\item[{Parameters}] \leavevmode
\sphinxstyleliteralstrong{\sphinxupquote{desired\_suffix}} \textendash{} File extension for given file path.

\end{description}\end{quote}

\end{fulllineitems}

\index{path\_validator() (polo.utils.io\_utils.RunSerializer static method)@\spxentry{path\_validator()}\spxextra{polo.utils.io\_utils.RunSerializer static method}}

\begin{fulllineitems}
\phantomsection\label{\detokenize{polo.utils:polo.utils.io_utils.RunSerializer.path_validator}}\pysiglinewithargsret{\sphinxbfcode{\sphinxupquote{static }}\sphinxbfcode{\sphinxupquote{path\_validator}}}{\emph{\DUrole{n}{path}}, \emph{\DUrole{n}{parent}\DUrole{o}{=}\DUrole{default_value}{False}}}{}
Tests to ensure a path exists. Passing parent = True will check for
the existence of the parent directory of the path.

\end{fulllineitems}


\end{fulllineitems}

\index{SceneExporter (class in polo.utils.io\_utils)@\spxentry{SceneExporter}\spxextra{class in polo.utils.io\_utils}}

\begin{fulllineitems}
\phantomsection\label{\detokenize{polo.utils:polo.utils.io_utils.SceneExporter}}\pysiglinewithargsret{\sphinxbfcode{\sphinxupquote{class }}\sphinxcode{\sphinxupquote{polo.utils.io\_utils.}}\sphinxbfcode{\sphinxupquote{SceneExporter}}}{\emph{\DUrole{n}{graphics\_scene}\DUrole{o}{=}\DUrole{default_value}{None}}, \emph{\DUrole{n}{file\_path}\DUrole{o}{=}\DUrole{default_value}{None}}}{}
Bases: \sphinxcode{\sphinxupquote{object}}
\index{write() (polo.utils.io\_utils.SceneExporter method)@\spxentry{write()}\spxextra{polo.utils.io\_utils.SceneExporter method}}

\begin{fulllineitems}
\phantomsection\label{\detokenize{polo.utils:polo.utils.io_utils.SceneExporter.write}}\pysiglinewithargsret{\sphinxbfcode{\sphinxupquote{write}}}{}{}
\end{fulllineitems}

\index{write\_image() (polo.utils.io\_utils.SceneExporter static method)@\spxentry{write\_image()}\spxextra{polo.utils.io\_utils.SceneExporter static method}}

\begin{fulllineitems}
\phantomsection\label{\detokenize{polo.utils:polo.utils.io_utils.SceneExporter.write_image}}\pysiglinewithargsret{\sphinxbfcode{\sphinxupquote{static }}\sphinxbfcode{\sphinxupquote{write\_image}}}{\emph{\DUrole{n}{scene}}, \emph{\DUrole{n}{file\_path}}}{}
Write the contents of a \sphinxcode{\sphinxupquote{QGraphicsScene}}
to a png file.
\begin{quote}\begin{description}
\item[{Parameters}] \leavevmode\begin{itemize}
\item {} 
\sphinxstyleliteralstrong{\sphinxupquote{scene}} (\sphinxstyleliteralemphasis{\sphinxupquote{QGraphicsScene}}) \textendash{} \sphinxcode{\sphinxupquote{QGraphicsScene}} to convert to image file.

\item {} 
\sphinxstyleliteralstrong{\sphinxupquote{file\_path}} (\sphinxstyleliteralemphasis{\sphinxupquote{str}}) \textendash{} Path to save image to.

\end{itemize}

\item[{Returns}] \leavevmode
File path to saved image if successful, Exception otherwise.

\item[{Return type}] \leavevmode
str or Exception

\end{description}\end{quote}

\end{fulllineitems}


\end{fulllineitems}

\index{XmlReader (class in polo.utils.io\_utils)@\spxentry{XmlReader}\spxextra{class in polo.utils.io\_utils}}

\begin{fulllineitems}
\phantomsection\label{\detokenize{polo.utils:polo.utils.io_utils.XmlReader}}\pysiglinewithargsret{\sphinxbfcode{\sphinxupquote{class }}\sphinxcode{\sphinxupquote{polo.utils.io\_utils.}}\sphinxbfcode{\sphinxupquote{XmlReader}}}{\emph{\DUrole{n}{xml\_path}\DUrole{o}{=}\DUrole{default_value}{None}}, \emph{\DUrole{n}{xml\_files}\DUrole{o}{=}\DUrole{default_value}{{[}{]}}}}{}
Bases: \sphinxcode{\sphinxupquote{object}}
\index{discover\_xml\_files() (polo.utils.io\_utils.XmlReader method)@\spxentry{discover\_xml\_files()}\spxextra{polo.utils.io\_utils.XmlReader method}}

\begin{fulllineitems}
\phantomsection\label{\detokenize{polo.utils:polo.utils.io_utils.XmlReader.discover_xml_files}}\pysiglinewithargsret{\sphinxbfcode{\sphinxupquote{discover\_xml\_files}}}{\emph{\DUrole{n}{parent\_dir}}}{}
Look for xml files in a given directory.
\begin{quote}\begin{description}
\item[{Parameters}] \leavevmode
\sphinxstyleliteralstrong{\sphinxupquote{parent\_dir}} (\sphinxstyleliteralemphasis{\sphinxupquote{str}}\sphinxstyleliteralemphasis{\sphinxupquote{ or }}\sphinxstyleliteralemphasis{\sphinxupquote{Path}}) \textendash{} Directory to look for xml files in

\item[{Returns}] \leavevmode
List of xml file paths, if any exist

\item[{Return type}] \leavevmode
list

\end{description}\end{quote}

\end{fulllineitems}

\index{find\_and\_read\_plate\_data() (polo.utils.io\_utils.XmlReader method)@\spxentry{find\_and\_read\_plate\_data()}\spxextra{polo.utils.io\_utils.XmlReader method}}

\begin{fulllineitems}
\phantomsection\label{\detokenize{polo.utils:polo.utils.io_utils.XmlReader.find_and_read_plate_data}}\pysiglinewithargsret{\sphinxbfcode{\sphinxupquote{find\_and\_read\_plate\_data}}}{\emph{\DUrole{n}{parent\_dir}}}{}
Find xml metadata files in a given directory. Read the
data from xml files that contain the \sphinxcode{\sphinxupquote{plate\_def}} key
string.
\begin{quote}\begin{description}
\item[{Parameters}] \leavevmode
\sphinxstyleliteralstrong{\sphinxupquote{parent\_dir}} (\sphinxstyleliteralemphasis{\sphinxupquote{Path}}\sphinxstyleliteralemphasis{\sphinxupquote{ or }}\sphinxstyleliteralemphasis{\sphinxupquote{str}}) \textendash{} Directory to look for xml files

\item[{Returns}] \leavevmode
Dict if xml file found and read successfully, False otherwise

\item[{Return type}] \leavevmode
dict or bool

\end{description}\end{quote}

\end{fulllineitems}

\index{get\_data\_from\_xml\_element() (polo.utils.io\_utils.XmlReader static method)@\spxentry{get\_data\_from\_xml\_element()}\spxextra{polo.utils.io\_utils.XmlReader static method}}

\begin{fulllineitems}
\phantomsection\label{\detokenize{polo.utils:polo.utils.io_utils.XmlReader.get_data_from_xml_element}}\pysiglinewithargsret{\sphinxbfcode{\sphinxupquote{static }}\sphinxbfcode{\sphinxupquote{get\_data\_from\_xml\_element}}}{\emph{\DUrole{n}{xml\_element}}}{}
Return the data stored in an \sphinxtitleref{xml\_element}. Helper method
for reading xml files.
\begin{quote}\begin{description}
\item[{Parameters}] \leavevmode
\sphinxstyleliteralstrong{\sphinxupquote{xml\_element}} (\sphinxstyleliteralemphasis{\sphinxupquote{{[}}}\sphinxstyleliteralemphasis{\sphinxupquote{type}}\sphinxstyleliteralemphasis{\sphinxupquote{{]}}}) \textendash{} xml element to read data from

\item[{Returns}] \leavevmode
Dictionary of data stored in xml element

\item[{Return type}] \leavevmode
dict

\end{description}\end{quote}

\end{fulllineitems}

\index{platedef\_key (polo.utils.io\_utils.XmlReader attribute)@\spxentry{platedef\_key}\spxextra{polo.utils.io\_utils.XmlReader attribute}}

\begin{fulllineitems}
\phantomsection\label{\detokenize{polo.utils:polo.utils.io_utils.XmlReader.platedef_key}}\pysigline{\sphinxbfcode{\sphinxupquote{platedef\_key}}\sphinxbfcode{\sphinxupquote{ = \textquotesingle{}platedef\textquotesingle{}}}}
\end{fulllineitems}

\index{read\_plate\_data\_xml() (polo.utils.io\_utils.XmlReader method)@\spxentry{read\_plate\_data\_xml()}\spxextra{polo.utils.io\_utils.XmlReader method}}

\begin{fulllineitems}
\phantomsection\label{\detokenize{polo.utils:polo.utils.io_utils.XmlReader.read_plate_data_xml}}\pysiglinewithargsret{\sphinxbfcode{\sphinxupquote{read\_plate\_data\_xml}}}{\emph{\DUrole{n}{xml\_path}\DUrole{o}{=}\DUrole{default_value}{None}}}{}
Read the data stored in an xml document. HWI includes metadata
about samples, imaging dates and other plate information in each
rar archive that is distrubted. This method is used to read that
data so it can be incorporated into \sphinxcode{\sphinxupquote{HWIRun}} objects.
\begin{quote}\begin{description}
\item[{Parameters}] \leavevmode
\sphinxstyleliteralstrong{\sphinxupquote{xml\_path}} (\sphinxstyleliteralemphasis{\sphinxupquote{str}}\sphinxstyleliteralemphasis{\sphinxupquote{ or }}\sphinxstyleliteralemphasis{\sphinxupquote{Path}}\sphinxstyleliteralemphasis{\sphinxupquote{, }}\sphinxstyleliteralemphasis{\sphinxupquote{optional}}) \textendash{} Path to xml file to read, defaults to None.
If None uses the xml path stored in \sphinxtitleref{xml\_path} attribute.

\item[{Returns}] \leavevmode
Dictionary of xml data if read was successful, Exception otherwise

\item[{Return type}] \leavevmode
dict or Exception

\end{description}\end{quote}

\end{fulllineitems}


\end{fulllineitems}

\index{XtalWriter (class in polo.utils.io\_utils)@\spxentry{XtalWriter}\spxextra{class in polo.utils.io\_utils}}

\begin{fulllineitems}
\phantomsection\label{\detokenize{polo.utils:polo.utils.io_utils.XtalWriter}}\pysiglinewithargsret{\sphinxbfcode{\sphinxupquote{class }}\sphinxcode{\sphinxupquote{polo.utils.io\_utils.}}\sphinxbfcode{\sphinxupquote{XtalWriter}}}{\emph{\DUrole{n}{run}}, \emph{\DUrole{n}{main\_window}}, \emph{\DUrole{o}{**}\DUrole{n}{kwargs}}}{}
Bases: {\hyperref[\detokenize{polo.utils:polo.utils.io_utils.RunSerializer}]{\sphinxcrossref{\sphinxcode{\sphinxupquote{polo.utils.io\_utils.RunSerializer}}}}}
\index{clean\_run\_for\_save() (polo.utils.io\_utils.XtalWriter static method)@\spxentry{clean\_run\_for\_save()}\spxextra{polo.utils.io\_utils.XtalWriter static method}}

\begin{fulllineitems}
\phantomsection\label{\detokenize{polo.utils:polo.utils.io_utils.XtalWriter.clean_run_for_save}}\pysiglinewithargsret{\sphinxbfcode{\sphinxupquote{static }}\sphinxbfcode{\sphinxupquote{clean\_run\_for\_save}}}{\emph{\DUrole{n}{run}}}{}
Remove circular references from the run passed through the \sphinxtitleref{run}
argument to avoid issues when writing to json files.
\begin{quote}\begin{description}
\item[{Parameters}] \leavevmode
\sphinxstyleliteralstrong{\sphinxupquote{run}} ({\hyperref[\detokenize{polo.crystallography:polo.crystallography.run.Run}]{\sphinxcrossref{\sphinxstyleliteralemphasis{\sphinxupquote{Run}}}}}\sphinxstyleliteralemphasis{\sphinxupquote{ or }}{\hyperref[\detokenize{polo.crystallography:polo.crystallography.run.HWIRun}]{\sphinxcrossref{\sphinxstyleliteralemphasis{\sphinxupquote{HWIRun}}}}}) \textendash{} Run to clean (remove circular references)

\item[{Returns}] \leavevmode
Run, free from circular references

\item[{Return type}] \leavevmode
{\hyperref[\detokenize{polo.crystallography:polo.crystallography.run.Run}]{\sphinxcrossref{Run}}} or {\hyperref[\detokenize{polo.crystallography:polo.crystallography.run.HWIRun}]{\sphinxcrossref{HWIRun}}}

\end{description}\end{quote}

\end{fulllineitems}

\index{file\_ext (polo.utils.io\_utils.XtalWriter attribute)@\spxentry{file\_ext}\spxextra{polo.utils.io\_utils.XtalWriter attribute}}

\begin{fulllineitems}
\phantomsection\label{\detokenize{polo.utils:polo.utils.io_utils.XtalWriter.file_ext}}\pysigline{\sphinxbfcode{\sphinxupquote{file\_ext}}\sphinxbfcode{\sphinxupquote{ = \textquotesingle{}.xtal\textquotesingle{}}}}
\end{fulllineitems}

\index{finished\_save() (polo.utils.io\_utils.XtalWriter method)@\spxentry{finished\_save()}\spxextra{polo.utils.io\_utils.XtalWriter method}}

\begin{fulllineitems}
\phantomsection\label{\detokenize{polo.utils:polo.utils.io_utils.XtalWriter.finished_save}}\pysiglinewithargsret{\sphinxbfcode{\sphinxupquote{finished\_save}}}{}{}
\end{fulllineitems}

\index{header\_flag (polo.utils.io\_utils.XtalWriter attribute)@\spxentry{header\_flag}\spxextra{polo.utils.io\_utils.XtalWriter attribute}}

\begin{fulllineitems}
\phantomsection\label{\detokenize{polo.utils:polo.utils.io_utils.XtalWriter.header_flag}}\pysigline{\sphinxbfcode{\sphinxupquote{header\_flag}}\sphinxbfcode{\sphinxupquote{ = \textquotesingle{}\textless{}\textgreater{}\textquotesingle{}}}}
\end{fulllineitems}

\index{header\_line (polo.utils.io\_utils.XtalWriter attribute)@\spxentry{header\_line}\spxextra{polo.utils.io\_utils.XtalWriter attribute}}

\begin{fulllineitems}
\phantomsection\label{\detokenize{polo.utils:polo.utils.io_utils.XtalWriter.header_line}}\pysigline{\sphinxbfcode{\sphinxupquote{header\_line}}\sphinxbfcode{\sphinxupquote{ = \textquotesingle{}\{\}\{\}:\{\}\textbackslash{}n\textquotesingle{}}}}
\end{fulllineitems}

\index{json\_encoder() (polo.utils.io\_utils.XtalWriter static method)@\spxentry{json\_encoder()}\spxextra{polo.utils.io\_utils.XtalWriter static method}}

\begin{fulllineitems}
\phantomsection\label{\detokenize{polo.utils:polo.utils.io_utils.XtalWriter.json_encoder}}\pysiglinewithargsret{\sphinxbfcode{\sphinxupquote{static }}\sphinxbfcode{\sphinxupquote{json\_encoder}}}{\emph{\DUrole{n}{obj}}}{}
Use instead of the default json encoder when writing an xtal file. 
If the encoded object is from a module within Polo will include a 
module and class identifier so it can be more easily deserialized 
when loaded back into the program.
\begin{quote}\begin{description}
\item[{Param}] \leavevmode
obj: An object to serialize to json.

\item[{Returns}] \leavevmode
A dictionary or string version of the passed object

\end{description}\end{quote}

\end{fulllineitems}

\index{run\_to\_dict() (polo.utils.io\_utils.XtalWriter method)@\spxentry{run\_to\_dict()}\spxextra{polo.utils.io\_utils.XtalWriter method}}

\begin{fulllineitems}
\phantomsection\label{\detokenize{polo.utils:polo.utils.io_utils.XtalWriter.run_to_dict}}\pysiglinewithargsret{\sphinxbfcode{\sphinxupquote{run\_to\_dict}}}{}{}
Create a json string from the run stored in the run attribute.
\begin{quote}\begin{description}
\item[{Returns}] \leavevmode
Run instance serialized to json

\item[{Return type}] \leavevmode
str

\end{description}\end{quote}

\end{fulllineitems}

\index{write\_xtal\_file() (polo.utils.io\_utils.XtalWriter method)@\spxentry{write\_xtal\_file()}\spxextra{polo.utils.io\_utils.XtalWriter method}}

\begin{fulllineitems}
\phantomsection\label{\detokenize{polo.utils:polo.utils.io_utils.XtalWriter.write_xtal_file}}\pysiglinewithargsret{\sphinxbfcode{\sphinxupquote{write\_xtal\_file}}}{\emph{\DUrole{n}{output\_path}\DUrole{o}{=}\DUrole{default_value}{None}}}{}
Method to serialize run object to xtal file format.
\begin{quote}\begin{description}
\item[{Parameters}] \leavevmode
\sphinxstyleliteralstrong{\sphinxupquote{output\_path}} (\sphinxstyleliteralemphasis{\sphinxupquote{str}}) \textendash{} Xtal file path

\item[{Returns}] \leavevmode
path to xtal file

\item[{Return type}] \leavevmode
str

\end{description}\end{quote}

\end{fulllineitems}

\index{write\_xtal\_file\_on\_thread() (polo.utils.io\_utils.XtalWriter method)@\spxentry{write\_xtal\_file\_on\_thread()}\spxextra{polo.utils.io\_utils.XtalWriter method}}

\begin{fulllineitems}
\phantomsection\label{\detokenize{polo.utils:polo.utils.io_utils.XtalWriter.write_xtal_file_on_thread}}\pysiglinewithargsret{\sphinxbfcode{\sphinxupquote{write\_xtal\_file\_on\_thread}}}{\emph{\DUrole{n}{output\_path}}}{}
Wrapper method around {\hyperref[\detokenize{polo.utils:polo.utils.io_utils.XtalWriter.write_xtal_file}]{\sphinxcrossref{\sphinxcode{\sphinxupquote{write\_xtal\_file()}}}}} 
that executes on a \sphinxcode{\sphinxupquote{QuickThread}} instance to prevent freezing the 
GUI when saving large xtal files
\begin{quote}\begin{description}
\item[{Parameters}] \leavevmode
\sphinxstyleliteralstrong{\sphinxupquote{output\_path}} (\sphinxstyleliteralemphasis{\sphinxupquote{str}}) \textendash{} Path to xtal file

\end{description}\end{quote}

\end{fulllineitems}

\index{xtal\_header() (polo.utils.io\_utils.XtalWriter property)@\spxentry{xtal\_header()}\spxextra{polo.utils.io\_utils.XtalWriter property}}

\begin{fulllineitems}
\phantomsection\label{\detokenize{polo.utils:polo.utils.io_utils.XtalWriter.xtal_header}}\pysigline{\sphinxbfcode{\sphinxupquote{property }}\sphinxbfcode{\sphinxupquote{xtal\_header}}}
Creates the header for an xtal file when called. Header lines are
indicated as such by the string in the header\_line constant,
which should be ‘\textless{}\textgreater{}’. The last line of the header will be a row
of equal signs and then the actual json content begins on the
next line.

\end{fulllineitems}


\end{fulllineitems}

\index{check\_for\_missing\_images() (in module polo.utils.io\_utils)@\spxentry{check\_for\_missing\_images()}\spxextra{in module polo.utils.io\_utils}}

\begin{fulllineitems}
\phantomsection\label{\detokenize{polo.utils:polo.utils.io_utils.check_for_missing_images}}\pysiglinewithargsret{\sphinxcode{\sphinxupquote{polo.utils.io\_utils.}}\sphinxbfcode{\sphinxupquote{check\_for\_missing\_images}}}{\emph{\DUrole{n}{dir\_path}}, \emph{\DUrole{n}{expected\_image\_count}}}{}
\end{fulllineitems}

\index{directory\_validator() (in module polo.utils.io\_utils)@\spxentry{directory\_validator()}\spxextra{in module polo.utils.io\_utils}}

\begin{fulllineitems}
\phantomsection\label{\detokenize{polo.utils:polo.utils.io_utils.directory_validator}}\pysiglinewithargsret{\sphinxcode{\sphinxupquote{polo.utils.io\_utils.}}\sphinxbfcode{\sphinxupquote{directory\_validator}}}{\emph{\DUrole{n}{dir\_path}}}{}
\end{fulllineitems}

\index{if\_dir\_not\_exists\_make() (in module polo.utils.io\_utils)@\spxentry{if\_dir\_not\_exists\_make()}\spxextra{in module polo.utils.io\_utils}}

\begin{fulllineitems}
\phantomsection\label{\detokenize{polo.utils:polo.utils.io_utils.if_dir_not_exists_make}}\pysiglinewithargsret{\sphinxcode{\sphinxupquote{polo.utils.io\_utils.}}\sphinxbfcode{\sphinxupquote{if\_dir\_not\_exists\_make}}}{\emph{\DUrole{n}{parent\_dir}}, \emph{\DUrole{n}{child\_dir}\DUrole{o}{=}\DUrole{default_value}{None}}}{}
If only parent\_dir is given attempts to make that directory. If parent
and child are given tries to make a directory child\_dir within parent dir.

\end{fulllineitems}

\index{list\_dir\_abs() (in module polo.utils.io\_utils)@\spxentry{list\_dir\_abs()}\spxextra{in module polo.utils.io\_utils}}

\begin{fulllineitems}
\phantomsection\label{\detokenize{polo.utils:polo.utils.io_utils.list_dir_abs}}\pysiglinewithargsret{\sphinxcode{\sphinxupquote{polo.utils.io\_utils.}}\sphinxbfcode{\sphinxupquote{list\_dir\_abs}}}{\emph{\DUrole{n}{parent\_dir}}, \emph{\DUrole{n}{allowed}\DUrole{o}{=}\DUrole{default_value}{False}}}{}
\end{fulllineitems}

\index{parse\_HWI\_filename\_meta() (in module polo.utils.io\_utils)@\spxentry{parse\_HWI\_filename\_meta()}\spxextra{in module polo.utils.io\_utils}}

\begin{fulllineitems}
\phantomsection\label{\detokenize{polo.utils:polo.utils.io_utils.parse_HWI_filename_meta}}\pysiglinewithargsret{\sphinxcode{\sphinxupquote{polo.utils.io\_utils.}}\sphinxbfcode{\sphinxupquote{parse\_HWI\_filename\_meta}}}{\emph{\DUrole{n}{HWI\_image\_file}}}{}
HWI images have a standard file nameing schema that gives info about when
they are taken and well number and that kind of thing. This function returns
that data

\end{fulllineitems}

\index{parse\_hwi\_dir\_metadata() (in module polo.utils.io\_utils)@\spxentry{parse\_hwi\_dir\_metadata()}\spxextra{in module polo.utils.io\_utils}}

\begin{fulllineitems}
\phantomsection\label{\detokenize{polo.utils:polo.utils.io_utils.parse_hwi_dir_metadata}}\pysiglinewithargsret{\sphinxcode{\sphinxupquote{polo.utils.io\_utils.}}\sphinxbfcode{\sphinxupquote{parse\_hwi\_dir\_metadata}}}{\emph{\DUrole{n}{dir\_name}}}{}
\end{fulllineitems}

\index{run\_name\_validator() (in module polo.utils.io\_utils)@\spxentry{run\_name\_validator()}\spxextra{in module polo.utils.io\_utils}}

\begin{fulllineitems}
\phantomsection\label{\detokenize{polo.utils:polo.utils.io_utils.run_name_validator}}\pysiglinewithargsret{\sphinxcode{\sphinxupquote{polo.utils.io\_utils.}}\sphinxbfcode{\sphinxupquote{run\_name\_validator}}}{\emph{\DUrole{n}{new\_run\_name}}, \emph{\DUrole{n}{current\_run\_names}}}{}
\end{fulllineitems}

\index{write\_screen\_html() (in module polo.utils.io\_utils)@\spxentry{write\_screen\_html()}\spxextra{in module polo.utils.io\_utils}}

\begin{fulllineitems}
\phantomsection\label{\detokenize{polo.utils:polo.utils.io_utils.write_screen_html}}\pysiglinewithargsret{\sphinxcode{\sphinxupquote{polo.utils.io\_utils.}}\sphinxbfcode{\sphinxupquote{write\_screen\_html}}}{\emph{\DUrole{n}{plate\_list}}, \emph{\DUrole{n}{well\_number}}, \emph{\DUrole{n}{run\_name}}, \emph{\DUrole{n}{x\_reagent}}, \emph{\DUrole{n}{y\_reagent}}, \emph{\DUrole{n}{well\_volume}}, \emph{\DUrole{n}{output\_path}}}{}
\end{fulllineitems}



\subsubsection{polo.utils.math\_utils module}
\label{\detokenize{polo.utils:module-polo.utils.math_utils}}\label{\detokenize{polo.utils:polo-utils-math-utils-module}}\index{module@\spxentry{module}!polo.utils.math\_utils@\spxentry{polo.utils.math\_utils}}\index{polo.utils.math\_utils@\spxentry{polo.utils.math\_utils}!module@\spxentry{module}}\index{best\_aspect\_ratio() (in module polo.utils.math\_utils)@\spxentry{best\_aspect\_ratio()}\spxextra{in module polo.utils.math\_utils}}

\begin{fulllineitems}
\phantomsection\label{\detokenize{polo.utils:polo.utils.math_utils.best_aspect_ratio}}\pysiglinewithargsret{\sphinxcode{\sphinxupquote{polo.utils.math\_utils.}}\sphinxbfcode{\sphinxupquote{best\_aspect\_ratio}}}{\emph{\DUrole{n}{w}}, \emph{\DUrole{n}{h}}, \emph{\DUrole{n}{n}}}{}
\end{fulllineitems}

\index{factors() (in module polo.utils.math\_utils)@\spxentry{factors()}\spxextra{in module polo.utils.math\_utils}}

\begin{fulllineitems}
\phantomsection\label{\detokenize{polo.utils:polo.utils.math_utils.factors}}\pysiglinewithargsret{\sphinxcode{\sphinxupquote{polo.utils.math\_utils.}}\sphinxbfcode{\sphinxupquote{factors}}}{\emph{\DUrole{n}{n}}}{}
\end{fulllineitems}

\index{get\_cell\_image\_dims() (in module polo.utils.math\_utils)@\spxentry{get\_cell\_image\_dims()}\spxextra{in module polo.utils.math\_utils}}

\begin{fulllineitems}
\phantomsection\label{\detokenize{polo.utils:polo.utils.math_utils.get_cell_image_dims}}\pysiglinewithargsret{\sphinxcode{\sphinxupquote{polo.utils.math\_utils.}}\sphinxbfcode{\sphinxupquote{get\_cell\_image\_dims}}}{\emph{\DUrole{n}{w}}, \emph{\DUrole{n}{h}}, \emph{\DUrole{n}{n}}}{}
\end{fulllineitems}

\index{get\_image\_cell\_size() (in module polo.utils.math\_utils)@\spxentry{get\_image\_cell\_size()}\spxextra{in module polo.utils.math\_utils}}

\begin{fulllineitems}
\phantomsection\label{\detokenize{polo.utils:polo.utils.math_utils.get_image_cell_size}}\pysiglinewithargsret{\sphinxcode{\sphinxupquote{polo.utils.math\_utils.}}\sphinxbfcode{\sphinxupquote{get\_image\_cell\_size}}}{\emph{\DUrole{n}{cell\_aspect}}, \emph{\DUrole{n}{w}}, \emph{\DUrole{n}{h}}}{}
\end{fulllineitems}



\subsubsection{polo.utils.unrar\_utils module}
\label{\detokenize{polo.utils:module-polo.utils.unrar_utils}}\label{\detokenize{polo.utils:polo-utils-unrar-utils-module}}\index{module@\spxentry{module}!polo.utils.unrar\_utils@\spxentry{polo.utils.unrar\_utils}}\index{polo.utils.unrar\_utils@\spxentry{polo.utils.unrar\_utils}!module@\spxentry{module}}\index{test\_for\_working\_unrar() (in module polo.utils.unrar\_utils)@\spxentry{test\_for\_working\_unrar()}\spxextra{in module polo.utils.unrar\_utils}}

\begin{fulllineitems}
\phantomsection\label{\detokenize{polo.utils:polo.utils.unrar_utils.test_for_working_unrar}}\pysiglinewithargsret{\sphinxcode{\sphinxupquote{polo.utils.unrar\_utils.}}\sphinxbfcode{\sphinxupquote{test\_for\_working\_unrar}}}{\emph{\DUrole{n}{unrar\_exe}\DUrole{o}{=}\DUrole{default_value}{\textquotesingle{}unrar\textquotesingle{}}}}{}
Tests if a working unrar installation exists on the machine.
\begin{quote}\begin{description}
\item[{Parameters}] \leavevmode
\sphinxstyleliteralstrong{\sphinxupquote{unrar\_exe}} (\sphinxstyleliteralemphasis{\sphinxupquote{Path}}\sphinxstyleliteralemphasis{\sphinxupquote{ or }}\sphinxstyleliteralemphasis{\sphinxupquote{str}}\sphinxstyleliteralemphasis{\sphinxupquote{, }}\sphinxstyleliteralemphasis{\sphinxupquote{optional}}) \textendash{} Path to unrar executable file, defaults to UNRAR\_EXE

\item[{Returns}] \leavevmode
True if working installation exists, False otherwise

\item[{Return type}] \leavevmode
bool

\end{description}\end{quote}

\end{fulllineitems}

\index{unrar\_archive() (in module polo.utils.unrar\_utils)@\spxentry{unrar\_archive()}\spxextra{in module polo.utils.unrar\_utils}}

\begin{fulllineitems}
\phantomsection\label{\detokenize{polo.utils:polo.utils.unrar_utils.unrar_archive}}\pysiglinewithargsret{\sphinxcode{\sphinxupquote{polo.utils.unrar\_utils.}}\sphinxbfcode{\sphinxupquote{unrar\_archive}}}{\emph{\DUrole{n}{rar\_path}}, \emph{\DUrole{n}{target\_dir}\DUrole{o}{=}\DUrole{default_value}{None}}}{}
De\sphinxhyphen{}compress a rar archive and return the path to the
uncompressed archive if it exists. All unrar functions
including this one are dependent of their being a working 
unrar installation. Unrar is included for both Windows and Mac
operating systems but not for Linux.
\begin{quote}\begin{description}
\item[{Parameters}] \leavevmode\begin{itemize}
\item {} 
\sphinxstyleliteralstrong{\sphinxupquote{rar\_path}} (\sphinxstyleliteralemphasis{\sphinxupquote{Path}}\sphinxstyleliteralemphasis{\sphinxupquote{ or }}\sphinxstyleliteralemphasis{\sphinxupquote{str}}) \textendash{} Path to rar archive file

\item {} 
\sphinxstyleliteralstrong{\sphinxupquote{target\_dir}} (\sphinxstyleliteralemphasis{\sphinxupquote{Path}}\sphinxstyleliteralemphasis{\sphinxupquote{ or }}\sphinxstyleliteralemphasis{\sphinxupquote{str}}\sphinxstyleliteralemphasis{\sphinxupquote{, }}\sphinxstyleliteralemphasis{\sphinxupquote{optional}}) \textendash{} Location to place the unrared file, defaults to None

\end{itemize}

\item[{Returns}] \leavevmode
Path if unrar is successful, error code if unrar fails or Exception if

\end{description}\end{quote}

exception is raised in the unrar process.
:rtype: Path, str or Exception

\end{fulllineitems}



\subsubsection{Module contents}
\label{\detokenize{polo.utils:module-polo.utils}}\label{\detokenize{polo.utils:module-contents}}\index{module@\spxentry{module}!polo.utils@\spxentry{polo.utils}}\index{polo.utils@\spxentry{polo.utils}!module@\spxentry{module}}

\subsection{polo.widgets package}
\label{\detokenize{polo.widgets:polo-widgets-package}}\label{\detokenize{polo.widgets::doc}}

\subsubsection{Submodules}
\label{\detokenize{polo.widgets:submodules}}

\subsubsection{polo.widgets.file\_browser module}
\label{\detokenize{polo.widgets:module-polo.widgets.file_browser}}\label{\detokenize{polo.widgets:polo-widgets-file-browser-module}}\index{module@\spxentry{module}!polo.widgets.file\_browser@\spxentry{polo.widgets.file\_browser}}\index{polo.widgets.file\_browser@\spxentry{polo.widgets.file\_browser}!module@\spxentry{module}}\index{fileBrowser (class in polo.widgets.file\_browser)@\spxentry{fileBrowser}\spxextra{class in polo.widgets.file\_browser}}

\begin{fulllineitems}
\phantomsection\label{\detokenize{polo.widgets:polo.widgets.file_browser.fileBrowser}}\pysiglinewithargsret{\sphinxbfcode{\sphinxupquote{class }}\sphinxcode{\sphinxupquote{polo.widgets.file\_browser.}}\sphinxbfcode{\sphinxupquote{fileBrowser}}}{\emph{\DUrole{n}{parent}\DUrole{o}{=}\DUrole{default_value}{None}}}{}
Bases: \sphinxcode{\sphinxupquote{PyQt5.QtWidgets.QTreeWidget}}
\index{DATA\_INDEX (polo.widgets.file\_browser.fileBrowser attribute)@\spxentry{DATA\_INDEX}\spxextra{polo.widgets.file\_browser.fileBrowser attribute}}

\begin{fulllineitems}
\phantomsection\label{\detokenize{polo.widgets:polo.widgets.file_browser.fileBrowser.DATA_INDEX}}\pysigline{\sphinxbfcode{\sphinxupquote{DATA\_INDEX}}\sphinxbfcode{\sphinxupquote{ = 5}}}
\end{fulllineitems}

\index{DIR\_ICON (polo.widgets.file\_browser.fileBrowser attribute)@\spxentry{DIR\_ICON}\spxextra{polo.widgets.file\_browser.fileBrowser attribute}}

\begin{fulllineitems}
\phantomsection\label{\detokenize{polo.widgets:polo.widgets.file_browser.fileBrowser.DIR_ICON}}\pysigline{\sphinxbfcode{\sphinxupquote{DIR\_ICON}}\sphinxbfcode{\sphinxupquote{ = \textquotesingle{}/home/ethan/Documents/github/Marco\_Polo/src/data/images/icons/dir.png\textquotesingle{}}}}
\end{fulllineitems}

\index{FILE\_ICON (polo.widgets.file\_browser.fileBrowser attribute)@\spxentry{FILE\_ICON}\spxextra{polo.widgets.file\_browser.fileBrowser attribute}}

\begin{fulllineitems}
\phantomsection\label{\detokenize{polo.widgets:polo.widgets.file_browser.fileBrowser.FILE_ICON}}\pysigline{\sphinxbfcode{\sphinxupquote{FILE\_ICON}}\sphinxbfcode{\sphinxupquote{ = \textquotesingle{}/home/ethan/Documents/github/Marco\_Polo/src/data/images/icons/file.png\textquotesingle{}}}}
\end{fulllineitems}

\index{get\_checked\_files() (polo.widgets.file\_browser.fileBrowser method)@\spxentry{get\_checked\_files()}\spxextra{polo.widgets.file\_browser.fileBrowser method}}

\begin{fulllineitems}
\phantomsection\label{\detokenize{polo.widgets:polo.widgets.file_browser.fileBrowser.get_checked_files}}\pysiglinewithargsret{\sphinxbfcode{\sphinxupquote{get\_checked\_files}}}{\emph{\DUrole{n}{home\_dir}}}{}
Traverse the file tree and return the full paths to files that have
been selected by the user.
\begin{quote}\begin{description}
\item[{Parameters}] \leavevmode
\sphinxstyleliteralstrong{\sphinxupquote{home\_dir}} (\sphinxstyleliteralemphasis{\sphinxupquote{str}}\sphinxstyleliteralemphasis{\sphinxupquote{ or }}\sphinxstyleliteralemphasis{\sphinxupquote{Path}}) \textendash{} User’s home directory. This path is the parent of all
returned files.

\item[{Returns}] \leavevmode
List of checked Paths

\item[{Return type}] \leavevmode
list

\end{description}\end{quote}

\end{fulllineitems}

\index{grow\_tree\_using\_mlsd() (polo.widgets.file\_browser.fileBrowser method)@\spxentry{grow\_tree\_using\_mlsd()}\spxextra{polo.widgets.file\_browser.fileBrowser method}}

\begin{fulllineitems}
\phantomsection\label{\detokenize{polo.widgets:polo.widgets.file_browser.fileBrowser.grow_tree_using_mlsd}}\pysiglinewithargsret{\sphinxbfcode{\sphinxupquote{grow\_tree\_using\_mlsd}}}{\emph{\DUrole{n}{ftp}}, \emph{\DUrole{n}{home\_dir}}, \emph{\DUrole{n}{set\_checkable}\DUrole{o}{=}\DUrole{default_value}{True}}}{}
Rescursively add child nodes to the tree by traversing
a user’s home directory at a remote ftp server. Filepaths
are read using mlsd formating.
\begin{quote}\begin{description}
\item[{Parameters}] \leavevmode\begin{itemize}
\item {} 
\sphinxstyleliteralstrong{\sphinxupquote{ftp}} (\sphinxstyleliteralemphasis{\sphinxupquote{FTP}}) \textendash{} FTP object with valid connection

\item {} 
\sphinxstyleliteralstrong{\sphinxupquote{home\_dir}} (\sphinxstyleliteralemphasis{\sphinxupquote{str}}\sphinxstyleliteralemphasis{\sphinxupquote{ or }}\sphinxstyleliteralemphasis{\sphinxupquote{Path}}) \textendash{} Path to the user’s home directory

\item {} 
\sphinxstyleliteralstrong{\sphinxupquote{set\_checkable}} (\sphinxstyleliteralemphasis{\sphinxupquote{bool}}\sphinxstyleliteralemphasis{\sphinxupquote{, }}\sphinxstyleliteralemphasis{\sphinxupquote{optional}}) \textendash{} Set files and dirs to checkable, defaults to True

\end{itemize}

\end{description}\end{quote}

\end{fulllineitems}


\end{fulllineitems}



\subsubsection{polo.widgets.map\_box module}
\label{\detokenize{polo.widgets:module-polo.widgets.map_box}}\label{\detokenize{polo.widgets:polo-widgets-map-box-module}}\index{module@\spxentry{module}!polo.widgets.map\_box@\spxentry{polo.widgets.map\_box}}\index{polo.widgets.map\_box@\spxentry{polo.widgets.map\_box}!module@\spxentry{module}}\index{MapBox (class in polo.widgets.map\_box)@\spxentry{MapBox}\spxextra{class in polo.widgets.map\_box}}

\begin{fulllineitems}
\phantomsection\label{\detokenize{polo.widgets:polo.widgets.map_box.MapBox}}\pysigline{\sphinxbfcode{\sphinxupquote{class }}\sphinxcode{\sphinxupquote{polo.widgets.map\_box.}}\sphinxbfcode{\sphinxupquote{MapBox}}}
Bases: \sphinxcode{\sphinxupquote{PyQt5.QtWidgets.QComboBox}}
\index{\_MapBox\_\_init() (polo.widgets.map\_box.MapBox method)@\spxentry{\_MapBox\_\_init()}\spxextra{polo.widgets.map\_box.MapBox method}}

\begin{fulllineitems}
\phantomsection\label{\detokenize{polo.widgets:polo.widgets.map_box.MapBox._MapBox__init}}\pysiglinewithargsret{\sphinxbfcode{\sphinxupquote{\_MapBox\_\_init}}}{\emph{parent=None}, \emph{mapping=\{\}}, \emph{sorter=\textless{}function MapBox.\textless{}lambda\textgreater{}\textgreater{}}}{}
\end{fulllineitems}

\index{current\_value() (polo.widgets.map\_box.MapBox method)@\spxentry{current\_value()}\spxextra{polo.widgets.map\_box.MapBox method}}

\begin{fulllineitems}
\phantomsection\label{\detokenize{polo.widgets:polo.widgets.map_box.MapBox.current_value}}\pysiglinewithargsret{\sphinxbfcode{\sphinxupquote{current\_value}}}{}{}
\end{fulllineitems}

\index{mapping() (polo.widgets.map\_box.MapBox property)@\spxentry{mapping()}\spxextra{polo.widgets.map\_box.MapBox property}}

\begin{fulllineitems}
\phantomsection\label{\detokenize{polo.widgets:polo.widgets.map_box.MapBox.mapping}}\pysigline{\sphinxbfcode{\sphinxupquote{property }}\sphinxbfcode{\sphinxupquote{mapping}}}
\end{fulllineitems}


\end{fulllineitems}



\subsubsection{polo.widgets.optimize\_widget module}
\label{\detokenize{polo.widgets:module-polo.widgets.optimize_widget}}\label{\detokenize{polo.widgets:polo-widgets-optimize-widget-module}}\index{module@\spxentry{module}!polo.widgets.optimize\_widget@\spxentry{polo.widgets.optimize\_widget}}\index{polo.widgets.optimize\_widget@\spxentry{polo.widgets.optimize\_widget}!module@\spxentry{module}}\index{OptimizeWidget (class in polo.widgets.optimize\_widget)@\spxentry{OptimizeWidget}\spxextra{class in polo.widgets.optimize\_widget}}

\begin{fulllineitems}
\phantomsection\label{\detokenize{polo.widgets:polo.widgets.optimize_widget.OptimizeWidget}}\pysiglinewithargsret{\sphinxbfcode{\sphinxupquote{class }}\sphinxcode{\sphinxupquote{polo.widgets.optimize\_widget.}}\sphinxbfcode{\sphinxupquote{OptimizeWidget}}}{\emph{\DUrole{n}{parent}}, \emph{\DUrole{n}{run}\DUrole{o}{=}\DUrole{default_value}{None}}}{}
Bases: \sphinxcode{\sphinxupquote{PyQt5.QtWidgets.QWidget}}
\index{GRID\_ICON (polo.widgets.optimize\_widget.OptimizeWidget attribute)@\spxentry{GRID\_ICON}\spxextra{polo.widgets.optimize\_widget.OptimizeWidget attribute}}

\begin{fulllineitems}
\phantomsection\label{\detokenize{polo.widgets:polo.widgets.optimize_widget.OptimizeWidget.GRID_ICON}}\pysigline{\sphinxbfcode{\sphinxupquote{GRID\_ICON}}\sphinxbfcode{\sphinxupquote{ = \textquotesingle{}/home/ethan/Documents/github/Marco\_Polo/src/data/images/icons/grid.png\textquotesingle{}}}}
The OptimizeWidget is a primary run interface widget that allows
users to create optimization screens around the crystallization conditions
that yielded xtal hits. The concept is very similar to the program MakeTray
available from Hampton Research. Currently, users cannot specify their
own conditions and are limited to the predetermined conditions of the
{\hyperref[\detokenize{polo.utils:polo.utils.io_utils.Menu}]{\sphinxcrossref{\sphinxcode{\sphinxupquote{Menu}}}}} that was selected when the run was originally imported
into Polo. Additionally, the OptimizeWidget is only available to \sphinxcode{\sphinxupquote{HWIRun}} instances
as the cocktail to well mapping cannot be inferred for other more
general \sphinxcode{\sphinxupquote{Run}} types.
\begin{quote}\begin{description}
\item[{Parameters}] \leavevmode\begin{itemize}
\item {} 
\sphinxstyleliteralstrong{\sphinxupquote{parent}} (\sphinxstyleliteralemphasis{\sphinxupquote{QWidget}}) \textendash{} Parent Widget

\item {} 
\sphinxstyleliteralstrong{\sphinxupquote{run}} ({\hyperref[\detokenize{polo.crystallography:polo.crystallography.run.HWIRun}]{\sphinxcrossref{\sphinxstyleliteralemphasis{\sphinxupquote{HWIRun}}}}}\sphinxstyleliteralemphasis{\sphinxupquote{, }}\sphinxstyleliteralemphasis{\sphinxupquote{optional}}) \textendash{} Run to screen for hits from, defaults to None

\end{itemize}

\end{description}\end{quote}

\end{fulllineitems}

\index{HTML\_ICON (polo.widgets.optimize\_widget.OptimizeWidget attribute)@\spxentry{HTML\_ICON}\spxextra{polo.widgets.optimize\_widget.OptimizeWidget attribute}}

\begin{fulllineitems}
\phantomsection\label{\detokenize{polo.widgets:polo.widgets.optimize_widget.OptimizeWidget.HTML_ICON}}\pysigline{\sphinxbfcode{\sphinxupquote{HTML\_ICON}}\sphinxbfcode{\sphinxupquote{ = \textquotesingle{}/home/ethan/Documents/github/Marco\_Polo/src/data/images/icons/html.png\textquotesingle{}}}}
\end{fulllineitems}

\index{\_check\_for\_overflow() (polo.widgets.optimize\_widget.OptimizeWidget method)@\spxentry{\_check\_for\_overflow()}\spxextra{polo.widgets.optimize\_widget.OptimizeWidget method}}

\begin{fulllineitems}
\phantomsection\label{\detokenize{polo.widgets:polo.widgets.optimize_widget.OptimizeWidget._check_for_overflow}}\pysiglinewithargsret{\sphinxbfcode{\sphinxupquote{\_check\_for\_overflow}}}{\emph{\DUrole{n}{volume\_list}}}{}
Private method to check if the volume of \sphinxcode{\sphinxupquote{Reagent}}
instancess in a given well exceeds the total well volume. 
If an overflow is detected, return False otherwise return the volume 
of H20 that should be added 
to the well as a {\hyperref[\detokenize{polo.crystallography:polo.crystallography.cocktail.UnitValue}]{\sphinxcrossref{\sphinxcode{\sphinxupquote{UnitValue}}}}}
instance.
\begin{quote}\begin{description}
\item[{Parameters}] \leavevmode
\sphinxstyleliteralstrong{\sphinxupquote{volume\_list}} (\sphinxstyleliteralemphasis{\sphinxupquote{list}}) \textendash{} List of \sphinxtitleref{UnitValues} that consitute the contents of a
well in the optimization plate

\item[{Returns}] \leavevmode
\sphinxtitleref{UnitValue} describing the volume of water that should be
added to the well if it does not overflow in liters, 
False otherwise

\item[{Return type}] \leavevmode
{\hyperref[\detokenize{polo.crystallography:polo.crystallography.cocktail.UnitValue}]{\sphinxcrossref{UnitValue}}} or False

\end{description}\end{quote}

\end{fulllineitems}

\index{\_error\_checker() (polo.widgets.optimize\_widget.OptimizeWidget method)@\spxentry{\_error\_checker()}\spxextra{polo.widgets.optimize\_widget.OptimizeWidget method}}

\begin{fulllineitems}
\phantomsection\label{\detokenize{polo.widgets:polo.widgets.optimize_widget.OptimizeWidget._error_checker}}\pysiglinewithargsret{\sphinxbfcode{\sphinxupquote{\_error\_checker}}}{}{}
Private method to check if all widgets and attributes have allowed 
values before calculating the actual grid screen. 
Show error message if there is a conflict.

\end{fulllineitems}

\index{\_export\_screen() (polo.widgets.optimize\_widget.OptimizeWidget method)@\spxentry{\_export\_screen()}\spxextra{polo.widgets.optimize\_widget.OptimizeWidget method}}

\begin{fulllineitems}
\phantomsection\label{\detokenize{polo.widgets:polo.widgets.optimize_widget.OptimizeWidget._export_screen}}\pysiglinewithargsret{\sphinxbfcode{\sphinxupquote{\_export\_screen}}}{}{}
Private method to write the current optimization screen to a
html file.

\end{fulllineitems}

\index{\_gradient() (polo.widgets.optimize\_widget.OptimizeWidget method)@\spxentry{\_gradient()}\spxextra{polo.widgets.optimize\_widget.OptimizeWidget method}}

\begin{fulllineitems}
\phantomsection\label{\detokenize{polo.widgets:polo.widgets.optimize_widget.OptimizeWidget._gradient}}\pysiglinewithargsret{\sphinxbfcode{\sphinxupquote{\_gradient}}}{\emph{\DUrole{n}{reagent}}, \emph{\DUrole{n}{num\_wells}}, \emph{\DUrole{n}{step}}, \emph{\DUrole{n}{stock}\DUrole{o}{=}\DUrole{default_value}{False}}}{}
Private method for calculating a concentration gradient for a
given \sphinxcode{\sphinxupquote{Reagent}} using a given step size as a proportion of the
\sphinxcode{\sphinxupquote{Reagent}} instance’s {\hyperref[\detokenize{polo.crystallography:polo.crystallography.cocktail.Reagent.concentration}]{\sphinxcrossref{\sphinxcode{\sphinxupquote{concentration}}}}}
attribute.
\begin{quote}\begin{description}
\item[{Parameters}] \leavevmode\begin{itemize}
\item {} 
\sphinxstyleliteralstrong{\sphinxupquote{reagent}} ({\hyperref[\detokenize{polo.crystallography:polo.crystallography.cocktail.Reagent}]{\sphinxcrossref{\sphinxstyleliteralemphasis{\sphinxupquote{Reagent}}}}}) \textendash{} Reagent to vary concentration

\item {} 
\sphinxstyleliteralstrong{\sphinxupquote{num\_wells}} (\sphinxstyleliteralemphasis{\sphinxupquote{int}}) \textendash{} Number of wells to vary concentration across

\item {} 
\sphinxstyleliteralstrong{\sphinxupquote{step}} (\sphinxstyleliteralemphasis{\sphinxupquote{float \textless{} 1}}) \textendash{} Proportion of hit concentration to vary each well by

\item {} 
\sphinxstyleliteralstrong{\sphinxupquote{stock}} (\sphinxstyleliteralemphasis{\sphinxupquote{bool}}\sphinxstyleliteralemphasis{\sphinxupquote{, }}\sphinxstyleliteralemphasis{\sphinxupquote{optional}}) \textendash{} If True, vary the stock volume not the hit
concentration unit, defaults to False

\end{itemize}

\item[{Returns}] \leavevmode
List of UnitValues that make up the \_gradient

\item[{Return type}] \leavevmode
list

\end{description}\end{quote}

\end{fulllineitems}

\index{\_handle\_reagent\_change() (polo.widgets.optimize\_widget.OptimizeWidget method)@\spxentry{\_handle\_reagent\_change()}\spxextra{polo.widgets.optimize\_widget.OptimizeWidget method}}

\begin{fulllineitems}
\phantomsection\label{\detokenize{polo.widgets:polo.widgets.optimize_widget.OptimizeWidget._handle_reagent_change}}\pysiglinewithargsret{\sphinxbfcode{\sphinxupquote{\_handle\_reagent\_change}}}{\emph{\DUrole{n}{x}\DUrole{o}{=}\DUrole{default_value}{False}}, \emph{\DUrole{n}{y}\DUrole{o}{=}\DUrole{default_value}{False}}, \emph{\DUrole{n}{const}\DUrole{o}{=}\DUrole{default_value}{False}}}{}
Private method that handles when a reagent is changed. The arguments
indicate which reagent has been changed.
\begin{quote}\begin{description}
\item[{Parameters}] \leavevmode\begin{itemize}
\item {} 
\sphinxstyleliteralstrong{\sphinxupquote{x}} (\sphinxstyleliteralemphasis{\sphinxupquote{bool}}\sphinxstyleliteralemphasis{\sphinxupquote{, }}\sphinxstyleliteralemphasis{\sphinxupquote{optional}}) \textendash{} If True update the x reagent, defaults to False

\item {} 
\sphinxstyleliteralstrong{\sphinxupquote{y}} (\sphinxstyleliteralemphasis{\sphinxupquote{bool}}\sphinxstyleliteralemphasis{\sphinxupquote{, }}\sphinxstyleliteralemphasis{\sphinxupquote{optional}}) \textendash{} If True update the y reagent, defaults to False

\item {} 
\sphinxstyleliteralstrong{\sphinxupquote{const}} (\sphinxstyleliteralemphasis{\sphinxupquote{bool}}\sphinxstyleliteralemphasis{\sphinxupquote{, }}\sphinxstyleliteralemphasis{\sphinxupquote{optional}}) \textendash{} If True update the constant reagents, defaults to False

\end{itemize}

\end{description}\end{quote}

\end{fulllineitems}

\index{\_make\_plate\_list() (polo.widgets.optimize\_widget.OptimizeWidget method)@\spxentry{\_make\_plate\_list()}\spxextra{polo.widgets.optimize\_widget.OptimizeWidget method}}

\begin{fulllineitems}
\phantomsection\label{\detokenize{polo.widgets:polo.widgets.optimize_widget.OptimizeWidget._make_plate_list}}\pysiglinewithargsret{\sphinxbfcode{\sphinxupquote{\_make\_plate\_list}}}{}{}
Private method that converts the concents of the 
\sphinxcode{\sphinxupquote{tableWidget}} UI (assuming that a optimization screen has been
already rendered to the user) to a list of lists that
is easier to write to html using the jinja2 template.
\begin{quote}\begin{description}
\item[{Returns}] \leavevmode
tableWidget contents converted to list

\item[{Return type}] \leavevmode
list

\end{description}\end{quote}

\end{fulllineitems}

\index{\_make\_well\_html() (polo.widgets.optimize\_widget.OptimizeWidget method)@\spxentry{\_make\_well\_html()}\spxextra{polo.widgets.optimize\_widget.OptimizeWidget method}}

\begin{fulllineitems}
\phantomsection\label{\detokenize{polo.widgets:polo.widgets.optimize_widget.OptimizeWidget._make_well_html}}\pysiglinewithargsret{\sphinxbfcode{\sphinxupquote{\_make\_well\_html}}}{\emph{\DUrole{n}{x\_con}}, \emph{\DUrole{n}{x\_stock}}, \emph{\DUrole{n}{y\_con}}, \emph{\DUrole{n}{y\_stock}}, \emph{\DUrole{n}{constants}}, \emph{\DUrole{n}{water}}}{}
Private method to format the information that describes the
contents of an individual well into prettier html that can be displayed
to the user in a \sphinxcode{\sphinxupquote{textBrowser}} widget.
\begin{quote}\begin{description}
\item[{Parameters}] \leavevmode\begin{itemize}
\item {} 
\sphinxstyleliteralstrong{\sphinxupquote{x\_con}} ({\hyperref[\detokenize{polo.crystallography:polo.crystallography.cocktail.UnitValue}]{\sphinxcrossref{\sphinxstyleliteralemphasis{\sphinxupquote{UnitValue}}}}}) \textendash{} Concentration of x reagent in this well

\item {} 
\sphinxstyleliteralstrong{\sphinxupquote{x\_stock}} ({\hyperref[\detokenize{polo.crystallography:polo.crystallography.cocktail.UnitValue}]{\sphinxcrossref{\sphinxstyleliteralemphasis{\sphinxupquote{UnitValue}}}}}) \textendash{} Volume of x reagent stock in this well

\item {} 
\sphinxstyleliteralstrong{\sphinxupquote{y\_con}} ({\hyperref[\detokenize{polo.crystallography:polo.crystallography.cocktail.UnitValue}]{\sphinxcrossref{\sphinxstyleliteralemphasis{\sphinxupquote{UnitValue}}}}}) \textendash{} Concentration of y reagent in this well

\item {} 
\sphinxstyleliteralstrong{\sphinxupquote{y\_stock}} ({\hyperref[\detokenize{polo.crystallography:polo.crystallography.cocktail.UnitValue}]{\sphinxcrossref{\sphinxstyleliteralemphasis{\sphinxupquote{UnitValue}}}}}) \textendash{} Volume of y reagent stock in this well

\item {} 
\sphinxstyleliteralstrong{\sphinxupquote{constants}} (\sphinxstyleliteralemphasis{\sphinxupquote{list of tuples}}) \textendash{} Tuples of constant reagents to be included in each well

\item {} 
\sphinxstyleliteralstrong{\sphinxupquote{water}} (\sphinxstyleliteralemphasis{\sphinxupquote{Signed Value}}) \textendash{} Volume of water to be added to this well

\end{itemize}

\item[{Returns}] \leavevmode
Html string to be rendered to the user

\item[{Return type}] \leavevmode
str

\end{description}\end{quote}

\end{fulllineitems}

\index{\_set\_constant\_reagents() (polo.widgets.optimize\_widget.OptimizeWidget method)@\spxentry{\_set\_constant\_reagents()}\spxextra{polo.widgets.optimize\_widget.OptimizeWidget method}}

\begin{fulllineitems}
\phantomsection\label{\detokenize{polo.widgets:polo.widgets.optimize_widget.OptimizeWidget._set_constant_reagents}}\pysiglinewithargsret{\sphinxbfcode{\sphinxupquote{\_set\_constant\_reagents}}}{}{}
Private method that populates the listWidget with constant 
reagents to display to the user.

\end{fulllineitems}

\index{\_set\_hit\_well\_choices() (polo.widgets.optimize\_widget.OptimizeWidget method)@\spxentry{\_set\_hit\_well\_choices()}\spxextra{polo.widgets.optimize\_widget.OptimizeWidget method}}

\begin{fulllineitems}
\phantomsection\label{\detokenize{polo.widgets:polo.widgets.optimize_widget.OptimizeWidget._set_hit_well_choices}}\pysiglinewithargsret{\sphinxbfcode{\sphinxupquote{\_set\_hit\_well\_choices}}}{}{}
Private method that sets the hit well comboBox widget choices 
based on the images in the {\hyperref[\detokenize{polo.widgets:polo.widgets.optimize_widget.OptimizeWidget.run}]{\sphinxcrossref{\sphinxcode{\sphinxupquote{run}}}}} attribute that 
are human classified as crystal.
Wells are identified in the comboBox by their well number.

\end{fulllineitems}

\index{\_set\_reagent\_choices() (polo.widgets.optimize\_widget.OptimizeWidget method)@\spxentry{\_set\_reagent\_choices()}\spxextra{polo.widgets.optimize\_widget.OptimizeWidget method}}

\begin{fulllineitems}
\phantomsection\label{\detokenize{polo.widgets:polo.widgets.optimize_widget.OptimizeWidget._set_reagent_choices}}\pysiglinewithargsret{\sphinxbfcode{\sphinxupquote{\_set\_reagent\_choices}}}{}{}
Private method that sets \sphinxcode{\sphinxupquote{Reagent}} choices for the x and y reagents
based on the currently selected well. \sphinxcode{\sphinxupquote{Reagents}} must come from the
class:\sphinxtitleref{Cocktail} instance associated with the selected well.

TODO: Add the option to vary pH instead of a reagent along either
axis. This would also mean that the constant reagents would need to
be updated.

\end{fulllineitems}

\index{\_set\_reagent\_stock\_con() (polo.widgets.optimize\_widget.OptimizeWidget method)@\spxentry{\_set\_reagent\_stock\_con()}\spxextra{polo.widgets.optimize\_widget.OptimizeWidget method}}

\begin{fulllineitems}
\phantomsection\label{\detokenize{polo.widgets:polo.widgets.optimize_widget.OptimizeWidget._set_reagent_stock_con}}\pysiglinewithargsret{\sphinxbfcode{\sphinxupquote{\_set\_reagent\_stock\_con}}}{}{}
Private method. If a \sphinxcode{\sphinxupquote{Reagent}} has already been assigned a 
stock concentration this method displays that concentration to the user 
through the appropriate \sphinxcode{\sphinxupquote{UnitCombobBox}} instance.
Only displays concentrations for the x and y reagents.

\end{fulllineitems}

\index{\_set\_reagent\_stock\_con\_values() (polo.widgets.optimize\_widget.OptimizeWidget method)@\spxentry{\_set\_reagent\_stock\_con\_values()}\spxextra{polo.widgets.optimize\_widget.OptimizeWidget method}}

\begin{fulllineitems}
\phantomsection\label{\detokenize{polo.widgets:polo.widgets.optimize_widget.OptimizeWidget._set_reagent_stock_con_values}}\pysiglinewithargsret{\sphinxbfcode{\sphinxupquote{\_set\_reagent\_stock\_con\_values}}}{\emph{\DUrole{n}{x}\DUrole{o}{=}\DUrole{default_value}{False}}, \emph{\DUrole{n}{y}\DUrole{o}{=}\DUrole{default_value}{False}}, \emph{\DUrole{n}{const}\DUrole{o}{=}\DUrole{default_value}{False}}}{}
Private method to update the stock concentations of current
reagents through their \sphinxcode{\sphinxupquote{stock\_con}}
attribute. The \sphinxcode{\sphinxupquote{Reagent}} to update
is indicated by the flag set to True at the time the method is
called. The stock concentration value is pulled from each reagent’s
respective \sphinxtitleref{unitComboBox} instance.
\begin{quote}\begin{description}
\item[{Parameters}] \leavevmode\begin{itemize}
\item {} 
\sphinxstyleliteralstrong{\sphinxupquote{x}} (\sphinxstyleliteralemphasis{\sphinxupquote{bool}}\sphinxstyleliteralemphasis{\sphinxupquote{, }}\sphinxstyleliteralemphasis{\sphinxupquote{optional}}) \textendash{} If True, set x reagent stock con, defaults to False

\item {} 
\sphinxstyleliteralstrong{\sphinxupquote{y}} (\sphinxstyleliteralemphasis{\sphinxupquote{bool}}\sphinxstyleliteralemphasis{\sphinxupquote{, }}\sphinxstyleliteralemphasis{\sphinxupquote{optional}}) \textendash{} If True, set the y reagent stock con, defaults to False

\item {} 
\sphinxstyleliteralstrong{\sphinxupquote{const}} (\sphinxstyleliteralemphasis{\sphinxupquote{bool}}\sphinxstyleliteralemphasis{\sphinxupquote{, }}\sphinxstyleliteralemphasis{\sphinxupquote{optional}}) \textendash{} If True, sets the constant reagents stock con,
defaults to False

\end{itemize}

\end{description}\end{quote}

\end{fulllineitems}

\index{\_set\_up\_unit\_comboboxes() (polo.widgets.optimize\_widget.OptimizeWidget method)@\spxentry{\_set\_up\_unit\_comboboxes()}\spxextra{polo.widgets.optimize\_widget.OptimizeWidget method}}

\begin{fulllineitems}
\phantomsection\label{\detokenize{polo.widgets:polo.widgets.optimize_widget.OptimizeWidget._set_up_unit_comboboxes}}\pysiglinewithargsret{\sphinxbfcode{\sphinxupquote{\_set\_up\_unit\_comboboxes}}}{}{}
Private method that sets the base unit and the scalers
of all \sphinxcode{\sphinxupquote{unitComboBox}} instances that are apart of the UI.

\end{fulllineitems}

\index{\_update\_current\_reagents() (polo.widgets.optimize\_widget.OptimizeWidget method)@\spxentry{\_update\_current\_reagents()}\spxextra{polo.widgets.optimize\_widget.OptimizeWidget method}}

\begin{fulllineitems}
\phantomsection\label{\detokenize{polo.widgets:polo.widgets.optimize_widget.OptimizeWidget._update_current_reagents}}\pysiglinewithargsret{\sphinxbfcode{\sphinxupquote{\_update\_current\_reagents}}}{\emph{\DUrole{n}{image\_index}\DUrole{o}{=}\DUrole{default_value}{None}}}{}
Private method that updates x and y reagent comboBox widgets to 
reflect what \sphinxcode{\sphinxupquote{Reagent}} instances are contained in the
currently selected well.
\begin{quote}\begin{description}
\item[{Parameters}] \leavevmode
\sphinxstyleliteralstrong{\sphinxupquote{image\_index}} (\sphinxstyleliteralemphasis{\sphinxupquote{int}}\sphinxstyleliteralemphasis{\sphinxupquote{, }}\sphinxstyleliteralemphasis{\sphinxupquote{optional}}) \textendash{} Index of the \sphinxcode{\sphinxupquote{Image}} to set \sphinxcode{\sphinxupquote{Reagent}} choices from,
defaults to None.

\end{description}\end{quote}

\end{fulllineitems}

\index{\_write\_optimization\_screen() (polo.widgets.optimize\_widget.OptimizeWidget method)@\spxentry{\_write\_optimization\_screen()}\spxextra{polo.widgets.optimize\_widget.OptimizeWidget method}}

\begin{fulllineitems}
\phantomsection\label{\detokenize{polo.widgets:polo.widgets.optimize_widget.OptimizeWidget._write_optimization_screen}}\pysiglinewithargsret{\sphinxbfcode{\sphinxupquote{\_write\_optimization\_screen}}}{}{}
Private method to write the current optimization screen
to the \sphinxcode{\sphinxupquote{tableWidget}} UI for display to the user.

\end{fulllineitems}

\index{adjust\_unit() (polo.widgets.optimize\_widget.OptimizeWidget method)@\spxentry{adjust\_unit()}\spxextra{polo.widgets.optimize\_widget.OptimizeWidget method}}

\begin{fulllineitems}
\phantomsection\label{\detokenize{polo.widgets:polo.widgets.optimize_widget.OptimizeWidget.adjust_unit}}\pysiglinewithargsret{\sphinxbfcode{\sphinxupquote{adjust\_unit}}}{\emph{\DUrole{n}{signed\_value}}, \emph{\DUrole{n}{new\_unit}}}{}
\end{fulllineitems}

\index{constant\_reagents() (polo.widgets.optimize\_widget.OptimizeWidget property)@\spxentry{constant\_reagents()}\spxextra{polo.widgets.optimize\_widget.OptimizeWidget property}}

\begin{fulllineitems}
\phantomsection\label{\detokenize{polo.widgets:polo.widgets.optimize_widget.OptimizeWidget.constant_reagents}}\pysigline{\sphinxbfcode{\sphinxupquote{property }}\sphinxbfcode{\sphinxupquote{constant\_reagents}}}
Retrieve a set of \sphinxcode{\sphinxupquote{Reagents}} that are not included as either the
x reagent or the y reagent but are still part of the crystallization
cocktail and therefore need to be included in the screen. Unlike
either the x or y reagents, constant reagents do not change their
concentration across the screening plate.

\end{fulllineitems}

\index{hit\_images() (polo.widgets.optimize\_widget.OptimizeWidget property)@\spxentry{hit\_images()}\spxextra{polo.widgets.optimize\_widget.OptimizeWidget property}}

\begin{fulllineitems}
\phantomsection\label{\detokenize{polo.widgets:polo.widgets.optimize_widget.OptimizeWidget.hit_images}}\pysigline{\sphinxbfcode{\sphinxupquote{property }}\sphinxbfcode{\sphinxupquote{hit\_images}}}
Retrieves a list of {\hyperref[\detokenize{polo.crystallography:polo.crystallography.image.Image}]{\sphinxcrossref{\sphinxcode{\sphinxupquote{Image}}}}}
object instances that have human
classification (\sphinxtitleref{human\_class} attribute) == ‘Crystals’. Used to
determine what wells to allow the user to optimize. Currently, only
allow the user to optimize wells they have marked as crystal.

\end{fulllineitems}

\index{run() (polo.widgets.optimize\_widget.OptimizeWidget property)@\spxentry{run()}\spxextra{polo.widgets.optimize\_widget.OptimizeWidget property}}

\begin{fulllineitems}
\phantomsection\label{\detokenize{polo.widgets:polo.widgets.optimize_widget.OptimizeWidget.run}}\pysigline{\sphinxbfcode{\sphinxupquote{property }}\sphinxbfcode{\sphinxupquote{run}}}
\end{fulllineitems}

\index{selected\_constant() (polo.widgets.optimize\_widget.OptimizeWidget property)@\spxentry{selected\_constant()}\spxextra{polo.widgets.optimize\_widget.OptimizeWidget property}}

\begin{fulllineitems}
\phantomsection\label{\detokenize{polo.widgets:polo.widgets.optimize_widget.OptimizeWidget.selected_constant}}\pysigline{\sphinxbfcode{\sphinxupquote{property }}\sphinxbfcode{\sphinxupquote{selected\_constant}}}
Return the constant \sphinxcode{\sphinxupquote{Reagent}} that is currently
selected by the user.
\begin{quote}\begin{description}
\item[{Returns}] \leavevmode
Currently selected constant reagent if exists and selected, 
None otherwise

\item[{Return type}] \leavevmode
{\hyperref[\detokenize{polo.crystallography:polo.crystallography.cocktail.Reagent}]{\sphinxcrossref{Reagent}}} or None

\end{description}\end{quote}

\end{fulllineitems}

\index{update() (polo.widgets.optimize\_widget.OptimizeWidget method)@\spxentry{update()}\spxextra{polo.widgets.optimize\_widget.OptimizeWidget method}}

\begin{fulllineitems}
\phantomsection\label{\detokenize{polo.widgets:polo.widgets.optimize_widget.OptimizeWidget.update}}\pysiglinewithargsret{\sphinxbfcode{\sphinxupquote{update}}}{}{}
Method to update reagents and selectable wells to the user after
they have made additional classifications that would increase or
decrease the pool of crystal classified images.

\end{fulllineitems}

\index{well\_volume() (polo.widgets.optimize\_widget.OptimizeWidget property)@\spxentry{well\_volume()}\spxextra{polo.widgets.optimize\_widget.OptimizeWidget property}}

\begin{fulllineitems}
\phantomsection\label{\detokenize{polo.widgets:polo.widgets.optimize_widget.OptimizeWidget.well_volume}}\pysigline{\sphinxbfcode{\sphinxupquote{property }}\sphinxbfcode{\sphinxupquote{well\_volume}}}
Returns the well volume set by the user modifed by whatever well
volume unit is currently selected.

\end{fulllineitems}

\index{x\_reagent() (polo.widgets.optimize\_widget.OptimizeWidget property)@\spxentry{x\_reagent()}\spxextra{polo.widgets.optimize\_widget.OptimizeWidget property}}

\begin{fulllineitems}
\phantomsection\label{\detokenize{polo.widgets:polo.widgets.optimize_widget.OptimizeWidget.x_reagent}}\pysigline{\sphinxbfcode{\sphinxupquote{property }}\sphinxbfcode{\sphinxupquote{x\_reagent}}}
Used to retrieve the \sphinxcode{\sphinxupquote{Reagent}} object that is to be varied along
the x axis of the screen.

\end{fulllineitems}

\index{x\_step() (polo.widgets.optimize\_widget.OptimizeWidget property)@\spxentry{x\_step()}\spxextra{polo.widgets.optimize\_widget.OptimizeWidget property}}

\begin{fulllineitems}
\phantomsection\label{\detokenize{polo.widgets:polo.widgets.optimize_widget.OptimizeWidget.x_step}}\pysigline{\sphinxbfcode{\sphinxupquote{property }}\sphinxbfcode{\sphinxupquote{x\_step}}}
The percent variance between x reagent wells.

\end{fulllineitems}

\index{x\_wells() (polo.widgets.optimize\_widget.OptimizeWidget property)@\spxentry{x\_wells()}\spxextra{polo.widgets.optimize\_widget.OptimizeWidget property}}

\begin{fulllineitems}
\phantomsection\label{\detokenize{polo.widgets:polo.widgets.optimize_widget.OptimizeWidget.x_wells}}\pysigline{\sphinxbfcode{\sphinxupquote{property }}\sphinxbfcode{\sphinxupquote{x\_wells}}}
Returns spinBox value that representing the number 
wells on the x axis of the screen.

\end{fulllineitems}

\index{y\_reagent() (polo.widgets.optimize\_widget.OptimizeWidget property)@\spxentry{y\_reagent()}\spxextra{polo.widgets.optimize\_widget.OptimizeWidget property}}

\begin{fulllineitems}
\phantomsection\label{\detokenize{polo.widgets:polo.widgets.optimize_widget.OptimizeWidget.y_reagent}}\pysigline{\sphinxbfcode{\sphinxupquote{property }}\sphinxbfcode{\sphinxupquote{y\_reagent}}}
Used to retreive the \sphinxcode{\sphinxupquote{Reagent}} object that is to be varied along
the y axis of the optimization plate

\end{fulllineitems}

\index{y\_step() (polo.widgets.optimize\_widget.OptimizeWidget property)@\spxentry{y\_step()}\spxextra{polo.widgets.optimize\_widget.OptimizeWidget property}}

\begin{fulllineitems}
\phantomsection\label{\detokenize{polo.widgets:polo.widgets.optimize_widget.OptimizeWidget.y_step}}\pysigline{\sphinxbfcode{\sphinxupquote{property }}\sphinxbfcode{\sphinxupquote{y\_step}}}
The percent variance between y reagent wells.

\end{fulllineitems}

\index{y\_wells() (polo.widgets.optimize\_widget.OptimizeWidget property)@\spxentry{y\_wells()}\spxextra{polo.widgets.optimize\_widget.OptimizeWidget property}}

\begin{fulllineitems}
\phantomsection\label{\detokenize{polo.widgets:polo.widgets.optimize_widget.OptimizeWidget.y_wells}}\pysigline{\sphinxbfcode{\sphinxupquote{property }}\sphinxbfcode{\sphinxupquote{y\_wells}}}
Returns spinBox value representing the number of
wells on the y axis of the screen.

\end{fulllineitems}


\end{fulllineitems}



\subsubsection{polo.widgets.plate\_inspector\_widget module}
\label{\detokenize{polo.widgets:module-polo.widgets.plate_inspector_widget}}\label{\detokenize{polo.widgets:polo-widgets-plate-inspector-widget-module}}\index{module@\spxentry{module}!polo.widgets.plate\_inspector\_widget@\spxentry{polo.widgets.plate\_inspector\_widget}}\index{polo.widgets.plate\_inspector\_widget@\spxentry{polo.widgets.plate\_inspector\_widget}!module@\spxentry{module}}\index{PlateInspectorWidget (class in polo.widgets.plate\_inspector\_widget)@\spxentry{PlateInspectorWidget}\spxextra{class in polo.widgets.plate\_inspector\_widget}}

\begin{fulllineitems}
\phantomsection\label{\detokenize{polo.widgets:polo.widgets.plate_inspector_widget.PlateInspectorWidget}}\pysiglinewithargsret{\sphinxbfcode{\sphinxupquote{class }}\sphinxcode{\sphinxupquote{polo.widgets.plate\_inspector\_widget.}}\sphinxbfcode{\sphinxupquote{PlateInspectorWidget}}}{\emph{\DUrole{n}{parent}}, \emph{\DUrole{n}{run}\DUrole{o}{=}\DUrole{default_value}{None}}}{}
Bases: \sphinxcode{\sphinxupquote{PyQt5.QtWidgets.QWidget}}
\index{\_apply\_color\_mapping() (polo.widgets.plate\_inspector\_widget.PlateInspectorWidget method)@\spxentry{\_apply\_color\_mapping()}\spxextra{polo.widgets.plate\_inspector\_widget.PlateInspectorWidget method}}

\begin{fulllineitems}
\phantomsection\label{\detokenize{polo.widgets:polo.widgets.plate_inspector_widget.PlateInspectorWidget._apply_color_mapping}}\pysiglinewithargsret{\sphinxbfcode{\sphinxupquote{\_apply\_color\_mapping}}}{}{}
Applies the current color mapping to displayed images. Images
are colored based on either their human or marco classification.

\end{fulllineitems}

\index{\_apply\_image\_filters() (polo.widgets.plate\_inspector\_widget.PlateInspectorWidget method)@\spxentry{\_apply\_image\_filters()}\spxextra{polo.widgets.plate\_inspector\_widget.PlateInspectorWidget method}}

\begin{fulllineitems}
\phantomsection\label{\detokenize{polo.widgets:polo.widgets.plate_inspector_widget.PlateInspectorWidget._apply_image_filters}}\pysiglinewithargsret{\sphinxbfcode{\sphinxupquote{\_apply\_image\_filters}}}{}{}
Wrapper function around \sphinxtitleref{plateViewer}
\sphinxcode{\sphinxupquote{deemphasize\_filtered\_images()}}
which changes the opacity of currently displayed images based on
their classifications.

\end{fulllineitems}

\index{\_parse\_label\_checkboxes() (polo.widgets.plate\_inspector\_widget.PlateInspectorWidget method)@\spxentry{\_parse\_label\_checkboxes()}\spxextra{polo.widgets.plate\_inspector\_widget.PlateInspectorWidget method}}

\begin{fulllineitems}
\phantomsection\label{\detokenize{polo.widgets:polo.widgets.plate_inspector_widget.PlateInspectorWidget._parse_label_checkboxes}}\pysiglinewithargsret{\sphinxbfcode{\sphinxupquote{\_parse\_label\_checkboxes}}}{}{}
Private method that reads values from \sphinxcode{\sphinxupquote{QCheckBox}}
instances related to image filtering.
Returns a dictionary where keys are the labels of the \sphinxcode{\sphinxupquote{QCheckBox}} instances
which should also be the possible image classifications and values
are the state of the \sphinxcode{\sphinxupquote{QCheckBox}} (True or False).

Returned dictionary will have following structure.

\begin{sphinxVerbatim}[commandchars=\\\{\}]
\PYG{p}{\PYGZob{}}
\PYG{l+s+s1}{\PYGZsq{}}\PYG{l+s+s1}{Crystals}\PYG{l+s+s1}{\PYGZsq{}}\PYG{p}{:} \PYG{k+kc}{True}\PYG{p}{,}
\PYG{l+s+s1}{\PYGZsq{}}\PYG{l+s+s1}{Clear}\PYG{l+s+s1}{\PYGZsq{}}\PYG{p}{:} \PYG{k+kc}{False}\PYG{p}{,}
\PYG{l+s+s1}{\PYGZsq{}}\PYG{l+s+s1}{Precipitate}\PYG{l+s+s1}{\PYGZsq{}}\PYG{p}{:} \PYG{k+kc}{True}\PYG{p}{,}
\PYG{l+s+s1}{\PYGZsq{}}\PYG{l+s+s1}{Other}\PYG{l+s+s1}{\PYGZsq{}}\PYG{p}{:} \PYG{k+kc}{False}
\PYG{p}{\PYGZcb{}}
\end{sphinxVerbatim}
\begin{quote}\begin{description}
\item[{Returns}] \leavevmode
Dict of \sphinxcode{\sphinxupquote{QCheckBox}} states.

\item[{Return type}] \leavevmode
dict

\end{description}\end{quote}

\end{fulllineitems}

\index{\_set\_alt\_spectrum\_buttons() (polo.widgets.plate\_inspector\_widget.PlateInspectorWidget method)@\spxentry{\_set\_alt\_spectrum\_buttons()}\spxextra{polo.widgets.plate\_inspector\_widget.PlateInspectorWidget method}}

\begin{fulllineitems}
\phantomsection\label{\detokenize{polo.widgets:polo.widgets.plate_inspector_widget.PlateInspectorWidget._set_alt_spectrum_buttons}}\pysiglinewithargsret{\sphinxbfcode{\sphinxupquote{\_set\_alt\_spectrum\_buttons}}}{}{}
Private helper function similar to 
\sphinxcode{\sphinxupquote{\_set\_time\_resolved\_buttons()}}
that determines if the navigation button that allows users to view 
alt spectrum images should be enabled. If conditions are not met then 
the button is disabled.

\end{fulllineitems}

\index{\_set\_color\_comboboxs() (polo.widgets.plate\_inspector\_widget.PlateInspectorWidget method)@\spxentry{\_set\_color\_comboboxs()}\spxextra{polo.widgets.plate\_inspector\_widget.PlateInspectorWidget method}}

\begin{fulllineitems}
\phantomsection\label{\detokenize{polo.widgets:polo.widgets.plate_inspector_widget.PlateInspectorWidget._set_color_comboboxs}}\pysiglinewithargsret{\sphinxbfcode{\sphinxupquote{\_set\_color\_comboboxs}}}{}{}
Private method that sets the label text associated with each color
selector \sphinxcode{\sphinxupquote{QComboBox}}. Should be called in the \sphinxtitleref{\_\_init\_\_} method before
the widget is shown to the user.

\end{fulllineitems}

\index{\_set\_color\_options() (polo.widgets.plate\_inspector\_widget.PlateInspectorWidget method)@\spxentry{\_set\_color\_options()}\spxextra{polo.widgets.plate\_inspector\_widget.PlateInspectorWidget method}}

\begin{fulllineitems}
\phantomsection\label{\detokenize{polo.widgets:polo.widgets.plate_inspector_widget.PlateInspectorWidget._set_color_options}}\pysiglinewithargsret{\sphinxbfcode{\sphinxupquote{\_set\_color\_options}}}{}{}
Private methods that uses the \sphinxcode{\sphinxupquote{COLORS}} constant to set the color options 
for each color selector \sphinxcode{\sphinxupquote{QComboBox}} instance in the image coloring tab.

\end{fulllineitems}

\index{\_set\_current\_page() (polo.widgets.plate\_inspector\_widget.PlateInspectorWidget method)@\spxentry{\_set\_current\_page()}\spxextra{polo.widgets.plate\_inspector\_widget.PlateInspectorWidget method}}

\begin{fulllineitems}
\phantomsection\label{\detokenize{polo.widgets:polo.widgets.plate_inspector_widget.PlateInspectorWidget._set_current_page}}\pysiglinewithargsret{\sphinxbfcode{\sphinxupquote{\_set\_current\_page}}}{\emph{\DUrole{n}{page\_number}}}{}
Set the current page number and show the view for that page by
calling \sphinxcode{\sphinxupquote{show\_current\_page()}}
\begin{quote}\begin{description}
\item[{Parameters}] \leavevmode
\sphinxstyleliteralstrong{\sphinxupquote{page\_number}} (\sphinxstyleliteralemphasis{\sphinxupquote{int}}) \textendash{} The new page number

\end{description}\end{quote}

\end{fulllineitems}

\index{\_set\_image\_count\_options() (polo.widgets.plate\_inspector\_widget.PlateInspectorWidget method)@\spxentry{\_set\_image\_count\_options()}\spxextra{polo.widgets.plate\_inspector\_widget.PlateInspectorWidget method}}

\begin{fulllineitems}
\phantomsection\label{\detokenize{polo.widgets:polo.widgets.plate_inspector_widget.PlateInspectorWidget._set_image_count_options}}\pysiglinewithargsret{\sphinxbfcode{\sphinxupquote{\_set\_image\_count\_options}}}{}{}
Private method to be called in the \sphinxtitleref{\_\_init\_\_} method that sets
the allowed number of images per page.

\end{fulllineitems}

\index{\_set\_images\_per\_page() (polo.widgets.plate\_inspector\_widget.PlateInspectorWidget method)@\spxentry{\_set\_images\_per\_page()}\spxextra{polo.widgets.plate\_inspector\_widget.PlateInspectorWidget method}}

\begin{fulllineitems}
\phantomsection\label{\detokenize{polo.widgets:polo.widgets.plate_inspector_widget.PlateInspectorWidget._set_images_per_page}}\pysiglinewithargsret{\sphinxbfcode{\sphinxupquote{\_set\_images\_per\_page}}}{}{}
Private method that tells the \sphinxcode{\sphinxupquote{plateViewer}} UI widget to set 
its {\hyperref[\detokenize{polo.widgets:polo.widgets.plate_inspector_widget.PlateInspectorWidget.images_per_page}]{\sphinxcrossref{\sphinxcode{\sphinxupquote{images\_per\_page}}}}} atttribute to the value specified in the 
images per page \sphinxcode{\sphinxupquote{QComboBox}}.

\end{fulllineitems}

\index{\_set\_plate\_label() (polo.widgets.plate\_inspector\_widget.PlateInspectorWidget method)@\spxentry{\_set\_plate\_label()}\spxextra{polo.widgets.plate\_inspector\_widget.PlateInspectorWidget method}}

\begin{fulllineitems}
\phantomsection\label{\detokenize{polo.widgets:polo.widgets.plate_inspector_widget.PlateInspectorWidget._set_plate_label}}\pysiglinewithargsret{\sphinxbfcode{\sphinxupquote{\_set\_plate\_label}}}{}{}
Private method to change the plate label to tell the user what view or 
“page” they are currently looking at.

\end{fulllineitems}

\index{\_set\_spin\_box\_range() (polo.widgets.plate\_inspector\_widget.PlateInspectorWidget method)@\spxentry{\_set\_spin\_box\_range()}\spxextra{polo.widgets.plate\_inspector\_widget.PlateInspectorWidget method}}

\begin{fulllineitems}
\phantomsection\label{\detokenize{polo.widgets:polo.widgets.plate_inspector_widget.PlateInspectorWidget._set_spin_box_range}}\pysiglinewithargsret{\sphinxbfcode{\sphinxupquote{\_set\_spin\_box\_range}}}{}{}
Set the allowed range for the page navigation spinbox.

\end{fulllineitems}

\index{\_set\_time\_resolved\_buttons() (polo.widgets.plate\_inspector\_widget.PlateInspectorWidget method)@\spxentry{\_set\_time\_resolved\_buttons()}\spxextra{polo.widgets.plate\_inspector\_widget.PlateInspectorWidget method}}

\begin{fulllineitems}
\phantomsection\label{\detokenize{polo.widgets:polo.widgets.plate_inspector_widget.PlateInspectorWidget._set_time_resolved_buttons}}\pysiglinewithargsret{\sphinxbfcode{\sphinxupquote{\_set\_time\_resolved\_buttons}}}{}{}
Private helper function that determines if navigation buttons 
that display alt spectrum images, previous and next date images 
can be used.

\end{fulllineitems}

\index{apply\_plate\_settings() (polo.widgets.plate\_inspector\_widget.PlateInspectorWidget method)@\spxentry{apply\_plate\_settings()}\spxextra{polo.widgets.plate\_inspector\_widget.PlateInspectorWidget method}}

\begin{fulllineitems}
\phantomsection\label{\detokenize{polo.widgets:polo.widgets.plate_inspector_widget.PlateInspectorWidget.apply_plate_settings}}\pysiglinewithargsret{\sphinxbfcode{\sphinxupquote{apply\_plate\_settings}}}{}{}
Parses \sphinxcode{\sphinxupquote{QCheckBox}} instances in the Plate View tab
to determine what behavior of the \sphinxcode{\sphinxupquote{plateViewer}} 
widget is requested by the user.

\end{fulllineitems}

\index{color\_mapping() (polo.widgets.plate\_inspector\_widget.PlateInspectorWidget property)@\spxentry{color\_mapping()}\spxextra{polo.widgets.plate\_inspector\_widget.PlateInspectorWidget property}}

\begin{fulllineitems}
\phantomsection\label{\detokenize{polo.widgets:polo.widgets.plate_inspector_widget.PlateInspectorWidget.color_mapping}}\pysigline{\sphinxbfcode{\sphinxupquote{property }}\sphinxbfcode{\sphinxupquote{color\_mapping}}}
Creates a color mapping dictionary that reflects the currently selected
color selector \sphinxcode{\sphinxupquote{QComboBox}} instances. The dictionary maps each image
classifications to a \sphinxcode{\sphinxupquote{QColor}} instance that can then be used
to color images in the plate viewer.

\end{fulllineitems}

\index{favorite() (polo.widgets.plate\_inspector\_widget.PlateInspectorWidget property)@\spxentry{favorite()}\spxextra{polo.widgets.plate\_inspector\_widget.PlateInspectorWidget property}}

\begin{fulllineitems}
\phantomsection\label{\detokenize{polo.widgets:polo.widgets.plate_inspector_widget.PlateInspectorWidget.favorite}}\pysigline{\sphinxbfcode{\sphinxupquote{property }}\sphinxbfcode{\sphinxupquote{favorite}}}
Status of the \sphinxtitleref{favorite} \sphinxcode{\sphinxupquote{QCheckBox}} filter.
\begin{quote}\begin{description}
\item[{Returns}] \leavevmode
State of the favorite \sphinxcode{\sphinxupquote{QCheckBox}}

\item[{Return type}] \leavevmode
bool

\end{description}\end{quote}

\end{fulllineitems}

\index{human() (polo.widgets.plate\_inspector\_widget.PlateInspectorWidget property)@\spxentry{human()}\spxextra{polo.widgets.plate\_inspector\_widget.PlateInspectorWidget property}}

\begin{fulllineitems}
\phantomsection\label{\detokenize{polo.widgets:polo.widgets.plate_inspector_widget.PlateInspectorWidget.human}}\pysigline{\sphinxbfcode{\sphinxupquote{property }}\sphinxbfcode{\sphinxupquote{human}}}
Status of human image classification \sphinxcode{\sphinxupquote{QCheckBox}}.
\begin{quote}\begin{description}
\item[{Returns}] \leavevmode
State of the human filter \sphinxcode{\sphinxupquote{QCheckBox}}

\item[{Return type}] \leavevmode
bool

\end{description}\end{quote}

\end{fulllineitems}

\index{images\_per\_page (polo.widgets.plate\_inspector\_widget.PlateInspectorWidget attribute)@\spxentry{images\_per\_page}\spxextra{polo.widgets.plate\_inspector\_widget.PlateInspectorWidget attribute}}

\begin{fulllineitems}
\phantomsection\label{\detokenize{polo.widgets:polo.widgets.plate_inspector_widget.PlateInspectorWidget.images_per_page}}\pysigline{\sphinxbfcode{\sphinxupquote{images\_per\_page}}\sphinxbfcode{\sphinxupquote{ = {[}16, 64, 96{]}}}}
\end{fulllineitems}

\index{marco() (polo.widgets.plate\_inspector\_widget.PlateInspectorWidget property)@\spxentry{marco()}\spxextra{polo.widgets.plate\_inspector\_widget.PlateInspectorWidget property}}

\begin{fulllineitems}
\phantomsection\label{\detokenize{polo.widgets:polo.widgets.plate_inspector_widget.PlateInspectorWidget.marco}}\pysigline{\sphinxbfcode{\sphinxupquote{property }}\sphinxbfcode{\sphinxupquote{marco}}}
Status of marco image classification \sphinxcode{\sphinxupquote{QCheckBox}}.
\begin{quote}\begin{description}
\item[{Returns}] \leavevmode
State of the marco filter \sphinxcode{\sphinxupquote{QCheckBox}}

\item[{Return type}] \leavevmode
bool

\end{description}\end{quote}

\end{fulllineitems}

\index{reset\_all() (polo.widgets.plate\_inspector\_widget.PlateInspectorWidget method)@\spxentry{reset\_all()}\spxextra{polo.widgets.plate\_inspector\_widget.PlateInspectorWidget method}}

\begin{fulllineitems}
\phantomsection\label{\detokenize{polo.widgets:polo.widgets.plate_inspector_widget.PlateInspectorWidget.reset_all}}\pysiglinewithargsret{\sphinxbfcode{\sphinxupquote{reset\_all}}}{}{}
Method to un\sphinxhyphen{}check all user selected settings.

\end{fulllineitems}

\index{run() (polo.widgets.plate\_inspector\_widget.PlateInspectorWidget property)@\spxentry{run()}\spxextra{polo.widgets.plate\_inspector\_widget.PlateInspectorWidget property}}

\begin{fulllineitems}
\phantomsection\label{\detokenize{polo.widgets:polo.widgets.plate_inspector_widget.PlateInspectorWidget.run}}\pysigline{\sphinxbfcode{\sphinxupquote{property }}\sphinxbfcode{\sphinxupquote{run}}}
\end{fulllineitems}

\index{selected\_classifications() (polo.widgets.plate\_inspector\_widget.PlateInspectorWidget property)@\spxentry{selected\_classifications()}\spxextra{polo.widgets.plate\_inspector\_widget.PlateInspectorWidget property}}

\begin{fulllineitems}
\phantomsection\label{\detokenize{polo.widgets:polo.widgets.plate_inspector_widget.PlateInspectorWidget.selected_classifications}}\pysigline{\sphinxbfcode{\sphinxupquote{property }}\sphinxbfcode{\sphinxupquote{selected\_classifications}}}
Image classifications that are
selected via the image filtering \sphinxcode{\sphinxupquote{QCheckBox}} instances. Also see the
\sphinxcode{\sphinxupquote{image\_type\_checkboxes}} property.
\begin{quote}\begin{description}
\item[{Returns}] \leavevmode
List of currently selected image classifications. Images
who’s classification is in this list should be shown / 
emphasized to the user.

\item[{Return type}] \leavevmode
list

\end{description}\end{quote}

\end{fulllineitems}

\index{set\_aspect\_ratio\_mode() (polo.widgets.plate\_inspector\_widget.PlateInspectorWidget method)@\spxentry{set\_aspect\_ratio\_mode()}\spxextra{polo.widgets.plate\_inspector\_widget.PlateInspectorWidget method}}

\begin{fulllineitems}
\phantomsection\label{\detokenize{polo.widgets:polo.widgets.plate_inspector_widget.PlateInspectorWidget.set_aspect_ratio_mode}}\pysiglinewithargsret{\sphinxbfcode{\sphinxupquote{set\_aspect\_ratio\_mode}}}{}{}
Sets the \sphinxtitleref{preserve\_aspect} attribute based on the status of
the preserve aspect ratio \sphinxcode{\sphinxupquote{QCheckBox}}. Preserving the 
aspect ratio results in displaying undistorted crystallization images 
but utilizes available display space less efficiently.

\end{fulllineitems}

\index{show\_current\_plate() (polo.widgets.plate\_inspector\_widget.PlateInspectorWidget method)@\spxentry{show\_current\_plate()}\spxextra{polo.widgets.plate\_inspector\_widget.PlateInspectorWidget method}}

\begin{fulllineitems}
\phantomsection\label{\detokenize{polo.widgets:polo.widgets.plate_inspector_widget.PlateInspectorWidget.show_current_plate}}\pysiglinewithargsret{\sphinxbfcode{\sphinxupquote{show\_current\_plate}}}{\emph{\DUrole{n}{next\_view}\DUrole{o}{=}\DUrole{default_value}{False}}, \emph{\DUrole{n}{prev\_view}\DUrole{o}{=}\DUrole{default_value}{False}}, \emph{\DUrole{n}{next\_date}\DUrole{o}{=}\DUrole{default_value}{False}}, \emph{\DUrole{n}{prev\_date}\DUrole{o}{=}\DUrole{default_value}{False}}, \emph{\DUrole{n}{alt\_spec}\DUrole{o}{=}\DUrole{default_value}{False}}}{}
Show the images belonging to the current plate view to the user.
\begin{quote}\begin{description}
\item[{Parameters}] \leavevmode\begin{itemize}
\item {} 
\sphinxstyleliteralstrong{\sphinxupquote{next\_date}} (\sphinxstyleliteralemphasis{\sphinxupquote{bool}}\sphinxstyleliteralemphasis{\sphinxupquote{, }}\sphinxstyleliteralemphasis{\sphinxupquote{optional}}) \textendash{} Flag, if True show equivalent images from future 
date, defaults to False

\item {} 
\sphinxstyleliteralstrong{\sphinxupquote{prev\_date}} (\sphinxstyleliteralemphasis{\sphinxupquote{bool}}\sphinxstyleliteralemphasis{\sphinxupquote{, }}\sphinxstyleliteralemphasis{\sphinxupquote{optional}}) \textendash{} Flag, if True shows equivalent images from past
date, defaults to False

\item {} 
\sphinxstyleliteralstrong{\sphinxupquote{alt\_spec}} (\sphinxstyleliteralemphasis{\sphinxupquote{bool}}\sphinxstyleliteralemphasis{\sphinxupquote{, }}\sphinxstyleliteralemphasis{\sphinxupquote{optional}}) \textendash{} Flag if True shows equivalent images in alternative
imaging spectrum, defaults to False

\end{itemize}

\end{description}\end{quote}

\end{fulllineitems}


\end{fulllineitems}



\subsubsection{polo.widgets.plate\_viewer module}
\label{\detokenize{polo.widgets:module-polo.widgets.plate_viewer}}\label{\detokenize{polo.widgets:polo-widgets-plate-viewer-module}}\index{module@\spxentry{module}!polo.widgets.plate\_viewer@\spxentry{polo.widgets.plate\_viewer}}\index{polo.widgets.plate\_viewer@\spxentry{polo.widgets.plate\_viewer}!module@\spxentry{module}}\index{PlateGraphicsItem (class in polo.widgets.plate\_viewer)@\spxentry{PlateGraphicsItem}\spxextra{class in polo.widgets.plate\_viewer}}

\begin{fulllineitems}
\phantomsection\label{\detokenize{polo.widgets:polo.widgets.plate_viewer.PlateGraphicsItem}}\pysiglinewithargsret{\sphinxbfcode{\sphinxupquote{class }}\sphinxcode{\sphinxupquote{polo.widgets.plate\_viewer.}}\sphinxbfcode{\sphinxupquote{PlateGraphicsItem}}}{\emph{\DUrole{n}{pixmap}}, \emph{\DUrole{n}{parent}\DUrole{o}{=}\DUrole{default_value}{None}}}{}
Bases: \sphinxcode{\sphinxupquote{PyQt5.QtWidgets.QGraphicsPixmapItem}}
\index{contextMenuEvent() (polo.widgets.plate\_viewer.PlateGraphicsItem method)@\spxentry{contextMenuEvent()}\spxextra{polo.widgets.plate\_viewer.PlateGraphicsItem method}}

\begin{fulllineitems}
\phantomsection\label{\detokenize{polo.widgets:polo.widgets.plate_viewer.PlateGraphicsItem.contextMenuEvent}}\pysiglinewithargsret{\sphinxbfcode{\sphinxupquote{contextMenuEvent}}}{\emph{\DUrole{n}{self}}, \emph{\DUrole{n}{QGraphicsSceneContextMenuEvent}}}{}
\end{fulllineitems}


\end{fulllineitems}

\index{plateViewer (class in polo.widgets.plate\_viewer)@\spxentry{plateViewer}\spxextra{class in polo.widgets.plate\_viewer}}

\begin{fulllineitems}
\phantomsection\label{\detokenize{polo.widgets:polo.widgets.plate_viewer.plateViewer}}\pysiglinewithargsret{\sphinxbfcode{\sphinxupquote{class }}\sphinxcode{\sphinxupquote{polo.widgets.plate\_viewer.}}\sphinxbfcode{\sphinxupquote{plateViewer}}}{\emph{\DUrole{n}{parent}}, \emph{\DUrole{n}{run}\DUrole{o}{=}\DUrole{default_value}{None}}, \emph{\DUrole{n}{images\_per\_page}\DUrole{o}{=}\DUrole{default_value}{24}}}{}
Bases: \sphinxcode{\sphinxupquote{PyQt5.QtWidgets.QGraphicsView}}
\index{\_get\_visible\_wells() (polo.widgets.plate\_viewer.plateViewer method)@\spxentry{\_get\_visible\_wells()}\spxextra{polo.widgets.plate\_viewer.plateViewer method}}

\begin{fulllineitems}
\phantomsection\label{\detokenize{polo.widgets:polo.widgets.plate_viewer.plateViewer._get_visible_wells}}\pysiglinewithargsret{\sphinxbfcode{\sphinxupquote{\_get\_visible\_wells}}}{\emph{\DUrole{n}{page}\DUrole{o}{=}\DUrole{default_value}{None}}}{}
Return indices of images that should be shown in the
current page. A page is equivalent to a subsection of a
larger screening plate.
\begin{quote}\begin{description}
\item[{Parameters}] \leavevmode
\sphinxstyleliteralstrong{\sphinxupquote{page}} (\sphinxstyleliteralemphasis{\sphinxupquote{int}}\sphinxstyleliteralemphasis{\sphinxupquote{, }}\sphinxstyleliteralemphasis{\sphinxupquote{optional}}) \textendash{} Page number to find images for, defaults to None

\item[{Yield}] \leavevmode
image index

\item[{Return type}] \leavevmode
int

\end{description}\end{quote}

\end{fulllineitems}

\index{\_make\_image\_label() (polo.widgets.plate\_viewer.plateViewer method)@\spxentry{\_make\_image\_label()}\spxextra{polo.widgets.plate\_viewer.plateViewer method}}

\begin{fulllineitems}
\phantomsection\label{\detokenize{polo.widgets:polo.widgets.plate_viewer.plateViewer._make_image_label}}\pysiglinewithargsret{\sphinxbfcode{\sphinxupquote{\_make\_image\_label}}}{\emph{\DUrole{n}{image}}, \emph{\DUrole{n}{label\_dict}}, \emph{\DUrole{n}{font\_size}\DUrole{o}{=}\DUrole{default_value}{35}}}{}
Private helper method for creating label strings to overlay onto
each image in the view.
\begin{quote}\begin{description}
\item[{Parameters}] \leavevmode\begin{itemize}
\item {} 
\sphinxstyleliteralstrong{\sphinxupquote{image}} ({\hyperref[\detokenize{polo.crystallography:polo.crystallography.image.Image}]{\sphinxcrossref{\sphinxstyleliteralemphasis{\sphinxupquote{Image}}}}}) \textendash{} Image to create label from

\item {} 
\sphinxstyleliteralstrong{\sphinxupquote{label\_dict}} (\sphinxstyleliteralemphasis{\sphinxupquote{dict}}) \textendash{} Dictionary of image attributes to include in the label

\item {} 
\sphinxstyleliteralstrong{\sphinxupquote{font\_size}} (\sphinxstyleliteralemphasis{\sphinxupquote{int}}\sphinxstyleliteralemphasis{\sphinxupquote{, }}\sphinxstyleliteralemphasis{\sphinxupquote{optional}}) \textendash{} Font size for the label, defaults to 35

\end{itemize}

\item[{Returns}] \leavevmode
QGraphicsTextItem with label text set

\item[{Return type}] \leavevmode
QGraphicsTextItem

\end{description}\end{quote}

\end{fulllineitems}

\index{\_set\_prerender\_info() (polo.widgets.plate\_viewer.plateViewer method)@\spxentry{\_set\_prerender\_info()}\spxextra{polo.widgets.plate\_viewer.plateViewer method}}

\begin{fulllineitems}
\phantomsection\label{\detokenize{polo.widgets:polo.widgets.plate_viewer.plateViewer._set_prerender_info}}\pysiglinewithargsret{\sphinxbfcode{\sphinxupquote{\_set\_prerender\_info}}}{\emph{\DUrole{n}{item}}, \emph{\DUrole{n}{image}}}{}
Private helper method that sets flags and the tooltip for
GraphicsItems before they are added to the GraphicsScene.
\begin{quote}\begin{description}
\item[{Parameters}] \leavevmode\begin{itemize}
\item {} 
\sphinxstyleliteralstrong{\sphinxupquote{item}} (\sphinxstyleliteralemphasis{\sphinxupquote{QGraphicsItem}}) \textendash{} GraphicsItem that is to be added to the scene

\item {} 
\sphinxstyleliteralstrong{\sphinxupquote{image}} ({\hyperref[\detokenize{polo.crystallography:polo.crystallography.image.Image}]{\sphinxcrossref{\sphinxstyleliteralemphasis{\sphinxupquote{Image}}}}}) \textendash{} Image who’s data will be used to label the GraphicsItem

\end{itemize}

\item[{Returns}] \leavevmode
QGraphicsItem

\item[{Return type}] \leavevmode
QGraphicsItem

\end{description}\end{quote}

\end{fulllineitems}

\index{aspect\_ratio() (polo.widgets.plate\_viewer.plateViewer property)@\spxentry{aspect\_ratio()}\spxextra{polo.widgets.plate\_viewer.plateViewer property}}

\begin{fulllineitems}
\phantomsection\label{\detokenize{polo.widgets:polo.widgets.plate_viewer.plateViewer.aspect_ratio}}\pysigline{\sphinxbfcode{\sphinxupquote{property }}\sphinxbfcode{\sphinxupquote{aspect\_ratio}}}
Current “best” aspect ratio for the view given the size of the
view and the number of images that need to be fit into the view.
\begin{quote}\begin{description}
\item[{Returns}] \leavevmode
Dimensions of the image grid, in images

\item[{Return type}] \leavevmode
tuple

\end{description}\end{quote}

\end{fulllineitems}

\index{changed\_images\_per\_page\_signal (polo.widgets.plate\_viewer.plateViewer attribute)@\spxentry{changed\_images\_per\_page\_signal}\spxextra{polo.widgets.plate\_viewer.plateViewer attribute}}

\begin{fulllineitems}
\phantomsection\label{\detokenize{polo.widgets:polo.widgets.plate_viewer.plateViewer.changed_images_per_page_signal}}\pysigline{\sphinxbfcode{\sphinxupquote{changed\_images\_per\_page\_signal}}}
\end{fulllineitems}

\index{changed\_page\_signal (polo.widgets.plate\_viewer.plateViewer attribute)@\spxentry{changed\_page\_signal}\spxextra{polo.widgets.plate\_viewer.plateViewer attribute}}

\begin{fulllineitems}
\phantomsection\label{\detokenize{polo.widgets:polo.widgets.plate_viewer.plateViewer.changed_page_signal}}\pysigline{\sphinxbfcode{\sphinxupquote{changed\_page\_signal}}}
\end{fulllineitems}

\index{current\_page() (polo.widgets.plate\_viewer.plateViewer property)@\spxentry{current\_page()}\spxextra{polo.widgets.plate\_viewer.plateViewer property}}

\begin{fulllineitems}
\phantomsection\label{\detokenize{polo.widgets:polo.widgets.plate_viewer.plateViewer.current_page}}\pysigline{\sphinxbfcode{\sphinxupquote{property }}\sphinxbfcode{\sphinxupquote{current\_page}}}
Current page
\begin{quote}\begin{description}
\item[{Returns}] \leavevmode
Current page

\item[{Return type}] \leavevmode
int

\end{description}\end{quote}

\end{fulllineitems}

\index{decolor\_all\_images() (polo.widgets.plate\_viewer.plateViewer method)@\spxentry{decolor\_all\_images()}\spxextra{polo.widgets.plate\_viewer.plateViewer method}}

\begin{fulllineitems}
\phantomsection\label{\detokenize{polo.widgets:polo.widgets.plate_viewer.plateViewer.decolor_all_images}}\pysiglinewithargsret{\sphinxbfcode{\sphinxupquote{decolor\_all\_images}}}{}{}
Removes all coloring from images in the \sphinxtitleref{\_scene} attribute.

\end{fulllineitems}

\index{emphasize\_all\_images() (polo.widgets.plate\_viewer.plateViewer method)@\spxentry{emphasize\_all\_images()}\spxextra{polo.widgets.plate\_viewer.plateViewer method}}

\begin{fulllineitems}
\phantomsection\label{\detokenize{polo.widgets:polo.widgets.plate_viewer.plateViewer.emphasize_all_images}}\pysiglinewithargsret{\sphinxbfcode{\sphinxupquote{emphasize\_all\_images}}}{}{}
Returns the opacity of all images in the \sphinxtitleref{\_scene} attribute
to 1, or fully opaque.

\end{fulllineitems}

\index{export\_current\_view() (polo.widgets.plate\_viewer.plateViewer method)@\spxentry{export\_current\_view()}\spxextra{polo.widgets.plate\_viewer.plateViewer method}}

\begin{fulllineitems}
\phantomsection\label{\detokenize{polo.widgets:polo.widgets.plate_viewer.plateViewer.export_current_view}}\pysiglinewithargsret{\sphinxbfcode{\sphinxupquote{export\_current\_view}}}{\emph{\DUrole{n}{save\_path}\DUrole{o}{=}\DUrole{default_value}{None}}}{}
Exports the current content of the QGraphicsScene \sphinxtitleref{\_scene} attribute
to a png file.
\begin{quote}\begin{description}
\item[{Parameters}] \leavevmode
\sphinxstyleliteralstrong{\sphinxupquote{save\_path}} (\sphinxstyleliteralemphasis{\sphinxupquote{str}}\sphinxstyleliteralemphasis{\sphinxupquote{ or }}\sphinxstyleliteralemphasis{\sphinxupquote{Path}}\sphinxstyleliteralemphasis{\sphinxupquote{, }}\sphinxstyleliteralemphasis{\sphinxupquote{optional}}) \textendash{} Path to save the image to, defaults to None. If kept 
as None opens a QFileDialog to get a save file path.

\end{description}\end{quote}

\end{fulllineitems}

\index{fitInView() (polo.widgets.plate\_viewer.plateViewer method)@\spxentry{fitInView()}\spxextra{polo.widgets.plate\_viewer.plateViewer method}}

\begin{fulllineitems}
\phantomsection\label{\detokenize{polo.widgets:polo.widgets.plate_viewer.plateViewer.fitInView}}\pysiglinewithargsret{\sphinxbfcode{\sphinxupquote{fitInView}}}{\emph{\DUrole{n}{scene}}, \emph{\DUrole{n}{preserve\_aspect}\DUrole{o}{=}\DUrole{default_value}{False}}}{}
Fit items added to \sphinxtitleref{\_scene} attribute into the available
display space.
\begin{quote}\begin{description}
\item[{Parameters}] \leavevmode\begin{itemize}
\item {} 
\sphinxstyleliteralstrong{\sphinxupquote{scene}} (\sphinxstyleliteralemphasis{\sphinxupquote{QGraphicsScene}}) \textendash{} QGraphicsScene to fit

\item {} 
\sphinxstyleliteralstrong{\sphinxupquote{preserve\_aspect}} (\sphinxstyleliteralemphasis{\sphinxupquote{bool}}\sphinxstyleliteralemphasis{\sphinxupquote{, }}\sphinxstyleliteralemphasis{\sphinxupquote{optional}}) \textendash{} If True, preserves the aspect ratio of
item is the scene, defaults to False

\end{itemize}

\end{description}\end{quote}

\end{fulllineitems}

\index{images\_per\_page() (polo.widgets.plate\_viewer.plateViewer property)@\spxentry{images\_per\_page()}\spxextra{polo.widgets.plate\_viewer.plateViewer property}}

\begin{fulllineitems}
\phantomsection\label{\detokenize{polo.widgets:polo.widgets.plate_viewer.plateViewer.images_per_page}}\pysigline{\sphinxbfcode{\sphinxupquote{property }}\sphinxbfcode{\sphinxupquote{images\_per\_page}}}
Number of images in the current page.
\begin{quote}\begin{description}
\item[{Returns}] \leavevmode
Number of images

\item[{Return type}] \leavevmode
int

\end{description}\end{quote}

\end{fulllineitems}

\index{pop\_out\_selected\_well() (polo.widgets.plate\_viewer.plateViewer method)@\spxentry{pop\_out\_selected\_well()}\spxextra{polo.widgets.plate\_viewer.plateViewer method}}

\begin{fulllineitems}
\phantomsection\label{\detokenize{polo.widgets:polo.widgets.plate_viewer.plateViewer.pop_out_selected_well}}\pysiglinewithargsret{\sphinxbfcode{\sphinxupquote{pop\_out\_selected\_well}}}{}{}
Helper method to handle image selection and open an ImagePopDialog
that displays the selected image in a pop out view.

\end{fulllineitems}

\index{run() (polo.widgets.plate\_viewer.plateViewer property)@\spxentry{run()}\spxextra{polo.widgets.plate\_viewer.plateViewer property}}

\begin{fulllineitems}
\phantomsection\label{\detokenize{polo.widgets:polo.widgets.plate_viewer.plateViewer.run}}\pysigline{\sphinxbfcode{\sphinxupquote{property }}\sphinxbfcode{\sphinxupquote{run}}}
The current run being displayed.
\begin{quote}\begin{description}
\item[{Returns}] \leavevmode
The current run

\item[{Return type}] \leavevmode
{\hyperref[\detokenize{polo.crystallography:polo.crystallography.run.HWIRun}]{\sphinxcrossref{HWIRun}}}

\end{description}\end{quote}

\end{fulllineitems}

\index{set\_scene\_colors\_from\_filters() (polo.widgets.plate\_viewer.plateViewer method)@\spxentry{set\_scene\_colors\_from\_filters()}\spxextra{polo.widgets.plate\_viewer.plateViewer method}}

\begin{fulllineitems}
\phantomsection\label{\detokenize{polo.widgets:polo.widgets.plate_viewer.plateViewer.set_scene_colors_from_filters}}\pysiglinewithargsret{\sphinxbfcode{\sphinxupquote{set\_scene\_colors\_from\_filters}}}{\emph{\DUrole{n}{color\_mapping}}, \emph{\DUrole{n}{strength}\DUrole{o}{=}\DUrole{default_value}{0.5}}, \emph{\DUrole{n}{human}\DUrole{o}{=}\DUrole{default_value}{False}}}{}
Set the color of images based on their current classifications. Very similar
to \sphinxcode{\sphinxupquote{set\_opacity\_from\_filters()}}.
Images can be colored by their MARCO or human classification.
\begin{quote}\begin{description}
\item[{Parameters}] \leavevmode\begin{itemize}
\item {} 
\sphinxstyleliteralstrong{\sphinxupquote{color\_mapping}} (\sphinxstyleliteralemphasis{\sphinxupquote{dict}}) \textendash{} Dictionary that maps image classifications to QColors

\item {} 
\sphinxstyleliteralstrong{\sphinxupquote{strength}} (\sphinxstyleliteralemphasis{\sphinxupquote{float}}\sphinxstyleliteralemphasis{\sphinxupquote{, }}\sphinxstyleliteralemphasis{\sphinxupquote{optional}}) \textendash{} Image color strength, defaults to 0.5

\item {} 
\sphinxstyleliteralstrong{\sphinxupquote{human}} (\sphinxstyleliteralemphasis{\sphinxupquote{bool}}\sphinxstyleliteralemphasis{\sphinxupquote{, }}\sphinxstyleliteralemphasis{\sphinxupquote{optional}}) \textendash{} If True, use the human classification to color images,
defaults to False

\end{itemize}

\end{description}\end{quote}

\end{fulllineitems}

\index{set\_scene\_opacity\_from\_filters() (polo.widgets.plate\_viewer.plateViewer method)@\spxentry{set\_scene\_opacity\_from\_filters()}\spxextra{polo.widgets.plate\_viewer.plateViewer method}}

\begin{fulllineitems}
\phantomsection\label{\detokenize{polo.widgets:polo.widgets.plate_viewer.plateViewer.set_scene_opacity_from_filters}}\pysiglinewithargsret{\sphinxbfcode{\sphinxupquote{set\_scene\_opacity\_from\_filters}}}{\emph{\DUrole{n}{image\_types}}, \emph{\DUrole{n}{human}\DUrole{o}{=}\DUrole{default_value}{False}}, \emph{\DUrole{n}{marco}\DUrole{o}{=}\DUrole{default_value}{False}}, \emph{\DUrole{n}{favorite}\DUrole{o}{=}\DUrole{default_value}{False}}, \emph{\DUrole{n}{filtered\_opacity}\DUrole{o}{=}\DUrole{default_value}{0.2}}}{}
Sets the opacity of all items in the current scene (\sphinxtitleref{\_scene} attribute)
based on image filtering criteria. Allows for highlighting images that
meet specific qualifications such has having a MARCO classification of
crystals. Images that do not meet the set filter requirements will have
their opacity set to the value specificed by the \sphinxtitleref{filtered\_opacity}
argument.
\begin{quote}\begin{description}
\item[{Parameters}] \leavevmode\begin{itemize}
\item {} 
\sphinxstyleliteralstrong{\sphinxupquote{image\_types}} (\sphinxstyleliteralemphasis{\sphinxupquote{set of list}}) \textendash{} Image classifications to select for.

\item {} 
\sphinxstyleliteralstrong{\sphinxupquote{human}} (\sphinxstyleliteralemphasis{\sphinxupquote{bool}}\sphinxstyleliteralemphasis{\sphinxupquote{, }}\sphinxstyleliteralemphasis{\sphinxupquote{optional}}) \textendash{} If True, use human classification to determine image
classification, defaults to False

\item {} 
\sphinxstyleliteralstrong{\sphinxupquote{marco}} (\sphinxstyleliteralemphasis{\sphinxupquote{bool}}\sphinxstyleliteralemphasis{\sphinxupquote{, }}\sphinxstyleliteralemphasis{\sphinxupquote{optional}}) \textendash{} If True, use MARCO classification to determine image
classification , defaults to False

\item {} 
\sphinxstyleliteralstrong{\sphinxupquote{favorite}} (\sphinxstyleliteralemphasis{\sphinxupquote{bool}}\sphinxstyleliteralemphasis{\sphinxupquote{, }}\sphinxstyleliteralemphasis{\sphinxupquote{optional}}) \textendash{} If True, image must be favorited to be
selected, defaults to False

\item {} 
\sphinxstyleliteralstrong{\sphinxupquote{filtered\_opacity}} (\sphinxstyleliteralemphasis{\sphinxupquote{float}}\sphinxstyleliteralemphasis{\sphinxupquote{, }}\sphinxstyleliteralemphasis{\sphinxupquote{optional}}) \textendash{} Set the opacity for images that do not need the
filtering requirements, defaults to 0.2

\end{itemize}

\end{description}\end{quote}

\end{fulllineitems}

\index{subgrid\_dict (polo.widgets.plate\_viewer.plateViewer attribute)@\spxentry{subgrid\_dict}\spxextra{polo.widgets.plate\_viewer.plateViewer attribute}}

\begin{fulllineitems}
\phantomsection\label{\detokenize{polo.widgets:polo.widgets.plate_viewer.plateViewer.subgrid_dict}}\pysigline{\sphinxbfcode{\sphinxupquote{subgrid\_dict}}\sphinxbfcode{\sphinxupquote{ = \{16: (4, 4), 64: (8, 8), 96: (8, 12), 1536: (32, 48)\}}}}
\end{fulllineitems}

\index{tile\_images\_onto\_scene() (polo.widgets.plate\_viewer.plateViewer method)@\spxentry{tile\_images\_onto\_scene()}\spxextra{polo.widgets.plate\_viewer.plateViewer method}}

\begin{fulllineitems}
\phantomsection\label{\detokenize{polo.widgets:polo.widgets.plate_viewer.plateViewer.tile_images_onto_scene}}\pysiglinewithargsret{\sphinxbfcode{\sphinxupquote{tile\_images\_onto\_scene}}}{\emph{\DUrole{n}{label\_dict}\DUrole{o}{=}\DUrole{default_value}{\{\}}}}{}
Calculates images that should be shown based on the current page
and the number of images per page. Then tiles these images into a grid,
adding them to \sphinxtitleref{\_scene} attribute.
\begin{quote}\begin{description}
\item[{Parameters}] \leavevmode
\sphinxstyleliteralstrong{\sphinxupquote{label\_dict}} (\sphinxstyleliteralemphasis{\sphinxupquote{dict}}\sphinxstyleliteralemphasis{\sphinxupquote{, }}\sphinxstyleliteralemphasis{\sphinxupquote{optional}}) \textendash{} Dictionary of Image attributes to pass along to
{\hyperref[\detokenize{polo.widgets:polo.widgets.plate_viewer.plateViewer._make_image_label}]{\sphinxcrossref{\sphinxcode{\sphinxupquote{\_make\_image\_label()}}}}}
to create image labels, defaults to \{\}

\end{description}\end{quote}

\end{fulllineitems}

\index{total\_pages() (polo.widgets.plate\_viewer.plateViewer property)@\spxentry{total\_pages()}\spxextra{polo.widgets.plate\_viewer.plateViewer property}}

\begin{fulllineitems}
\phantomsection\label{\detokenize{polo.widgets:polo.widgets.plate_viewer.plateViewer.total_pages}}\pysigline{\sphinxbfcode{\sphinxupquote{property }}\sphinxbfcode{\sphinxupquote{total\_pages}}}
Total number of pages based on the number of images per page
and the number of images in the current run.
\begin{quote}\begin{description}
\item[{Returns}] \leavevmode
Number of pages

\item[{Return type}] \leavevmode
int

\end{description}\end{quote}

\end{fulllineitems}

\index{view\_dims() (polo.widgets.plate\_viewer.plateViewer property)@\spxentry{view\_dims()}\spxextra{polo.widgets.plate\_viewer.plateViewer property}}

\begin{fulllineitems}
\phantomsection\label{\detokenize{polo.widgets:polo.widgets.plate_viewer.plateViewer.view_dims}}\pysigline{\sphinxbfcode{\sphinxupquote{property }}\sphinxbfcode{\sphinxupquote{view\_dims}}}
Current view dimensions in pixels.
\begin{quote}\begin{description}
\item[{Returns}] \leavevmode
Width and height of the view

\item[{Return type}] \leavevmode
tuple

\end{description}\end{quote}

\end{fulllineitems}

\index{well\_index\_to\_subgrid() (polo.widgets.plate\_viewer.plateViewer static method)@\spxentry{well\_index\_to\_subgrid()}\spxextra{polo.widgets.plate\_viewer.plateViewer static method}}

\begin{fulllineitems}
\phantomsection\label{\detokenize{polo.widgets:polo.widgets.plate_viewer.plateViewer.well_index_to_subgrid}}\pysiglinewithargsret{\sphinxbfcode{\sphinxupquote{static }}\sphinxbfcode{\sphinxupquote{well\_index\_to\_subgrid}}}{\emph{\DUrole{n}{i}}, \emph{\DUrole{n}{c\_r}}, \emph{\DUrole{n}{c\_c}}, \emph{\DUrole{n}{p\_r}}, \emph{\DUrole{n}{p\_c}}}{}
Find the linear index of the subgrid that a particular index belongs to
within a larger grid. For example, ou are given a list of length 16.
The list is reshaped into a 4 x 4
2D list. We divide the new grid into 4 quadrants each 2 X 2 and label them
with an index (0, 1, 2, 3). Given an index of the original list we want to
find the subgrid it belongs to.
\begin{quote}\begin{description}
\item[{Parameters}] \leavevmode\begin{itemize}
\item {} 
\sphinxstyleliteralstrong{\sphinxupquote{i}} (\sphinxstyleliteralemphasis{\sphinxupquote{int}}) \textendash{} Index of point to locate in the 1D list

\item {} 
\sphinxstyleliteralstrong{\sphinxupquote{c\_r}} (\sphinxstyleliteralemphasis{\sphinxupquote{int}}) \textendash{} Number of rows in each subgrid

\item {} 
\sphinxstyleliteralstrong{\sphinxupquote{c\_c}} (\sphinxstyleliteralemphasis{\sphinxupquote{int}}) \textendash{} Number of columns in each subgrid

\item {} 
\sphinxstyleliteralstrong{\sphinxupquote{p\_r}} (\sphinxstyleliteralemphasis{\sphinxupquote{int}}) \textendash{} Number or rows in the entire grid

\item {} 
\sphinxstyleliteralstrong{\sphinxupquote{p\_c}} (\sphinxstyleliteralemphasis{\sphinxupquote{int}}) \textendash{} Number of columns in the entire grid

\end{itemize}

\item[{Returns}] \leavevmode
Index of the subgrid the index \sphinxtitleref{i} belongs to

\item[{Return type}] \leavevmode
int

\end{description}\end{quote}

\end{fulllineitems}

\index{wheelEvent() (polo.widgets.plate\_viewer.plateViewer method)@\spxentry{wheelEvent()}\spxextra{polo.widgets.plate\_viewer.plateViewer method}}

\begin{fulllineitems}
\phantomsection\label{\detokenize{polo.widgets:polo.widgets.plate_viewer.plateViewer.wheelEvent}}\pysiglinewithargsret{\sphinxbfcode{\sphinxupquote{wheelEvent}}}{\emph{\DUrole{n}{event}}}{}
Handle Qt wheelEvents by setting the \sphinxtitleref{\_zoom} attribute. Allows users
to zoom in and out of the current view.
\begin{quote}\begin{description}
\item[{Parameters}] \leavevmode
\sphinxstyleliteralstrong{\sphinxupquote{event}} (\sphinxstyleliteralemphasis{\sphinxupquote{QEvent}}) \textendash{} event

\end{description}\end{quote}

\end{fulllineitems}


\end{fulllineitems}



\subsubsection{polo.widgets.plate\_visualizer module}
\label{\detokenize{polo.widgets:module-polo.widgets.plate_visualizer}}\label{\detokenize{polo.widgets:polo-widgets-plate-visualizer-module}}\index{module@\spxentry{module}!polo.widgets.plate\_visualizer@\spxentry{polo.widgets.plate\_visualizer}}\index{polo.widgets.plate\_visualizer@\spxentry{polo.widgets.plate\_visualizer}!module@\spxentry{module}}\index{PlateVisualizer (class in polo.widgets.plate\_visualizer)@\spxentry{PlateVisualizer}\spxextra{class in polo.widgets.plate\_visualizer}}

\begin{fulllineitems}
\phantomsection\label{\detokenize{polo.widgets:polo.widgets.plate_visualizer.PlateVisualizer}}\pysiglinewithargsret{\sphinxbfcode{\sphinxupquote{class }}\sphinxcode{\sphinxupquote{polo.widgets.plate\_visualizer.}}\sphinxbfcode{\sphinxupquote{PlateVisualizer}}}{\emph{\DUrole{n}{parent}\DUrole{o}{=}\DUrole{default_value}{None}}}{}
Bases: \sphinxcode{\sphinxupquote{PyQt5.QtWidgets.QGraphicsView}}

The PlateVisualizer is a small widget to assist users understand
what part of the screening plate they are currently viewing. It renders
a grid of rectangles (blocks) that each represent one view (page) in the
\sphinxtitleref{PlateInspector} widget. The page that is currently being viewed is 
highlighted to show the user what part of the plate they are looking at.
\begin{quote}\begin{description}
\item[{Parameters}] \leavevmode
\sphinxstyleliteralstrong{\sphinxupquote{parent}} (\sphinxstyleliteralemphasis{\sphinxupquote{QWidget}}\sphinxstyleliteralemphasis{\sphinxupquote{, }}\sphinxstyleliteralemphasis{\sphinxupquote{optional}}) \textendash{} Parent widget, defaults to None

\end{description}\end{quote}
\index{\_block\_size() (polo.widgets.plate\_visualizer.PlateVisualizer method)@\spxentry{\_block\_size()}\spxextra{polo.widgets.plate\_visualizer.PlateVisualizer method}}

\begin{fulllineitems}
\phantomsection\label{\detokenize{polo.widgets:polo.widgets.plate_visualizer.PlateVisualizer._block_size}}\pysiglinewithargsret{\sphinxbfcode{\sphinxupquote{\_block\_size}}}{\emph{\DUrole{n}{x}}, \emph{\DUrole{n}{y}}}{}
Private method to calculate the size of individual blocks
to render in the QGraphicsView.
\begin{quote}\begin{description}
\item[{Parameters}] \leavevmode\begin{itemize}
\item {} 
\sphinxstyleliteralstrong{\sphinxupquote{x}} (\sphinxstyleliteralemphasis{\sphinxupquote{int}}) \textendash{} Length of x\sphinxhyphen{}axis in blocks

\item {} 
\sphinxstyleliteralstrong{\sphinxupquote{y}} (\sphinxstyleliteralemphasis{\sphinxupquote{int}}) \textendash{} Length of y\sphinxhyphen{}axis in blocks

\end{itemize}

\item[{Returns}] \leavevmode
tuple, length of block x\sphinxhyphen{}axis in pixels,
length of block y\sphinxhyphen{}axis in pixels

\item[{Return type}] \leavevmode
tuple

\end{description}\end{quote}

\end{fulllineitems}

\index{\_handle\_block\_selection() (polo.widgets.plate\_visualizer.PlateVisualizer method)@\spxentry{\_handle\_block\_selection()}\spxextra{polo.widgets.plate\_visualizer.PlateVisualizer method}}

\begin{fulllineitems}
\phantomsection\label{\detokenize{polo.widgets:polo.widgets.plate_visualizer.PlateVisualizer._handle_block_selection}}\pysiglinewithargsret{\sphinxbfcode{\sphinxupquote{\_handle\_block\_selection}}}{}{}
Private helper method to handle when a user selects a block.
In theory should open the view that the selected block corresponds
to but currently having some issues with this causing segmentation
faults so it is disabled for now.

\end{fulllineitems}

\index{\_highlight\_block() (polo.widgets.plate\_visualizer.PlateVisualizer method)@\spxentry{\_highlight\_block()}\spxextra{polo.widgets.plate\_visualizer.PlateVisualizer method}}

\begin{fulllineitems}
\phantomsection\label{\detokenize{polo.widgets:polo.widgets.plate_visualizer.PlateVisualizer._highlight_block}}\pysiglinewithargsret{\sphinxbfcode{\sphinxupquote{\_highlight\_block}}}{\emph{\DUrole{n}{block}}}{}
Private method that highlights a block in the
QGraphicsScene.
\begin{quote}\begin{description}
\item[{Parameters}] \leavevmode
\sphinxstyleliteralstrong{\sphinxupquote{block}} (\sphinxstyleliteralemphasis{\sphinxupquote{QGraphicsRectItem}}) \textendash{} Block to highlight

\end{description}\end{quote}

\end{fulllineitems}

\index{block\_dims() (polo.widgets.plate\_visualizer.PlateVisualizer static method)@\spxentry{block\_dims()}\spxextra{polo.widgets.plate\_visualizer.PlateVisualizer static method}}

\begin{fulllineitems}
\phantomsection\label{\detokenize{polo.widgets:polo.widgets.plate_visualizer.PlateVisualizer.block_dims}}\pysiglinewithargsret{\sphinxbfcode{\sphinxupquote{static }}\sphinxbfcode{\sphinxupquote{block\_dims}}}{\emph{\DUrole{n}{plate\_x}}, \emph{\DUrole{n}{plate\_y}}, \emph{\DUrole{n}{grid\_x}}, \emph{\DUrole{n}{grid\_y}}}{}
Helper method to calculate the size of plate section
blocks
\begin{quote}\begin{description}
\item[{Parameters}] \leavevmode\begin{itemize}
\item {} 
\sphinxstyleliteralstrong{\sphinxupquote{plate\_x}} (\sphinxstyleliteralemphasis{\sphinxupquote{int}}) \textendash{} Number of wells plate has on its x axis

\item {} 
\sphinxstyleliteralstrong{\sphinxupquote{plate\_y}} (\sphinxstyleliteralemphasis{\sphinxupquote{int}}) \textendash{} Number of wells plate has on it s y axis

\item {} 
\sphinxstyleliteralstrong{\sphinxupquote{grid\_x}} (\sphinxstyleliteralemphasis{\sphinxupquote{int}}) \textendash{} Number of wells in the subgrid on its x axis

\item {} 
\sphinxstyleliteralstrong{\sphinxupquote{grid\_y}} (\sphinxstyleliteralemphasis{\sphinxupquote{int}}) \textendash{} Number of wells in the subgrid on its y axis

\end{itemize}

\item[{Returns}] \leavevmode
tuple, first item being length of x axis in
blocks and second being length of y axis in blocks

\item[{Return type}] \leavevmode
tuple

\end{description}\end{quote}

\end{fulllineitems}

\index{default\_brush (polo.widgets.plate\_visualizer.PlateVisualizer attribute)@\spxentry{default\_brush}\spxextra{polo.widgets.plate\_visualizer.PlateVisualizer attribute}}

\begin{fulllineitems}
\phantomsection\label{\detokenize{polo.widgets:polo.widgets.plate_visualizer.PlateVisualizer.default_brush}}\pysigline{\sphinxbfcode{\sphinxupquote{default\_brush}}\sphinxbfcode{\sphinxupquote{ = \textless{}PyQt5.QtGui.QBrush object\textgreater{}}}}
\end{fulllineitems}

\index{default\_pen (polo.widgets.plate\_visualizer.PlateVisualizer attribute)@\spxentry{default\_pen}\spxextra{polo.widgets.plate\_visualizer.PlateVisualizer attribute}}

\begin{fulllineitems}
\phantomsection\label{\detokenize{polo.widgets:polo.widgets.plate_visualizer.PlateVisualizer.default_pen}}\pysigline{\sphinxbfcode{\sphinxupquote{default\_pen}}\sphinxbfcode{\sphinxupquote{ = \textless{}PyQt5.QtGui.QPen object\textgreater{}}}}
\end{fulllineitems}

\index{plate\_size (polo.widgets.plate\_visualizer.PlateVisualizer attribute)@\spxentry{plate\_size}\spxextra{polo.widgets.plate\_visualizer.PlateVisualizer attribute}}

\begin{fulllineitems}
\phantomsection\label{\detokenize{polo.widgets:polo.widgets.plate_visualizer.PlateVisualizer.plate_size}}\pysigline{\sphinxbfcode{\sphinxupquote{plate\_size}}\sphinxbfcode{\sphinxupquote{ = (32, 48)}}}
\end{fulllineitems}

\index{plate\_view\_requested (polo.widgets.plate\_visualizer.PlateVisualizer attribute)@\spxentry{plate\_view\_requested}\spxextra{polo.widgets.plate\_visualizer.PlateVisualizer attribute}}

\begin{fulllineitems}
\phantomsection\label{\detokenize{polo.widgets:polo.widgets.plate_visualizer.PlateVisualizer.plate_view_requested}}\pysigline{\sphinxbfcode{\sphinxupquote{plate\_view\_requested}}}
\end{fulllineitems}

\index{selected\_brush (polo.widgets.plate\_visualizer.PlateVisualizer attribute)@\spxentry{selected\_brush}\spxextra{polo.widgets.plate\_visualizer.PlateVisualizer attribute}}

\begin{fulllineitems}
\phantomsection\label{\detokenize{polo.widgets:polo.widgets.plate_visualizer.PlateVisualizer.selected_brush}}\pysigline{\sphinxbfcode{\sphinxupquote{selected\_brush}}\sphinxbfcode{\sphinxupquote{ = \textless{}PyQt5.QtGui.QBrush object\textgreater{}}}}
\end{fulllineitems}

\index{set\_selected\_block() (polo.widgets.plate\_visualizer.PlateVisualizer method)@\spxentry{set\_selected\_block()}\spxextra{polo.widgets.plate\_visualizer.PlateVisualizer method}}

\begin{fulllineitems}
\phantomsection\label{\detokenize{polo.widgets:polo.widgets.plate_visualizer.PlateVisualizer.set_selected_block}}\pysiglinewithargsret{\sphinxbfcode{\sphinxupquote{set\_selected\_block}}}{\emph{\DUrole{n}{block\_id}}}{}
Sets the currently selected block based on its ID.
\begin{quote}\begin{description}
\item[{Parameters}] \leavevmode
\sphinxstyleliteralstrong{\sphinxupquote{block\_id}} (\sphinxstyleliteralemphasis{\sphinxupquote{int}}) \textendash{} Block ID

\end{description}\end{quote}

\end{fulllineitems}

\index{setup\_view() (polo.widgets.plate\_visualizer.PlateVisualizer method)@\spxentry{setup\_view()}\spxextra{polo.widgets.plate\_visualizer.PlateVisualizer method}}

\begin{fulllineitems}
\phantomsection\label{\detokenize{polo.widgets:polo.widgets.plate_visualizer.PlateVisualizer.setup_view}}\pysiglinewithargsret{\sphinxbfcode{\sphinxupquote{setup\_view}}}{\emph{\DUrole{n}{grid\_cords}}, \emph{\DUrole{n}{plate\_size}\DUrole{o}{=}\DUrole{default_value}{None}}}{}
set up the intail view based on the current plate
size (normally 32 * 48 wells for 1536 well plate) and
the subgrid size in wells.
\begin{quote}\begin{description}
\item[{Parameters}] \leavevmode\begin{itemize}
\item {} 
\sphinxstyleliteralstrong{\sphinxupquote{grid\_cords}} (\sphinxstyleliteralemphasis{\sphinxupquote{tuple}}) \textendash{} Subgrid size tuple (x, y) in wells

\item {} 
\sphinxstyleliteralstrong{\sphinxupquote{plate\_size}} (\sphinxstyleliteralemphasis{\sphinxupquote{tuple}}\sphinxstyleliteralemphasis{\sphinxupquote{, }}\sphinxstyleliteralemphasis{\sphinxupquote{optional}}) \textendash{} Size of entire plate (x, y) in wells, defaults to None.
If None used the default 1536 well plate size of
32 * 48.

\end{itemize}

\end{description}\end{quote}

\end{fulllineitems}


\end{fulllineitems}



\subsubsection{polo.widgets.run\_organizer module}
\label{\detokenize{polo.widgets:module-polo.widgets.run_organizer}}\label{\detokenize{polo.widgets:polo-widgets-run-organizer-module}}\index{module@\spxentry{module}!polo.widgets.run\_organizer@\spxentry{polo.widgets.run\_organizer}}\index{polo.widgets.run\_organizer@\spxentry{polo.widgets.run\_organizer}!module@\spxentry{module}}\index{RunOrganizer (class in polo.widgets.run\_organizer)@\spxentry{RunOrganizer}\spxextra{class in polo.widgets.run\_organizer}}

\begin{fulllineitems}
\phantomsection\label{\detokenize{polo.widgets:polo.widgets.run_organizer.RunOrganizer}}\pysiglinewithargsret{\sphinxbfcode{\sphinxupquote{class }}\sphinxcode{\sphinxupquote{polo.widgets.run\_organizer.}}\sphinxbfcode{\sphinxupquote{RunOrganizer}}}{\emph{\DUrole{n}{parent}\DUrole{o}{=}\DUrole{default_value}{None}}, \emph{\DUrole{n}{auto\_link\_runs}\DUrole{o}{=}\DUrole{default_value}{True}}}{}
Bases: \sphinxcode{\sphinxupquote{PyQt5.QtWidgets.QWidget}}

Widget for organizing and importing runs into Polo.
\begin{quote}\begin{description}
\item[{Parameters}] \leavevmode\begin{itemize}
\item {} 
\sphinxstyleliteralstrong{\sphinxupquote{parent}} (\sphinxstyleliteralemphasis{\sphinxupquote{QWidget}}\sphinxstyleliteralemphasis{\sphinxupquote{, }}\sphinxstyleliteralemphasis{\sphinxupquote{optional}}) \textendash{} Parent widget, defaults to None

\item {} 
\sphinxstyleliteralstrong{\sphinxupquote{auto\_link\_runs}} (\sphinxstyleliteralemphasis{\sphinxupquote{bool}}\sphinxstyleliteralemphasis{\sphinxupquote{, }}\sphinxstyleliteralemphasis{\sphinxupquote{optional}}) \textendash{} If True runs are automatically
linked as they are loaded in, defaults to True

\end{itemize}

\end{description}\end{quote}
\index{\_add\_run\_from\_directory() (polo.widgets.run\_organizer.RunOrganizer method)@\spxentry{\_add\_run\_from\_directory()}\spxextra{polo.widgets.run\_organizer.RunOrganizer method}}

\begin{fulllineitems}
\phantomsection\label{\detokenize{polo.widgets:polo.widgets.run_organizer.RunOrganizer._add_run_from_directory}}\pysiglinewithargsret{\sphinxbfcode{\sphinxupquote{\_add\_run\_from\_directory}}}{\emph{\DUrole{n}{dir\_path}}}{}
Private method to add a run to the runTree from a directory path.
\begin{quote}\begin{description}
\item[{Parameters}] \leavevmode
\sphinxstyleliteralstrong{\sphinxupquote{dir\_path}} (\sphinxstyleliteralemphasis{\sphinxupquote{str}}\sphinxstyleliteralemphasis{\sphinxupquote{ or }}\sphinxstyleliteralemphasis{\sphinxupquote{Path}}) \textendash{} Path to directory to import

\item[{Returns}] \leavevmode
Run or HWIRun created from directory if successful

\item[{Return type}] \leavevmode
{\hyperref[\detokenize{polo.crystallography:polo.crystallography.run.Run}]{\sphinxcrossref{Run}}} or {\hyperref[\detokenize{polo.crystallography:polo.crystallography.run.HWIRun}]{\sphinxcrossref{HWIRun}}}

\end{description}\end{quote}

\end{fulllineitems}

\index{\_add\_runs\_to\_tree() (polo.widgets.run\_organizer.RunOrganizer method)@\spxentry{\_add\_runs\_to\_tree()}\spxextra{polo.widgets.run\_organizer.RunOrganizer method}}

\begin{fulllineitems}
\phantomsection\label{\detokenize{polo.widgets:polo.widgets.run_organizer.RunOrganizer._add_runs_to_tree}}\pysiglinewithargsret{\sphinxbfcode{\sphinxupquote{\_add\_runs\_to\_tree}}}{\emph{\DUrole{n}{runs}}}{}
Private method to add a set of runs to the runTree.
\begin{quote}\begin{description}
\item[{Parameters}] \leavevmode
\sphinxstyleliteralstrong{\sphinxupquote{runs}} (\sphinxstyleliteralemphasis{\sphinxupquote{list}}) \textendash{} List of runs to add to the runTree

\end{description}\end{quote}

\end{fulllineitems}

\index{\_check\_for\_existing\_backup() (polo.widgets.run\_organizer.RunOrganizer method)@\spxentry{\_check\_for\_existing\_backup()}\spxextra{polo.widgets.run\_organizer.RunOrganizer method}}

\begin{fulllineitems}
\phantomsection\label{\detokenize{polo.widgets:polo.widgets.run_organizer.RunOrganizer._check_for_existing_backup}}\pysiglinewithargsret{\sphinxbfcode{\sphinxupquote{\_check\_for\_existing\_backup}}}{\emph{\DUrole{n}{run}}}{}
Check the directory specified by the \sphinxtitleref{BACKUP\_DIR} constant for
a backup mso file that matches the run passed through the \sphinxtitleref{run}
argument. Run’s are matched to mso backups by their run name so it
the user has renamed their run after the backup is saved it will not
be found.

See {\hyperref[\detokenize{polo.widgets:polo.widgets.run_organizer.RunOrganizer.backup_classifications}]{\sphinxcrossref{\sphinxcode{\sphinxupquote{backup\_classifications()}}}}}
for details on how the mso files are written.
\begin{quote}\begin{description}
\item[{Parameters}] \leavevmode
\sphinxstyleliteralstrong{\sphinxupquote{run}} ({\hyperref[\detokenize{polo.crystallography:polo.crystallography.run.HWIRun}]{\sphinxcrossref{\sphinxstyleliteralemphasis{\sphinxupquote{HWIRun}}}}}) \textendash{} Run to search for mso backup with

\item[{Returns}] \leavevmode
Path to mso backup if one exists that matches the \sphinxtitleref{run}, else
return None

\item[{Return type}] \leavevmode
str or None

\end{description}\end{quote}

\end{fulllineitems}

\index{\_clear\_current\_run() (polo.widgets.run\_organizer.RunOrganizer method)@\spxentry{\_clear\_current\_run()}\spxextra{polo.widgets.run\_organizer.RunOrganizer method}}

\begin{fulllineitems}
\phantomsection\label{\detokenize{polo.widgets:polo.widgets.run_organizer.RunOrganizer._clear_current_run}}\pysiglinewithargsret{\sphinxbfcode{\sphinxupquote{\_clear\_current\_run}}}{\emph{\DUrole{n}{run\_list}}}{}
Clear out the current run from other widgets by emiting a
\sphinxtitleref{opening\_run} signal with a list that does not contain
a Run or HWIRun object.
\begin{quote}\begin{description}
\item[{Parameters}] \leavevmode
\sphinxstyleliteralstrong{\sphinxupquote{run\_list}} (\sphinxstyleliteralemphasis{\sphinxupquote{list}}) \textendash{} List of runs

\end{description}\end{quote}

\end{fulllineitems}

\index{\_handle\_classification\_request() (polo.widgets.run\_organizer.RunOrganizer method)@\spxentry{\_handle\_classification\_request()}\spxextra{polo.widgets.run\_organizer.RunOrganizer method}}

\begin{fulllineitems}
\phantomsection\label{\detokenize{polo.widgets:polo.widgets.run_organizer.RunOrganizer._handle_classification_request}}\pysiglinewithargsret{\sphinxbfcode{\sphinxupquote{\_handle\_classification\_request}}}{}{}
Private method to open a classification thread of the currently selected run.
Calls  {\hyperref[\detokenize{polo.widgets:polo.widgets.run_organizer.RunOrganizer._open_classification_thread}]{\sphinxcrossref{\sphinxcode{\sphinxupquote{\_open\_classification\_thread()}}}}} to
start the classification thread.

\end{fulllineitems}

\index{\_handle\_opening\_run() (polo.widgets.run\_organizer.RunOrganizer method)@\spxentry{\_handle\_opening\_run()}\spxextra{polo.widgets.run\_organizer.RunOrganizer method}}

\begin{fulllineitems}
\phantomsection\label{\detokenize{polo.widgets:polo.widgets.run_organizer.RunOrganizer._handle_opening_run}}\pysiglinewithargsret{\sphinxbfcode{\sphinxupquote{\_handle\_opening\_run}}}{\emph{\DUrole{o}{*}\DUrole{n}{args}}}{}
Private method that signal to other widgets that the current run should be opened
for analysis and viewing by emiting the \sphinxtitleref{opening\_run} signal containing
the selected run.

\end{fulllineitems}

\index{\_open\_classification\_thread() (polo.widgets.run\_organizer.RunOrganizer method)@\spxentry{\_open\_classification\_thread()}\spxextra{polo.widgets.run\_organizer.RunOrganizer method}}

\begin{fulllineitems}
\phantomsection\label{\detokenize{polo.widgets:polo.widgets.run_organizer.RunOrganizer._open_classification_thread}}\pysiglinewithargsret{\sphinxbfcode{\sphinxupquote{\_open\_classification\_thread}}}{\emph{\DUrole{n}{run}}}{}
Private method to create and run a classification thread which will run
the MARCO model on all images in the run passed to \sphinxtitleref{run} argument.
\begin{quote}\begin{description}
\item[{Parameters}] \leavevmode
\sphinxstyleliteralstrong{\sphinxupquote{run}} ({\hyperref[\detokenize{polo.crystallography:polo.crystallography.run.Run}]{\sphinxcrossref{\sphinxstyleliteralemphasis{\sphinxupquote{Run}}}}}\sphinxstyleliteralemphasis{\sphinxupquote{ or }}{\hyperref[\detokenize{polo.crystallography:polo.crystallography.run.HWIRun}]{\sphinxcrossref{\sphinxstyleliteralemphasis{\sphinxupquote{HWIRun}}}}}) \textendash{} Run or HWIRun instance to run MARCO on

\end{description}\end{quote}

\end{fulllineitems}

\index{\_remove\_run() (polo.widgets.run\_organizer.RunOrganizer method)@\spxentry{\_remove\_run()}\spxextra{polo.widgets.run\_organizer.RunOrganizer method}}

\begin{fulllineitems}
\phantomsection\label{\detokenize{polo.widgets:polo.widgets.run_organizer.RunOrganizer._remove_run}}\pysiglinewithargsret{\sphinxbfcode{\sphinxupquote{\_remove\_run}}}{\emph{\DUrole{n}{run}\DUrole{o}{=}\DUrole{default_value}{None}}}{}
Private method to completely remove a run from Polo.
\begin{quote}\begin{description}
\item[{Parameters}] \leavevmode
\sphinxstyleliteralstrong{\sphinxupquote{run}} ({\hyperref[\detokenize{polo.crystallography:polo.crystallography.run.Run}]{\sphinxcrossref{\sphinxstyleliteralemphasis{\sphinxupquote{Run}}}}}\sphinxstyleliteralemphasis{\sphinxupquote{ or }}{\hyperref[\detokenize{polo.crystallography:polo.crystallography.run.HWIRun}]{\sphinxcrossref{\sphinxstyleliteralemphasis{\sphinxupquote{HWIRun}}}}}\sphinxstyleliteralemphasis{\sphinxupquote{, }}\sphinxstyleliteralemphasis{\sphinxupquote{optional}}) \textendash{} Run to remove from current session, defaults to None

\end{description}\end{quote}

\end{fulllineitems}

\index{\_set\_estimated\_classification\_time() (polo.widgets.run\_organizer.RunOrganizer method)@\spxentry{\_set\_estimated\_classification\_time()}\spxextra{polo.widgets.run\_organizer.RunOrganizer method}}

\begin{fulllineitems}
\phantomsection\label{\detokenize{polo.widgets:polo.widgets.run_organizer.RunOrganizer._set_estimated_classification_time}}\pysiglinewithargsret{\sphinxbfcode{\sphinxupquote{\_set\_estimated\_classification\_time}}}{\emph{\DUrole{n}{time}}, \emph{\DUrole{n}{num\_images\_remain}}}{}
Display the estimated classification time to the user. Time remaining
is calculated by multiplying the time it took to classify a representative
image by the number of images that remain to be classified.
\begin{quote}\begin{description}
\item[{Parameters}] \leavevmode\begin{itemize}
\item {} 
\sphinxstyleliteralstrong{\sphinxupquote{time}} (\sphinxstyleliteralemphasis{\sphinxupquote{int}}) \textendash{} Time to classify latest image

\item {} 
\sphinxstyleliteralstrong{\sphinxupquote{num\_images\_remain}} (\sphinxstyleliteralemphasis{\sphinxupquote{int}}) \textendash{} Number of images that still require classification

\end{itemize}

\end{description}\end{quote}

\end{fulllineitems}

\index{\_set\_progress\_value() (polo.widgets.run\_organizer.RunOrganizer method)@\spxentry{\_set\_progress\_value()}\spxextra{polo.widgets.run\_organizer.RunOrganizer method}}

\begin{fulllineitems}
\phantomsection\label{\detokenize{polo.widgets:polo.widgets.run_organizer.RunOrganizer._set_progress_value}}\pysiglinewithargsret{\sphinxbfcode{\sphinxupquote{\_set\_progress\_value}}}{\emph{\DUrole{n}{val}}}{}
Private helper method to increment the classification
progress bar.
\begin{quote}\begin{description}
\item[{Parameters}] \leavevmode
\sphinxstyleliteralstrong{\sphinxupquote{val}} (\sphinxstyleliteralemphasis{\sphinxupquote{int}}) \textendash{} Value to set progress bar to

\end{description}\end{quote}

\end{fulllineitems}

\index{add\_run\_from\_directory() (polo.widgets.run\_organizer.RunOrganizer method)@\spxentry{add\_run\_from\_directory()}\spxextra{polo.widgets.run\_organizer.RunOrganizer method}}

\begin{fulllineitems}
\phantomsection\label{\detokenize{polo.widgets:polo.widgets.run_organizer.RunOrganizer.add_run_from_directory}}\pysiglinewithargsret{\sphinxbfcode{\sphinxupquote{add\_run\_from\_directory}}}{\emph{\DUrole{n}{dir\_path}}}{}
\end{fulllineitems}

\index{backup\_classifications() (polo.widgets.run\_organizer.RunOrganizer method)@\spxentry{backup\_classifications()}\spxextra{polo.widgets.run\_organizer.RunOrganizer method}}

\begin{fulllineitems}
\phantomsection\label{\detokenize{polo.widgets:polo.widgets.run_organizer.RunOrganizer.backup_classifications}}\pysiglinewithargsret{\sphinxbfcode{\sphinxupquote{backup\_classifications}}}{\emph{\DUrole{n}{run}}}{}
Write the human classifications of the images in the \sphinxtitleref{run} argument
to an mso file and store it in the directory specified by the
\sphinxcode{\sphinxupquote{BACKUP\_DIR}} constant. Does not store MARCO classifications because
these can be much more easily recreated than human classifications.
Additionally, when a run is loaded back in and a backup mso exists
for it Polo assumes the classifications in that mso file are human
classifications.

Currently only \sphinxcode{\sphinxupquote{HWIRun}} instances can be written as mso files because of mso’s
integration with cocktail data and well assignments. Need a different
format for non\sphinxhyphen{}HWI runs that would map filenames to classifications
and ignore cocktail data / well assignments.
\begin{quote}\begin{description}
\item[{Parameters}] \leavevmode
\sphinxstyleliteralstrong{\sphinxupquote{run}} ({\hyperref[\detokenize{polo.crystallography:polo.crystallography.run.HWIRun}]{\sphinxcrossref{\sphinxstyleliteralemphasis{\sphinxupquote{HWIRun}}}}}) \textendash{} \sphinxcode{\sphinxupquote{HWIRun}} to backup human classifications

\end{description}\end{quote}

\end{fulllineitems}

\index{backup\_classifications\_on\_thread() (polo.widgets.run\_organizer.RunOrganizer method)@\spxentry{backup\_classifications\_on\_thread()}\spxextra{polo.widgets.run\_organizer.RunOrganizer method}}

\begin{fulllineitems}
\phantomsection\label{\detokenize{polo.widgets:polo.widgets.run_organizer.RunOrganizer.backup_classifications_on_thread}}\pysiglinewithargsret{\sphinxbfcode{\sphinxupquote{backup\_classifications\_on\_thread}}}{\emph{\DUrole{n}{run}}}{}
Does the exact same thing as 
{\hyperref[\detokenize{polo.widgets:polo.widgets.run_organizer.RunOrganizer.backup_classifications}]{\sphinxcrossref{\sphinxcode{\sphinxupquote{backup\_classifications()}}}}} 
except excutes the job on a \sphinxtitleref{QuickThread} instance to avoid slow
computers complaining about the GUI being frozen. This has been
especially prevelant on Windows machines.
\begin{quote}\begin{description}
\item[{Parameters}] \leavevmode
\sphinxstyleliteralstrong{\sphinxupquote{run}} ({\hyperref[\detokenize{polo.crystallography:polo.crystallography.run.HWIRun}]{\sphinxcrossref{\sphinxstyleliteralemphasis{\sphinxupquote{HWIRun}}}}}) \textendash{} Run to save as mso file

\end{description}\end{quote}

\end{fulllineitems}

\index{classify\_run (polo.widgets.run\_organizer.RunOrganizer attribute)@\spxentry{classify\_run}\spxextra{polo.widgets.run\_organizer.RunOrganizer attribute}}

\begin{fulllineitems}
\phantomsection\label{\detokenize{polo.widgets:polo.widgets.run_organizer.RunOrganizer.classify_run}}\pysigline{\sphinxbfcode{\sphinxupquote{classify\_run}}}
\end{fulllineitems}

\index{finished\_ftp\_download() (polo.widgets.run\_organizer.RunOrganizer method)@\spxentry{finished\_ftp\_download()}\spxextra{polo.widgets.run\_organizer.RunOrganizer method}}

\begin{fulllineitems}
\phantomsection\label{\detokenize{polo.widgets:polo.widgets.run_organizer.RunOrganizer.finished_ftp_download}}\pysiglinewithargsret{\sphinxbfcode{\sphinxupquote{finished\_ftp\_download}}}{}{}
\end{fulllineitems}

\index{ftp\_download\_status (polo.widgets.run\_organizer.RunOrganizer attribute)@\spxentry{ftp\_download\_status}\spxextra{polo.widgets.run\_organizer.RunOrganizer attribute}}

\begin{fulllineitems}
\phantomsection\label{\detokenize{polo.widgets:polo.widgets.run_organizer.RunOrganizer.ftp_download_status}}\pysigline{\sphinxbfcode{\sphinxupquote{ftp\_download\_status}}}
\end{fulllineitems}

\index{handle\_ftp\_download() (polo.widgets.run\_organizer.RunOrganizer method)@\spxentry{handle\_ftp\_download()}\spxextra{polo.widgets.run\_organizer.RunOrganizer method}}

\begin{fulllineitems}
\phantomsection\label{\detokenize{polo.widgets:polo.widgets.run_organizer.RunOrganizer.handle_ftp_download}}\pysiglinewithargsret{\sphinxbfcode{\sphinxupquote{handle\_ftp\_download}}}{\emph{\DUrole{n}{file\_path}}}{}
\end{fulllineitems}

\index{import\_run\_from\_dialog() (polo.widgets.run\_organizer.RunOrganizer method)@\spxentry{import\_run\_from\_dialog()}\spxextra{polo.widgets.run\_organizer.RunOrganizer method}}

\begin{fulllineitems}
\phantomsection\label{\detokenize{polo.widgets:polo.widgets.run_organizer.RunOrganizer.import_run_from_dialog}}\pysiglinewithargsret{\sphinxbfcode{\sphinxupquote{import\_run\_from\_dialog}}}{}{}
Import a run from a file dialog.

\end{fulllineitems}

\index{import\_run\_from\_ftp() (polo.widgets.run\_organizer.RunOrganizer method)@\spxentry{import\_run\_from\_ftp()}\spxextra{polo.widgets.run\_organizer.RunOrganizer method}}

\begin{fulllineitems}
\phantomsection\label{\detokenize{polo.widgets:polo.widgets.run_organizer.RunOrganizer.import_run_from_ftp}}\pysiglinewithargsret{\sphinxbfcode{\sphinxupquote{import\_run\_from\_ftp}}}{}{}
Import runs from an FTP server. If an FTP download thread is not already
running creates an FTPDialog instances and opens it to the user. FTP functions
are then taken over by the FTPDialog until it is closed.

\end{fulllineitems}

\index{import\_saved\_runs() (polo.widgets.run\_organizer.RunOrganizer method)@\spxentry{import\_saved\_runs()}\spxextra{polo.widgets.run\_organizer.RunOrganizer method}}

\begin{fulllineitems}
\phantomsection\label{\detokenize{polo.widgets:polo.widgets.run_organizer.RunOrganizer.import_saved_runs}}\pysiglinewithargsret{\sphinxbfcode{\sphinxupquote{import\_saved\_runs}}}{\emph{\DUrole{n}{xtal\_files}\DUrole{o}{=}\DUrole{default_value}{{[}{]}}}}{}
Import runs saved to xtal files.
\begin{quote}\begin{description}
\item[{Parameters}] \leavevmode
\sphinxstyleliteralstrong{\sphinxupquote{xtal\_files}} (\sphinxstyleliteralemphasis{\sphinxupquote{list}}\sphinxstyleliteralemphasis{\sphinxupquote{, }}\sphinxstyleliteralemphasis{\sphinxupquote{optional}}) \textendash{} List of xtal files to import runs from, defaults to {[}{]}

\end{description}\end{quote}

\end{fulllineitems}

\index{opening\_run (polo.widgets.run\_organizer.RunOrganizer attribute)@\spxentry{opening\_run}\spxextra{polo.widgets.run\_organizer.RunOrganizer attribute}}

\begin{fulllineitems}
\phantomsection\label{\detokenize{polo.widgets:polo.widgets.run_organizer.RunOrganizer.opening_run}}\pysigline{\sphinxbfcode{\sphinxupquote{opening\_run}}}
\end{fulllineitems}

\index{remove\_run() (polo.widgets.run\_organizer.RunOrganizer method)@\spxentry{remove\_run()}\spxextra{polo.widgets.run\_organizer.RunOrganizer method}}

\begin{fulllineitems}
\phantomsection\label{\detokenize{polo.widgets:polo.widgets.run_organizer.RunOrganizer.remove_run}}\pysiglinewithargsret{\sphinxbfcode{\sphinxupquote{remove\_run}}}{\emph{\DUrole{n}{run}\DUrole{o}{=}\DUrole{default_value}{None}}}{}
\end{fulllineitems}


\end{fulllineitems}



\subsubsection{polo.widgets.run\_tree module}
\label{\detokenize{polo.widgets:module-polo.widgets.run_tree}}\label{\detokenize{polo.widgets:polo-widgets-run-tree-module}}\index{module@\spxentry{module}!polo.widgets.run\_tree@\spxentry{polo.widgets.run\_tree}}\index{polo.widgets.run\_tree@\spxentry{polo.widgets.run\_tree}!module@\spxentry{module}}\index{RunTree (class in polo.widgets.run\_tree)@\spxentry{RunTree}\spxextra{class in polo.widgets.run\_tree}}

\begin{fulllineitems}
\phantomsection\label{\detokenize{polo.widgets:polo.widgets.run_tree.RunTree}}\pysiglinewithargsret{\sphinxbfcode{\sphinxupquote{class }}\sphinxcode{\sphinxupquote{polo.widgets.run\_tree.}}\sphinxbfcode{\sphinxupquote{RunTree}}}{\emph{\DUrole{n}{parent}\DUrole{o}{=}\DUrole{default_value}{None}}, \emph{\DUrole{n}{auto\_link}\DUrole{o}{=}\DUrole{default_value}{True}}}{}
Bases: \sphinxcode{\sphinxupquote{PyQt5.QtWidgets.QTreeWidget}}

Inherits the \sphinxcode{\sphinxupquote{QTreeWidget}} class and acts as the sample and
run display. The User uses the RunTree to open and classify runs
they load into Polo.
\begin{quote}\begin{description}
\item[{Parameters}] \leavevmode\begin{itemize}
\item {} 
\sphinxstyleliteralstrong{\sphinxupquote{parent}} (\sphinxstyleliteralemphasis{\sphinxupquote{QWidget}}\sphinxstyleliteralemphasis{\sphinxupquote{, }}\sphinxstyleliteralemphasis{\sphinxupquote{optional}}) \textendash{} Parent widget, defaults to None

\item {} 
\sphinxstyleliteralstrong{\sphinxupquote{auto\_link}} (\sphinxstyleliteralemphasis{\sphinxupquote{bool}}\sphinxstyleliteralemphasis{\sphinxupquote{, }}\sphinxstyleliteralemphasis{\sphinxupquote{optional}}) \textendash{} If True automatically link runs together, defaults to True

\end{itemize}

\end{description}\end{quote}
\index{\_add\_classifications\_from\_mso\_slot() (polo.widgets.run\_tree.RunTree method)@\spxentry{\_add\_classifications\_from\_mso\_slot()}\spxextra{polo.widgets.run\_tree.RunTree method}}

\begin{fulllineitems}
\phantomsection\label{\detokenize{polo.widgets:polo.widgets.run_tree.RunTree._add_classifications_from_mso_slot}}\pysiglinewithargsret{\sphinxbfcode{\sphinxupquote{\_add\_classifications\_from\_mso\_slot}}}{\emph{\DUrole{n}{event}\DUrole{o}{=}\DUrole{default_value}{None}}}{}
Add classifications to an existing \sphinxcode{\sphinxupquote{Run}} from the contents of an
MSO file. Intended to be connected to the \sphinxtitleref{classify\_from\_mso}
QAction that is defined in the :
{\hyperref[\detokenize{polo.widgets:polo.widgets.run_tree.RunTree.contextMenuEvent}]{\sphinxcrossref{\sphinxcode{\sphinxupquote{contextMenuEvent()}}}}}
method.
\begin{quote}\begin{description}
\item[{Parameters}] \leavevmode
\sphinxstyleliteralstrong{\sphinxupquote{event}} (\sphinxstyleliteralemphasis{\sphinxupquote{QEvent}}\sphinxstyleliteralemphasis{\sphinxupquote{, }}\sphinxstyleliteralemphasis{\sphinxupquote{optional}}) \textendash{} QEvent, defaults to None

\end{description}\end{quote}

\end{fulllineitems}

\index{\_add\_run\_node() (polo.widgets.run\_tree.RunTree method)@\spxentry{\_add\_run\_node()}\spxextra{polo.widgets.run\_tree.RunTree method}}

\begin{fulllineitems}
\phantomsection\label{\detokenize{polo.widgets:polo.widgets.run_tree.RunTree._add_run_node}}\pysiglinewithargsret{\sphinxbfcode{\sphinxupquote{\_add\_run\_node}}}{\emph{\DUrole{n}{run}}, \emph{\DUrole{n}{tree}\DUrole{o}{=}\DUrole{default_value}{None}}}{}
Private method that adds a new run node.
\begin{quote}\begin{description}
\item[{Parameters}] \leavevmode\begin{itemize}
\item {} 
\sphinxstyleliteralstrong{\sphinxupquote{run}} ({\hyperref[\detokenize{polo.crystallography:polo.crystallography.run.Run}]{\sphinxcrossref{\sphinxstyleliteralemphasis{\sphinxupquote{Run}}}}}\sphinxstyleliteralemphasis{\sphinxupquote{ or }}{\hyperref[\detokenize{polo.crystallography:polo.crystallography.run.HWIRun}]{\sphinxcrossref{\sphinxstyleliteralemphasis{\sphinxupquote{HWIRun}}}}}) \textendash{} Run to add to the tree

\item {} 
\sphinxstyleliteralstrong{\sphinxupquote{tree}} (\sphinxstyleliteralemphasis{\sphinxupquote{QTreeWidgetItem}}\sphinxstyleliteralemphasis{\sphinxupquote{, }}\sphinxstyleliteralemphasis{\sphinxupquote{optional}}) \textendash{} \sphinxtitleref{QTreeWidgetItem} to act as parent node, defaults to None.
If None uses the root as the parent node.

\end{itemize}

\item[{Returns}] \leavevmode
Node added to the tree

\item[{Return type}] \leavevmode
QTreeWidgetItem

\end{description}\end{quote}

\end{fulllineitems}

\index{\_edit\_data\_slot() (polo.widgets.run\_tree.RunTree method)@\spxentry{\_edit\_data\_slot()}\spxextra{polo.widgets.run\_tree.RunTree method}}

\begin{fulllineitems}
\phantomsection\label{\detokenize{polo.widgets:polo.widgets.run_tree.RunTree._edit_data_slot}}\pysiglinewithargsret{\sphinxbfcode{\sphinxupquote{\_edit\_data\_slot}}}{\emph{\DUrole{n}{event}\DUrole{o}{=}\DUrole{default_value}{None}}}{}
Private method used to update the data in a \sphinxcode{\sphinxupquote{Run}} after it has
been modified by the user through the \sphinxcode{\sphinxupquote{RunUpdater}} dialog.
\begin{quote}\begin{description}
\item[{Parameters}] \leavevmode
\sphinxstyleliteralstrong{\sphinxupquote{event}} (\sphinxstyleliteralemphasis{\sphinxupquote{QEvent}}\sphinxstyleliteralemphasis{\sphinxupquote{, }}\sphinxstyleliteralemphasis{\sphinxupquote{optional}}) \textendash{} QEvent, defaults to None

\end{description}\end{quote}

\end{fulllineitems}

\index{\_get\_run\_node() (polo.widgets.run\_tree.RunTree method)@\spxentry{\_get\_run\_node()}\spxextra{polo.widgets.run\_tree.RunTree method}}

\begin{fulllineitems}
\phantomsection\label{\detokenize{polo.widgets:polo.widgets.run_tree.RunTree._get_run_node}}\pysiglinewithargsret{\sphinxbfcode{\sphinxupquote{\_get\_run\_node}}}{\emph{\DUrole{n}{run}}}{}
Private helper method that returns the \sphinxcode{\sphinxupquote{QTreeWidgetItem}}
corresponding to a given \sphinxcode{\sphinxupquote{Run}}. Returns None if a node cannot be found.
\begin{quote}\begin{description}
\item[{Parameters}] \leavevmode
\sphinxstyleliteralstrong{\sphinxupquote{run}} ({\hyperref[\detokenize{polo.crystallography:polo.crystallography.run.Run}]{\sphinxcrossref{\sphinxstyleliteralemphasis{\sphinxupquote{Run}}}}}\sphinxstyleliteralemphasis{\sphinxupquote{ or }}{\hyperref[\detokenize{polo.crystallography:polo.crystallography.run.HWIRun}]{\sphinxcrossref{\sphinxstyleliteralemphasis{\sphinxupquote{HWIRun}}}}}) \textendash{} Run to search for

\item[{Returns}] \leavevmode
Given run’s corresponding \sphinxcode{\sphinxupquote{QTreeWidgetItem}} if it exists

\item[{Return type}] \leavevmode
QTreeWidgetItem

\end{description}\end{quote}

\end{fulllineitems}

\index{\_open\_run\_slot() (polo.widgets.run\_tree.RunTree method)@\spxentry{\_open\_run\_slot()}\spxextra{polo.widgets.run\_tree.RunTree method}}

\begin{fulllineitems}
\phantomsection\label{\detokenize{polo.widgets:polo.widgets.run_tree.RunTree._open_run_slot}}\pysiglinewithargsret{\sphinxbfcode{\sphinxupquote{\_open\_run\_slot}}}{\emph{\DUrole{n}{event}\DUrole{o}{=}\DUrole{default_value}{None}}}{}
Private method that emits the {\hyperref[\detokenize{polo.widgets:polo.widgets.run_tree.RunTree.opening_run}]{\sphinxcrossref{\sphinxcode{\sphinxupquote{opening\_run}}}}} signal when called. This
signal can be connected to other widgets to communicate that the user
has selected a run and wants to open it for analysis.
\begin{quote}\begin{description}
\item[{Parameters}] \leavevmode
\sphinxstyleliteralstrong{\sphinxupquote{event}} (\sphinxstyleliteralemphasis{\sphinxupquote{QEvent}}\sphinxstyleliteralemphasis{\sphinxupquote{, }}\sphinxstyleliteralemphasis{\sphinxupquote{optional}}) \textendash{} QEvent, defaults to None

\end{description}\end{quote}

\end{fulllineitems}

\index{\_remove\_run() (polo.widgets.run\_tree.RunTree method)@\spxentry{\_remove\_run()}\spxextra{polo.widgets.run\_tree.RunTree method}}

\begin{fulllineitems}
\phantomsection\label{\detokenize{polo.widgets:polo.widgets.run_tree.RunTree._remove_run}}\pysiglinewithargsret{\sphinxbfcode{\sphinxupquote{\_remove\_run}}}{\emph{\DUrole{n}{run\_name}}}{}~\begin{description}
\item[{Private method to remove a \sphinxcode{\sphinxupquote{Run}} completely from the}] \leavevmode
Polo interface.

\end{description}
\begin{quote}\begin{description}
\item[{Parameters}] \leavevmode
\sphinxstyleliteralstrong{\sphinxupquote{run\_name}} (\sphinxstyleliteralemphasis{\sphinxupquote{str}}) \textendash{} Run name of \sphinxcode{\sphinxupquote{Run}} instance to remove

\end{description}\end{quote}

\end{fulllineitems}

\index{\_remove\_run\_slot() (polo.widgets.run\_tree.RunTree method)@\spxentry{\_remove\_run\_slot()}\spxextra{polo.widgets.run\_tree.RunTree method}}

\begin{fulllineitems}
\phantomsection\label{\detokenize{polo.widgets:polo.widgets.run_tree.RunTree._remove_run_slot}}\pysiglinewithargsret{\sphinxbfcode{\sphinxupquote{\_remove\_run\_slot}}}{\emph{\DUrole{n}{event}\DUrole{o}{=}\DUrole{default_value}{None}}}{}
Slot to connect to contextMenu popup to remove the selected run.

\end{fulllineitems}

\index{add\_classified\_run() (polo.widgets.run\_tree.RunTree method)@\spxentry{add\_classified\_run()}\spxextra{polo.widgets.run\_tree.RunTree method}}

\begin{fulllineitems}
\phantomsection\label{\detokenize{polo.widgets:polo.widgets.run_tree.RunTree.add_classified_run}}\pysiglinewithargsret{\sphinxbfcode{\sphinxupquote{add\_classified\_run}}}{\emph{\DUrole{n}{run}}}{}
Marks a \sphinxcode{\sphinxupquote{Run}} instance as classified by adding it to the
\sphinxcode{\sphinxupquote{classified\_status}} dictionary.
\begin{quote}\begin{description}
\item[{Parameters}] \leavevmode
\sphinxstyleliteralstrong{\sphinxupquote{run}} ({\hyperref[\detokenize{polo.crystallography:polo.crystallography.run.Run}]{\sphinxcrossref{\sphinxstyleliteralemphasis{\sphinxupquote{Run}}}}}\sphinxstyleliteralemphasis{\sphinxupquote{ or }}{\hyperref[\detokenize{polo.crystallography:polo.crystallography.run.HWIRun}]{\sphinxcrossref{\sphinxstyleliteralemphasis{\sphinxupquote{HWIRun}}}}}) \textendash{} Run to mark as classified

\end{description}\end{quote}

\end{fulllineitems}

\index{add\_run\_to\_tree() (polo.widgets.run\_tree.RunTree method)@\spxentry{add\_run\_to\_tree()}\spxextra{polo.widgets.run\_tree.RunTree method}}

\begin{fulllineitems}
\phantomsection\label{\detokenize{polo.widgets:polo.widgets.run_tree.RunTree.add_run_to_tree}}\pysiglinewithargsret{\sphinxbfcode{\sphinxupquote{add\_run\_to\_tree}}}{\emph{\DUrole{n}{new\_run}}}{}
Add a new \sphinxcode{\sphinxupquote{Run}} instance to the tree. Uses the \sphinxcode{\sphinxupquote{Run}} instance’s 
\sphinxcode{\sphinxupquote{sampleName}} attribute to determine what sample node 
the \sphinxcode{\sphinxupquote{Run}} instance should be added
to. If the sample name does not exist in the tree a new sample node is added.
If the \sphinxcode{\sphinxupquote{Run}} instance lacks the \sphinxcode{\sphinxupquote{sampleName}} attribute as is the case for 
non\sphinxhyphen{}HWIRuns the \sphinxcode{\sphinxupquote{sampleName}} attribute is set to “Non\sphinxhyphen{}HWI Runs”. 
If the \sphinxcode{\sphinxupquote{Run}} instance is an \sphinxcode{\sphinxupquote{HWIRun}} and lacks the 
\sphinxcode{\sphinxupquote{sampleName}} attribute \sphinxcode{\sphinxupquote{sampleName}} is
set to “Sampleless Runs”.
\begin{quote}\begin{description}
\item[{Parameters}] \leavevmode
\sphinxstyleliteralstrong{\sphinxupquote{new\_run}} ({\hyperref[\detokenize{polo.crystallography:polo.crystallography.run.Run}]{\sphinxcrossref{\sphinxstyleliteralemphasis{\sphinxupquote{Run}}}}}\sphinxstyleliteralemphasis{\sphinxupquote{ or }}{\hyperref[\detokenize{polo.crystallography:polo.crystallography.run.HWIRun}]{\sphinxcrossref{\sphinxstyleliteralemphasis{\sphinxupquote{HWIRun}}}}}) \textendash{} Run to add to the tree

\end{description}\end{quote}

\end{fulllineitems}

\index{add\_sample() (polo.widgets.run\_tree.RunTree method)@\spxentry{add\_sample()}\spxextra{polo.widgets.run\_tree.RunTree method}}

\begin{fulllineitems}
\phantomsection\label{\detokenize{polo.widgets:polo.widgets.run_tree.RunTree.add_sample}}\pysiglinewithargsret{\sphinxbfcode{\sphinxupquote{add\_sample}}}{\emph{\DUrole{n}{sample\_name}}, \emph{\DUrole{o}{*}\DUrole{n}{args}}}{}
Adds a new sample to the tree. Samples are the
highest level node in the \sphinxtitleref{RunTree}.
\begin{quote}\begin{description}
\item[{Parameters}] \leavevmode
\sphinxstyleliteralstrong{\sphinxupquote{sample\_name}} (\sphinxstyleliteralemphasis{\sphinxupquote{str}}) \textendash{} Name of sample to add, acts as key so should 
be unique.

\end{description}\end{quote}

\end{fulllineitems}

\index{contextMenuEvent() (polo.widgets.run\_tree.RunTree method)@\spxentry{contextMenuEvent()}\spxextra{polo.widgets.run\_tree.RunTree method}}

\begin{fulllineitems}
\phantomsection\label{\detokenize{polo.widgets:polo.widgets.run_tree.RunTree.contextMenuEvent}}\pysiglinewithargsret{\sphinxbfcode{\sphinxupquote{contextMenuEvent}}}{\emph{\DUrole{n}{event}}}{}
Handle left click events by creating a popup context menu.
\begin{quote}\begin{description}
\item[{Parameters}] \leavevmode
\sphinxstyleliteralstrong{\sphinxupquote{event}} (\sphinxstyleliteralemphasis{\sphinxupquote{QEvent}}) \textendash{} QEvent

\end{description}\end{quote}

\end{fulllineitems}

\index{current\_run\_names() (polo.widgets.run\_tree.RunTree property)@\spxentry{current\_run\_names()}\spxextra{polo.widgets.run\_tree.RunTree property}}

\begin{fulllineitems}
\phantomsection\label{\detokenize{polo.widgets:polo.widgets.run_tree.RunTree.current_run_names}}\pysigline{\sphinxbfcode{\sphinxupquote{property }}\sphinxbfcode{\sphinxupquote{current\_run\_names}}}
List of all currently loaded \sphinxcode{\sphinxupquote{Run}} names.
\begin{quote}\begin{description}
\item[{Returns}] \leavevmode
List of :class:Run\textasciigrave{} names

\item[{Return type}] \leavevmode
list

\end{description}\end{quote}

\end{fulllineitems}

\index{link\_sample() (polo.widgets.run\_tree.RunTree method)@\spxentry{link\_sample()}\spxextra{polo.widgets.run\_tree.RunTree method}}

\begin{fulllineitems}
\phantomsection\label{\detokenize{polo.widgets:polo.widgets.run_tree.RunTree.link_sample}}\pysiglinewithargsret{\sphinxbfcode{\sphinxupquote{link\_sample}}}{\emph{\DUrole{n}{sample\_name}}}{}
Links all \sphinxcode{\sphinxupquote{Run}} instances in a given sample together by both date
and spectrum using the {\hyperref[\detokenize{polo.utils:polo.utils.io_utils.RunLinker.the_big_link}]{\sphinxcrossref{\sphinxcode{\sphinxupquote{the\_big\_link()}}}}}
method.
\begin{quote}\begin{description}
\item[{Parameters}] \leavevmode
\sphinxstyleliteralstrong{\sphinxupquote{sample\_name}} (\sphinxstyleliteralemphasis{\sphinxupquote{str}}) \textendash{} Name of the sample who’s runs should be linked

\end{description}\end{quote}

\end{fulllineitems}

\index{opening\_run (polo.widgets.run\_tree.RunTree attribute)@\spxentry{opening\_run}\spxextra{polo.widgets.run\_tree.RunTree attribute}}

\begin{fulllineitems}
\phantomsection\label{\detokenize{polo.widgets:polo.widgets.run_tree.RunTree.opening_run}}\pysigline{\sphinxbfcode{\sphinxupquote{opening\_run}}}
\end{fulllineitems}

\index{remove\_run\_from\_view() (polo.widgets.run\_tree.RunTree method)@\spxentry{remove\_run\_from\_view()}\spxextra{polo.widgets.run\_tree.RunTree method}}

\begin{fulllineitems}
\phantomsection\label{\detokenize{polo.widgets:polo.widgets.run_tree.RunTree.remove_run_from_view}}\pysiglinewithargsret{\sphinxbfcode{\sphinxupquote{remove\_run\_from\_view}}}{\emph{\DUrole{n}{run\_name}}}{}
Remove a \sphinxcode{\sphinxupquote{Run}} instance using its \sphinxcode{\sphinxupquote{run\_name}} attribute.
Does not effect any other widgets. Calling this method only 
removes the \sphinxcode{\sphinxupquote{Run}} instance from
the display. If a \sphinxcode{\sphinxupquote{Run}} instance is removed from successfully
it is returned.
\begin{quote}\begin{description}
\item[{Parameters}] \leavevmode
\sphinxstyleliteralstrong{\sphinxupquote{run\_name}} (\sphinxstyleliteralemphasis{\sphinxupquote{str}}) \textendash{} Name of run to remove

\item[{Returns}] \leavevmode
Removed run

\item[{Return type}] \leavevmode
{\hyperref[\detokenize{polo.crystallography:polo.crystallography.run.Run}]{\sphinxcrossref{Run}}} or {\hyperref[\detokenize{polo.crystallography:polo.crystallography.run.HWIRun}]{\sphinxcrossref{HWIRun}}}

\end{description}\end{quote}

\end{fulllineitems}

\index{remove\_run\_signal (polo.widgets.run\_tree.RunTree attribute)@\spxentry{remove\_run\_signal}\spxextra{polo.widgets.run\_tree.RunTree attribute}}

\begin{fulllineitems}
\phantomsection\label{\detokenize{polo.widgets:polo.widgets.run_tree.RunTree.remove_run_signal}}\pysigline{\sphinxbfcode{\sphinxupquote{remove\_run\_signal}}}
\end{fulllineitems}

\index{save\_run\_signal (polo.widgets.run\_tree.RunTree attribute)@\spxentry{save\_run\_signal}\spxextra{polo.widgets.run\_tree.RunTree attribute}}

\begin{fulllineitems}
\phantomsection\label{\detokenize{polo.widgets:polo.widgets.run_tree.RunTree.save_run_signal}}\pysigline{\sphinxbfcode{\sphinxupquote{save\_run\_signal}}}
\end{fulllineitems}

\index{selected\_run() (polo.widgets.run\_tree.RunTree property)@\spxentry{selected\_run()}\spxextra{polo.widgets.run\_tree.RunTree property}}

\begin{fulllineitems}
\phantomsection\label{\detokenize{polo.widgets:polo.widgets.run_tree.RunTree.selected_run}}\pysigline{\sphinxbfcode{\sphinxupquote{property }}\sphinxbfcode{\sphinxupquote{selected\_run}}}
The: class:\sphinxtitleref{Run} that is currently selected. If no \sphinxcode{\sphinxupquote{Run}} is
selected returns False.
\begin{quote}\begin{description}
\item[{Returns}] \leavevmode
The currently selected run, if one exists, otherwise returns False

\item[{Return type}] \leavevmode
{\hyperref[\detokenize{polo.crystallography:polo.crystallography.run.Run}]{\sphinxcrossref{Run}}}, {\hyperref[\detokenize{polo.crystallography:polo.crystallography.run.HWIRun}]{\sphinxcrossref{HWIRun}}} or False

\end{description}\end{quote}

\end{fulllineitems}


\end{fulllineitems}



\subsubsection{polo.widgets.slideshow\_inspector module}
\label{\detokenize{polo.widgets:module-polo.widgets.slideshow_inspector}}\label{\detokenize{polo.widgets:polo-widgets-slideshow-inspector-module}}\index{module@\spxentry{module}!polo.widgets.slideshow\_inspector@\spxentry{polo.widgets.slideshow\_inspector}}\index{polo.widgets.slideshow\_inspector@\spxentry{polo.widgets.slideshow\_inspector}!module@\spxentry{module}}\index{slideshowInspector (class in polo.widgets.slideshow\_inspector)@\spxentry{slideshowInspector}\spxextra{class in polo.widgets.slideshow\_inspector}}

\begin{fulllineitems}
\phantomsection\label{\detokenize{polo.widgets:polo.widgets.slideshow_inspector.slideshowInspector}}\pysiglinewithargsret{\sphinxbfcode{\sphinxupquote{class }}\sphinxcode{\sphinxupquote{polo.widgets.slideshow\_inspector.}}\sphinxbfcode{\sphinxupquote{slideshowInspector}}}{\emph{\DUrole{n}{parent}}, \emph{\DUrole{n}{run}\DUrole{o}{=}\DUrole{default_value}{None}}}{}
Bases: \sphinxcode{\sphinxupquote{PyQt5.QtWidgets.QWidget}}

The slideshowInspector widget is a primary run interface that allows
users to view their screening images in a standard slideshow format. If
multiple imaging runs of the sample sample exist it also allows the user to
navigate between or simultaneously view these images.
\begin{quote}\begin{description}
\item[{Parameters}] \leavevmode\begin{itemize}
\item {} 
\sphinxstyleliteralstrong{\sphinxupquote{parent}} (\sphinxstyleliteralemphasis{\sphinxupquote{QtWidget}}) \textendash{} Parent widget

\item {} 
\sphinxstyleliteralstrong{\sphinxupquote{run}} ({\hyperref[\detokenize{polo.crystallography:polo.crystallography.run.Run}]{\sphinxcrossref{\sphinxstyleliteralemphasis{\sphinxupquote{Run}}}}}\sphinxstyleliteralemphasis{\sphinxupquote{ or }}{\hyperref[\detokenize{polo.crystallography:polo.crystallography.run.HWIRun}]{\sphinxcrossref{\sphinxstyleliteralemphasis{\sphinxupquote{HWIRun}}}}}\sphinxstyleliteralemphasis{\sphinxupquote{, }}\sphinxstyleliteralemphasis{\sphinxupquote{optional}}) \textendash{} Run to show to the user, defaults to None

\end{itemize}

\end{description}\end{quote}
\index{\_classify\_image() (polo.widgets.slideshow\_inspector.slideshowInspector method)@\spxentry{\_classify\_image()}\spxextra{polo.widgets.slideshow\_inspector.slideshowInspector method}}

\begin{fulllineitems}
\phantomsection\label{\detokenize{polo.widgets:polo.widgets.slideshow_inspector.slideshowInspector._classify_image}}\pysiglinewithargsret{\sphinxbfcode{\sphinxupquote{\_classify\_image}}}{\emph{\DUrole{n}{classification}}}{}
Private method to change the human classification of the current
image.
\begin{quote}\begin{description}
\item[{Parameters}] \leavevmode
\sphinxstyleliteralstrong{\sphinxupquote{classification}} (\sphinxstyleliteralemphasis{\sphinxupquote{str}}) \textendash{} Image classification

\end{description}\end{quote}

\end{fulllineitems}

\index{\_display\_current\_image() (polo.widgets.slideshow\_inspector.slideshowInspector method)@\spxentry{\_display\_current\_image()}\spxextra{polo.widgets.slideshow\_inspector.slideshowInspector method}}

\begin{fulllineitems}
\phantomsection\label{\detokenize{polo.widgets:polo.widgets.slideshow_inspector.slideshowInspector._display_current_image}}\pysiglinewithargsret{\sphinxbfcode{\sphinxupquote{\_display\_current\_image}}}{}{}
Private method that displays the current image as 
determined by the \sphinxtitleref{current\_image} attribute of the \sphinxtitleref{slideshowViewer}
widget and populates any widgets that display current image metadata.

\end{fulllineitems}

\index{\_mark\_current\_image\_as\_favorite() (polo.widgets.slideshow\_inspector.slideshowInspector method)@\spxentry{\_mark\_current\_image\_as\_favorite()}\spxextra{polo.widgets.slideshow\_inspector.slideshowInspector method}}

\begin{fulllineitems}
\phantomsection\label{\detokenize{polo.widgets:polo.widgets.slideshow_inspector.slideshowInspector._mark_current_image_as_favorite}}\pysiglinewithargsret{\sphinxbfcode{\sphinxupquote{\_mark\_current\_image\_as\_favorite}}}{}{}
Private method that sets the favorite label on the current
image to the current value of the favorite \sphinxcode{\sphinxupquote{QCheckBox}}.
\begin{quote}\begin{description}
\item[{Parameters}] \leavevmode
\sphinxstyleliteralstrong{\sphinxupquote{favorite\_status}} (\sphinxstyleliteralemphasis{\sphinxupquote{bool}}) \textendash{} Whether this image is a favorite or not

\end{description}\end{quote}

\end{fulllineitems}

\index{\_navigate\_carousel() (polo.widgets.slideshow\_inspector.slideshowInspector method)@\spxentry{\_navigate\_carousel()}\spxextra{polo.widgets.slideshow\_inspector.slideshowInspector method}}

\begin{fulllineitems}
\phantomsection\label{\detokenize{polo.widgets:polo.widgets.slideshow_inspector.slideshowInspector._navigate_carousel}}\pysiglinewithargsret{\sphinxbfcode{\sphinxupquote{\_navigate\_carousel}}}{\emph{\DUrole{n}{next\_image}\DUrole{o}{=}\DUrole{default_value}{False}}, \emph{\DUrole{n}{prev\_image}\DUrole{o}{=}\DUrole{default_value}{False}}}{}
Private method to control the carousel using boolean flags. Calls 
\sphinxcode{\sphinxupquote{carousel\_controls()}}.
\begin{quote}\begin{description}
\item[{Parameters}] \leavevmode\begin{itemize}
\item {} 
\sphinxstyleliteralstrong{\sphinxupquote{next\_image}} (\sphinxstyleliteralemphasis{\sphinxupquote{bool}}\sphinxstyleliteralemphasis{\sphinxupquote{, }}\sphinxstyleliteralemphasis{\sphinxupquote{optional}}) \textendash{} If True navigates to next image in carousel, 
defaults to False

\item {} 
\sphinxstyleliteralstrong{\sphinxupquote{prev\_image}} (\sphinxstyleliteralemphasis{\sphinxupquote{bool}}\sphinxstyleliteralemphasis{\sphinxupquote{, }}\sphinxstyleliteralemphasis{\sphinxupquote{optional}}) \textendash{} If True navigates to previous image in carousel,
defaults to False

\end{itemize}

\end{description}\end{quote}

\end{fulllineitems}

\index{\_set\_alt\_image() (polo.widgets.slideshow\_inspector.slideshowInspector method)@\spxentry{\_set\_alt\_image()}\spxextra{polo.widgets.slideshow\_inspector.slideshowInspector method}}

\begin{fulllineitems}
\phantomsection\label{\detokenize{polo.widgets:polo.widgets.slideshow_inspector.slideshowInspector._set_alt_image}}\pysiglinewithargsret{\sphinxbfcode{\sphinxupquote{\_set\_alt\_image}}}{\emph{\DUrole{n}{next\_date}\DUrole{o}{=}\DUrole{default_value}{False}}, \emph{\DUrole{n}{prev\_date}\DUrole{o}{=}\DUrole{default_value}{False}}, \emph{\DUrole{n}{alt\_spec}\DUrole{o}{=}\DUrole{default_value}{False}}}{}
Display an image linked to the current image based on
boolean flags.
\begin{quote}\begin{description}
\item[{Parameters}] \leavevmode\begin{itemize}
\item {} 
\sphinxstyleliteralstrong{\sphinxupquote{next\_date}} (\sphinxstyleliteralemphasis{\sphinxupquote{bool}}\sphinxstyleliteralemphasis{\sphinxupquote{, }}\sphinxstyleliteralemphasis{\sphinxupquote{optional}}) \textendash{} If True show the current image’s next
image by date, defaults to False

\item {} 
\sphinxstyleliteralstrong{\sphinxupquote{prev\_date}} (\sphinxstyleliteralemphasis{\sphinxupquote{bool}}\sphinxstyleliteralemphasis{\sphinxupquote{, }}\sphinxstyleliteralemphasis{\sphinxupquote{optional}}) \textendash{} If True, show the current image’s previous
image by date, defaults to False

\item {} 
\sphinxstyleliteralstrong{\sphinxupquote{alt\_spec}} (\sphinxstyleliteralemphasis{\sphinxupquote{bool}}\sphinxstyleliteralemphasis{\sphinxupquote{, }}\sphinxstyleliteralemphasis{\sphinxupquote{optional}}) \textendash{} If True, show the current image’s alt
spectrum image, defaults to False

\end{itemize}

\end{description}\end{quote}

\end{fulllineitems}

\index{\_set\_alt\_spectrum\_buttons() (polo.widgets.slideshow\_inspector.slideshowInspector method)@\spxentry{\_set\_alt\_spectrum\_buttons()}\spxextra{polo.widgets.slideshow\_inspector.slideshowInspector method}}

\begin{fulllineitems}
\phantomsection\label{\detokenize{polo.widgets:polo.widgets.slideshow_inspector.slideshowInspector._set_alt_spectrum_buttons}}\pysiglinewithargsret{\sphinxbfcode{\sphinxupquote{\_set\_alt\_spectrum\_buttons}}}{}{}
Turns alt spectrum functions on or off depending on contents
of the \sphinxtitleref{Run} instance referenced by the \sphinxtitleref{run} attribute.
Alt spectrum buttons will be enabled if the \sphinxtitleref{run} is a part
of an alt spectrum linked list. This means another \sphinxtitleref{Run}
instance is referenced by it’s \sphinxtitleref{alt\_spectrum} attribute.

\end{fulllineitems}

\index{\_set\_classification\_button\_labels() (polo.widgets.slideshow\_inspector.slideshowInspector method)@\spxentry{\_set\_classification\_button\_labels()}\spxextra{polo.widgets.slideshow\_inspector.slideshowInspector method}}

\begin{fulllineitems}
\phantomsection\label{\detokenize{polo.widgets:polo.widgets.slideshow_inspector.slideshowInspector._set_classification_button_labels}}\pysiglinewithargsret{\sphinxbfcode{\sphinxupquote{\_set\_classification\_button\_labels}}}{}{}
Private method that sets the labels of image classification
buttons based on the \sphinxcode{\sphinxupquote{IMAGE\_CLASSIFICATIONS}} constant. Should be called
in the \sphinxtitleref{\_\_init\_\_} method.

\end{fulllineitems}

\index{\_set\_favorite\_checkbox() (polo.widgets.slideshow\_inspector.slideshowInspector method)@\spxentry{\_set\_favorite\_checkbox()}\spxextra{polo.widgets.slideshow\_inspector.slideshowInspector method}}

\begin{fulllineitems}
\phantomsection\label{\detokenize{polo.widgets:polo.widgets.slideshow_inspector.slideshowInspector._set_favorite_checkbox}}\pysiglinewithargsret{\sphinxbfcode{\sphinxupquote{\_set\_favorite\_checkbox}}}{}{}
Private method that sets the value of the favorite \sphinxcode{\sphinxupquote{QCheckBox}} based
on whether the current image is marked as a favorite or not.
Should be used when loading an image into the view.

An image is considered a favorite if it’s \sphinxtitleref{favorite} attribute ==
True.

\end{fulllineitems}

\index{\_set\_image\_class\_checkbox\_labels() (polo.widgets.slideshow\_inspector.slideshowInspector method)@\spxentry{\_set\_image\_class\_checkbox\_labels()}\spxextra{polo.widgets.slideshow\_inspector.slideshowInspector method}}

\begin{fulllineitems}
\phantomsection\label{\detokenize{polo.widgets:polo.widgets.slideshow_inspector.slideshowInspector._set_image_class_checkbox_labels}}\pysiglinewithargsret{\sphinxbfcode{\sphinxupquote{\_set\_image\_class\_checkbox\_labels}}}{}{}
Private method to the \sphinxcode{\sphinxupquote{QCheckBox}} labels for imaging filtering
from the \sphinxtitleref{IMAGE\_CLASSIFICATIONS} constant. Should be called in
the \sphinxtitleref{\_\_init\_\_} method.

\end{fulllineitems}

\index{\_set\_image\_name() (polo.widgets.slideshow\_inspector.slideshowInspector method)@\spxentry{\_set\_image\_name()}\spxextra{polo.widgets.slideshow\_inspector.slideshowInspector method}}

\begin{fulllineitems}
\phantomsection\label{\detokenize{polo.widgets:polo.widgets.slideshow_inspector.slideshowInspector._set_image_name}}\pysiglinewithargsret{\sphinxbfcode{\sphinxupquote{\_set\_image\_name}}}{}{}
Private method that sets current image label to the
image’s filepath basename.

\end{fulllineitems}

\index{\_set\_slideshow\_mode() (polo.widgets.slideshow\_inspector.slideshowInspector method)@\spxentry{\_set\_slideshow\_mode()}\spxextra{polo.widgets.slideshow\_inspector.slideshowInspector method}}

\begin{fulllineitems}
\phantomsection\label{\detokenize{polo.widgets:polo.widgets.slideshow_inspector.slideshowInspector._set_slideshow_mode}}\pysiglinewithargsret{\sphinxbfcode{\sphinxupquote{\_set\_slideshow\_mode}}}{\emph{\DUrole{n}{show\_all\_dates}\DUrole{o}{=}\DUrole{default_value}{False}}, \emph{\DUrole{n}{show\_all\_specs}\DUrole{o}{=}\DUrole{default_value}{False}}}{}
Private method to set the slideshowViewer mode. Either to display
a single image, all dates or all spectrums.
\begin{quote}\begin{description}
\item[{Parameters}] \leavevmode\begin{itemize}
\item {} 
\sphinxstyleliteralstrong{\sphinxupquote{show\_all\_dates}} (\sphinxstyleliteralemphasis{\sphinxupquote{bool}}\sphinxstyleliteralemphasis{\sphinxupquote{, }}\sphinxstyleliteralemphasis{\sphinxupquote{optional}}) \textendash{} If True sets slideshowViewer to show all
dates, defaults to False

\item {} 
\sphinxstyleliteralstrong{\sphinxupquote{show\_all\_specs}} (\sphinxstyleliteralemphasis{\sphinxupquote{bool}}\sphinxstyleliteralemphasis{\sphinxupquote{, }}\sphinxstyleliteralemphasis{\sphinxupquote{optional}}) \textendash{} If True sets slideshowViewer to show all
spectrums, defaults to False

\end{itemize}

\end{description}\end{quote}

\end{fulllineitems}

\index{\_set\_time\_resolved\_functions() (polo.widgets.slideshow\_inspector.slideshowInspector method)@\spxentry{\_set\_time\_resolved\_functions()}\spxextra{polo.widgets.slideshow\_inspector.slideshowInspector method}}

\begin{fulllineitems}
\phantomsection\label{\detokenize{polo.widgets:polo.widgets.slideshow_inspector.slideshowInspector._set_time_resolved_functions}}\pysiglinewithargsret{\sphinxbfcode{\sphinxupquote{\_set\_time\_resolved\_functions}}}{}{}
Private method that turns time resolved functions on or off 
depending on contents of the \sphinxtitleref{Run} instance referenced by 
the \sphinxtitleref{run} attribute. Time resolved functions are enabled 
when the \sphinxtitleref{run} is part of a time resolved linked list. 
This means another \sphinxtitleref{Run} instance is referenced by 
it’s \sphinxtitleref{next\_run} and / or \sphinxtitleref{previous\_run} attributes.

\end{fulllineitems}

\index{\_show\_image\_from\_well\_number() (polo.widgets.slideshow\_inspector.slideshowInspector method)@\spxentry{\_show\_image\_from\_well\_number()}\spxextra{polo.widgets.slideshow\_inspector.slideshowInspector method}}

\begin{fulllineitems}
\phantomsection\label{\detokenize{polo.widgets:polo.widgets.slideshow_inspector.slideshowInspector._show_image_from_well_number}}\pysiglinewithargsret{\sphinxbfcode{\sphinxupquote{\_show\_image\_from\_well\_number}}}{\emph{\DUrole{n}{well\_number}}}{}
Private method to display an image by well number.
\begin{quote}\begin{description}
\item[{Parameters}] \leavevmode
\sphinxstyleliteralstrong{\sphinxupquote{well\_number}} (\sphinxstyleliteralemphasis{\sphinxupquote{int}}) \textendash{} Well number of image to display

\end{description}\end{quote}

\end{fulllineitems}

\index{\_submit\_filters() (polo.widgets.slideshow\_inspector.slideshowInspector method)@\spxentry{\_submit\_filters()}\spxextra{polo.widgets.slideshow\_inspector.slideshowInspector method}}

\begin{fulllineitems}
\phantomsection\label{\detokenize{polo.widgets:polo.widgets.slideshow_inspector.slideshowInspector._submit_filters}}\pysiglinewithargsret{\sphinxbfcode{\sphinxupquote{\_submit\_filters}}}{}{}
Private method that passes the current user selected
image filters to the slideshowViewer so the current
slideshow contents can be adjusted to reflect the
new filters. Displays the current image after filtering.

\end{fulllineitems}

\index{current\_image() (polo.widgets.slideshow\_inspector.slideshowInspector property)@\spxentry{current\_image()}\spxextra{polo.widgets.slideshow\_inspector.slideshowInspector property}}

\begin{fulllineitems}
\phantomsection\label{\detokenize{polo.widgets:polo.widgets.slideshow_inspector.slideshowInspector.current_image}}\pysigline{\sphinxbfcode{\sphinxupquote{property }}\sphinxbfcode{\sphinxupquote{current\_image}}}
Current {\hyperref[\detokenize{polo.crystallography:polo.crystallography.image.Image}]{\sphinxcrossref{\sphinxcode{\sphinxupquote{Image}}}}} object being displayed in the \sphinxtitleref{slideshowViewer}
widget.
\begin{quote}\begin{description}
\item[{Returns}] \leavevmode
The current image

\item[{Return type}] \leavevmode
{\hyperref[\detokenize{polo.crystallography:polo.crystallography.image.Image}]{\sphinxcrossref{Image}}}

\end{description}\end{quote}

\end{fulllineitems}

\index{current\_sort\_function() (polo.widgets.slideshow\_inspector.slideshowInspector property)@\spxentry{current\_sort\_function()}\spxextra{polo.widgets.slideshow\_inspector.slideshowInspector property}}

\begin{fulllineitems}
\phantomsection\label{\detokenize{polo.widgets:polo.widgets.slideshow_inspector.slideshowInspector.current_sort_function}}\pysigline{\sphinxbfcode{\sphinxupquote{property }}\sphinxbfcode{\sphinxupquote{current\_sort\_function}}}
Return a function to use for image sorting based on current user
radiobutton sort selections.
\begin{quote}\begin{description}
\item[{Returns}] \leavevmode
Sort function

\item[{Return type}] \leavevmode
func

\end{description}\end{quote}

\end{fulllineitems}

\index{export\_current\_view() (polo.widgets.slideshow\_inspector.slideshowInspector method)@\spxentry{export\_current\_view()}\spxextra{polo.widgets.slideshow\_inspector.slideshowInspector method}}

\begin{fulllineitems}
\phantomsection\label{\detokenize{polo.widgets:polo.widgets.slideshow_inspector.slideshowInspector.export_current_view}}\pysiglinewithargsret{\sphinxbfcode{\sphinxupquote{export\_current\_view}}}{}{}
Export the current view to a png file. Show the user a message box
to tell them if the export succeeded or failed.

\end{fulllineitems}

\index{favorites() (polo.widgets.slideshow\_inspector.slideshowInspector property)@\spxentry{favorites()}\spxextra{polo.widgets.slideshow\_inspector.slideshowInspector property}}

\begin{fulllineitems}
\phantomsection\label{\detokenize{polo.widgets:polo.widgets.slideshow_inspector.slideshowInspector.favorites}}\pysigline{\sphinxbfcode{\sphinxupquote{property }}\sphinxbfcode{\sphinxupquote{favorites}}}
Returns the state of the favorite \sphinxcode{\sphinxupquote{QCheckBox}}.
\begin{quote}\begin{description}
\item[{Returns}] \leavevmode
Favorite \sphinxcode{\sphinxupquote{QCheckBox}} state

\item[{Return type}] \leavevmode
bool

\end{description}\end{quote}

\end{fulllineitems}

\index{human() (polo.widgets.slideshow\_inspector.slideshowInspector property)@\spxentry{human()}\spxextra{polo.widgets.slideshow\_inspector.slideshowInspector property}}

\begin{fulllineitems}
\phantomsection\label{\detokenize{polo.widgets:polo.widgets.slideshow_inspector.slideshowInspector.human}}\pysigline{\sphinxbfcode{\sphinxupquote{property }}\sphinxbfcode{\sphinxupquote{human}}}
State of the human classifier \sphinxcode{\sphinxupquote{QCheckBox}}. If True, assume the user
wants their selected image classifications to be in reference to image’s
human classification.
\begin{quote}\begin{description}
\item[{Returns}] \leavevmode
State of the \sphinxcode{\sphinxupquote{QCheckBox}}

\item[{Return type}] \leavevmode
bool

\end{description}\end{quote}

\end{fulllineitems}

\index{marco() (polo.widgets.slideshow\_inspector.slideshowInspector property)@\spxentry{marco()}\spxextra{polo.widgets.slideshow\_inspector.slideshowInspector property}}

\begin{fulllineitems}
\phantomsection\label{\detokenize{polo.widgets:polo.widgets.slideshow_inspector.slideshowInspector.marco}}\pysigline{\sphinxbfcode{\sphinxupquote{property }}\sphinxbfcode{\sphinxupquote{marco}}}
State of the MARCO classifier \sphinxcode{\sphinxupquote{QCheckBox}}. If True, assume the user
wants their selected image classifications to be in reference to image’s
MARCO classification.
\begin{quote}\begin{description}
\item[{Returns}] \leavevmode
State of the \sphinxcode{\sphinxupquote{QCheckBox}}

\item[{Return type}] \leavevmode
bool

\end{description}\end{quote}

\end{fulllineitems}

\index{run() (polo.widgets.slideshow\_inspector.slideshowInspector property)@\spxentry{run()}\spxextra{polo.widgets.slideshow\_inspector.slideshowInspector property}}

\begin{fulllineitems}
\phantomsection\label{\detokenize{polo.widgets:polo.widgets.slideshow_inspector.slideshowInspector.run}}\pysigline{\sphinxbfcode{\sphinxupquote{property }}\sphinxbfcode{\sphinxupquote{run}}}
\end{fulllineitems}

\index{selected\_classifications() (polo.widgets.slideshow\_inspector.slideshowInspector property)@\spxentry{selected\_classifications()}\spxextra{polo.widgets.slideshow\_inspector.slideshowInspector property}}

\begin{fulllineitems}
\phantomsection\label{\detokenize{polo.widgets:polo.widgets.slideshow_inspector.slideshowInspector.selected_classifications}}\pysigline{\sphinxbfcode{\sphinxupquote{property }}\sphinxbfcode{\sphinxupquote{selected\_classifications}}}
Returns image classification keywords for any image classification
\sphinxcode{\sphinxupquote{QCheckBox}} instances that are checked.
\begin{quote}\begin{description}
\item[{Returns}] \leavevmode
List of selected images classifications

\item[{Return type}] \leavevmode
list

\end{description}\end{quote}

\end{fulllineitems}

\index{sort\_images\_by\_cocktail\_number() (polo.widgets.slideshow\_inspector.slideshowInspector static method)@\spxentry{sort\_images\_by\_cocktail\_number()}\spxextra{polo.widgets.slideshow\_inspector.slideshowInspector static method}}

\begin{fulllineitems}
\phantomsection\label{\detokenize{polo.widgets:polo.widgets.slideshow_inspector.slideshowInspector.sort_images_by_cocktail_number}}\pysiglinewithargsret{\sphinxbfcode{\sphinxupquote{static }}\sphinxbfcode{\sphinxupquote{sort\_images\_by\_cocktail\_number}}}{\emph{\DUrole{n}{images}}}{}
Helper method that sorts a collection of images by their
cocktail number. Returns False if the images cannot be sorted
by this parameter.
\begin{quote}\begin{description}
\item[{Parameters}] \leavevmode
\sphinxstyleliteralstrong{\sphinxupquote{images}} (\sphinxstyleliteralemphasis{\sphinxupquote{list}}) \textendash{} List of images to be sorted

\item[{Returns}] \leavevmode
List of images sorted by cocktail number, False if cannot be sorted

\item[{Return type}] \leavevmode
list, bool

\end{description}\end{quote}

\end{fulllineitems}

\index{sort\_images\_by\_marco\_confidence() (polo.widgets.slideshow\_inspector.slideshowInspector static method)@\spxentry{sort\_images\_by\_marco\_confidence()}\spxextra{polo.widgets.slideshow\_inspector.slideshowInspector static method}}

\begin{fulllineitems}
\phantomsection\label{\detokenize{polo.widgets:polo.widgets.slideshow_inspector.slideshowInspector.sort_images_by_marco_confidence}}\pysiglinewithargsret{\sphinxbfcode{\sphinxupquote{static }}\sphinxbfcode{\sphinxupquote{sort\_images\_by\_marco\_confidence}}}{\emph{\DUrole{n}{images}}}{}
Helper method to sort a collection of images by their MARCO
classification confidence. Does not descriminate based on
image classification.
\begin{quote}\begin{description}
\item[{Parameters}] \leavevmode
\sphinxstyleliteralstrong{\sphinxupquote{images}} (\sphinxstyleliteralemphasis{\sphinxupquote{list}}) \textendash{} List of images to sort

\item[{Returns}] \leavevmode
List of images sorted by prediction confidence

\item[{Return type}] \leavevmode
list

\end{description}\end{quote}

\end{fulllineitems}

\index{sort\_images\_by\_well\_number() (polo.widgets.slideshow\_inspector.slideshowInspector static method)@\spxentry{sort\_images\_by\_well\_number()}\spxextra{polo.widgets.slideshow\_inspector.slideshowInspector static method}}

\begin{fulllineitems}
\phantomsection\label{\detokenize{polo.widgets:polo.widgets.slideshow_inspector.slideshowInspector.sort_images_by_well_number}}\pysiglinewithargsret{\sphinxbfcode{\sphinxupquote{static }}\sphinxbfcode{\sphinxupquote{sort\_images\_by\_well\_number}}}{\emph{\DUrole{n}{images}}}{}
Helper method to sort a collection of images by their well number.
If images cannot be sorted by well number (which in theory shouldn’t happen)
returns False
\begin{quote}\begin{description}
\item[{Parameters}] \leavevmode
\sphinxstyleliteralstrong{\sphinxupquote{images}} (\sphinxstyleliteralemphasis{\sphinxupquote{list}}) \textendash{} List of images to be sorted

\item[{Returns}] \leavevmode
List os images sorted by well number

\item[{Return type}] \leavevmode
list

\end{description}\end{quote}

\end{fulllineitems}


\end{fulllineitems}



\subsubsection{polo.widgets.slideshow\_viewer module}
\label{\detokenize{polo.widgets:module-polo.widgets.slideshow_viewer}}\label{\detokenize{polo.widgets:polo-widgets-slideshow-viewer-module}}\index{module@\spxentry{module}!polo.widgets.slideshow\_viewer@\spxentry{polo.widgets.slideshow\_viewer}}\index{polo.widgets.slideshow\_viewer@\spxentry{polo.widgets.slideshow\_viewer}!module@\spxentry{module}}\index{Carousel (class in polo.widgets.slideshow\_viewer)@\spxentry{Carousel}\spxextra{class in polo.widgets.slideshow\_viewer}}

\begin{fulllineitems}
\phantomsection\label{\detokenize{polo.widgets:polo.widgets.slideshow_viewer.Carousel}}\pysigline{\sphinxbfcode{\sphinxupquote{class }}\sphinxcode{\sphinxupquote{polo.widgets.slideshow\_viewer.}}\sphinxbfcode{\sphinxupquote{Carousel}}}
Bases: \sphinxcode{\sphinxupquote{object}}

The Carousel class handles navigation between \sphinxtitleref{Slide} instances.
\index{add\_slides() (polo.widgets.slideshow\_viewer.Carousel method)@\spxentry{add\_slides()}\spxextra{polo.widgets.slideshow\_viewer.Carousel method}}

\begin{fulllineitems}
\phantomsection\label{\detokenize{polo.widgets:polo.widgets.slideshow_viewer.Carousel.add_slides}}\pysiglinewithargsret{\sphinxbfcode{\sphinxupquote{add\_slides}}}{\emph{\DUrole{n}{ordered\_images}}, \emph{\DUrole{n}{sort\_function}\DUrole{o}{=}\DUrole{default_value}{None}}}{}
Sets up linked list consisting of nodes of Slide instances. The list
is circular and bi\sphinxhyphen{}directional. Sets self.current\_slide to the first
slide in the linked list. The order of the slides in the linked list
will reflect the order of the images in the \sphinxtitleref{ordered\_images} argument.
\begin{quote}\begin{description}
\item[{Parameters}] \leavevmode
\sphinxstyleliteralstrong{\sphinxupquote{ordered\_images}} \textendash{} a list of Image objects to create the linked list            from. The order of the images will be reflected by the linked list.

\item[{Returns}] \leavevmode
First slide in linked list

\item[{Return type}] \leavevmode
{\hyperref[\detokenize{polo.widgets:polo.widgets.slideshow_viewer.Slide}]{\sphinxcrossref{Slide}}}

\end{description}\end{quote}

\end{fulllineitems}

\index{controls() (polo.widgets.slideshow\_viewer.Carousel method)@\spxentry{controls()}\spxextra{polo.widgets.slideshow\_viewer.Carousel method}}

\begin{fulllineitems}
\phantomsection\label{\detokenize{polo.widgets:polo.widgets.slideshow_viewer.Carousel.controls}}\pysiglinewithargsret{\sphinxbfcode{\sphinxupquote{controls}}}{\emph{\DUrole{n}{next\_slide}\DUrole{o}{=}\DUrole{default_value}{False}}, \emph{\DUrole{n}{prev\_slide}\DUrole{o}{=}\DUrole{default_value}{False}}}{}
Controls the navigation through the slides
in the carousel. Does not control access to alternative
images that may be available to the user.
\begin{quote}\begin{description}
\item[{Parameters}] \leavevmode\begin{itemize}
\item {} 
\sphinxstyleliteralstrong{\sphinxupquote{next\_slide}} (\sphinxstyleliteralemphasis{\sphinxupquote{bool}}) \textendash{} If set to True, tells the carousel to
advance one Slide

\item {} 
\sphinxstyleliteralstrong{\sphinxupquote{prev\_slide}} (\sphinxstyleliteralemphasis{\sphinxupquote{bool}}) \textendash{} If set to True, tells the carousel to
retreat by one Slide

\end{itemize}

\end{description}\end{quote}

\end{fulllineitems}

\index{current\_slide() (polo.widgets.slideshow\_viewer.Carousel property)@\spxentry{current\_slide()}\spxextra{polo.widgets.slideshow\_viewer.Carousel property}}

\begin{fulllineitems}
\phantomsection\label{\detokenize{polo.widgets:polo.widgets.slideshow_viewer.Carousel.current_slide}}\pysigline{\sphinxbfcode{\sphinxupquote{property }}\sphinxbfcode{\sphinxupquote{current\_slide}}}
Current slide, the slide that should be displayed to the user.
\begin{quote}\begin{description}
\item[{Returns}] \leavevmode
The current slide

\item[{Return type}] \leavevmode
{\hyperref[\detokenize{polo.widgets:polo.widgets.slideshow_viewer.Slide}]{\sphinxcrossref{Slide}}}

\end{description}\end{quote}

\end{fulllineitems}


\end{fulllineitems}

\index{PhotoViewer (class in polo.widgets.slideshow\_viewer)@\spxentry{PhotoViewer}\spxextra{class in polo.widgets.slideshow\_viewer}}

\begin{fulllineitems}
\phantomsection\label{\detokenize{polo.widgets:polo.widgets.slideshow_viewer.PhotoViewer}}\pysiglinewithargsret{\sphinxbfcode{\sphinxupquote{class }}\sphinxcode{\sphinxupquote{polo.widgets.slideshow\_viewer.}}\sphinxbfcode{\sphinxupquote{PhotoViewer}}}{\emph{\DUrole{n}{parent}}}{}
Bases: \sphinxcode{\sphinxupquote{PyQt5.QtWidgets.QGraphicsView}}
\index{add\_pixmap() (polo.widgets.slideshow\_viewer.PhotoViewer method)@\spxentry{add\_pixmap()}\spxextra{polo.widgets.slideshow\_viewer.PhotoViewer method}}

\begin{fulllineitems}
\phantomsection\label{\detokenize{polo.widgets:polo.widgets.slideshow_viewer.PhotoViewer.add_pixmap}}\pysiglinewithargsret{\sphinxbfcode{\sphinxupquote{add\_pixmap}}}{\emph{\DUrole{n}{pixmap}}}{}
Adds a \sphinxtitleref{Pixmap} instances to the current sene.
\begin{quote}\begin{description}
\item[{Parameters}] \leavevmode
\sphinxstyleliteralstrong{\sphinxupquote{pixmap}} (\sphinxstyleliteralemphasis{\sphinxupquote{Pixmap}}) \textendash{} Pixmap to add to the sene

\end{description}\end{quote}

\end{fulllineitems}

\index{fitInView() (polo.widgets.slideshow\_viewer.PhotoViewer method)@\spxentry{fitInView()}\spxextra{polo.widgets.slideshow\_viewer.PhotoViewer method}}

\begin{fulllineitems}
\phantomsection\label{\detokenize{polo.widgets:polo.widgets.slideshow_viewer.PhotoViewer.fitInView}}\pysiglinewithargsret{\sphinxbfcode{\sphinxupquote{fitInView}}}{\emph{\DUrole{n}{self}}, \emph{\DUrole{n}{QRectF}}, \emph{\DUrole{n}{mode}\DUrole{p}{:} \DUrole{n}{Qt.AspectRatioMode} \DUrole{o}{=} \DUrole{default_value}{Qt.IgnoreAspectRatio}}}{}
fitInView(self, QGraphicsItem, mode: Qt.AspectRatioMode = Qt.IgnoreAspectRatio)
fitInView(self, float, float, float, float, mode: Qt.AspectRatioMode = Qt.IgnoreAspectRatio)

\end{fulllineitems}

\index{hasPhoto() (polo.widgets.slideshow\_viewer.PhotoViewer method)@\spxentry{hasPhoto()}\spxextra{polo.widgets.slideshow\_viewer.PhotoViewer method}}

\begin{fulllineitems}
\phantomsection\label{\detokenize{polo.widgets:polo.widgets.slideshow_viewer.PhotoViewer.hasPhoto}}\pysiglinewithargsret{\sphinxbfcode{\sphinxupquote{hasPhoto}}}{}{}
\end{fulllineitems}

\index{mousePressEvent() (polo.widgets.slideshow\_viewer.PhotoViewer method)@\spxentry{mousePressEvent()}\spxextra{polo.widgets.slideshow\_viewer.PhotoViewer method}}

\begin{fulllineitems}
\phantomsection\label{\detokenize{polo.widgets:polo.widgets.slideshow_viewer.PhotoViewer.mousePressEvent}}\pysiglinewithargsret{\sphinxbfcode{\sphinxupquote{mousePressEvent}}}{\emph{\DUrole{n}{event}}}{}
Handles mouse press events.
\begin{quote}\begin{description}
\item[{Parameters}] \leavevmode
\sphinxstyleliteralstrong{\sphinxupquote{event}} (\sphinxstyleliteralemphasis{\sphinxupquote{QEvent}}) \textendash{} Mouse press event

\end{description}\end{quote}

\end{fulllineitems}

\index{photoClicked (polo.widgets.slideshow\_viewer.PhotoViewer attribute)@\spxentry{photoClicked}\spxextra{polo.widgets.slideshow\_viewer.PhotoViewer attribute}}

\begin{fulllineitems}
\phantomsection\label{\detokenize{polo.widgets:polo.widgets.slideshow_viewer.PhotoViewer.photoClicked}}\pysigline{\sphinxbfcode{\sphinxupquote{photoClicked}}}
\end{fulllineitems}

\index{toggleDragMode() (polo.widgets.slideshow\_viewer.PhotoViewer method)@\spxentry{toggleDragMode()}\spxextra{polo.widgets.slideshow\_viewer.PhotoViewer method}}

\begin{fulllineitems}
\phantomsection\label{\detokenize{polo.widgets:polo.widgets.slideshow_viewer.PhotoViewer.toggleDragMode}}\pysiglinewithargsret{\sphinxbfcode{\sphinxupquote{toggleDragMode}}}{}{}
Turns drag mode on and off.

\end{fulllineitems}

\index{wheelEvent() (polo.widgets.slideshow\_viewer.PhotoViewer method)@\spxentry{wheelEvent()}\spxextra{polo.widgets.slideshow\_viewer.PhotoViewer method}}

\begin{fulllineitems}
\phantomsection\label{\detokenize{polo.widgets:polo.widgets.slideshow_viewer.PhotoViewer.wheelEvent}}\pysiglinewithargsret{\sphinxbfcode{\sphinxupquote{wheelEvent}}}{\emph{\DUrole{n}{event}}}{}
Handles mouse wheel events to allow for scaling for zooming in and
out of the currently displayed image.
\begin{quote}\begin{description}
\item[{Parameters}] \leavevmode
\sphinxstyleliteralstrong{\sphinxupquote{event}} (\sphinxstyleliteralemphasis{\sphinxupquote{QEvent}}) \textendash{} Mouse scroll wheel event

\end{description}\end{quote}

\end{fulllineitems}


\end{fulllineitems}

\index{Slide (class in polo.widgets.slideshow\_viewer)@\spxentry{Slide}\spxextra{class in polo.widgets.slideshow\_viewer}}

\begin{fulllineitems}
\phantomsection\label{\detokenize{polo.widgets:polo.widgets.slideshow_viewer.Slide}}\pysiglinewithargsret{\sphinxbfcode{\sphinxupquote{class }}\sphinxcode{\sphinxupquote{polo.widgets.slideshow\_viewer.}}\sphinxbfcode{\sphinxupquote{Slide}}}{\emph{\DUrole{n}{image}}, \emph{\DUrole{n}{next\_slide}\DUrole{o}{=}\DUrole{default_value}{None}}, \emph{\DUrole{n}{prev\_slide}\DUrole{o}{=}\DUrole{default_value}{None}}, \emph{\DUrole{n}{slide\_number}\DUrole{o}{=}\DUrole{default_value}{None}}}{}
Bases: \sphinxcode{\sphinxupquote{object}}

Acts like a slide in a slideshow carousel. Holds an Image object instance
as the contents of the slide. Forms a linked list with other slides through
the \sphinxtitleref{next\_slide} and \sphinxtitleref{prev\_slide} attributes which act as the forwards
and backwards pointers to other slides.
\begin{quote}\begin{description}
\item[{Parameters}] \leavevmode\begin{itemize}
\item {} 
\sphinxstyleliteralstrong{\sphinxupquote{image}} ({\hyperref[\detokenize{polo.crystallography:polo.crystallography.image.Image}]{\sphinxcrossref{\sphinxstyleliteralemphasis{\sphinxupquote{Image}}}}}) \textendash{} Image that this slide will display

\item {} 
\sphinxstyleliteralstrong{\sphinxupquote{next\_slide}} ({\hyperref[\detokenize{polo.widgets:polo.widgets.slideshow_viewer.Slide}]{\sphinxcrossref{\sphinxstyleliteralemphasis{\sphinxupquote{Slide}}}}}\sphinxstyleliteralemphasis{\sphinxupquote{, }}\sphinxstyleliteralemphasis{\sphinxupquote{optional}}) \textendash{} Next slide in the slideshow, defaults to None

\item {} 
\sphinxstyleliteralstrong{\sphinxupquote{prev\_slide}} ({\hyperref[\detokenize{polo.widgets:polo.widgets.slideshow_viewer.Slide}]{\sphinxcrossref{\sphinxstyleliteralemphasis{\sphinxupquote{Slide}}}}}\sphinxstyleliteralemphasis{\sphinxupquote{, }}\sphinxstyleliteralemphasis{\sphinxupquote{optional}}) \textendash{} Previous slide in the slideshow, defaults to None

\item {} 
\sphinxstyleliteralstrong{\sphinxupquote{slide\_number}} (\sphinxstyleliteralemphasis{\sphinxupquote{int}}\sphinxstyleliteralemphasis{\sphinxupquote{, }}\sphinxstyleliteralemphasis{\sphinxupquote{optional}}) \textendash{} Index of this slide in the slideshow, defaults to None

\end{itemize}

\end{description}\end{quote}

\end{fulllineitems}

\index{SlideshowViewer (class in polo.widgets.slideshow\_viewer)@\spxentry{SlideshowViewer}\spxextra{class in polo.widgets.slideshow\_viewer}}

\begin{fulllineitems}
\phantomsection\label{\detokenize{polo.widgets:polo.widgets.slideshow_viewer.SlideshowViewer}}\pysiglinewithargsret{\sphinxbfcode{\sphinxupquote{class }}\sphinxcode{\sphinxupquote{polo.widgets.slideshow\_viewer.}}\sphinxbfcode{\sphinxupquote{SlideshowViewer}}}{\emph{\DUrole{n}{parent}}, \emph{\DUrole{n}{run}\DUrole{o}{=}\DUrole{default_value}{None}}, \emph{\DUrole{n}{current\_image}\DUrole{o}{=}\DUrole{default_value}{None}}}{}
Bases: {\hyperref[\detokenize{polo.widgets:polo.widgets.slideshow_viewer.PhotoViewer}]{\sphinxcrossref{\sphinxcode{\sphinxupquote{polo.widgets.slideshow\_viewer.PhotoViewer}}}}}
\index{\_add\_text\_to\_scene() (polo.widgets.slideshow\_viewer.SlideshowViewer method)@\spxentry{\_add\_text\_to\_scene()}\spxextra{polo.widgets.slideshow\_viewer.SlideshowViewer method}}

\begin{fulllineitems}
\phantomsection\label{\detokenize{polo.widgets:polo.widgets.slideshow_viewer.SlideshowViewer._add_text_to_scene}}\pysiglinewithargsret{\sphinxbfcode{\sphinxupquote{\_add\_text\_to\_scene}}}{\emph{\DUrole{n}{text}}, \emph{\DUrole{n}{x}}, \emph{\DUrole{n}{y}}, \emph{\DUrole{n}{size}\DUrole{o}{=}\DUrole{default_value}{40}}}{}
Private method to add text on top of an image. Adds the text to
the current scene at the position specified by the \sphinxtitleref{x} and \sphinxtitleref{y}
arguments.
\begin{quote}\begin{description}
\item[{Parameters}] \leavevmode\begin{itemize}
\item {} 
\sphinxstyleliteralstrong{\sphinxupquote{text}} (\sphinxstyleliteralemphasis{\sphinxupquote{str}}) \textendash{} Text to add to image

\item {} 
\sphinxstyleliteralstrong{\sphinxupquote{x}} (\sphinxstyleliteralemphasis{\sphinxupquote{int}}) \textendash{} X cordinate of text

\item {} 
\sphinxstyleliteralstrong{\sphinxupquote{y}} (\sphinxstyleliteralemphasis{\sphinxupquote{int}}) \textendash{} Y cordinate of text

\item {} 
\sphinxstyleliteralstrong{\sphinxupquote{size}} (\sphinxstyleliteralemphasis{\sphinxupquote{int}}\sphinxstyleliteralemphasis{\sphinxupquote{, }}\sphinxstyleliteralemphasis{\sphinxupquote{optional}}) \textendash{} Size of text, defaults to 40

\end{itemize}

\end{description}\end{quote}

\end{fulllineitems}

\index{\_set\_all\_dates\_scene() (polo.widgets.slideshow\_viewer.SlideshowViewer method)@\spxentry{\_set\_all\_dates\_scene()}\spxextra{polo.widgets.slideshow\_viewer.SlideshowViewer method}}

\begin{fulllineitems}
\phantomsection\label{\detokenize{polo.widgets:polo.widgets.slideshow_viewer.SlideshowViewer._set_all_dates_scene}}\pysiglinewithargsret{\sphinxbfcode{\sphinxupquote{\_set\_all\_dates\_scene}}}{\emph{\DUrole{n}{image}}}{}
Private method that creates a time resolved view from the {\hyperref[\detokenize{polo.crystallography:polo.crystallography.image.Image}]{\sphinxcrossref{\sphinxcode{\sphinxupquote{Image}}}}} 
instance passed through the \sphinxtitleref{image} argument.
\begin{quote}\begin{description}
\item[{Parameters}] \leavevmode
\sphinxstyleliteralstrong{\sphinxupquote{image}} ({\hyperref[\detokenize{polo.crystallography:polo.crystallography.image.Image}]{\sphinxcrossref{\sphinxstyleliteralemphasis{\sphinxupquote{Image}}}}}) \textendash{} Image to create time resolved view from

\end{description}\end{quote}

\end{fulllineitems}

\index{\_set\_all\_spectrums\_scene() (polo.widgets.slideshow\_viewer.SlideshowViewer method)@\spxentry{\_set\_all\_spectrums\_scene()}\spxextra{polo.widgets.slideshow\_viewer.SlideshowViewer method}}

\begin{fulllineitems}
\phantomsection\label{\detokenize{polo.widgets:polo.widgets.slideshow_viewer.SlideshowViewer._set_all_spectrums_scene}}\pysiglinewithargsret{\sphinxbfcode{\sphinxupquote{\_set\_all\_spectrums\_scene}}}{\emph{\DUrole{n}{image}}}{}
Private method that creates a view that includes all alt spectrum
images the {\hyperref[\detokenize{polo.crystallography:polo.crystallography.image.Image}]{\sphinxcrossref{\sphinxcode{\sphinxupquote{Image}}}}} instance is linked to.
\begin{quote}\begin{description}
\item[{Parameters}] \leavevmode
\sphinxstyleliteralstrong{\sphinxupquote{image}} ({\hyperref[\detokenize{polo.crystallography:polo.crystallography.image.Image}]{\sphinxcrossref{\sphinxstyleliteralemphasis{\sphinxupquote{Image}}}}}) \textendash{} Image to create the view from

\end{description}\end{quote}

\end{fulllineitems}

\index{\_set\_single\_image\_scene() (polo.widgets.slideshow\_viewer.SlideshowViewer method)@\spxentry{\_set\_single\_image\_scene()}\spxextra{polo.widgets.slideshow\_viewer.SlideshowViewer method}}

\begin{fulllineitems}
\phantomsection\label{\detokenize{polo.widgets:polo.widgets.slideshow_viewer.SlideshowViewer._set_single_image_scene}}\pysiglinewithargsret{\sphinxbfcode{\sphinxupquote{\_set\_single\_image\_scene}}}{\emph{\DUrole{n}{image}}}{}
Private method that creates a standard single image view from the 
{\hyperref[\detokenize{polo.crystallography:polo.crystallography.image.Image}]{\sphinxcrossref{\sphinxcode{\sphinxupquote{Image}}}}} instance passed to the \sphinxtitleref{image} argument.
\begin{quote}\begin{description}
\item[{Parameters}] \leavevmode
\sphinxstyleliteralstrong{\sphinxupquote{image}} ({\hyperref[\detokenize{polo.crystallography:polo.crystallography.image.Image}]{\sphinxcrossref{\sphinxstyleliteralemphasis{\sphinxupquote{Image}}}}}) \textendash{} Image to display

\end{description}\end{quote}

\end{fulllineitems}

\index{arrange\_multi\_image\_scene() (polo.widgets.slideshow\_viewer.SlideshowViewer method)@\spxentry{arrange\_multi\_image\_scene()}\spxextra{polo.widgets.slideshow\_viewer.SlideshowViewer method}}

\begin{fulllineitems}
\phantomsection\label{\detokenize{polo.widgets:polo.widgets.slideshow_viewer.SlideshowViewer.arrange_multi_image_scene}}\pysiglinewithargsret{\sphinxbfcode{\sphinxupquote{arrange\_multi\_image\_scene}}}{\emph{\DUrole{n}{image\_list}}, \emph{\DUrole{n}{render\_date}\DUrole{o}{=}\DUrole{default_value}{False}}}{}
Helper method to arrange multiple images into the same
view.
\begin{quote}\begin{description}
\item[{Parameters}] \leavevmode\begin{itemize}
\item {} 
\sphinxstyleliteralstrong{\sphinxupquote{image\_list}} (\sphinxstyleliteralemphasis{\sphinxupquote{list}}) \textendash{} List of images to add to the view

\item {} 
\sphinxstyleliteralstrong{\sphinxupquote{render\_date}} (\sphinxstyleliteralemphasis{\sphinxupquote{bool}}\sphinxstyleliteralemphasis{\sphinxupquote{, }}\sphinxstyleliteralemphasis{\sphinxupquote{optional}}) \textendash{} If True adds a date label to each image, defaults to False

\end{itemize}

\end{description}\end{quote}

\end{fulllineitems}

\index{carousel\_controls() (polo.widgets.slideshow\_viewer.SlideshowViewer method)@\spxentry{carousel\_controls()}\spxextra{polo.widgets.slideshow\_viewer.SlideshowViewer method}}

\begin{fulllineitems}
\phantomsection\label{\detokenize{polo.widgets:polo.widgets.slideshow_viewer.SlideshowViewer.carousel_controls}}\pysiglinewithargsret{\sphinxbfcode{\sphinxupquote{carousel\_controls}}}{\emph{\DUrole{n}{next\_image}\DUrole{o}{=}\DUrole{default_value}{False}}, \emph{\DUrole{n}{previous\_image}\DUrole{o}{=}\DUrole{default_value}{False}}}{}
Wrapper around the {\hyperref[\detokenize{polo.widgets:polo.widgets.slideshow_viewer.Carousel.controls}]{\sphinxcrossref{\sphinxcode{\sphinxupquote{controls()}}}}}
method that allows image navigation. Does not actually display the
image.
\begin{quote}\begin{description}
\item[{Parameters}] \leavevmode\begin{itemize}
\item {} 
\sphinxstyleliteralstrong{\sphinxupquote{next\_image}} (\sphinxstyleliteralemphasis{\sphinxupquote{bool}}) \textendash{} If True, tells carousel to advance by one slide.

\item {} 
\sphinxstyleliteralstrong{\sphinxupquote{previous\_image}} (\sphinxstyleliteralemphasis{\sphinxupquote{bool}}) \textendash{} If True, tells carousel to retreat by one slide.

\end{itemize}

\end{description}\end{quote}

:returns The current image.
:rtype Image

\end{fulllineitems}

\index{classify\_current\_image() (polo.widgets.slideshow\_viewer.SlideshowViewer method)@\spxentry{classify\_current\_image()}\spxextra{polo.widgets.slideshow\_viewer.SlideshowViewer method}}

\begin{fulllineitems}
\phantomsection\label{\detokenize{polo.widgets:polo.widgets.slideshow_viewer.SlideshowViewer.classify_current_image}}\pysiglinewithargsret{\sphinxbfcode{\sphinxupquote{classify\_current\_image}}}{\emph{\DUrole{n}{classification}}}{}
Changes the human classification of the current image.

\end{fulllineitems}

\index{display\_current\_image() (polo.widgets.slideshow\_viewer.SlideshowViewer method)@\spxentry{display\_current\_image()}\spxextra{polo.widgets.slideshow\_viewer.SlideshowViewer method}}

\begin{fulllineitems}
\phantomsection\label{\detokenize{polo.widgets:polo.widgets.slideshow_viewer.SlideshowViewer.display_current_image}}\pysiglinewithargsret{\sphinxbfcode{\sphinxupquote{display\_current\_image}}}{}{}
Renders the Image instance currently stored in the \sphinxtitleref{current\_image}
attribute.

\end{fulllineitems}

\index{get\_cur\_img\_cocktail\_str() (polo.widgets.slideshow\_viewer.SlideshowViewer method)@\spxentry{get\_cur\_img\_cocktail\_str()}\spxextra{polo.widgets.slideshow\_viewer.SlideshowViewer method}}

\begin{fulllineitems}
\phantomsection\label{\detokenize{polo.widgets:polo.widgets.slideshow_viewer.SlideshowViewer.get_cur_img_cocktail_str}}\pysiglinewithargsret{\sphinxbfcode{\sphinxupquote{get\_cur\_img\_cocktail\_str}}}{}{}
Retruns the \sphinxtitleref{current\_image} cocktail information
as a string.
\begin{quote}\begin{description}
\item[{Returns}] \leavevmode
Cocktail information string

\item[{Return type}] \leavevmode
str

\end{description}\end{quote}

\end{fulllineitems}

\index{get\_cur\_img\_meta\_str() (polo.widgets.slideshow\_viewer.SlideshowViewer method)@\spxentry{get\_cur\_img\_meta\_str()}\spxextra{polo.widgets.slideshow\_viewer.SlideshowViewer method}}

\begin{fulllineitems}
\phantomsection\label{\detokenize{polo.widgets:polo.widgets.slideshow_viewer.SlideshowViewer.get_cur_img_meta_str}}\pysiglinewithargsret{\sphinxbfcode{\sphinxupquote{get\_cur\_img\_meta\_str}}}{}{}
Returns the \sphinxtitleref{current\_image} metadata as a string.
\begin{quote}\begin{description}
\item[{Returns}] \leavevmode
Metadata string

\item[{Return type}] \leavevmode
str

\end{description}\end{quote}

\end{fulllineitems}

\index{photoClicked (polo.widgets.slideshow\_viewer.SlideshowViewer attribute)@\spxentry{photoClicked}\spxextra{polo.widgets.slideshow\_viewer.SlideshowViewer attribute}}

\begin{fulllineitems}
\phantomsection\label{\detokenize{polo.widgets:polo.widgets.slideshow_viewer.SlideshowViewer.photoClicked}}\pysigline{\sphinxbfcode{\sphinxupquote{photoClicked}}}
Wrapper class around QGraphicsView and displays image to the user
in the slideshow viewer tab of the main window.
\begin{quote}\begin{description}
\item[{Parameters}] \leavevmode\begin{itemize}
\item {} 
\sphinxstyleliteralstrong{\sphinxupquote{run}} ({\hyperref[\detokenize{polo.crystallography:polo.crystallography.run.Run}]{\sphinxcrossref{\sphinxstyleliteralemphasis{\sphinxupquote{Run}}}}}) \textendash{} Current run whose images are to be shown by the viewer.

\item {} 
\sphinxstyleliteralstrong{\sphinxupquote{parent}} (\sphinxstyleliteralemphasis{\sphinxupquote{QWidget}}) \textendash{} Parent Widget of this instance.

\item {} 
\sphinxstyleliteralstrong{\sphinxupquote{current\_image}} ({\hyperref[\detokenize{polo.crystallography:polo.crystallography.image.Image}]{\sphinxcrossref{\sphinxstyleliteralemphasis{\sphinxupquote{Image}}}}}) \textendash{} Image that is currently displayed by the viewer.

\end{itemize}

\end{description}\end{quote}

\end{fulllineitems}

\index{run() (polo.widgets.slideshow\_viewer.SlideshowViewer property)@\spxentry{run()}\spxextra{polo.widgets.slideshow\_viewer.SlideshowViewer property}}

\begin{fulllineitems}
\phantomsection\label{\detokenize{polo.widgets:polo.widgets.slideshow_viewer.SlideshowViewer.run}}\pysigline{\sphinxbfcode{\sphinxupquote{property }}\sphinxbfcode{\sphinxupquote{run}}}
\end{fulllineitems}

\index{set\_alt\_image() (polo.widgets.slideshow\_viewer.SlideshowViewer method)@\spxentry{set\_alt\_image()}\spxextra{polo.widgets.slideshow\_viewer.SlideshowViewer method}}

\begin{fulllineitems}
\phantomsection\label{\detokenize{polo.widgets:polo.widgets.slideshow_viewer.SlideshowViewer.set_alt_image}}\pysiglinewithargsret{\sphinxbfcode{\sphinxupquote{set\_alt\_image}}}{\emph{\DUrole{n}{next\_date}\DUrole{o}{=}\DUrole{default_value}{False}}, \emph{\DUrole{n}{prev\_date}\DUrole{o}{=}\DUrole{default_value}{False}}, \emph{\DUrole{n}{alt\_spec}\DUrole{o}{=}\DUrole{default_value}{False}}}{}
Sets the \sphinxtitleref{current\_image} attribute to a linked image specified by
one of the three boolean flags.
\begin{quote}\begin{description}
\item[{Parameters}] \leavevmode\begin{itemize}
\item {} 
\sphinxstyleliteralstrong{\sphinxupquote{next\_date}} \textendash{} If True, sets the \sphinxtitleref{current\_image}
to the next image by date

\item {} 
\sphinxstyleliteralstrong{\sphinxupquote{prev\_date}} \textendash{} If True, sets the \sphinxtitleref{current\_image}
to the previous image by date

\item {} 
\sphinxstyleliteralstrong{\sphinxupquote{alt\_spec}} \textendash{} If True, sets the \sphinxtitleref{current\_image}
to an alt spectrum image

\end{itemize}

\end{description}\end{quote}

\end{fulllineitems}

\index{set\_current\_image\_by\_well\_number() (polo.widgets.slideshow\_viewer.SlideshowViewer method)@\spxentry{set\_current\_image\_by\_well\_number()}\spxextra{polo.widgets.slideshow\_viewer.SlideshowViewer method}}

\begin{fulllineitems}
\phantomsection\label{\detokenize{polo.widgets:polo.widgets.slideshow_viewer.SlideshowViewer.set_current_image_by_well_number}}\pysiglinewithargsret{\sphinxbfcode{\sphinxupquote{set\_current\_image\_by\_well\_number}}}{\emph{\DUrole{n}{well\_number}}}{}
Set the current image to the {\hyperref[\detokenize{polo.crystallography:polo.crystallography.image.Image}]{\sphinxcrossref{\sphinxcode{\sphinxupquote{Image}}}}} instance associated with a
specific well number.
\begin{quote}\begin{description}
\item[{Parameters}] \leavevmode
\sphinxstyleliteralstrong{\sphinxupquote{well\_number}} (\sphinxstyleliteralemphasis{\sphinxupquote{int}}) \textendash{} Well number to display

\end{description}\end{quote}

\end{fulllineitems}

\index{update\_slides\_from\_filters() (polo.widgets.slideshow\_viewer.SlideshowViewer method)@\spxentry{update\_slides\_from\_filters()}\spxextra{polo.widgets.slideshow\_viewer.SlideshowViewer method}}

\begin{fulllineitems}
\phantomsection\label{\detokenize{polo.widgets:polo.widgets.slideshow_viewer.SlideshowViewer.update_slides_from_filters}}\pysiglinewithargsret{\sphinxbfcode{\sphinxupquote{update\_slides\_from\_filters}}}{\emph{\DUrole{n}{image\_types}}, \emph{\DUrole{n}{human}}, \emph{\DUrole{n}{marco}}, \emph{\DUrole{n}{favorite}\DUrole{o}{=}\DUrole{default_value}{False}}, \emph{\DUrole{n}{sort\_function}\DUrole{o}{=}\DUrole{default_value}{None}}}{}
Creates new \sphinxtitleref{Carousel} slides based on selected image filters.
Sets the \sphinxtitleref{current\_image} attribute to the {\hyperref[\detokenize{polo.crystallography:polo.crystallography.image.Image}]{\sphinxcrossref{\sphinxcode{\sphinxupquote{Image}}}}} instance at the 
the \sphinxtitleref{current slide} attribute of \sphinxtitleref{\_carousel} attribute.
\begin{quote}\begin{description}
\item[{Parameters}] \leavevmode\begin{itemize}
\item {} 
\sphinxstyleliteralstrong{\sphinxupquote{image\_types}} (\sphinxstyleliteralemphasis{\sphinxupquote{set}}\sphinxstyleliteralemphasis{\sphinxupquote{ or }}\sphinxstyleliteralemphasis{\sphinxupquote{list}}) \textendash{} Set of image classifications to include in results.

\item {} 
\sphinxstyleliteralstrong{\sphinxupquote{human}} (\sphinxstyleliteralemphasis{\sphinxupquote{bool}}) \textendash{} If True, \sphinxtitleref{image\_types} refers to human classification 
of the image.

\item {} 
\sphinxstyleliteralstrong{\sphinxupquote{marco}} (\sphinxstyleliteralemphasis{\sphinxupquote{bool}}) \textendash{} If True, \sphinxtitleref{image\_types} refers to the machine 
(MARCO) classification of the image.

\end{itemize}

\end{description}\end{quote}

\end{fulllineitems}


\end{fulllineitems}



\subsubsection{polo.widgets.table\_inspector module}
\label{\detokenize{polo.widgets:module-polo.widgets.table_inspector}}\label{\detokenize{polo.widgets:polo-widgets-table-inspector-module}}\index{module@\spxentry{module}!polo.widgets.table\_inspector@\spxentry{polo.widgets.table\_inspector}}\index{polo.widgets.table\_inspector@\spxentry{polo.widgets.table\_inspector}!module@\spxentry{module}}\index{TableInspector (class in polo.widgets.table\_inspector)@\spxentry{TableInspector}\spxextra{class in polo.widgets.table\_inspector}}

\begin{fulllineitems}
\phantomsection\label{\detokenize{polo.widgets:polo.widgets.table_inspector.TableInspector}}\pysiglinewithargsret{\sphinxbfcode{\sphinxupquote{class }}\sphinxcode{\sphinxupquote{polo.widgets.table\_inspector.}}\sphinxbfcode{\sphinxupquote{TableInspector}}}{\emph{\DUrole{n}{parent}\DUrole{o}{=}\DUrole{default_value}{None}}, \emph{\DUrole{n}{run}\DUrole{o}{=}\DUrole{default_value}{None}}}{}
Bases: \sphinxcode{\sphinxupquote{PyQt5.QtWidgets.QWidget}}

TableInspector class displays a run’s data in a spreadsheet
type view.
\begin{quote}\begin{description}
\item[{Parameters}] \leavevmode\begin{itemize}
\item {} 
\sphinxstyleliteralstrong{\sphinxupquote{parent}} (\sphinxstyleliteralemphasis{\sphinxupquote{QWidget}}\sphinxstyleliteralemphasis{\sphinxupquote{, }}\sphinxstyleliteralemphasis{\sphinxupquote{optional}}) \textendash{} Parent widget, defaults to None

\item {} 
\sphinxstyleliteralstrong{\sphinxupquote{run}} ({\hyperref[\detokenize{polo.crystallography:polo.crystallography.run.Run}]{\sphinxcrossref{\sphinxstyleliteralemphasis{\sphinxupquote{Run}}}}}\sphinxstyleliteralemphasis{\sphinxupquote{ or }}{\hyperref[\detokenize{polo.crystallography:polo.crystallography.run.HWIRun}]{\sphinxcrossref{\sphinxstyleliteralemphasis{\sphinxupquote{HWIRun}}}}}\sphinxstyleliteralemphasis{\sphinxupquote{, }}\sphinxstyleliteralemphasis{\sphinxupquote{optional}}) \textendash{} Run to display in the table, defaults to None

\end{itemize}

\end{description}\end{quote}
\index{\_assign\_checkboxes\_to\_class() (polo.widgets.table\_inspector.TableInspector method)@\spxentry{\_assign\_checkboxes\_to\_class()}\spxextra{polo.widgets.table\_inspector.TableInspector method}}

\begin{fulllineitems}
\phantomsection\label{\detokenize{polo.widgets:polo.widgets.table_inspector.TableInspector._assign_checkboxes_to_class}}\pysiglinewithargsret{\sphinxbfcode{\sphinxupquote{\_assign\_checkboxes\_to\_class}}}{}{}
Private method that assigns filtering checkboxs to an
image classification from the \sphinxtitleref{IMAGE\_CLASSIFICATION} constant.

\end{fulllineitems}

\index{\_set\_column\_options() (polo.widgets.table\_inspector.TableInspector method)@\spxentry{\_set\_column\_options()}\spxextra{polo.widgets.table\_inspector.TableInspector method}}

\begin{fulllineitems}
\phantomsection\label{\detokenize{polo.widgets:polo.widgets.table_inspector.TableInspector._set_column_options}}\pysiglinewithargsret{\sphinxbfcode{\sphinxupquote{\_set\_column\_options}}}{}{}
Private method that sets the availabe columns to display
based on the attributes of the run stored in the \sphinxtitleref{run} attribute.
Adds a \sphinxcode{\sphinxupquote{QCheckBox}} widget for each attribute.

TODO: formating for private attributes to make them prettier

\end{fulllineitems}

\index{run() (polo.widgets.table\_inspector.TableInspector property)@\spxentry{run()}\spxextra{polo.widgets.table\_inspector.TableInspector property}}

\begin{fulllineitems}
\phantomsection\label{\detokenize{polo.widgets:polo.widgets.table_inspector.TableInspector.run}}\pysigline{\sphinxbfcode{\sphinxupquote{property }}\sphinxbfcode{\sphinxupquote{run}}}
\end{fulllineitems}

\index{selected\_classifications() (polo.widgets.table\_inspector.TableInspector property)@\spxentry{selected\_classifications()}\spxextra{polo.widgets.table\_inspector.TableInspector property}}

\begin{fulllineitems}
\phantomsection\label{\detokenize{polo.widgets:polo.widgets.table_inspector.TableInspector.selected_classifications}}\pysigline{\sphinxbfcode{\sphinxupquote{property }}\sphinxbfcode{\sphinxupquote{selected\_classifications}}}
Return image classifications that are currently selected.
\begin{quote}\begin{description}
\item[{Returns}] \leavevmode
List of selected image classifications

\item[{Return type}] \leavevmode
set

\end{description}\end{quote}

\end{fulllineitems}

\index{selected\_headers() (polo.widgets.table\_inspector.TableInspector property)@\spxentry{selected\_headers()}\spxextra{polo.widgets.table\_inspector.TableInspector property}}

\begin{fulllineitems}
\phantomsection\label{\detokenize{polo.widgets:polo.widgets.table_inspector.TableInspector.selected_headers}}\pysigline{\sphinxbfcode{\sphinxupquote{property }}\sphinxbfcode{\sphinxupquote{selected\_headers}}}
Return the headers that have been selected by the user.
\begin{quote}\begin{description}
\item[{Returns}] \leavevmode
Names of column headers that are currently selected

\item[{Return type}] \leavevmode
set

\end{description}\end{quote}

\end{fulllineitems}

\index{update\_table\_view() (polo.widgets.table\_inspector.TableInspector method)@\spxentry{update\_table\_view()}\spxextra{polo.widgets.table\_inspector.TableInspector method}}

\begin{fulllineitems}
\phantomsection\label{\detokenize{polo.widgets:polo.widgets.table_inspector.TableInspector.update_table_view}}\pysiglinewithargsret{\sphinxbfcode{\sphinxupquote{update\_table\_view}}}{}{}
Private method that updates the data being displayed
in the tableViewer.

\end{fulllineitems}


\end{fulllineitems}



\subsubsection{polo.widgets.table\_viewer module}
\label{\detokenize{polo.widgets:module-polo.widgets.table_viewer}}\label{\detokenize{polo.widgets:polo-widgets-table-viewer-module}}\index{module@\spxentry{module}!polo.widgets.table\_viewer@\spxentry{polo.widgets.table\_viewer}}\index{polo.widgets.table\_viewer@\spxentry{polo.widgets.table\_viewer}!module@\spxentry{module}}\index{TableViewer (class in polo.widgets.table\_viewer)@\spxentry{TableViewer}\spxextra{class in polo.widgets.table\_viewer}}

\begin{fulllineitems}
\phantomsection\label{\detokenize{polo.widgets:polo.widgets.table_viewer.TableViewer}}\pysiglinewithargsret{\sphinxbfcode{\sphinxupquote{class }}\sphinxcode{\sphinxupquote{polo.widgets.table\_viewer.}}\sphinxbfcode{\sphinxupquote{TableViewer}}}{\emph{\DUrole{n}{parent}}, \emph{\DUrole{n}{run}\DUrole{o}{=}\DUrole{default_value}{None}}}{}
Bases: \sphinxcode{\sphinxupquote{PyQt5.QtWidgets.QTableWidget}}

TableViewer instances override QTableWidget and provide a
more convenient interface for translating the data in \sphinxtitleref{Run} and \sphinxtitleref{HWIRun} 
objects into a table format.
\begin{quote}\begin{description}
\item[{Parameters}] \leavevmode\begin{itemize}
\item {} 
\sphinxstyleliteralstrong{\sphinxupquote{parent}} (\sphinxstyleliteralemphasis{\sphinxupquote{QtWidget}}) \textendash{} Parent widget

\item {} 
\sphinxstyleliteralstrong{\sphinxupquote{run}} ({\hyperref[\detokenize{polo.crystallography:polo.crystallography.run.Run}]{\sphinxcrossref{\sphinxstyleliteralemphasis{\sphinxupquote{Run}}}}}\sphinxstyleliteralemphasis{\sphinxupquote{, }}\sphinxstyleliteralemphasis{\sphinxupquote{optional}}) \textendash{} Run to show in this table view, defaults to None

\end{itemize}

\end{description}\end{quote}
\index{fieldnames() (polo.widgets.table\_viewer.TableViewer property)@\spxentry{fieldnames()}\spxextra{polo.widgets.table\_viewer.TableViewer property}}

\begin{fulllineitems}
\phantomsection\label{\detokenize{polo.widgets:polo.widgets.table_viewer.TableViewer.fieldnames}}\pysigline{\sphinxbfcode{\sphinxupquote{property }}\sphinxbfcode{\sphinxupquote{fieldnames}}}
Return the fieldnames for the current run. Should only be
used when setting the values for the listWidget in a \sphinxtitleref{tableInspector}
instance as is expensive to call.
\begin{quote}\begin{description}
\item[{Returns}] \leavevmode
list of fieldnames

\item[{Return type}] \leavevmode
list

\end{description}\end{quote}

\end{fulllineitems}

\index{filter() (polo.widgets.table\_viewer.TableViewer static method)@\spxentry{filter()}\spxextra{polo.widgets.table\_viewer.TableViewer static method}}

\begin{fulllineitems}
\phantomsection\label{\detokenize{polo.widgets:polo.widgets.table_viewer.TableViewer.filter}}\pysiglinewithargsret{\sphinxbfcode{\sphinxupquote{static }}\sphinxbfcode{\sphinxupquote{filter}}}{\emph{\DUrole{n}{row}}, \emph{\DUrole{n}{image\_classes}}, \emph{\DUrole{n}{human}}, \emph{\DUrole{n}{marco}}}{}
Helper method to determine if a row should be included based on
the image filters the user has selected
\begin{quote}\begin{description}
\item[{Parameters}] \leavevmode\begin{itemize}
\item {} 
\sphinxstyleliteralstrong{\sphinxupquote{row}} (\sphinxstyleliteralemphasis{\sphinxupquote{dict}}) \textendash{} row data

\item {} 
\sphinxstyleliteralstrong{\sphinxupquote{image\_classes}} (\sphinxstyleliteralemphasis{\sphinxupquote{set}}\sphinxstyleliteralemphasis{\sphinxupquote{ or }}\sphinxstyleliteralemphasis{\sphinxupquote{list}}) \textendash{} types of images to include, i.e Crystals, Clear

\item {} 
\sphinxstyleliteralstrong{\sphinxupquote{human}} (\sphinxstyleliteralemphasis{\sphinxupquote{Bool}}) \textendash{} If image\_classes should be in reference to human classifier

\item {} 
\sphinxstyleliteralstrong{\sphinxupquote{marco}} (\sphinxstyleliteralemphasis{\sphinxupquote{Bool}}) \textendash{} If image\_classes should be in reference to machine classifier

\end{itemize}

\item[{Returns}] \leavevmode
If image should be filtered, False means do not filter image

\item[{Return type}] \leavevmode
Bool

\end{description}\end{quote}

\end{fulllineitems}

\index{make\_header\_map() (polo.widgets.table\_viewer.TableViewer method)@\spxentry{make\_header\_map()}\spxextra{polo.widgets.table\_viewer.TableViewer method}}

\begin{fulllineitems}
\phantomsection\label{\detokenize{polo.widgets:polo.widgets.table_viewer.TableViewer.make_header_map}}\pysiglinewithargsret{\sphinxbfcode{\sphinxupquote{make\_header\_map}}}{\emph{\DUrole{n}{headers}}}{}
Helper method to map header keywords to their index (order). 
This method is required as headers are delivered as a set and 
we want them to be presented in a consistent order to the user.
\begin{quote}\begin{description}
\item[{Parameters}] \leavevmode
\sphinxstyleliteralstrong{\sphinxupquote{headers}} (\sphinxstyleliteralemphasis{\sphinxupquote{set}}) \textendash{} Set of headers strings

\item[{Returns}] \leavevmode
Dictionary of header strings mapped to indices

\item[{Return type}] \leavevmode
dict

\end{description}\end{quote}

\end{fulllineitems}

\index{populate\_table() (polo.widgets.table\_viewer.TableViewer method)@\spxentry{populate\_table()}\spxextra{polo.widgets.table\_viewer.TableViewer method}}

\begin{fulllineitems}
\phantomsection\label{\detokenize{polo.widgets:polo.widgets.table_viewer.TableViewer.populate_table}}\pysiglinewithargsret{\sphinxbfcode{\sphinxupquote{populate\_table}}}{\emph{\DUrole{n}{image\_classes}}, \emph{\DUrole{n}{human}}, \emph{\DUrole{n}{marco}}}{}
Populates the table and displays data to the user based on their
header and image filtering selections.
\begin{quote}\begin{description}
\item[{Parameters}] \leavevmode\begin{itemize}
\item {} 
\sphinxstyleliteralstrong{\sphinxupquote{image\_classes}} (\sphinxstyleliteralemphasis{\sphinxupquote{set}}\sphinxstyleliteralemphasis{\sphinxupquote{ or }}\sphinxstyleliteralemphasis{\sphinxupquote{list}}) \textendash{} types of images to include, i.e Crystals, Clear

\item {} 
\sphinxstyleliteralstrong{\sphinxupquote{human}} (\sphinxstyleliteralemphasis{\sphinxupquote{Bool}}) \textendash{} If image\_classes should be in reference to human classifier

\item {} 
\sphinxstyleliteralstrong{\sphinxupquote{marco}} (\sphinxstyleliteralemphasis{\sphinxupquote{Bool}}) \textendash{} If image\_classes should be in reference to machine classifier

\end{itemize}

\item[{Returns}] \leavevmode
If image should be filtered, False means do not filter image

\end{description}\end{quote}

\end{fulllineitems}

\index{run() (polo.widgets.table\_viewer.TableViewer property)@\spxentry{run()}\spxextra{polo.widgets.table\_viewer.TableViewer property}}

\begin{fulllineitems}
\phantomsection\label{\detokenize{polo.widgets:polo.widgets.table_viewer.TableViewer.run}}\pysigline{\sphinxbfcode{\sphinxupquote{property }}\sphinxbfcode{\sphinxupquote{run}}}
Return the run object
\begin{quote}\begin{description}
\item[{Returns}] \leavevmode
Run object

\item[{Return type}] \leavevmode
{\hyperref[\detokenize{polo.crystallography:polo.crystallography.run.Run}]{\sphinxcrossref{Run}}}

\end{description}\end{quote}

\end{fulllineitems}

\index{table\_data() (polo.widgets.table\_viewer.TableViewer property)@\spxentry{table\_data()}\spxextra{polo.widgets.table\_viewer.TableViewer property}}

\begin{fulllineitems}
\phantomsection\label{\detokenize{polo.widgets:polo.widgets.table_viewer.TableViewer.table_data}}\pysigline{\sphinxbfcode{\sphinxupquote{property }}\sphinxbfcode{\sphinxupquote{table\_data}}}
Property to retrieve the current table fieldnames and table data
using \sphinxtitleref{get\_csv\_data} function of the \sphinxtitleref{RunCsvWriter} class.
\begin{quote}\begin{description}
\item[{Returns}] \leavevmode
fieldnames, table data

\item[{Return type}] \leavevmode
tuple

\end{description}\end{quote}

\end{fulllineitems}


\end{fulllineitems}



\subsubsection{polo.widgets.unit\_combo module}
\label{\detokenize{polo.widgets:module-polo.widgets.unit_combo}}\label{\detokenize{polo.widgets:polo-widgets-unit-combo-module}}\index{module@\spxentry{module}!polo.widgets.unit\_combo@\spxentry{polo.widgets.unit\_combo}}\index{polo.widgets.unit\_combo@\spxentry{polo.widgets.unit\_combo}!module@\spxentry{module}}\index{UnitComboBox (class in polo.widgets.unit\_combo)@\spxentry{UnitComboBox}\spxextra{class in polo.widgets.unit\_combo}}

\begin{fulllineitems}
\phantomsection\label{\detokenize{polo.widgets:polo.widgets.unit_combo.UnitComboBox}}\pysiglinewithargsret{\sphinxbfcode{\sphinxupquote{class }}\sphinxcode{\sphinxupquote{polo.widgets.unit\_combo.}}\sphinxbfcode{\sphinxupquote{UnitComboBox}}}{\emph{\DUrole{n}{parent}\DUrole{o}{=}\DUrole{default_value}{None}}, \emph{\DUrole{n}{base\_unit}\DUrole{o}{=}\DUrole{default_value}{None}}, \emph{\DUrole{n}{scalers}\DUrole{o}{=}\DUrole{default_value}{\{\}}}}{}
Bases: \sphinxcode{\sphinxupquote{PyQt5.QtWidgets.QWidget}}

Widget that is a combination of a spinbox and a
comboBox that allows a user to select a value using the
spinBox and a unit using the comboBox.

Example:

Lets say we want to create a UnitComboBox that allows someone
to select a Molar concentration using micro\sphinxhyphen{}molar, milli\sphinxhyphen{}molar,
centi\sphinxhyphen{}molar or molar.

\begin{sphinxVerbatim}[commandchars=\\\{\}]
\PYG{c+c1}{\PYGZsh{} create the scaler dictionary}
\PYG{n}{s} \PYG{o}{=} \PYG{p}{\PYGZob{}}
    \PYG{l+s+s1}{\PYGZsq{}}\PYG{l+s+s1}{u}\PYG{l+s+s1}{\PYGZsq{}}\PYG{p}{:} \PYG{l+m+mf}{1e\PYGZhy{}6}\PYG{p}{,} \PYG{l+s+s1}{\PYGZsq{}}\PYG{l+s+s1}{m}\PYG{l+s+s1}{\PYGZsq{}}\PYG{p}{:} \PYG{l+m+mf}{1e\PYGZhy{}3}\PYG{p}{,} \PYG{l+s+s1}{\PYGZsq{}}\PYG{l+s+s1}{c}\PYG{l+s+s1}{\PYGZsq{}}\PYG{p}{:} \PYG{l+m+mf}{1e\PYGZhy{}2}
\PYG{p}{\PYGZcb{}}
\PYG{c+c1}{\PYGZsh{} values are in reference to the base unit}
\PYG{n}{unit\PYGZus{}combo} \PYG{o}{=} \PYG{n}{UnitComboBox}\PYG{p}{(}
    \PYG{n}{parent}\PYG{o}{=}\PYG{k+kc}{None}\PYG{p}{,} \PYG{n}{base\PYGZus{}unit}\PYG{o}{=}\PYG{l+s+s1}{\PYGZsq{}}\PYG{l+s+s1}{M}\PYG{l+s+s1}{\PYGZsq{}}\PYG{p}{,} \PYG{n}{scalers}\PYG{o}{=}\PYG{n}{s}
    \PYG{p}{)}
\end{sphinxVerbatim}
\begin{quote}\begin{description}
\item[{Parameters}] \leavevmode\begin{itemize}
\item {} 
\sphinxstyleliteralstrong{\sphinxupquote{parent}} (\sphinxstyleliteralemphasis{\sphinxupquote{QWidget}}\sphinxstyleliteralemphasis{\sphinxupquote{, }}\sphinxstyleliteralemphasis{\sphinxupquote{optional}}) \textendash{} Parent widget, defaults to None

\item {} 
\sphinxstyleliteralstrong{\sphinxupquote{base\_unit}} (\sphinxstyleliteralemphasis{\sphinxupquote{str}}\sphinxstyleliteralemphasis{\sphinxupquote{, }}\sphinxstyleliteralemphasis{\sphinxupquote{optional}}) \textendash{} Base unit string, defaults to None

\item {} 
\sphinxstyleliteralstrong{\sphinxupquote{scalers}} (\sphinxstyleliteralemphasis{\sphinxupquote{dict}}\sphinxstyleliteralemphasis{\sphinxupquote{, }}\sphinxstyleliteralemphasis{\sphinxupquote{optional}}) \textendash{} Dictionary of prefixes to apply to the baseunit.
Keys should be string prefixes and values should
be value that scales the baseunit, defaults to \{\}

\end{itemize}

\end{description}\end{quote}
\index{\_set\_unit\_combobox\_text() (polo.widgets.unit\_combo.UnitComboBox method)@\spxentry{\_set\_unit\_combobox\_text()}\spxextra{polo.widgets.unit\_combo.UnitComboBox method}}

\begin{fulllineitems}
\phantomsection\label{\detokenize{polo.widgets:polo.widgets.unit_combo.UnitComboBox._set_unit_combobox_text}}\pysiglinewithargsret{\sphinxbfcode{\sphinxupquote{\_set\_unit\_combobox\_text}}}{}{}
Private method to add units to the unit comboBox based
on the \sphinxtitleref{base\_unit} and the \sphinxtitleref{scalers} attributes.
\begin{quote}\begin{description}
\item[{Returns}] \leavevmode
Items added to the comboBox

\item[{Return type}] \leavevmode
list

\end{description}\end{quote}

\end{fulllineitems}

\index{get\_value() (polo.widgets.unit\_combo.UnitComboBox method)@\spxentry{get\_value()}\spxextra{polo.widgets.unit\_combo.UnitComboBox method}}

\begin{fulllineitems}
\phantomsection\label{\detokenize{polo.widgets:polo.widgets.unit_combo.UnitComboBox.get_value}}\pysiglinewithargsret{\sphinxbfcode{\sphinxupquote{get\_value}}}{}{}
Return a UnitValue constructed from the value of the
spinBox value and unit from the comboBox.
\begin{quote}\begin{description}
\item[{Returns}] \leavevmode
UnitValue constructed from current spinBox 
value and comboBox unit

\item[{Return type}] \leavevmode
{\hyperref[\detokenize{polo.crystallography:polo.crystallography.cocktail.UnitValue}]{\sphinxcrossref{UnitValue}}}

\end{description}\end{quote}

\end{fulllineitems}

\index{saved\_scalers (polo.widgets.unit\_combo.UnitComboBox attribute)@\spxentry{saved\_scalers}\spxextra{polo.widgets.unit\_combo.UnitComboBox attribute}}

\begin{fulllineitems}
\phantomsection\label{\detokenize{polo.widgets:polo.widgets.unit_combo.UnitComboBox.saved_scalers}}\pysigline{\sphinxbfcode{\sphinxupquote{saved\_scalers}}\sphinxbfcode{\sphinxupquote{ = \{\textquotesingle{}c\textquotesingle{}: 0.01, \textquotesingle{}m\textquotesingle{}: 0.001, \textquotesingle{}u\textquotesingle{}: 1e\sphinxhyphen{}06\}}}}
\end{fulllineitems}

\index{scalers() (polo.widgets.unit\_combo.UnitComboBox property)@\spxentry{scalers()}\spxextra{polo.widgets.unit\_combo.UnitComboBox property}}

\begin{fulllineitems}
\phantomsection\label{\detokenize{polo.widgets:polo.widgets.unit_combo.UnitComboBox.scalers}}\pysigline{\sphinxbfcode{\sphinxupquote{property }}\sphinxbfcode{\sphinxupquote{scalers}}}
The current scalers.
\begin{quote}\begin{description}
\item[{Returns}] \leavevmode
List of scaler values

\item[{Return type}] \leavevmode
list

\end{description}\end{quote}

\end{fulllineitems}

\index{set\_value() (polo.widgets.unit\_combo.UnitComboBox method)@\spxentry{set\_value()}\spxextra{polo.widgets.unit\_combo.UnitComboBox method}}

\begin{fulllineitems}
\phantomsection\label{\detokenize{polo.widgets:polo.widgets.unit_combo.UnitComboBox.set_value}}\pysiglinewithargsret{\sphinxbfcode{\sphinxupquote{set\_value}}}{\emph{\DUrole{n}{value}}, \emph{\DUrole{o}{*}\DUrole{n}{args}}}{}
Set the spinBox value and the comboBox unit based on the value and
unit of a \sphinxtitleref{UnitValue} instance
\begin{quote}\begin{description}
\item[{Parameters}] \leavevmode
\sphinxstyleliteralstrong{\sphinxupquote{value}} ({\hyperref[\detokenize{polo.crystallography:polo.crystallography.cocktail.UnitValue}]{\sphinxcrossref{\sphinxstyleliteralemphasis{\sphinxupquote{UnitValue}}}}}) \textendash{} UnitValue

\end{description}\end{quote}

\end{fulllineitems}

\index{set\_zero() (polo.widgets.unit\_combo.UnitComboBox method)@\spxentry{set\_zero()}\spxextra{polo.widgets.unit\_combo.UnitComboBox method}}

\begin{fulllineitems}
\phantomsection\label{\detokenize{polo.widgets:polo.widgets.unit_combo.UnitComboBox.set_zero}}\pysiglinewithargsret{\sphinxbfcode{\sphinxupquote{set\_zero}}}{}{}
Set the spinbox value to 0

\end{fulllineitems}

\index{sorted\_scalers() (polo.widgets.unit\_combo.UnitComboBox property)@\spxentry{sorted\_scalers()}\spxextra{polo.widgets.unit\_combo.UnitComboBox property}}

\begin{fulllineitems}
\phantomsection\label{\detokenize{polo.widgets:polo.widgets.unit_combo.UnitComboBox.sorted_scalers}}\pysigline{\sphinxbfcode{\sphinxupquote{property }}\sphinxbfcode{\sphinxupquote{sorted\_scalers}}}
Scalers sorted by their magnitude.
\begin{quote}\begin{description}
\item[{Returns}] \leavevmode
List of scalers

\item[{Return type}] \leavevmode
list

\end{description}\end{quote}

\end{fulllineitems}

\index{unit\_combobox\_text() (polo.widgets.unit\_combo.UnitComboBox property)@\spxentry{unit\_combobox\_text()}\spxextra{polo.widgets.unit\_combo.UnitComboBox property}}

\begin{fulllineitems}
\phantomsection\label{\detokenize{polo.widgets:polo.widgets.unit_combo.UnitComboBox.unit_combobox_text}}\pysigline{\sphinxbfcode{\sphinxupquote{property }}\sphinxbfcode{\sphinxupquote{unit\_combobox\_text}}}
The text in unit comboBox which corresponds to a specific scaler.
\begin{quote}\begin{description}
\item[{Returns}] \leavevmode
List of all scalers in the unit comboBox

\item[{Return type}] \leavevmode
list

\end{description}\end{quote}

\end{fulllineitems}

\index{unit\_text\_parser() (polo.widgets.unit\_combo.UnitComboBox method)@\spxentry{unit\_text\_parser()}\spxextra{polo.widgets.unit\_combo.UnitComboBox method}}

\begin{fulllineitems}
\phantomsection\label{\detokenize{polo.widgets:polo.widgets.unit_combo.UnitComboBox.unit_text_parser}}\pysiglinewithargsret{\sphinxbfcode{\sphinxupquote{unit\_text\_parser}}}{\emph{\DUrole{n}{unit\_text}\DUrole{o}{=}\DUrole{default_value}{None}}}{}
\end{fulllineitems}


\end{fulllineitems}



\subsubsection{Module contents}
\label{\detokenize{polo.widgets:module-polo.widgets}}\label{\detokenize{polo.widgets:module-contents}}\index{module@\spxentry{module}!polo.widgets@\spxentry{polo.widgets}}\index{polo.widgets@\spxentry{polo.widgets}!module@\spxentry{module}}

\subsection{polo.windows package}
\label{\detokenize{polo.windows:polo-windows-package}}\label{\detokenize{polo.windows::doc}}

\subsubsection{Submodules}
\label{\detokenize{polo.windows:submodules}}

\subsubsection{polo.windows.ftp\_dialog module}
\label{\detokenize{polo.windows:module-polo.windows.ftp_dialog}}\label{\detokenize{polo.windows:polo-windows-ftp-dialog-module}}\index{module@\spxentry{module}!polo.windows.ftp\_dialog@\spxentry{polo.windows.ftp\_dialog}}\index{polo.windows.ftp\_dialog@\spxentry{polo.windows.ftp\_dialog}!module@\spxentry{module}}\index{FTPDialog (class in polo.windows.ftp\_dialog)@\spxentry{FTPDialog}\spxextra{class in polo.windows.ftp\_dialog}}

\begin{fulllineitems}
\phantomsection\label{\detokenize{polo.windows:polo.windows.ftp_dialog.FTPDialog}}\pysiglinewithargsret{\sphinxbfcode{\sphinxupquote{class }}\sphinxcode{\sphinxupquote{polo.windows.ftp\_dialog.}}\sphinxbfcode{\sphinxupquote{FTPDialog}}}{\emph{\DUrole{n}{ftp\_connection}\DUrole{o}{=}\DUrole{default_value}{None}}, \emph{\DUrole{n}{parent}\DUrole{o}{=}\DUrole{default_value}{None}}}{}
Bases: \sphinxcode{\sphinxupquote{PyQt5.QtWidgets.QDialog}}

FTPDialog class acts as the interface for interacting
with a remote FTP server. Allows for browsing and downloading
files stored on the server.
\begin{quote}\begin{description}
\item[{Parameters}] \leavevmode\begin{itemize}
\item {} 
\sphinxstyleliteralstrong{\sphinxupquote{ftp\_connection}} (\sphinxstyleliteralemphasis{\sphinxupquote{FTP}}\sphinxstyleliteralemphasis{\sphinxupquote{, }}\sphinxstyleliteralemphasis{\sphinxupquote{optional}}) \textendash{} Existing FTP connection, defaults to None

\item {} 
\sphinxstyleliteralstrong{\sphinxupquote{parent}} (\sphinxstyleliteralemphasis{\sphinxupquote{QWidget}}\sphinxstyleliteralemphasis{\sphinxupquote{, }}\sphinxstyleliteralemphasis{\sphinxupquote{optional}}) \textendash{} Parent widget, defaults to None

\end{itemize}

\end{description}\end{quote}
\index{CONNECTED\_ICON (polo.windows.ftp\_dialog.FTPDialog attribute)@\spxentry{CONNECTED\_ICON}\spxextra{polo.windows.ftp\_dialog.FTPDialog attribute}}

\begin{fulllineitems}
\phantomsection\label{\detokenize{polo.windows:polo.windows.ftp_dialog.FTPDialog.CONNECTED_ICON}}\pysigline{\sphinxbfcode{\sphinxupquote{CONNECTED\_ICON}}\sphinxbfcode{\sphinxupquote{ = \textquotesingle{}/home/ethan/Documents/github/Marco\_Polo/src/data/images/icons/connected.png\textquotesingle{}}}}
\end{fulllineitems}

\index{DISCONNECTED\_ICON (polo.windows.ftp\_dialog.FTPDialog attribute)@\spxentry{DISCONNECTED\_ICON}\spxextra{polo.windows.ftp\_dialog.FTPDialog attribute}}

\begin{fulllineitems}
\phantomsection\label{\detokenize{polo.windows:polo.windows.ftp_dialog.FTPDialog.DISCONNECTED_ICON}}\pysigline{\sphinxbfcode{\sphinxupquote{DISCONNECTED\_ICON}}\sphinxbfcode{\sphinxupquote{ = \textquotesingle{}/home/ethan/Documents/github/Marco\_Polo/src/data/images/icons/disconnected.png\textquotesingle{}}}}
\end{fulllineitems}

\index{DOWNLOAD\_ICON (polo.windows.ftp\_dialog.FTPDialog attribute)@\spxentry{DOWNLOAD\_ICON}\spxextra{polo.windows.ftp\_dialog.FTPDialog attribute}}

\begin{fulllineitems}
\phantomsection\label{\detokenize{polo.windows:polo.windows.ftp_dialog.FTPDialog.DOWNLOAD_ICON}}\pysigline{\sphinxbfcode{\sphinxupquote{DOWNLOAD\_ICON}}\sphinxbfcode{\sphinxupquote{ = \textquotesingle{}/home/ethan/Documents/github/Marco\_Polo/src/data/images/icons/download.png\textquotesingle{}}}}
\end{fulllineitems}

\index{connect\_ftp() (polo.windows.ftp\_dialog.FTPDialog method)@\spxentry{connect\_ftp()}\spxextra{polo.windows.ftp\_dialog.FTPDialog method}}

\begin{fulllineitems}
\phantomsection\label{\detokenize{polo.windows:polo.windows.ftp_dialog.FTPDialog.connect_ftp}}\pysiglinewithargsret{\sphinxbfcode{\sphinxupquote{connect\_ftp}}}{}{}
Attempt to establish a connection to an ftp server. If the connection is
successful then recursively walk through the user’s home directory
and display available directories and files via the
\sphinxtitleref{fileBrowser} widget. If the user has an extremely large number of
files this can take a while. If the connection fails show the user
the error code thrown by ftplib.

\end{fulllineitems}

\index{download\_selected\_files() (polo.windows.ftp\_dialog.FTPDialog method)@\spxentry{download\_selected\_files()}\spxextra{polo.windows.ftp\_dialog.FTPDialog method}}

\begin{fulllineitems}
\phantomsection\label{\detokenize{polo.windows:polo.windows.ftp_dialog.FTPDialog.download_selected_files}}\pysiglinewithargsret{\sphinxbfcode{\sphinxupquote{download\_selected\_files}}}{}{}
Signals to the \sphinxtitleref{fileBrowser} widget to download all files / dirs the
user has selected. Downloading occurs in the background and the FTP
browser dialog is closed after a download has successfully begun.
Another download should not be initiated while one is
already in progress.

\end{fulllineitems}

\index{host() (polo.windows.ftp\_dialog.FTPDialog property)@\spxentry{host()}\spxextra{polo.windows.ftp\_dialog.FTPDialog property}}

\begin{fulllineitems}
\phantomsection\label{\detokenize{polo.windows:polo.windows.ftp_dialog.FTPDialog.host}}\pysigline{\sphinxbfcode{\sphinxupquote{property }}\sphinxbfcode{\sphinxupquote{host}}}
Get user entered FTP host.
\begin{quote}\begin{description}
\item[{Returns}] \leavevmode
host address

\item[{Return type}] \leavevmode
str

\end{description}\end{quote}

\end{fulllineitems}

\index{password() (polo.windows.ftp\_dialog.FTPDialog property)@\spxentry{password()}\spxextra{polo.windows.ftp\_dialog.FTPDialog property}}

\begin{fulllineitems}
\phantomsection\label{\detokenize{polo.windows:polo.windows.ftp_dialog.FTPDialog.password}}\pysigline{\sphinxbfcode{\sphinxupquote{property }}\sphinxbfcode{\sphinxupquote{password}}}
Return user entered password.
\begin{quote}\begin{description}
\item[{Returns}] \leavevmode
password

\item[{Return type}] \leavevmode
str

\end{description}\end{quote}

\end{fulllineitems}

\index{set\_connection\_status() (polo.windows.ftp\_dialog.FTPDialog method)@\spxentry{set\_connection\_status()}\spxextra{polo.windows.ftp\_dialog.FTPDialog method}}

\begin{fulllineitems}
\phantomsection\label{\detokenize{polo.windows:polo.windows.ftp_dialog.FTPDialog.set_connection_status}}\pysiglinewithargsret{\sphinxbfcode{\sphinxupquote{set\_connection\_status}}}{\emph{\DUrole{n}{connected}\DUrole{o}{=}\DUrole{default_value}{False}}}{}
Change the Qlabel that displays the current connection status
to the user.
\begin{quote}\begin{description}
\item[{Parameters}] \leavevmode
\sphinxstyleliteralstrong{\sphinxupquote{connected}} (\sphinxstyleliteralemphasis{\sphinxupquote{bool}}\sphinxstyleliteralemphasis{\sphinxupquote{, }}\sphinxstyleliteralemphasis{\sphinxupquote{optional}}) \textendash{} If FTP connection is successful, defaults to False

\end{description}\end{quote}

\end{fulllineitems}

\index{username() (polo.windows.ftp\_dialog.FTPDialog property)@\spxentry{username()}\spxextra{polo.windows.ftp\_dialog.FTPDialog property}}

\begin{fulllineitems}
\phantomsection\label{\detokenize{polo.windows:polo.windows.ftp_dialog.FTPDialog.username}}\pysigline{\sphinxbfcode{\sphinxupquote{property }}\sphinxbfcode{\sphinxupquote{username}}}
Return username.
\begin{quote}\begin{description}
\item[{Returns}] \leavevmode
username

\item[{Return type}] \leavevmode
str

\end{description}\end{quote}

\end{fulllineitems}


\end{fulllineitems}



\subsubsection{polo.windows.image\_pop\_dialog module}
\label{\detokenize{polo.windows:module-polo.windows.image_pop_dialog}}\label{\detokenize{polo.windows:polo-windows-image-pop-dialog-module}}\index{module@\spxentry{module}!polo.windows.image\_pop\_dialog@\spxentry{polo.windows.image\_pop\_dialog}}\index{polo.windows.image\_pop\_dialog@\spxentry{polo.windows.image\_pop\_dialog}!module@\spxentry{module}}\index{ImagePopDialog (class in polo.windows.image\_pop\_dialog)@\spxentry{ImagePopDialog}\spxextra{class in polo.windows.image\_pop\_dialog}}

\begin{fulllineitems}
\phantomsection\label{\detokenize{polo.windows:polo.windows.image_pop_dialog.ImagePopDialog}}\pysiglinewithargsret{\sphinxbfcode{\sphinxupquote{class }}\sphinxcode{\sphinxupquote{polo.windows.image\_pop\_dialog.}}\sphinxbfcode{\sphinxupquote{ImagePopDialog}}}{\emph{\DUrole{n}{image}}, \emph{\DUrole{n}{parent}\DUrole{o}{=}\DUrole{default_value}{None}}}{}
Bases: \sphinxcode{\sphinxupquote{PyQt5.QtWidgets.QDialog}}

Pop up that displays a selected image in a larger view. Intended
to be used with the \sphinxtitleref{PlateViewer} widget when a user selects an
image from the grid.
\begin{quote}\begin{description}
\item[{Parameters}] \leavevmode
\sphinxstyleliteralstrong{\sphinxupquote{image}} ({\hyperref[\detokenize{polo.crystallography:polo.crystallography.image.Image}]{\sphinxcrossref{\sphinxstyleliteralemphasis{\sphinxupquote{Image}}}}}) \textendash{} Image to show

\end{description}\end{quote}
\index{\_change\_favorite\_status() (polo.windows.image\_pop\_dialog.ImagePopDialog method)@\spxentry{\_change\_favorite\_status()}\spxextra{polo.windows.image\_pop\_dialog.ImagePopDialog method}}

\begin{fulllineitems}
\phantomsection\label{\detokenize{polo.windows:polo.windows.image_pop_dialog.ImagePopDialog._change_favorite_status}}\pysiglinewithargsret{\sphinxbfcode{\sphinxupquote{\_change\_favorite\_status}}}{}{}
Private method that updates the favorite status of the current 
{\hyperref[\detokenize{polo.windows:polo.windows.image_pop_dialog.ImagePopDialog.image}]{\sphinxcrossref{\sphinxcode{\sphinxupquote{image}}}}}
attribute to the state of the favorite \sphinxcode{\sphinxupquote{QRadioButton}}.

\end{fulllineitems}

\index{\_set\_allowed\_navigation\_functions() (polo.windows.image\_pop\_dialog.ImagePopDialog method)@\spxentry{\_set\_allowed\_navigation\_functions()}\spxextra{polo.windows.image\_pop\_dialog.ImagePopDialog method}}

\begin{fulllineitems}
\phantomsection\label{\detokenize{polo.windows:polo.windows.image_pop_dialog.ImagePopDialog._set_allowed_navigation_functions}}\pysiglinewithargsret{\sphinxbfcode{\sphinxupquote{\_set\_allowed\_navigation\_functions}}}{}{}
Private method to enable or disable navigation by date or spectrum buttons
based on the content of the current image. 
Tests the {\hyperref[\detokenize{polo.crystallography:polo.crystallography.image.Image}]{\sphinxcrossref{\sphinxcode{\sphinxupquote{Image}}}}} instance
referenced by the {\hyperref[\detokenize{polo.windows:polo.windows.image_pop_dialog.ImagePopDialog.image}]{\sphinxcrossref{\sphinxcode{\sphinxupquote{image}}}}}
attribute to determine if it is linked to
a future date, previous date or alt spectrum image through it’s
\sphinxcode{\sphinxupquote{next\_image}}
, \sphinxcode{\sphinxupquote{previous\_image}}     
and \sphinxcode{\sphinxupquote{alt\_image}} attributes
respectively. If an attribute == None, then the button that
requires that attribute will be disabled.

\end{fulllineitems}

\index{\_set\_cocktail\_details() (polo.windows.image\_pop\_dialog.ImagePopDialog method)@\spxentry{\_set\_cocktail\_details()}\spxextra{polo.windows.image\_pop\_dialog.ImagePopDialog method}}

\begin{fulllineitems}
\phantomsection\label{\detokenize{polo.windows:polo.windows.image_pop_dialog.ImagePopDialog._set_cocktail_details}}\pysiglinewithargsret{\sphinxbfcode{\sphinxupquote{\_set\_cocktail\_details}}}{}{}
Private method that shows the 
{\hyperref[\detokenize{polo.windows:polo.windows.image_pop_dialog.ImagePopDialog.image}]{\sphinxcrossref{\sphinxcode{\sphinxupquote{image}}}}}
attribute metadata in the text display widgets.

\end{fulllineitems}

\index{\_set\_groupbox\_title() (polo.windows.image\_pop\_dialog.ImagePopDialog method)@\spxentry{\_set\_groupbox\_title()}\spxextra{polo.windows.image\_pop\_dialog.ImagePopDialog method}}

\begin{fulllineitems}
\phantomsection\label{\detokenize{polo.windows:polo.windows.image_pop_dialog.ImagePopDialog._set_groupbox_title}}\pysiglinewithargsret{\sphinxbfcode{\sphinxupquote{\_set\_groupbox\_title}}}{}{}
Private method that set the the title of main groupbox to the 
basename of the {\hyperref[\detokenize{polo.crystallography:polo.crystallography.image.Image.path}]{\sphinxcrossref{\sphinxcode{\sphinxupquote{path}}}}}
attribute of the {\hyperref[\detokenize{polo.crystallography:polo.crystallography.image.Image}]{\sphinxcrossref{\sphinxcode{\sphinxupquote{Image}}}}} instance
referenced by the {\hyperref[\detokenize{polo.windows:polo.windows.image_pop_dialog.ImagePopDialog.image}]{\sphinxcrossref{\sphinxcode{\sphinxupquote{image}}}}}
attribute.

\end{fulllineitems}

\index{\_set\_image\_details() (polo.windows.image\_pop\_dialog.ImagePopDialog method)@\spxentry{\_set\_image\_details()}\spxextra{polo.windows.image\_pop\_dialog.ImagePopDialog method}}

\begin{fulllineitems}
\phantomsection\label{\detokenize{polo.windows:polo.windows.image_pop_dialog.ImagePopDialog._set_image_details}}\pysiglinewithargsret{\sphinxbfcode{\sphinxupquote{\_set\_image\_details}}}{}{}
Private method that displays the 
{\hyperref[\detokenize{polo.crystallography:polo.crystallography.image.Image}]{\sphinxcrossref{\sphinxcode{\sphinxupquote{Image}}}}} instance referenced
by the {\hyperref[\detokenize{polo.windows:polo.windows.image_pop_dialog.ImagePopDialog.image}]{\sphinxcrossref{\sphinxcode{\sphinxupquote{image}}}}}
attribute.

\end{fulllineitems}

\index{classify\_image() (polo.windows.image\_pop\_dialog.ImagePopDialog method)@\spxentry{classify\_image()}\spxextra{polo.windows.image\_pop\_dialog.ImagePopDialog method}}

\begin{fulllineitems}
\phantomsection\label{\detokenize{polo.windows:polo.windows.image_pop_dialog.ImagePopDialog.classify_image}}\pysiglinewithargsret{\sphinxbfcode{\sphinxupquote{classify\_image}}}{\emph{\DUrole{n}{crystals}\DUrole{o}{=}\DUrole{default_value}{False}}, \emph{\DUrole{n}{clear}\DUrole{o}{=}\DUrole{default_value}{False}}, \emph{\DUrole{n}{precipitate}\DUrole{o}{=}\DUrole{default_value}{False}}, \emph{\DUrole{n}{other}\DUrole{o}{=}\DUrole{default_value}{False}}}{}
Set the human classification of the 
{\hyperref[\detokenize{polo.crystallography:polo.crystallography.image.Image}]{\sphinxcrossref{\sphinxcode{\sphinxupquote{Image}}}}} instances
referenced by the {\hyperref[\detokenize{polo.windows:polo.windows.image_pop_dialog.ImagePopDialog.image}]{\sphinxcrossref{\sphinxcode{\sphinxupquote{image}}}}}
attribute.
\begin{quote}\begin{description}
\item[{Parameters}] \leavevmode\begin{itemize}
\item {} 
\sphinxstyleliteralstrong{\sphinxupquote{crystals}} (\sphinxstyleliteralemphasis{\sphinxupquote{bool}}\sphinxstyleliteralemphasis{\sphinxupquote{, }}\sphinxstyleliteralemphasis{\sphinxupquote{optional}}) \textendash{} If True, classify the \sphinxtitleref{image} as crystal,
default False

\item {} 
\sphinxstyleliteralstrong{\sphinxupquote{clear}} (\sphinxstyleliteralemphasis{\sphinxupquote{bool}}\sphinxstyleliteralemphasis{\sphinxupquote{, }}\sphinxstyleliteralemphasis{\sphinxupquote{optional}}) \textendash{} If True, classify the \sphinxtitleref{image} as clear,
defaults to False

\item {} 
\sphinxstyleliteralstrong{\sphinxupquote{precipitate}} (\sphinxstyleliteralemphasis{\sphinxupquote{bool}}\sphinxstyleliteralemphasis{\sphinxupquote{, }}\sphinxstyleliteralemphasis{\sphinxupquote{optional}}) \textendash{} If True, classify the \sphinxtitleref{image} as precipitate, 
defaults to False

\item {} 
\sphinxstyleliteralstrong{\sphinxupquote{other}} (\sphinxstyleliteralemphasis{\sphinxupquote{bool}}\sphinxstyleliteralemphasis{\sphinxupquote{, }}\sphinxstyleliteralemphasis{\sphinxupquote{optional}}) \textendash{} If True, classify as the \sphinxtitleref{image} as other,
defaults to False

\end{itemize}

\end{description}\end{quote}

\end{fulllineitems}

\index{image() (polo.windows.image\_pop\_dialog.ImagePopDialog property)@\spxentry{image()}\spxextra{polo.windows.image\_pop\_dialog.ImagePopDialog property}}

\begin{fulllineitems}
\phantomsection\label{\detokenize{polo.windows:polo.windows.image_pop_dialog.ImagePopDialog.image}}\pysigline{\sphinxbfcode{\sphinxupquote{property }}\sphinxbfcode{\sphinxupquote{image}}}~\begin{description}
\item[{The {\hyperref[\detokenize{polo.crystallography:polo.crystallography.image.Image}]{\sphinxcrossref{\sphinxcode{\sphinxupquote{Image}}}}}}] \leavevmode
being displayed.

\end{description}
\begin{quote}\begin{description}
\item[{Returns}] \leavevmode
The {\hyperref[\detokenize{polo.crystallography:polo.crystallography.image.Image}]{\sphinxcrossref{\sphinxcode{\sphinxupquote{Image}}}}} instance to be displayed

\item[{Return type}] \leavevmode
{\hyperref[\detokenize{polo.crystallography:polo.crystallography.image.Image}]{\sphinxcrossref{Image}}}

\end{description}\end{quote}

\end{fulllineitems}

\index{show() (polo.windows.image\_pop\_dialog.ImagePopDialog method)@\spxentry{show()}\spxextra{polo.windows.image\_pop\_dialog.ImagePopDialog method}}

\begin{fulllineitems}
\phantomsection\label{\detokenize{polo.windows:polo.windows.image_pop_dialog.ImagePopDialog.show}}\pysiglinewithargsret{\sphinxbfcode{\sphinxupquote{show}}}{}{}
Shows the dialog window.

\end{fulllineitems}

\index{show\_alt\_image() (polo.windows.image\_pop\_dialog.ImagePopDialog method)@\spxentry{show\_alt\_image()}\spxextra{polo.windows.image\_pop\_dialog.ImagePopDialog method}}

\begin{fulllineitems}
\phantomsection\label{\detokenize{polo.windows:polo.windows.image_pop_dialog.ImagePopDialog.show_alt_image}}\pysiglinewithargsret{\sphinxbfcode{\sphinxupquote{show\_alt\_image}}}{\emph{\DUrole{n}{next\_date}\DUrole{o}{=}\DUrole{default_value}{False}}, \emph{\DUrole{n}{prev\_date}\DUrole{o}{=}\DUrole{default_value}{False}}, \emph{\DUrole{n}{alt}\DUrole{o}{=}\DUrole{default_value}{False}}}{}
Show a linked image based on boolean flags.
\begin{quote}\begin{description}
\item[{Parameters}] \leavevmode\begin{itemize}
\item {} 
\sphinxstyleliteralstrong{\sphinxupquote{next\_date}} (\sphinxstyleliteralemphasis{\sphinxupquote{bool}}\sphinxstyleliteralemphasis{\sphinxupquote{, }}\sphinxstyleliteralemphasis{\sphinxupquote{optional}}) \textendash{} If True, set
{\hyperref[\detokenize{polo.windows:polo.windows.image_pop_dialog.ImagePopDialog.image}]{\sphinxcrossref{\sphinxcode{\sphinxupquote{image}}}}} 
attribute to next the available imaging date, defaults to False

\item {} 
\sphinxstyleliteralstrong{\sphinxupquote{prev\_date}} (\sphinxstyleliteralemphasis{\sphinxupquote{bool}}\sphinxstyleliteralemphasis{\sphinxupquote{, }}\sphinxstyleliteralemphasis{\sphinxupquote{optional}}) \textendash{} If True, set 
{\hyperref[\detokenize{polo.windows:polo.windows.image_pop_dialog.ImagePopDialog.image}]{\sphinxcrossref{\sphinxcode{\sphinxupquote{image}}}}}
attribute to previous 
imaging date, defaults to False

\item {} 
\sphinxstyleliteralstrong{\sphinxupquote{alt}} (\sphinxstyleliteralemphasis{\sphinxupquote{bool}}\sphinxstyleliteralemphasis{\sphinxupquote{, }}\sphinxstyleliteralemphasis{\sphinxupquote{optional}}) \textendash{} If True, set 
{\hyperref[\detokenize{polo.windows:polo.windows.image_pop_dialog.ImagePopDialog.image}]{\sphinxcrossref{\sphinxcode{\sphinxupquote{image}}}}}
attribute to an alt spectrum
image, defaults to False

\end{itemize}

\end{description}\end{quote}

\end{fulllineitems}

\index{show\_image() (polo.windows.image\_pop\_dialog.ImagePopDialog method)@\spxentry{show\_image()}\spxextra{polo.windows.image\_pop\_dialog.ImagePopDialog method}}

\begin{fulllineitems}
\phantomsection\label{\detokenize{polo.windows:polo.windows.image_pop_dialog.ImagePopDialog.show_image}}\pysiglinewithargsret{\sphinxbfcode{\sphinxupquote{show\_image}}}{}{}
Show the {\hyperref[\detokenize{polo.crystallography:polo.crystallography.image.Image}]{\sphinxcrossref{\sphinxcode{\sphinxupquote{Image}}}}}
instance referenced by the
{\hyperref[\detokenize{polo.windows:polo.windows.image_pop_dialog.ImagePopDialog.image}]{\sphinxcrossref{\sphinxcode{\sphinxupquote{image}}}}} attribute.

\end{fulllineitems}


\end{fulllineitems}



\subsubsection{polo.windows.log\_dialog module}
\label{\detokenize{polo.windows:module-polo.windows.log_dialog}}\label{\detokenize{polo.windows:polo-windows-log-dialog-module}}\index{module@\spxentry{module}!polo.windows.log\_dialog@\spxentry{polo.windows.log\_dialog}}\index{polo.windows.log\_dialog@\spxentry{polo.windows.log\_dialog}!module@\spxentry{module}}\index{LogDialog (class in polo.windows.log\_dialog)@\spxentry{LogDialog}\spxextra{class in polo.windows.log\_dialog}}

\begin{fulllineitems}
\phantomsection\label{\detokenize{polo.windows:polo.windows.log_dialog.LogDialog}}\pysiglinewithargsret{\sphinxbfcode{\sphinxupquote{class }}\sphinxcode{\sphinxupquote{polo.windows.log\_dialog.}}\sphinxbfcode{\sphinxupquote{LogDialog}}}{\emph{\DUrole{n}{parent}\DUrole{o}{=}\DUrole{default_value}{None}}}{}
Bases: \sphinxcode{\sphinxupquote{PyQt5.QtWidgets.QDialog}}

Small dialog for displaying the contents of the Polo log file.
\index{clear\_log() (polo.windows.log\_dialog.LogDialog method)@\spxentry{clear\_log()}\spxextra{polo.windows.log\_dialog.LogDialog method}}

\begin{fulllineitems}
\phantomsection\label{\detokenize{polo.windows:polo.windows.log_dialog.LogDialog.clear_log}}\pysiglinewithargsret{\sphinxbfcode{\sphinxupquote{clear\_log}}}{}{}
Deletes the contents of the log file.

\end{fulllineitems}

\index{display\_log\_text() (polo.windows.log\_dialog.LogDialog method)@\spxentry{display\_log\_text()}\spxextra{polo.windows.log\_dialog.LogDialog method}}

\begin{fulllineitems}
\phantomsection\label{\detokenize{polo.windows:polo.windows.log_dialog.LogDialog.display_log_text}}\pysiglinewithargsret{\sphinxbfcode{\sphinxupquote{display\_log\_text}}}{}{}
Opens the log file and writes the contents to textBrowser widget
for display to the user.

\end{fulllineitems}

\index{save\_log\_file() (polo.windows.log\_dialog.LogDialog method)@\spxentry{save\_log\_file()}\spxextra{polo.windows.log\_dialog.LogDialog method}}

\begin{fulllineitems}
\phantomsection\label{\detokenize{polo.windows:polo.windows.log_dialog.LogDialog.save_log_file}}\pysiglinewithargsret{\sphinxbfcode{\sphinxupquote{save\_log\_file}}}{}{}
Saves the current log file contents to a new location.

\end{fulllineitems}


\end{fulllineitems}



\subsubsection{polo.windows.main\_window module}
\label{\detokenize{polo.windows:module-polo.windows.main_window}}\label{\detokenize{polo.windows:polo-windows-main-window-module}}\index{module@\spxentry{module}!polo.windows.main\_window@\spxentry{polo.windows.main\_window}}\index{polo.windows.main\_window@\spxentry{polo.windows.main\_window}!module@\spxentry{module}}\index{MainWindow (class in polo.windows.main\_window)@\spxentry{MainWindow}\spxextra{class in polo.windows.main\_window}}

\begin{fulllineitems}
\phantomsection\label{\detokenize{polo.windows:polo.windows.main_window.MainWindow}}\pysigline{\sphinxbfcode{\sphinxupquote{class }}\sphinxcode{\sphinxupquote{polo.windows.main\_window.}}\sphinxbfcode{\sphinxupquote{MainWindow}}}
Bases: \sphinxcode{\sphinxupquote{PyQt5.QtWidgets.QMainWindow}}, \sphinxcode{\sphinxupquote{polo.designer.UI\_main\_window.Ui\_MainWindow}}

QMainWindow that ultimately is the parent of all other
included widgets.
\index{BAR\_COLORS (polo.windows.main\_window.MainWindow attribute)@\spxentry{BAR\_COLORS}\spxextra{polo.windows.main\_window.MainWindow attribute}}

\begin{fulllineitems}
\phantomsection\label{\detokenize{polo.windows:polo.windows.main_window.MainWindow.BAR_COLORS}}\pysigline{\sphinxbfcode{\sphinxupquote{BAR\_COLORS}}\sphinxbfcode{\sphinxupquote{ = {[}15, 13, 14, 4{]}}}}
\end{fulllineitems}

\index{CRYSTAL\_ICON (polo.windows.main\_window.MainWindow attribute)@\spxentry{CRYSTAL\_ICON}\spxextra{polo.windows.main\_window.MainWindow attribute}}

\begin{fulllineitems}
\phantomsection\label{\detokenize{polo.windows:polo.windows.main_window.MainWindow.CRYSTAL_ICON}}\pysigline{\sphinxbfcode{\sphinxupquote{CRYSTAL\_ICON}}\sphinxbfcode{\sphinxupquote{ = \textquotesingle{}/home/ethan/Documents/github/Marco\_Polo/src/data/images/icons/crystal.png\textquotesingle{}}}}
\end{fulllineitems}

\index{\_handle\_delete\_backups() (polo.windows.main\_window.MainWindow method)@\spxentry{\_handle\_delete\_backups()}\spxextra{polo.windows.main\_window.MainWindow method}}

\begin{fulllineitems}
\phantomsection\label{\detokenize{polo.windows:polo.windows.main_window.MainWindow._handle_delete_backups}}\pysiglinewithargsret{\sphinxbfcode{\sphinxupquote{\_handle\_delete\_backups}}}{}{}
Private method that handles a user request to delete all backup 
mso files. If backups cannot be deleted shows a message box indicating
failure to delete.

\end{fulllineitems}

\index{\_handle\_export() (polo.windows.main\_window.MainWindow method)@\spxentry{\_handle\_export()}\spxextra{polo.windows.main\_window.MainWindow method}}

\begin{fulllineitems}
\phantomsection\label{\detokenize{polo.windows:polo.windows.main_window.MainWindow._handle_export}}\pysiglinewithargsret{\sphinxbfcode{\sphinxupquote{\_handle\_export}}}{\emph{\DUrole{n}{action}}, \emph{\DUrole{n}{export\_path}\DUrole{o}{=}\DUrole{default_value}{None}}}{}
Private method to handle when a user requests to export a run
to a non\sphinxhyphen{}xtal file format.
\begin{quote}\begin{description}
\item[{Parameters}] \leavevmode\begin{itemize}
\item {} 
\sphinxstyleliteralstrong{\sphinxupquote{action}} (\sphinxstyleliteralemphasis{\sphinxupquote{QAction}}) \textendash{} QAction that describes the export type the user has requested

\item {} 
\sphinxstyleliteralstrong{\sphinxupquote{export\_path}} (\sphinxstyleliteralemphasis{\sphinxupquote{str}}\sphinxstyleliteralemphasis{\sphinxupquote{ or }}\sphinxstyleliteralemphasis{\sphinxupquote{Path}}\sphinxstyleliteralemphasis{\sphinxupquote{, }}\sphinxstyleliteralemphasis{\sphinxupquote{optional}}) \textendash{} Path to export file to, defaults to None

\end{itemize}

\end{description}\end{quote}

\end{fulllineitems}

\index{\_handle\_file\_menu() (polo.windows.main\_window.MainWindow method)@\spxentry{\_handle\_file\_menu()}\spxextra{polo.windows.main\_window.MainWindow method}}

\begin{fulllineitems}
\phantomsection\label{\detokenize{polo.windows:polo.windows.main_window.MainWindow._handle_file_menu}}\pysiglinewithargsret{\sphinxbfcode{\sphinxupquote{\_handle\_file\_menu}}}{\emph{\DUrole{n}{selection}}}{}
Private method that handles user interaction with the file menu;
this usually means saving a run as an xtal file.
\begin{quote}\begin{description}
\item[{Parameters}] \leavevmode
\sphinxstyleliteralstrong{\sphinxupquote{selection}} (\sphinxstyleliteralemphasis{\sphinxupquote{QAction}}) \textendash{} QAction that describes user selection

\end{description}\end{quote}

\end{fulllineitems}

\index{\_handle\_help\_menu() (polo.windows.main\_window.MainWindow method)@\spxentry{\_handle\_help\_menu()}\spxextra{polo.windows.main\_window.MainWindow method}}

\begin{fulllineitems}
\phantomsection\label{\detokenize{polo.windows:polo.windows.main_window.MainWindow._handle_help_menu}}\pysiglinewithargsret{\sphinxbfcode{\sphinxupquote{\_handle\_help\_menu}}}{\emph{\DUrole{n}{action}}}{}
Private method that handles user interaction with the help menu. 
All selections open links to various pages of the documentation website.
\begin{quote}\begin{description}
\item[{Parameters}] \leavevmode
\sphinxstyleliteralstrong{\sphinxupquote{action}} (\sphinxstyleliteralemphasis{\sphinxupquote{QAction}}) \textendash{} QAction that describes the user’s selection

\end{description}\end{quote}

\end{fulllineitems}

\index{\_handle\_image\_import() (polo.windows.main\_window.MainWindow method)@\spxentry{\_handle\_image\_import()}\spxextra{polo.windows.main\_window.MainWindow method}}

\begin{fulllineitems}
\phantomsection\label{\detokenize{polo.windows:polo.windows.main_window.MainWindow._handle_image_import}}\pysiglinewithargsret{\sphinxbfcode{\sphinxupquote{\_handle\_image\_import}}}{\emph{\DUrole{n}{selection}}}{}
Private method that handles when the user attempts to import images into Polo. 
Effectively a wrapper around other methods that provide the functionality to
each option in the import menu.
\begin{quote}\begin{description}
\item[{Parameters}] \leavevmode
\sphinxstyleliteralstrong{\sphinxupquote{selection}} \textendash{} QAction. QAction from user menu selection.

\end{description}\end{quote}

\end{fulllineitems}

\index{\_handle\_opening\_run() (polo.windows.main\_window.MainWindow method)@\spxentry{\_handle\_opening\_run()}\spxextra{polo.windows.main\_window.MainWindow method}}

\begin{fulllineitems}
\phantomsection\label{\detokenize{polo.windows:polo.windows.main_window.MainWindow._handle_opening_run}}\pysiglinewithargsret{\sphinxbfcode{\sphinxupquote{\_handle\_opening\_run}}}{\emph{\DUrole{n}{new\_run}}}{}
Private method that handles opening a run. For the most part,
this means setting the \sphinxcode{\sphinxupquote{run}} attribute of other widgets to the
\sphinxtitleref{new\_run} argument. The setter methods of these widgets should then handle
updating their interfaces to reflect the new run being
opened. Also calls {\hyperref[\detokenize{polo.windows:polo.windows.main_window.MainWindow._tab_limiter}]{\sphinxcrossref{\sphinxcode{\sphinxupquote{\_tab\_limiter()}}}}}
and {\hyperref[\detokenize{polo.windows:polo.windows.main_window.MainWindow._plot_limiter}]{\sphinxcrossref{\sphinxcode{\sphinxupquote{\_plot\_limiter()}}}}} to set 
allowed functions for the user based on the type of run they open.

Additionally, if this is not the first run to be opened, before
the \sphinxtitleref{new\_run} is set as the \sphinxcode{\sphinxupquote{current\_run}} the pixmaps of the 
\sphinxcode{\sphinxupquote{current\_run}} are unloaded to free up memory.
\begin{quote}\begin{description}
\item[{Parameters}] \leavevmode
\sphinxstyleliteralstrong{\sphinxupquote{q}} (\sphinxstyleliteralemphasis{\sphinxupquote{list}}) \textendash{} List containing the run to be opened. Likely originating from
the \sphinxcode{\sphinxupquote{RunOrganizer}} widget.

\end{description}\end{quote}

\end{fulllineitems}

\index{\_handle\_plot\_selection() (polo.windows.main\_window.MainWindow method)@\spxentry{\_handle\_plot\_selection()}\spxextra{polo.windows.main\_window.MainWindow method}}

\begin{fulllineitems}
\phantomsection\label{\detokenize{polo.windows:polo.windows.main_window.MainWindow._handle_plot_selection}}\pysiglinewithargsret{\sphinxbfcode{\sphinxupquote{\_handle\_plot\_selection}}}{}{}
Private method to handle user plot selections.

TODO: Move all plot methods into their own widget

\end{fulllineitems}

\index{\_handle\_tool\_menu() (polo.windows.main\_window.MainWindow method)@\spxentry{\_handle\_tool\_menu()}\spxextra{polo.windows.main\_window.MainWindow method}}

\begin{fulllineitems}
\phantomsection\label{\detokenize{polo.windows:polo.windows.main_window.MainWindow._handle_tool_menu}}\pysiglinewithargsret{\sphinxbfcode{\sphinxupquote{\_handle\_tool\_menu}}}{\emph{\DUrole{n}{selection}}}{}
Private method that handles selection of 
all options available to the user in 
the \sphinxtitleref{Tools} section of the main window menu.
\begin{quote}\begin{description}
\item[{Parameters}] \leavevmode
\sphinxstyleliteralstrong{\sphinxupquote{selection}} (\sphinxstyleliteralemphasis{\sphinxupquote{QAction}}) \textendash{} User’s menu selection

\end{description}\end{quote}

\end{fulllineitems}

\index{\_on\_changed\_tab() (polo.windows.main\_window.MainWindow method)@\spxentry{\_on\_changed\_tab()}\spxextra{polo.windows.main\_window.MainWindow method}}

\begin{fulllineitems}
\phantomsection\label{\detokenize{polo.windows:polo.windows.main_window.MainWindow._on_changed_tab}}\pysiglinewithargsret{\sphinxbfcode{\sphinxupquote{\_on\_changed\_tab}}}{\emph{\DUrole{n}{i}}}{}
Private method that handles GUI behavior when a user
switches from one tab to another.
\begin{quote}\begin{description}
\item[{Parameters}] \leavevmode
\sphinxstyleliteralstrong{\sphinxupquote{i}} \textendash{} Int. The index of the current tab, after user has changed tabs.

\end{description}\end{quote}

\end{fulllineitems}

\index{\_plot\_limiter() (polo.windows.main\_window.MainWindow method)@\spxentry{\_plot\_limiter()}\spxextra{polo.windows.main\_window.MainWindow method}}

\begin{fulllineitems}
\phantomsection\label{\detokenize{polo.windows:polo.windows.main_window.MainWindow._plot_limiter}}\pysiglinewithargsret{\sphinxbfcode{\sphinxupquote{\_plot\_limiter}}}{}{}
Private method to limit the types of plots that can be shown
based on the type of the \sphinxtitleref{current\_run}.

\end{fulllineitems}

\index{\_save\_file\_dialog() (polo.windows.main\_window.MainWindow method)@\spxentry{\_save\_file\_dialog()}\spxextra{polo.windows.main\_window.MainWindow method}}

\begin{fulllineitems}
\phantomsection\label{\detokenize{polo.windows:polo.windows.main_window.MainWindow._save_file_dialog}}\pysiglinewithargsret{\sphinxbfcode{\sphinxupquote{\_save\_file\_dialog}}}{}{}
Private method to open a QFileDialog to get a location
to save a run to.
\begin{quote}\begin{description}
\item[{Returns}] \leavevmode
Path to save file to

\item[{Return type}] \leavevmode
str

\end{description}\end{quote}

\end{fulllineitems}

\index{\_set\_tab\_icons() (polo.windows.main\_window.MainWindow method)@\spxentry{\_set\_tab\_icons()}\spxextra{polo.windows.main\_window.MainWindow method}}

\begin{fulllineitems}
\phantomsection\label{\detokenize{polo.windows:polo.windows.main_window.MainWindow._set_tab_icons}}\pysiglinewithargsret{\sphinxbfcode{\sphinxupquote{\_set\_tab\_icons}}}{}{}
Private method that assigns icons to each of the main run 
interface tabs. Should be called in the \sphinxtitleref{\_\_init\_\_} method before
the main window is shown to the user.

\end{fulllineitems}

\index{\_tab\_limiter() (polo.windows.main\_window.MainWindow method)@\spxentry{\_tab\_limiter()}\spxextra{polo.windows.main\_window.MainWindow method}}

\begin{fulllineitems}
\phantomsection\label{\detokenize{polo.windows:polo.windows.main_window.MainWindow._tab_limiter}}\pysiglinewithargsret{\sphinxbfcode{\sphinxupquote{\_tab\_limiter}}}{}{}
Private method that limits the interfaces that a user is allowed
to interact with based on the type of \sphinxcode{\sphinxupquote{Run}} they have loaded and
selected. Currently, \sphinxcode{\sphinxupquote{Run}} functionality is limited due to the fact
cocktails cannot be mapped to images.

\end{fulllineitems}

\index{closeEvent() (polo.windows.main\_window.MainWindow method)@\spxentry{closeEvent()}\spxextra{polo.windows.main\_window.MainWindow method}}

\begin{fulllineitems}
\phantomsection\label{\detokenize{polo.windows:polo.windows.main_window.MainWindow.closeEvent}}\pysiglinewithargsret{\sphinxbfcode{\sphinxupquote{closeEvent}}}{\emph{\DUrole{n}{event}}}{}
Handle main window close events. Writes mso backup files of
all loaded runs that have human classifications so they can be
restored later.
\begin{quote}\begin{description}
\item[{Parameters}] \leavevmode
\sphinxstyleliteralstrong{\sphinxupquote{event}} (\sphinxstyleliteralemphasis{\sphinxupquote{QEvent}}) \textendash{} QEvent

\end{description}\end{quote}

\end{fulllineitems}

\index{delete\_all\_backups() (polo.windows.main\_window.MainWindow static method)@\spxentry{delete\_all\_backups()}\spxextra{polo.windows.main\_window.MainWindow static method}}

\begin{fulllineitems}
\phantomsection\label{\detokenize{polo.windows:polo.windows.main_window.MainWindow.delete_all_backups}}\pysiglinewithargsret{\sphinxbfcode{\sphinxupquote{static }}\sphinxbfcode{\sphinxupquote{delete\_all\_backups}}}{}{}
Deletes all backup mso files.
\begin{quote}\begin{description}
\item[{Raises}] \leavevmode
\sphinxstyleliteralstrong{\sphinxupquote{e}} \textendash{} Any exceptions thrown by the function call

\item[{Returns}] \leavevmode
True, if backups are deleted

\item[{Return type}] \leavevmode
bool

\end{description}\end{quote}

\end{fulllineitems}

\index{get\_widget\_dims() (polo.windows.main\_window.MainWindow static method)@\spxentry{get\_widget\_dims()}\spxextra{polo.windows.main\_window.MainWindow static method}}

\begin{fulllineitems}
\phantomsection\label{\detokenize{polo.windows:polo.windows.main_window.MainWindow.get_widget_dims}}\pysiglinewithargsret{\sphinxbfcode{\sphinxupquote{static }}\sphinxbfcode{\sphinxupquote{get\_widget\_dims}}}{\emph{\DUrole{n}{self}}, \emph{\DUrole{n}{widget}}}{}
Returns the width and height of a \sphinxcode{\sphinxupquote{QWidget}}
as a tuple.
\begin{quote}\begin{description}
\item[{Parameters}] \leavevmode
\sphinxstyleliteralstrong{\sphinxupquote{widget}} (\sphinxstyleliteralemphasis{\sphinxupquote{QWidget}}) \textendash{} QWidget

\item[{Returns}] \leavevmode
width and height of the widget

\item[{Return type}] \leavevmode
tuple

\end{description}\end{quote}

\end{fulllineitems}

\index{layout\_widget\_lister() (polo.windows.main\_window.MainWindow static method)@\spxentry{layout\_widget\_lister()}\spxextra{polo.windows.main\_window.MainWindow static method}}

\begin{fulllineitems}
\phantomsection\label{\detokenize{polo.windows:polo.windows.main_window.MainWindow.layout_widget_lister}}\pysiglinewithargsret{\sphinxbfcode{\sphinxupquote{static }}\sphinxbfcode{\sphinxupquote{layout\_widget\_lister}}}{\emph{\DUrole{n}{self}}, \emph{\DUrole{n}{layout}}}{}
List all widgets in a given layout.
\begin{quote}\begin{description}
\item[{Parameters}] \leavevmode
\sphinxstyleliteralstrong{\sphinxupquote{layout}} (\sphinxstyleliteralemphasis{\sphinxupquote{QLayout}}) \textendash{} QLayout that contains widgets

\item[{Returns}] \leavevmode
Tuple of widgets in the given layout

\item[{Return type}] \leavevmode
tuple

\end{description}\end{quote}

\end{fulllineitems}


\end{fulllineitems}



\subsubsection{polo.windows.pptx\_dialog module}
\label{\detokenize{polo.windows:module-polo.windows.pptx_dialog}}\label{\detokenize{polo.windows:polo-windows-pptx-dialog-module}}\index{module@\spxentry{module}!polo.windows.pptx\_dialog@\spxentry{polo.windows.pptx\_dialog}}\index{polo.windows.pptx\_dialog@\spxentry{polo.windows.pptx\_dialog}!module@\spxentry{module}}\index{PptxDesignerDialog (class in polo.windows.pptx\_dialog)@\spxentry{PptxDesignerDialog}\spxextra{class in polo.windows.pptx\_dialog}}

\begin{fulllineitems}
\phantomsection\label{\detokenize{polo.windows:polo.windows.pptx_dialog.PptxDesignerDialog}}\pysiglinewithargsret{\sphinxbfcode{\sphinxupquote{class }}\sphinxcode{\sphinxupquote{polo.windows.pptx\_dialog.}}\sphinxbfcode{\sphinxupquote{PptxDesignerDialog}}}{\emph{\DUrole{n}{runs}}, \emph{\DUrole{n}{parent}\DUrole{o}{=}\DUrole{default_value}{None}}}{}
Bases: \sphinxcode{\sphinxupquote{PyQt5.QtWidgets.QDialog}}
\index{\_browse\_and\_update\_line\_edit() (polo.windows.pptx\_dialog.PptxDesignerDialog method)@\spxentry{\_browse\_and\_update\_line\_edit()}\spxextra{polo.windows.pptx\_dialog.PptxDesignerDialog method}}

\begin{fulllineitems}
\phantomsection\label{\detokenize{polo.windows:polo.windows.pptx_dialog.PptxDesignerDialog._browse_and_update_line_edit}}\pysiglinewithargsret{\sphinxbfcode{\sphinxupquote{\_browse\_and\_update\_line\_edit}}}{}{}
Private method that calls 
\sphinxcode{\sphinxupquote{\_get\_save\_path()}}
to open a file browser. If the user selects a save path using the file
browser then displays this path in the filepath \sphinxcode{\sphinxupquote{QLineEdit}} widget.

\end{fulllineitems}

\index{\_get\_save\_path() (polo.windows.pptx\_dialog.PptxDesignerDialog method)@\spxentry{\_get\_save\_path()}\spxextra{polo.windows.pptx\_dialog.PptxDesignerDialog method}}

\begin{fulllineitems}
\phantomsection\label{\detokenize{polo.windows:polo.windows.pptx_dialog.PptxDesignerDialog._get_save_path}}\pysiglinewithargsret{\sphinxbfcode{\sphinxupquote{\_get\_save\_path}}}{}{}
Private method that opens a file browser and returns the selected
save filepath.
\begin{quote}\begin{description}
\item[{Returns}] \leavevmode
Filepath if one is specified by the user, empty string otherwise

\item[{Return type}] \leavevmode
str

\end{description}\end{quote}

\end{fulllineitems}

\index{\_parse\_image\_classifications() (polo.windows.pptx\_dialog.PptxDesignerDialog method)@\spxentry{\_parse\_image\_classifications()}\spxextra{polo.windows.pptx\_dialog.PptxDesignerDialog method}}

\begin{fulllineitems}
\phantomsection\label{\detokenize{polo.windows:polo.windows.pptx_dialog.PptxDesignerDialog._parse_image_classifications}}\pysiglinewithargsret{\sphinxbfcode{\sphinxupquote{\_parse\_image\_classifications}}}{}{}
Private method to get all currently selected image classifications
by reading the state of all image classification 
\sphinxcode{\sphinxupquote{QCheckBox}} instances.
\begin{quote}\begin{description}
\item[{Returns}] \leavevmode
Set of all selected image classifications

\item[{Return type}] \leavevmode
set

\end{description}\end{quote}

\end{fulllineitems}

\index{\_set\_up\_image\_classification\_checkboxes() (polo.windows.pptx\_dialog.PptxDesignerDialog method)@\spxentry{\_set\_up\_image\_classification\_checkboxes()}\spxextra{polo.windows.pptx\_dialog.PptxDesignerDialog method}}

\begin{fulllineitems}
\phantomsection\label{\detokenize{polo.windows:polo.windows.pptx_dialog.PptxDesignerDialog._set_up_image_classification_checkboxes}}\pysiglinewithargsret{\sphinxbfcode{\sphinxupquote{\_set\_up\_image\_classification\_checkboxes}}}{}{}
Private method that sets up the labels for the image classifications
\sphinxcode{\sphinxupquote{QCheckBox}} instances. Should be called in the \sphinxtitleref{\_\_init\_\_} function before
displaying the dialog to the user.
\begin{quote}\begin{description}
\item[{Returns}] \leavevmode
Dictionary of image classifications which map to the \sphinxcode{\sphinxupquote{QCheckBox}} 
that corresponds to that image classification

\item[{Return type}] \leavevmode
dict

\end{description}\end{quote}

\end{fulllineitems}

\index{\_validate\_typed\_path() (polo.windows.pptx\_dialog.PptxDesignerDialog method)@\spxentry{\_validate\_typed\_path()}\spxextra{polo.windows.pptx\_dialog.PptxDesignerDialog method}}

\begin{fulllineitems}
\phantomsection\label{\detokenize{polo.windows:polo.windows.pptx_dialog.PptxDesignerDialog._validate_typed_path}}\pysiglinewithargsret{\sphinxbfcode{\sphinxupquote{\_validate\_typed\_path}}}{}{}
Private method that validates that a filepath in the filepath
\sphinxcode{\sphinxupquote{QLineEdit}} widget is actually a valid path that a pptx file could be
saved there.
\begin{quote}\begin{description}
\item[{Returns}] \leavevmode
True if the path is valid, otherwise returns None and shows
a message box to the user.

\item[{Return type}] \leavevmode
True or None

\end{description}\end{quote}

\end{fulllineitems}

\index{\_write\_presentation() (polo.windows.pptx\_dialog.PptxDesignerDialog method)@\spxentry{\_write\_presentation()}\spxextra{polo.windows.pptx\_dialog.PptxDesignerDialog method}}

\begin{fulllineitems}
\phantomsection\label{\detokenize{polo.windows:polo.windows.pptx_dialog.PptxDesignerDialog._write_presentation}}\pysiglinewithargsret{\sphinxbfcode{\sphinxupquote{\_write\_presentation}}}{\emph{\DUrole{n}{run}\DUrole{o}{=}\DUrole{default_value}{None}}}{}
Private method that actually does the work of generating a
presentation from a \sphinxcode{\sphinxupquote{Run}} or \sphinxcode{\sphinxupquote{HWIRun}} instance.
\begin{quote}\begin{description}
\item[{Parameters}] \leavevmode
\sphinxstyleliteralstrong{\sphinxupquote{run}} ({\hyperref[\detokenize{polo.crystallography:polo.crystallography.run.Run}]{\sphinxcrossref{\sphinxstyleliteralemphasis{\sphinxupquote{Run}}}}}\sphinxstyleliteralemphasis{\sphinxupquote{ or }}{\hyperref[\detokenize{polo.crystallography:polo.crystallography.run.HWIRun}]{\sphinxcrossref{\sphinxstyleliteralemphasis{\sphinxupquote{HWIRun}}}}}\sphinxstyleliteralemphasis{\sphinxupquote{, }}\sphinxstyleliteralemphasis{\sphinxupquote{optional}}) \textendash{} Run to create a presentation from, defaults to None

\item[{Returns}] \leavevmode
Path to the pptx presentation is write is successful, 
Exception otherwise.

\item[{Return type}] \leavevmode
str or Exception

\end{description}\end{quote}

\end{fulllineitems}

\index{all\_dates() (polo.windows.pptx\_dialog.PptxDesignerDialog property)@\spxentry{all\_dates()}\spxextra{polo.windows.pptx\_dialog.PptxDesignerDialog property}}

\begin{fulllineitems}
\phantomsection\label{\detokenize{polo.windows:polo.windows.pptx_dialog.PptxDesignerDialog.all_dates}}\pysigline{\sphinxbfcode{\sphinxupquote{property }}\sphinxbfcode{\sphinxupquote{all\_dates}}}
The state of the “Include all Dates” \sphinxcode{\sphinxupquote{QCheckBox}}. If it is checked this
indicates that a time resolved slide should be included in the
presentation.
\begin{quote}\begin{description}
\item[{Returns}] \leavevmode
State of the \sphinxcode{\sphinxupquote{QCheckBox}}

\item[{Return type}] \leavevmode
bool

\end{description}\end{quote}

\end{fulllineitems}

\index{all\_specs() (polo.windows.pptx\_dialog.PptxDesignerDialog property)@\spxentry{all\_specs()}\spxextra{polo.windows.pptx\_dialog.PptxDesignerDialog property}}

\begin{fulllineitems}
\phantomsection\label{\detokenize{polo.windows:polo.windows.pptx_dialog.PptxDesignerDialog.all_specs}}\pysigline{\sphinxbfcode{\sphinxupquote{property }}\sphinxbfcode{\sphinxupquote{all\_specs}}}
The state of the “Include all Spectrums” \sphinxcode{\sphinxupquote{QCheckBox}}. If it is checked this
indicates that a multi\sphinxhyphen{}spectrum slide should be included in the
presentation.
\begin{quote}\begin{description}
\item[{Returns}] \leavevmode
State of the \sphinxcode{\sphinxupquote{QCheckBox}}

\item[{Return type}] \leavevmode
bool

\end{description}\end{quote}

\end{fulllineitems}

\index{check\_for\_warnings() (polo.windows.pptx\_dialog.PptxDesignerDialog method)@\spxentry{check\_for\_warnings()}\spxextra{polo.windows.pptx\_dialog.PptxDesignerDialog method}}

\begin{fulllineitems}
\phantomsection\label{\detokenize{polo.windows:polo.windows.pptx_dialog.PptxDesignerDialog.check_for_warnings}}\pysiglinewithargsret{\sphinxbfcode{\sphinxupquote{check\_for\_warnings}}}{}{}
\end{fulllineitems}

\index{human() (polo.windows.pptx\_dialog.PptxDesignerDialog property)@\spxentry{human()}\spxextra{polo.windows.pptx\_dialog.PptxDesignerDialog property}}

\begin{fulllineitems}
\phantomsection\label{\detokenize{polo.windows:polo.windows.pptx_dialog.PptxDesignerDialog.human}}\pysigline{\sphinxbfcode{\sphinxupquote{property }}\sphinxbfcode{\sphinxupquote{human}}}
State of the human classifier \sphinxcode{\sphinxupquote{QCheckBox}}.
\begin{quote}\begin{description}
\item[{Returns}] \leavevmode
State of the \sphinxcode{\sphinxupquote{QCheckBox}}

\item[{Return type}] \leavevmode
bool

\end{description}\end{quote}

\end{fulllineitems}

\index{marco() (polo.windows.pptx\_dialog.PptxDesignerDialog property)@\spxentry{marco()}\spxextra{polo.windows.pptx\_dialog.PptxDesignerDialog property}}

\begin{fulllineitems}
\phantomsection\label{\detokenize{polo.windows:polo.windows.pptx_dialog.PptxDesignerDialog.marco}}\pysigline{\sphinxbfcode{\sphinxupquote{property }}\sphinxbfcode{\sphinxupquote{marco}}}
State of the MARCO classifier \sphinxcode{\sphinxupquote{QCheckBox}}.
\begin{quote}\begin{description}
\item[{Returns}] \leavevmode
State of the \sphinxcode{\sphinxupquote{QCheckBox}}

\item[{Return type}] \leavevmode
bool

\end{description}\end{quote}

\end{fulllineitems}

\index{set\_default\_titles() (polo.windows.pptx\_dialog.PptxDesignerDialog method)@\spxentry{set\_default\_titles()}\spxextra{polo.windows.pptx\_dialog.PptxDesignerDialog method}}

\begin{fulllineitems}
\phantomsection\label{\detokenize{polo.windows:polo.windows.pptx_dialog.PptxDesignerDialog.set_default_titles}}\pysiglinewithargsret{\sphinxbfcode{\sphinxupquote{set\_default\_titles}}}{}{}
\end{fulllineitems}

\index{setup\_run\_tree() (polo.windows.pptx\_dialog.PptxDesignerDialog method)@\spxentry{setup\_run\_tree()}\spxextra{polo.windows.pptx\_dialog.PptxDesignerDialog method}}

\begin{fulllineitems}
\phantomsection\label{\detokenize{polo.windows:polo.windows.pptx_dialog.PptxDesignerDialog.setup_run_tree}}\pysiglinewithargsret{\sphinxbfcode{\sphinxupquote{setup\_run\_tree}}}{}{}
\end{fulllineitems}

\index{subtitle() (polo.windows.pptx\_dialog.PptxDesignerDialog property)@\spxentry{subtitle()}\spxextra{polo.windows.pptx\_dialog.PptxDesignerDialog property}}

\begin{fulllineitems}
\phantomsection\label{\detokenize{polo.windows:polo.windows.pptx_dialog.PptxDesignerDialog.subtitle}}\pysigline{\sphinxbfcode{\sphinxupquote{property }}\sphinxbfcode{\sphinxupquote{subtitle}}}
Subtitle the user has entered for the presentation via the subtitle
\sphinxcode{\sphinxupquote{QLineEdit}} widget. If no string has been entered will return the empty
string.
\begin{quote}\begin{description}
\item[{Returns}] \leavevmode
The presentation subtitle

\item[{Return type}] \leavevmode
str

\end{description}\end{quote}

\end{fulllineitems}

\index{title() (polo.windows.pptx\_dialog.PptxDesignerDialog property)@\spxentry{title()}\spxextra{polo.windows.pptx\_dialog.PptxDesignerDialog property}}

\begin{fulllineitems}
\phantomsection\label{\detokenize{polo.windows:polo.windows.pptx_dialog.PptxDesignerDialog.title}}\pysigline{\sphinxbfcode{\sphinxupquote{property }}\sphinxbfcode{\sphinxupquote{title}}}
Title the user has entered for the presentation via the title
\sphinxcode{\sphinxupquote{QLineEdit}} widget. If no string has been entered will return the empty
string.
\begin{quote}\begin{description}
\item[{Returns}] \leavevmode
The presentation title

\item[{Return type}] \leavevmode
str

\end{description}\end{quote}

\end{fulllineitems}


\end{fulllineitems}



\subsubsection{polo.windows.run\_importer module}
\label{\detokenize{polo.windows:module-polo.windows.run_importer}}\label{\detokenize{polo.windows:polo-windows-run-importer-module}}\index{module@\spxentry{module}!polo.windows.run\_importer@\spxentry{polo.windows.run\_importer}}\index{polo.windows.run\_importer@\spxentry{polo.windows.run\_importer}!module@\spxentry{module}}\index{ImportCandidate (class in polo.windows.run\_importer)@\spxentry{ImportCandidate}\spxextra{class in polo.windows.run\_importer}}

\begin{fulllineitems}
\phantomsection\label{\detokenize{polo.windows:polo.windows.run_importer.ImportCandidate}}\pysiglinewithargsret{\sphinxbfcode{\sphinxupquote{class }}\sphinxcode{\sphinxupquote{polo.windows.run\_importer.}}\sphinxbfcode{\sphinxupquote{ImportCandidate}}}{\emph{\DUrole{n}{path}}}{}
Bases: \sphinxcode{\sphinxupquote{object}}
\index{assign\_run\_type() (polo.windows.run\_importer.ImportCandidate method)@\spxentry{assign\_run\_type()}\spxextra{polo.windows.run\_importer.ImportCandidate method}}

\begin{fulllineitems}
\phantomsection\label{\detokenize{polo.windows:polo.windows.run_importer.ImportCandidate.assign_run_type}}\pysiglinewithargsret{\sphinxbfcode{\sphinxupquote{assign\_run\_type}}}{}{}
If the {\hyperref[\detokenize{polo.windows:polo.windows.run_importer.ImportCandidate.path}]{\sphinxcrossref{\sphinxcode{\sphinxupquote{path}}}}} attribute is verified as importable assigns a run class
(Run or HWIRun) to the \sphinxcode{\sphinxupquote{import\_type}} attribute. Later on in the import
pipeline this attribute tells other methods how the {\hyperref[\detokenize{polo.windows:polo.windows.run_importer.ImportCandidate}]{\sphinxcrossref{\sphinxcode{\sphinxupquote{ImportCandidate}}}}}
should be imported as different operations are required to create \sphinxtitleref{Run}
instances then \sphinxtitleref{HWIRun} instances.

The {\hyperref[\detokenize{polo.windows:polo.windows.run_importer.ImportCandidate}]{\sphinxcrossref{\sphinxcode{\sphinxupquote{ImportCandidate}}}}} is assigned to a HWIRun is its metadata is
successfully parsed.
\begin{quote}\begin{description}
\item[{Returns}] \leavevmode
The import type

\item[{Return type}] \leavevmode
{\hyperref[\detokenize{polo.crystallography:polo.crystallography.run.Run}]{\sphinxcrossref{Run}}} or {\hyperref[\detokenize{polo.crystallography:polo.crystallography.run.HWIRun}]{\sphinxcrossref{HWIRun}}}

\end{description}\end{quote}

\end{fulllineitems}

\index{cocktail\_menu() (polo.windows.run\_importer.ImportCandidate property)@\spxentry{cocktail\_menu()}\spxextra{polo.windows.run\_importer.ImportCandidate property}}

\begin{fulllineitems}
\phantomsection\label{\detokenize{polo.windows:polo.windows.run_importer.ImportCandidate.cocktail_menu}}\pysigline{\sphinxbfcode{\sphinxupquote{property }}\sphinxbfcode{\sphinxupquote{cocktail\_menu}}}
If the {\hyperref[\detokenize{polo.windows:polo.windows.run_importer.ImportCandidate}]{\sphinxcrossref{\sphinxcode{\sphinxupquote{ImportCandidate}}}}} instances’s \sphinxcode{\sphinxupquote{import\_type}} 
attribute is the \sphinxcode{\sphinxupquote{HWIRun\textasciigrave{}class and the candidate has a valid 
date then returns a :class:\textasciigrave{}polo.utils.io\_utils.Menu}}
instance that was in use at the {\hyperref[\detokenize{polo.windows:polo.windows.run_importer.ImportCandidate}]{\sphinxcrossref{\sphinxcode{\sphinxupquote{ImportCandidate}}}}} 
instanes’s date. If anyof these conditions are not meet then
returns None.
\begin{quote}\begin{description}
\item[{Returns}] \leavevmode
{\hyperref[\detokenize{polo.utils:polo.utils.io_utils.Menu}]{\sphinxcrossref{\sphinxcode{\sphinxupquote{polo.utils.io\_utils.Menu}}}}} or None

\item[{Return type}] \leavevmode
{\hyperref[\detokenize{polo.utils:polo.utils.io_utils.Menu}]{\sphinxcrossref{\sphinxcode{\sphinxupquote{polo.utils.io\_utils.Menu}}}}} or None

\end{description}\end{quote}

\end{fulllineitems}

\index{is\_rar() (polo.windows.run\_importer.ImportCandidate property)@\spxentry{is\_rar()}\spxextra{polo.windows.run\_importer.ImportCandidate property}}

\begin{fulllineitems}
\phantomsection\label{\detokenize{polo.windows:polo.windows.run_importer.ImportCandidate.is_rar}}\pysigline{\sphinxbfcode{\sphinxupquote{property }}\sphinxbfcode{\sphinxupquote{is\_rar}}}
Import candidate is a rar file. True if the file is rar file, False
otherwise.
\begin{quote}\begin{description}
\item[{Returns}] \leavevmode
Rar status

\item[{Return type}] \leavevmode
bool

\end{description}\end{quote}

\end{fulllineitems}

\index{path() (polo.windows.run\_importer.ImportCandidate property)@\spxentry{path()}\spxextra{polo.windows.run\_importer.ImportCandidate property}}

\begin{fulllineitems}
\phantomsection\label{\detokenize{polo.windows:polo.windows.run_importer.ImportCandidate.path}}\pysigline{\sphinxbfcode{\sphinxupquote{property }}\sphinxbfcode{\sphinxupquote{path}}}
Return the {\hyperref[\detokenize{polo.windows:polo.windows.run_importer.ImportCandidate}]{\sphinxcrossref{\sphinxcode{\sphinxupquote{ImportCandidate}}}}} instances’s path.
\begin{quote}\begin{description}
\item[{Returns}] \leavevmode
Path to the file which will actually be imported

\item[{Return type}] \leavevmode
str

\end{description}\end{quote}

\end{fulllineitems}

\index{read\_xmldata() (polo.windows.run\_importer.ImportCandidate method)@\spxentry{read\_xmldata()}\spxextra{polo.windows.run\_importer.ImportCandidate method}}

\begin{fulllineitems}
\phantomsection\label{\detokenize{polo.windows:polo.windows.run_importer.ImportCandidate.read_xmldata}}\pysiglinewithargsret{\sphinxbfcode{\sphinxupquote{read\_xmldata}}}{\emph{\DUrole{n}{dir\_path}}}{}
Attempts to read any xml metadata files in the path referenced by
the \sphinxcode{\sphinxupquote{dir\_path}} argument.
\begin{quote}\begin{description}
\item[{Parameters}] \leavevmode
\sphinxstyleliteralstrong{\sphinxupquote{dir\_path}} (\sphinxstyleliteralemphasis{\sphinxupquote{str}}\sphinxstyleliteralemphasis{\sphinxupquote{ or }}\sphinxstyleliteralemphasis{\sphinxupquote{Path}}) \textendash{} Path to check for xml files in

\item[{Returns}] \leavevmode
Dictionary of data pulled from xml files. If no data is found
then returns an empty dictionary.

\item[{Return type}] \leavevmode
dict

\end{description}\end{quote}

\end{fulllineitems}

\index{set\_cocktail\_menu() (polo.windows.run\_importer.ImportCandidate method)@\spxentry{set\_cocktail\_menu()}\spxextra{polo.windows.run\_importer.ImportCandidate method}}

\begin{fulllineitems}
\phantomsection\label{\detokenize{polo.windows:polo.windows.run_importer.ImportCandidate.set_cocktail_menu}}\pysiglinewithargsret{\sphinxbfcode{\sphinxupquote{set\_cocktail\_menu}}}{}{}
\end{fulllineitems}

\index{unrar() (polo.windows.run\_importer.ImportCandidate method)@\spxentry{unrar()}\spxextra{polo.windows.run\_importer.ImportCandidate method}}

\begin{fulllineitems}
\phantomsection\label{\detokenize{polo.windows:polo.windows.run_importer.ImportCandidate.unrar}}\pysiglinewithargsret{\sphinxbfcode{\sphinxupquote{unrar}}}{}{}
If the {\hyperref[\detokenize{polo.windows:polo.windows.run_importer.ImportCandidate}]{\sphinxcrossref{\sphinxcode{\sphinxupquote{ImportCandidate}}}}} instances’s {\hyperref[\detokenize{polo.windows:polo.windows.run_importer.ImportCandidate.is_rar}]{\sphinxcrossref{\sphinxcode{\sphinxupquote{is\_rar}}}}}
property is True then this method will attempt to de\sphinxhyphen{}compress the 
rar archive referenced by the instance’s 
{\hyperref[\detokenize{polo.windows:polo.windows.run_importer.ImportCandidate.path}]{\sphinxcrossref{\sphinxcode{\sphinxupquote{path}}}}} attribute.
\begin{quote}\begin{description}
\item[{Returns}] \leavevmode
Path to un\sphinxhyphen{}compressed rar archive if unrar was successful, 
otherwise returns False

\item[{Return type}] \leavevmode
Path or str

\end{description}\end{quote}

\end{fulllineitems}

\index{verify\_path() (polo.windows.run\_importer.ImportCandidate method)@\spxentry{verify\_path()}\spxextra{polo.windows.run\_importer.ImportCandidate method}}

\begin{fulllineitems}
\phantomsection\label{\detokenize{polo.windows:polo.windows.run_importer.ImportCandidate.verify_path}}\pysiglinewithargsret{\sphinxbfcode{\sphinxupquote{verify\_path}}}{}{}
Verifies that the path referenced by the {\hyperref[\detokenize{polo.windows:polo.windows.run_importer.ImportCandidate.path}]{\sphinxcrossref{\sphinxcode{\sphinxupquote{path}}}}} attribute could
potentially be imported into Polo.
\begin{quote}\begin{description}
\item[{Returns}] \leavevmode
True if {\hyperref[\detokenize{polo.windows:polo.windows.run_importer.ImportCandidate.path}]{\sphinxcrossref{\sphinxcode{\sphinxupquote{path}}}}} is verified, False otherwise

\item[{Return type}] \leavevmode
bool

\end{description}\end{quote}

\end{fulllineitems}


\end{fulllineitems}

\index{RunImporterDialog (class in polo.windows.run\_importer)@\spxentry{RunImporterDialog}\spxextra{class in polo.windows.run\_importer}}

\begin{fulllineitems}
\phantomsection\label{\detokenize{polo.windows:polo.windows.run_importer.RunImporterDialog}}\pysiglinewithargsret{\sphinxbfcode{\sphinxupquote{class }}\sphinxcode{\sphinxupquote{polo.windows.run\_importer.}}\sphinxbfcode{\sphinxupquote{RunImporterDialog}}}{\emph{\DUrole{n}{current\_run\_names}}, \emph{\DUrole{n}{parent}\DUrole{o}{=}\DUrole{default_value}{None}}}{}
Bases: \sphinxcode{\sphinxupquote{PyQt5.QtWidgets.QDialog}}

RunImporterDialog instances are the user interface for importing
runs from rar archives or directories stored on the local machine.
\begin{quote}\begin{description}
\item[{Parameters}] \leavevmode
\sphinxstyleliteralstrong{\sphinxupquote{current\_run\_names}} (\sphinxstyleliteralemphasis{\sphinxupquote{list}}\sphinxstyleliteralemphasis{\sphinxupquote{ or }}\sphinxstyleliteralemphasis{\sphinxupquote{set}}) \textendash{} Runnames that are already in use by the
current Polo session (Run names should be unique)

\end{description}\end{quote}
\index{HWI\_INDEX (polo.windows.run\_importer.RunImporterDialog attribute)@\spxentry{HWI\_INDEX}\spxextra{polo.windows.run\_importer.RunImporterDialog attribute}}

\begin{fulllineitems}
\phantomsection\label{\detokenize{polo.windows:polo.windows.run_importer.RunImporterDialog.HWI_INDEX}}\pysigline{\sphinxbfcode{\sphinxupquote{HWI\_INDEX}}\sphinxbfcode{\sphinxupquote{ = 0}}}
\end{fulllineitems}

\index{NON\_HWI\_INDEX (polo.windows.run\_importer.RunImporterDialog attribute)@\spxentry{NON\_HWI\_INDEX}\spxextra{polo.windows.run\_importer.RunImporterDialog attribute}}

\begin{fulllineitems}
\phantomsection\label{\detokenize{polo.windows:polo.windows.run_importer.RunImporterDialog.NON_HWI_INDEX}}\pysigline{\sphinxbfcode{\sphinxupquote{NON\_HWI\_INDEX}}\sphinxbfcode{\sphinxupquote{ = 1}}}
\end{fulllineitems}

\index{RAW\_INDEX (polo.windows.run\_importer.RunImporterDialog attribute)@\spxentry{RAW\_INDEX}\spxextra{polo.windows.run\_importer.RunImporterDialog attribute}}

\begin{fulllineitems}
\phantomsection\label{\detokenize{polo.windows:polo.windows.run_importer.RunImporterDialog.RAW_INDEX}}\pysigline{\sphinxbfcode{\sphinxupquote{RAW\_INDEX}}\sphinxbfcode{\sphinxupquote{ = 2}}}
\end{fulllineitems}

\index{\_add\_import\_candidates() (polo.windows.run\_importer.RunImporterDialog method)@\spxentry{\_add\_import\_candidates()}\spxextra{polo.windows.run\_importer.RunImporterDialog method}}

\begin{fulllineitems}
\phantomsection\label{\detokenize{polo.windows:polo.windows.run_importer.RunImporterDialog._add_import_candidates}}\pysiglinewithargsret{\sphinxbfcode{\sphinxupquote{\_add\_import\_candidates}}}{\emph{\DUrole{n}{new\_candidates}}}{}
Adds {\hyperref[\detokenize{polo.windows:polo.windows.run_importer.ImportCandidate}]{\sphinxcrossref{\sphinxcode{\sphinxupquote{ImportCandidate}}}}} instances to the 
\sphinxcode{\sphinxupquote{import\_candidates}} attribute.
\begin{quote}\begin{description}
\item[{Parameters}] \leavevmode
\sphinxstyleliteralstrong{\sphinxupquote{new\_candidates}} (\sphinxstyleliteralemphasis{\sphinxupquote{list}}) \textendash{} List of \sphinxtitleref{ImportCandidates}

\end{description}\end{quote}

\end{fulllineitems}

\index{\_could\_not\_import\_message() (polo.windows.run\_importer.RunImporterDialog method)@\spxentry{\_could\_not\_import\_message()}\spxextra{polo.windows.run\_importer.RunImporterDialog method}}

\begin{fulllineitems}
\phantomsection\label{\detokenize{polo.windows:polo.windows.run_importer.RunImporterDialog._could_not_import_message}}\pysiglinewithargsret{\sphinxbfcode{\sphinxupquote{\_could\_not\_import\_message}}}{\emph{\DUrole{n}{prefix}}, \emph{\DUrole{n}{paths}}}{}
Private method that creates a message box popup for when imports fail.
\begin{quote}\begin{description}
\item[{Parameters}] \leavevmode\begin{itemize}
\item {} 
\sphinxstyleliteralstrong{\sphinxupquote{prefix}} (\sphinxstyleliteralemphasis{\sphinxupquote{str}}) \textendash{} First part of the error message. Something
like “Could not import the following files:”

\item {} 
\sphinxstyleliteralstrong{\sphinxupquote{paths}} (\sphinxstyleliteralemphasis{\sphinxupquote{list}}) \textendash{} List of filepaths that could not be imported

\end{itemize}

\item[{Returns}] \leavevmode
QMessageBox

\item[{Return type}] \leavevmode
QMessageBox

\end{description}\end{quote}

\end{fulllineitems}

\index{\_disable\_hwi\_import\_tools() (polo.windows.run\_importer.RunImporterDialog method)@\spxentry{\_disable\_hwi\_import\_tools()}\spxextra{polo.windows.run\_importer.RunImporterDialog method}}

\begin{fulllineitems}
\phantomsection\label{\detokenize{polo.windows:polo.windows.run_importer.RunImporterDialog._disable_hwi_import_tools}}\pysiglinewithargsret{\sphinxbfcode{\sphinxupquote{\_disable\_hwi\_import\_tools}}}{}{}
Private method to disable widgets that should only be used for
\sphinxcode{\sphinxupquote{HWIRun}} imports.

\end{fulllineitems}

\index{\_display\_candidate\_paths() (polo.windows.run\_importer.RunImporterDialog method)@\spxentry{\_display\_candidate\_paths()}\spxextra{polo.windows.run\_importer.RunImporterDialog method}}

\begin{fulllineitems}
\phantomsection\label{\detokenize{polo.windows:polo.windows.run_importer.RunImporterDialog._display_candidate_paths}}\pysiglinewithargsret{\sphinxbfcode{\sphinxupquote{\_display\_candidate\_paths}}}{}{}
Private method that updates the dialog’s \sphinxcode{\sphinxupquote{QListWidget}} with the
file paths of the current {\hyperref[\detokenize{polo.windows:polo.windows.run_importer.ImportCandidate}]{\sphinxcrossref{\sphinxcode{\sphinxupquote{ImportCandidate}}}}} instances referenced
by the  \sphinxcode{\sphinxupquote{import\_candidates}} attribute.

\end{fulllineitems}

\index{\_display\_cocktail\_files() (polo.windows.run\_importer.RunImporterDialog method)@\spxentry{\_display\_cocktail\_files()}\spxextra{polo.windows.run\_importer.RunImporterDialog method}}

\begin{fulllineitems}
\phantomsection\label{\detokenize{polo.windows:polo.windows.run_importer.RunImporterDialog._display_cocktail_files}}\pysiglinewithargsret{\sphinxbfcode{\sphinxupquote{\_display\_cocktail\_files}}}{\emph{\DUrole{n}{menu\_type}\DUrole{o}{=}\DUrole{default_value}{None}}}{}
Private method that displays the available cocktail files to the
user via the {\hyperref[\detokenize{polo.utils:polo.utils.io_utils.Menu}]{\sphinxcrossref{\sphinxcode{\sphinxupquote{Menu}}}}} \sphinxcode{\sphinxupquote{QComboBox}} widget.
\begin{quote}\begin{description}
\item[{Parameters}] \leavevmode
\sphinxstyleliteralstrong{\sphinxupquote{menu\_type}} (\sphinxstyleliteralemphasis{\sphinxupquote{str}}\sphinxstyleliteralemphasis{\sphinxupquote{, }}\sphinxstyleliteralemphasis{\sphinxupquote{optional}}) \textendash{} Key for which kind of cocktail screens to display, defaults to None. 
“m” for membrane screens and “s” for soluble screens.

\end{description}\end{quote}

\end{fulllineitems}

\index{\_enable\_hwi\_import\_tools() (polo.windows.run\_importer.RunImporterDialog method)@\spxentry{\_enable\_hwi\_import\_tools()}\spxextra{polo.windows.run\_importer.RunImporterDialog method}}

\begin{fulllineitems}
\phantomsection\label{\detokenize{polo.windows:polo.windows.run_importer.RunImporterDialog._enable_hwi_import_tools}}\pysiglinewithargsret{\sphinxbfcode{\sphinxupquote{\_enable\_hwi\_import\_tools}}}{}{}
Private method to enable widgets that should only be used
for \sphinxcode{\sphinxupquote{HWIRun}} imports.

\end{fulllineitems}

\index{\_handle\_candidate\_change() (polo.windows.run\_importer.RunImporterDialog method)@\spxentry{\_handle\_candidate\_change()}\spxextra{polo.windows.run\_importer.RunImporterDialog method}}

\begin{fulllineitems}
\phantomsection\label{\detokenize{polo.windows:polo.windows.run_importer.RunImporterDialog._handle_candidate_change}}\pysiglinewithargsret{\sphinxbfcode{\sphinxupquote{\_handle\_candidate\_change}}}{}{}
Private method that calls 
{\hyperref[\detokenize{polo.windows:polo.windows.run_importer.RunImporterDialog._update_selected_candidate}]{\sphinxcrossref{\sphinxcode{\sphinxupquote{\_update\_selected\_candidate()}}}}}
and then  {\hyperref[\detokenize{polo.windows:polo.windows.run_importer.RunImporterDialog._populate_fields}]{\sphinxcrossref{\sphinxcode{\sphinxupquote{\_populate\_fields()}}}}}. 
This updates the data of the previously selected 
{\hyperref[\detokenize{polo.windows:polo.windows.run_importer.ImportCandidate}]{\sphinxcrossref{\sphinxcode{\sphinxupquote{ImportCandidate}}}}} if it has been changed and then
updates data display widgets with the information from the currently selected
{\hyperref[\detokenize{polo.windows:polo.windows.run_importer.ImportCandidate}]{\sphinxcrossref{\sphinxcode{\sphinxupquote{ImportCandidate}}}}} instance.

\end{fulllineitems}

\index{\_import\_files() (polo.windows.run\_importer.RunImporterDialog method)@\spxentry{\_import\_files()}\spxextra{polo.windows.run\_importer.RunImporterDialog method}}

\begin{fulllineitems}
\phantomsection\label{\detokenize{polo.windows:polo.windows.run_importer.RunImporterDialog._import_files}}\pysiglinewithargsret{\sphinxbfcode{\sphinxupquote{\_import\_files}}}{\emph{\DUrole{n}{rar}\DUrole{o}{=}\DUrole{default_value}{True}}}{}
Private method that attempts to import a collection of file paths
specified by the user. If importing a rar archive the method 
creates a {\hyperref[\detokenize{polo.threads:polo.threads.thread.QuickThread}]{\sphinxcrossref{\sphinxcode{\sphinxupquote{polo.threads.thread.QuickThread}}}}} 
instance and runs all rar operations on that 
thread to avoid freezing the GUI on slower machines. 
Imported runs are added to the \sphinxcode{\sphinxupquote{import\_candidates}} attribute
dictionary and then displayed to the user by calling
\sphinxcode{\sphinxupquote{\_display\_candidate\_paths()}}.
\begin{quote}\begin{description}
\item[{Parameters}] \leavevmode
\sphinxstyleliteralstrong{\sphinxupquote{rar}} (\sphinxstyleliteralemphasis{\sphinxupquote{bool}}\sphinxstyleliteralemphasis{\sphinxupquote{, }}\sphinxstyleliteralemphasis{\sphinxupquote{optional}}) \textendash{} If True opens the filebrowser for rar archives and filters
out all other import types, defaults to True

\end{description}\end{quote}

\end{fulllineitems}

\index{\_import\_run() (polo.windows.run\_importer.RunImporterDialog method)@\spxentry{\_import\_run()}\spxextra{polo.windows.run\_importer.RunImporterDialog method}}

\begin{fulllineitems}
\phantomsection\label{\detokenize{polo.windows:polo.windows.run_importer.RunImporterDialog._import_run}}\pysiglinewithargsret{\sphinxbfcode{\sphinxupquote{\_import\_run}}}{\emph{\DUrole{n}{import\_candidate}}}{}
Private helper method that is called by
{\hyperref[\detokenize{polo.windows:polo.windows.run_importer.RunImporterDialog._import_runs}]{\sphinxcrossref{\sphinxcode{\sphinxupquote{\_import\_runs()}}}}} that
attempts to import a run from an :class:{\color{red}\bfseries{}\textasciigrave{}}ImportCandidate.
\begin{quote}\begin{description}
\item[{Parameters}] \leavevmode
\sphinxstyleliteralstrong{\sphinxupquote{import\_candidate}} ({\hyperref[\detokenize{polo.windows:polo.windows.run_importer.ImportCandidate}]{\sphinxcrossref{\sphinxstyleliteralemphasis{\sphinxupquote{ImportCandidate}}}}}) \textendash{} {\hyperref[\detokenize{polo.windows:polo.windows.run_importer.ImportCandidate}]{\sphinxcrossref{\sphinxcode{\sphinxupquote{ImportCandidate}}}}} to create run from

\item[{Returns}] \leavevmode
Run or HWIRun if successful

\item[{Return type}] \leavevmode
{\hyperref[\detokenize{polo.crystallography:polo.crystallography.run.Run}]{\sphinxcrossref{Run}}} or {\hyperref[\detokenize{polo.crystallography:polo.crystallography.run.HWIRun}]{\sphinxcrossref{HWIRun}}}

\end{description}\end{quote}

\end{fulllineitems}

\index{\_import\_runs() (polo.windows.run\_importer.RunImporterDialog method)@\spxentry{\_import\_runs()}\spxextra{polo.windows.run\_importer.RunImporterDialog method}}

\begin{fulllineitems}
\phantomsection\label{\detokenize{polo.windows:polo.windows.run_importer.RunImporterDialog._import_runs}}\pysiglinewithargsret{\sphinxbfcode{\sphinxupquote{\_import\_runs}}}{}{}
Private method that attempts to create run objects from all available
{\hyperref[\detokenize{polo.windows:polo.windows.run_importer.ImportCandidate}]{\sphinxcrossref{\sphinxcode{\sphinxupquote{ImportCandidate}}}}} instances.

\end{fulllineitems}

\index{\_open\_browser() (polo.windows.run\_importer.RunImporterDialog method)@\spxentry{\_open\_browser()}\spxextra{polo.windows.run\_importer.RunImporterDialog method}}

\begin{fulllineitems}
\phantomsection\label{\detokenize{polo.windows:polo.windows.run_importer.RunImporterDialog._open_browser}}\pysiglinewithargsret{\sphinxbfcode{\sphinxupquote{\_open\_browser}}}{\emph{\DUrole{n}{rar}\DUrole{o}{=}\DUrole{default_value}{True}}}{}
Private method that opens a \sphinxcode{\sphinxupquote{QFileBrowser}} instance that allows the 
user to select files for import. 
The allowed filetype is set using the \sphinxtitleref{rar} flag.
\begin{quote}\begin{description}
\item[{Parameters}] \leavevmode
\sphinxstyleliteralstrong{\sphinxupquote{rar}} (\sphinxstyleliteralemphasis{\sphinxupquote{bool}}\sphinxstyleliteralemphasis{\sphinxupquote{, }}\sphinxstyleliteralemphasis{\sphinxupquote{optional}}) \textendash{} If True, allow user to only import Rar archive files 
defaults to True. If False
only allows the user to import directories.

\item[{Returns}] \leavevmode
List of files the user has selected for import

\item[{Return type}] \leavevmode
list

\end{description}\end{quote}

\end{fulllineitems}

\index{\_populate\_fields() (polo.windows.run\_importer.RunImporterDialog method)@\spxentry{\_populate\_fields()}\spxextra{polo.windows.run\_importer.RunImporterDialog method}}

\begin{fulllineitems}
\phantomsection\label{\detokenize{polo.windows:polo.windows.run_importer.RunImporterDialog._populate_fields}}\pysiglinewithargsret{\sphinxbfcode{\sphinxupquote{\_populate\_fields}}}{\emph{\DUrole{n}{import\_candidate}}}{}
Private method to display {\hyperref[\detokenize{polo.windows:polo.windows.run_importer.ImportCandidate}]{\sphinxcrossref{\sphinxcode{\sphinxupquote{ImportCandidate}}}}} 
data to the user.
\begin{quote}\begin{description}
\item[{Parameters}] \leavevmode
\sphinxstyleliteralstrong{\sphinxupquote{import\_candidate}} ({\hyperref[\detokenize{polo.windows:polo.windows.run_importer.ImportCandidate}]{\sphinxcrossref{\sphinxstyleliteralemphasis{\sphinxupquote{ImportCandidate}}}}}) \textendash{} ImportCandidate to display

\end{description}\end{quote}

\end{fulllineitems}

\index{\_remove\_run() (polo.windows.run\_importer.RunImporterDialog method)@\spxentry{\_remove\_run()}\spxextra{polo.windows.run\_importer.RunImporterDialog method}}

\begin{fulllineitems}
\phantomsection\label{\detokenize{polo.windows:polo.windows.run_importer.RunImporterDialog._remove_run}}\pysiglinewithargsret{\sphinxbfcode{\sphinxupquote{\_remove\_run}}}{}{}
Removes a run as an import candidate and refreshes the \sphinxcode{\sphinxupquote{QlistWidget}}
to reflect the removal.

\end{fulllineitems}

\index{\_restore\_defaults() (polo.windows.run\_importer.RunImporterDialog method)@\spxentry{\_restore\_defaults()}\spxextra{polo.windows.run\_importer.RunImporterDialog method}}

\begin{fulllineitems}
\phantomsection\label{\detokenize{polo.windows:polo.windows.run_importer.RunImporterDialog._restore_defaults}}\pysiglinewithargsret{\sphinxbfcode{\sphinxupquote{\_restore\_defaults}}}{}{}
Restore suggested import settings for an {\hyperref[\detokenize{polo.windows:polo.windows.run_importer.ImportCandidate}]{\sphinxcrossref{\sphinxcode{\sphinxupquote{ImportCandidate}}}}} in case
the user has changed them and then wants to undo those changes.

\end{fulllineitems}

\index{\_set\_cocktail\_menu() (polo.windows.run\_importer.RunImporterDialog method)@\spxentry{\_set\_cocktail\_menu()}\spxextra{polo.windows.run\_importer.RunImporterDialog method}}

\begin{fulllineitems}
\phantomsection\label{\detokenize{polo.windows:polo.windows.run_importer.RunImporterDialog._set_cocktail_menu}}\pysiglinewithargsret{\sphinxbfcode{\sphinxupquote{\_set\_cocktail\_menu}}}{}{}
Private method that sets the cocktail \sphinxcode{\sphinxupquote{QComboBox}} based on the
{\hyperref[\detokenize{polo.utils:polo.utils.io_utils.Menu}]{\sphinxcrossref{\sphinxcode{\sphinxupquote{Menu}}}}} instance referenced by the 
\sphinxcode{\sphinxupquote{selected\_candidate\textasciigrave{}attribute. This method is used to convey to
the user which
:class:\textasciigrave{}\textasciitilde{}polo.utils.io\_utils.Menu}} has been selected for a given
{\hyperref[\detokenize{polo.windows:polo.windows.run_importer.ImportCandidate}]{\sphinxcrossref{\sphinxcode{\sphinxupquote{ImportCandidate}}}}}.

\end{fulllineitems}

\index{\_set\_cocktail\_menu\_type\_radiobuttons() (polo.windows.run\_importer.RunImporterDialog method)@\spxentry{\_set\_cocktail\_menu\_type\_radiobuttons()}\spxextra{polo.windows.run\_importer.RunImporterDialog method}}

\begin{fulllineitems}
\phantomsection\label{\detokenize{polo.windows:polo.windows.run_importer.RunImporterDialog._set_cocktail_menu_type_radiobuttons}}\pysiglinewithargsret{\sphinxbfcode{\sphinxupquote{\_set\_cocktail\_menu\_type\_radiobuttons}}}{\emph{\DUrole{n}{type\_}}}{}
Private method that sets the {\hyperref[\detokenize{polo.utils:polo.utils.io_utils.Menu}]{\sphinxcrossref{\sphinxcode{\sphinxupquote{Menu}}}}} 
type \sphinxcode{\sphinxupquote{QRadioButtons}}
given a {\hyperref[\detokenize{polo.utils:polo.utils.io_utils.Menu}]{\sphinxcrossref{\sphinxcode{\sphinxupquote{Menu}}}}} type key.
\begin{quote}\begin{description}
\item[{Parameters}] \leavevmode
\sphinxstyleliteralstrong{\sphinxupquote{type}} (\sphinxstyleliteralemphasis{\sphinxupquote{str}}) \textendash{} Menu type key. If \sphinxtitleref{type\_} == ‘s’ then soluble
menu radioButton state is set to True. If ‘type\_\textasciigrave{} == ‘m’ then
membrane radiobutton state is set to True

\end{description}\end{quote}

\end{fulllineitems}

\index{\_set\_image\_spectrum() (polo.windows.run\_importer.RunImporterDialog method)@\spxentry{\_set\_image\_spectrum()}\spxextra{polo.windows.run\_importer.RunImporterDialog method}}

\begin{fulllineitems}
\phantomsection\label{\detokenize{polo.windows:polo.windows.run_importer.RunImporterDialog._set_image_spectrum}}\pysiglinewithargsret{\sphinxbfcode{\sphinxupquote{\_set\_image\_spectrum}}}{\emph{\DUrole{n}{spectrum}}}{}
Private method that sets the image spectrum comboBox
based on the \sphinxtitleref{spectrum} argument. Should be used to display
the inferred spectrum of an import candidate to the user when
that candidate is selected.
\begin{quote}\begin{description}
\item[{Parameters}] \leavevmode
\sphinxstyleliteralstrong{\sphinxupquote{spectrum}} (\sphinxstyleliteralemphasis{\sphinxupquote{str}}) \textendash{} Spectrum key

\end{description}\end{quote}

\end{fulllineitems}

\index{\_test\_candidate\_paths() (polo.windows.run\_importer.RunImporterDialog method)@\spxentry{\_test\_candidate\_paths()}\spxextra{polo.windows.run\_importer.RunImporterDialog method}}

\begin{fulllineitems}
\phantomsection\label{\detokenize{polo.windows:polo.windows.run_importer.RunImporterDialog._test_candidate_paths}}\pysiglinewithargsret{\sphinxbfcode{\sphinxupquote{\_test\_candidate\_paths}}}{\emph{\DUrole{n}{file\_paths}}}{}
Private method that validates filepaths to ensure they could be
imported into Polo.
\begin{quote}\begin{description}
\item[{Parameters}] \leavevmode
\sphinxstyleliteralstrong{\sphinxupquote{file\_paths}} (\sphinxstyleliteralemphasis{\sphinxupquote{list}}) \textendash{} List of filepaths to be imported

\item[{Returns}] \leavevmode
Tuple with first item being verified paths and second being
list of paths that failed verification tests.

\item[{Return type}] \leavevmode
tuple

\end{description}\end{quote}

\end{fulllineitems}

\index{\_unrar\_candidate\_paths() (polo.windows.run\_importer.RunImporterDialog method)@\spxentry{\_unrar\_candidate\_paths()}\spxextra{polo.windows.run\_importer.RunImporterDialog method}}

\begin{fulllineitems}
\phantomsection\label{\detokenize{polo.windows:polo.windows.run_importer.RunImporterDialog._unrar_candidate_paths}}\pysiglinewithargsret{\sphinxbfcode{\sphinxupquote{\_unrar\_candidate\_paths}}}{\emph{\DUrole{n}{candidate\_paths}}}{}
Private method that attempts to un\sphinxhyphen{}compress a collection of rar
archive files.
\begin{quote}\begin{description}
\item[{Parameters}] \leavevmode
\sphinxstyleliteralstrong{\sphinxupquote{candidate\_paths}} (\sphinxstyleliteralemphasis{\sphinxupquote{list}}) \textendash{} List of filepaths to unrar

\item[{Returns}] \leavevmode
Tuple, first being list of paths that were successfully unrared and the
second being list of filepaths that could not be unrared

\item[{Return type}] \leavevmode
tuple

\end{description}\end{quote}

\end{fulllineitems}

\index{\_update\_candidate\_run\_data() (polo.windows.run\_importer.RunImporterDialog method)@\spxentry{\_update\_candidate\_run\_data()}\spxextra{polo.windows.run\_importer.RunImporterDialog method}}

\begin{fulllineitems}
\phantomsection\label{\detokenize{polo.windows:polo.windows.run_importer.RunImporterDialog._update_candidate_run_data}}\pysiglinewithargsret{\sphinxbfcode{\sphinxupquote{\_update\_candidate\_run\_data}}}{}{}
Private method that allows the user to update an {\hyperref[\detokenize{polo.windows:polo.windows.run_importer.ImportCandidate}]{\sphinxcrossref{\sphinxcode{\sphinxupquote{ImportCandidate}}}}}
instance’s data from the widgets in the \sphinxtitleref{RunImporterDialog} by updating
the dictionary referenced by an {\hyperref[\detokenize{polo.windows:polo.windows.run_importer.ImportCandidate}]{\sphinxcrossref{\sphinxcode{\sphinxupquote{ImportCandidate}}}}} instacnes’s 
\sphinxcode{\sphinxupquote{data}} attribute with user entered values.

\end{fulllineitems}

\index{\_update\_selected\_candidate() (polo.windows.run\_importer.RunImporterDialog method)@\spxentry{\_update\_selected\_candidate()}\spxextra{polo.windows.run\_importer.RunImporterDialog method}}

\begin{fulllineitems}
\phantomsection\label{\detokenize{polo.windows:polo.windows.run_importer.RunImporterDialog._update_selected_candidate}}\pysiglinewithargsret{\sphinxbfcode{\sphinxupquote{\_update\_selected\_candidate}}}{}{}
Private method that updates currently selected {\hyperref[\detokenize{polo.windows:polo.windows.run_importer.ImportCandidate}]{\sphinxcrossref{\sphinxcode{\sphinxupquote{ImportCandidate}}}}} by
calling {\hyperref[\detokenize{polo.windows:polo.windows.run_importer.RunImporterDialog._update_candidate_run_data}]{\sphinxcrossref{\sphinxcode{\sphinxupquote{\_update\_candidate\_run\_data()}}}}}
and then updating the display by calling
{\hyperref[\detokenize{polo.windows:polo.windows.run_importer.RunImporterDialog._populate_fields}]{\sphinxcrossref{\sphinxcode{\sphinxupquote{\_populate\_fields()}}}}}.

\end{fulllineitems}

\index{\_verify\_run\_name() (polo.windows.run\_importer.RunImporterDialog method)@\spxentry{\_verify\_run\_name()}\spxextra{polo.windows.run\_importer.RunImporterDialog method}}

\begin{fulllineitems}
\phantomsection\label{\detokenize{polo.windows:polo.windows.run_importer.RunImporterDialog._verify_run_name}}\pysiglinewithargsret{\sphinxbfcode{\sphinxupquote{\_verify\_run\_name}}}{}{}
Private method to verify a run name. If run name fails verification
clears the runname \sphinxcode{\sphinxupquote{QLineEdit}} widget and shows an error message to the user.

\end{fulllineitems}

\index{all\_run\_names() (polo.windows.run\_importer.RunImporterDialog property)@\spxentry{all\_run\_names()}\spxextra{polo.windows.run\_importer.RunImporterDialog property}}

\begin{fulllineitems}
\phantomsection\label{\detokenize{polo.windows:polo.windows.run_importer.RunImporterDialog.all_run_names}}\pysigline{\sphinxbfcode{\sphinxupquote{property }}\sphinxbfcode{\sphinxupquote{all\_run\_names}}}
All run names of all current :class:ImportCandidate
instances.
\begin{quote}\begin{description}
\item[{Returns}] \leavevmode
Set of all run names

\item[{Return type}] \leavevmode
set

\end{description}\end{quote}

\end{fulllineitems}

\index{selected\_candidate() (polo.windows.run\_importer.RunImporterDialog property)@\spxentry{selected\_candidate()}\spxextra{polo.windows.run\_importer.RunImporterDialog property}}

\begin{fulllineitems}
\phantomsection\label{\detokenize{polo.windows:polo.windows.run_importer.RunImporterDialog.selected_candidate}}\pysigline{\sphinxbfcode{\sphinxupquote{property }}\sphinxbfcode{\sphinxupquote{selected\_candidate}}}
The currently selected {\hyperref[\detokenize{polo.windows:polo.windows.run_importer.ImportCandidate}]{\sphinxcrossref{\sphinxcode{\sphinxupquote{ImportCandidate}}}}} if one exists, otherwise
returns None.
\begin{quote}\begin{description}
\item[{Returns}] \leavevmode
Currently selected candidate

\item[{Return type}] \leavevmode
{\hyperref[\detokenize{polo.windows:polo.windows.run_importer.ImportCandidate}]{\sphinxcrossref{ImportCandidate}}}

\end{description}\end{quote}

\end{fulllineitems}

\index{selection\_dict() (polo.windows.run\_importer.RunImporterDialog property)@\spxentry{selection\_dict()}\spxextra{polo.windows.run\_importer.RunImporterDialog property}}

\begin{fulllineitems}
\phantomsection\label{\detokenize{polo.windows:polo.windows.run_importer.RunImporterDialog.selection_dict}}\pysigline{\sphinxbfcode{\sphinxupquote{property }}\sphinxbfcode{\sphinxupquote{selection\_dict}}}
Returns a dictionary who’s keys are \sphinxcode{\sphinxupquote{Run}} attributes and values
are the values of {\hyperref[\detokenize{polo.windows:polo.windows.run_importer.RunImporterDialog}]{\sphinxcrossref{\sphinxcode{\sphinxupquote{RunImporterDialog}}}}} widgets that correspond to
these attributes.

Example of the dictionary returned below.

\begin{sphinxVerbatim}[commandchars=\\\{\}]
\PYG{p}{\PYGZob{}}
    \PYG{l+s+s1}{\PYGZsq{}}\PYG{l+s+s1}{cocktail\PYGZus{}menu}\PYG{l+s+s1}{\PYGZsq{}}\PYG{p}{:} \PYG{n}{Menu}\PYG{p}{,}
    \PYG{l+s+s1}{\PYGZsq{}}\PYG{l+s+s1}{date}\PYG{l+s+s1}{\PYGZsq{}}\PYG{p}{:} \PYG{n}{datetime}\PYG{p}{,}
    \PYG{l+s+s1}{\PYGZsq{}}\PYG{l+s+s1}{run\PYGZus{}name}\PYG{l+s+s1}{\PYGZsq{}}\PYG{p}{:} \PYG{n+nb}{str}\PYG{p}{,}
    \PYG{l+s+s1}{\PYGZsq{}}\PYG{l+s+s1}{image\PYGZus{}spectrum}\PYG{l+s+s1}{\PYGZsq{}}\PYG{p}{:} \PYG{n+nb}{str}
\PYG{p}{\PYGZcb{}}
\end{sphinxVerbatim}
\begin{quote}\begin{description}
\item[{Returns}] \leavevmode
dict

\item[{Return type}] \leavevmode
dict

\end{description}\end{quote}

\end{fulllineitems}


\end{fulllineitems}



\subsubsection{polo.windows.run\_updater\_dialog module}
\label{\detokenize{polo.windows:module-polo.windows.run_updater_dialog}}\label{\detokenize{polo.windows:polo-windows-run-updater-dialog-module}}\index{module@\spxentry{module}!polo.windows.run\_updater\_dialog@\spxentry{polo.windows.run\_updater\_dialog}}\index{polo.windows.run\_updater\_dialog@\spxentry{polo.windows.run\_updater\_dialog}!module@\spxentry{module}}\index{RunUpdaterDialog (class in polo.windows.run\_updater\_dialog)@\spxentry{RunUpdaterDialog}\spxextra{class in polo.windows.run\_updater\_dialog}}

\begin{fulllineitems}
\phantomsection\label{\detokenize{polo.windows:polo.windows.run_updater_dialog.RunUpdaterDialog}}\pysiglinewithargsret{\sphinxbfcode{\sphinxupquote{class }}\sphinxcode{\sphinxupquote{polo.windows.run\_updater\_dialog.}}\sphinxbfcode{\sphinxupquote{RunUpdaterDialog}}}{\emph{\DUrole{n}{run}}, \emph{\DUrole{n}{run\_names}}, \emph{\DUrole{n}{parent}\DUrole{o}{=}\DUrole{default_value}{None}}}{}
Bases: \sphinxcode{\sphinxupquote{PyQt5.QtWidgets.QDialog}}

Small dialog for updating basic information about a run after
it has been imported. Includes updating the plate ID, the cocktail
menu used and the image spectrum.
\begin{quote}\begin{description}
\item[{Parameters}] \leavevmode\begin{itemize}
\item {} 
\sphinxstyleliteralstrong{\sphinxupquote{run}} ({\hyperref[\detokenize{polo.crystallography:polo.crystallography.run.Run}]{\sphinxcrossref{\sphinxstyleliteralemphasis{\sphinxupquote{Run}}}}}\sphinxstyleliteralemphasis{\sphinxupquote{ or }}{\hyperref[\detokenize{polo.crystallography:polo.crystallography.run.HWIRun}]{\sphinxcrossref{\sphinxstyleliteralemphasis{\sphinxupquote{HWIRun}}}}}) \textendash{} Run to update

\item {} 
\sphinxstyleliteralstrong{\sphinxupquote{run\_names}} (\sphinxstyleliteralemphasis{\sphinxupquote{list}}\sphinxstyleliteralemphasis{\sphinxupquote{ or }}\sphinxstyleliteralemphasis{\sphinxupquote{set}}) \textendash{} Names of already loaded runs.

\item {} 
\sphinxstyleliteralstrong{\sphinxupquote{parent}} (\sphinxstyleliteralemphasis{\sphinxupquote{QWidget}}\sphinxstyleliteralemphasis{\sphinxupquote{, }}\sphinxstyleliteralemphasis{\sphinxupquote{optional}}) \textendash{} Parent widget, defaults to None

\end{itemize}

\end{description}\end{quote}
\index{\_select\_run\_menu() (polo.windows.run\_updater\_dialog.RunUpdaterDialog method)@\spxentry{\_select\_run\_menu()}\spxextra{polo.windows.run\_updater\_dialog.RunUpdaterDialog method}}

\begin{fulllineitems}
\phantomsection\label{\detokenize{polo.windows:polo.windows.run_updater_dialog.RunUpdaterDialog._select_run_menu}}\pysiglinewithargsret{\sphinxbfcode{\sphinxupquote{\_select\_run\_menu}}}{}{}
Private method that sets the current index of the  \sphinxcode{\sphinxupquote{QComboBox}}
based on the current \sphinxcode{\sphinxupquote{cocktail\_menu}} attribute of the \sphinxcode{\sphinxupquote{Run}}
instance  referenced by the {\hyperref[\detokenize{polo.windows:polo.windows.run_updater_dialog.RunUpdaterDialog.run}]{\sphinxcrossref{\sphinxcode{\sphinxupquote{run}}}}} attribute.

\end{fulllineitems}

\index{\_set\_cocktail\_menu() (polo.windows.run\_updater\_dialog.RunUpdaterDialog method)@\spxentry{\_set\_cocktail\_menu()}\spxextra{polo.windows.run\_updater\_dialog.RunUpdaterDialog method}}

\begin{fulllineitems}
\phantomsection\label{\detokenize{polo.windows:polo.windows.run_updater_dialog.RunUpdaterDialog._set_cocktail_menu}}\pysiglinewithargsret{\sphinxbfcode{\sphinxupquote{\_set\_cocktail\_menu}}}{}{}
Private method that display cocktails in the 
{\hyperref[\detokenize{polo.utils:polo.utils.io_utils.Menu}]{\sphinxcrossref{\sphinxcode{\sphinxupquote{Menu}}}}} \sphinxcode{\sphinxupquote{QComboBox}} based on
the current menu type selection. Either displays
soluble or membrane cocktail menus.

\end{fulllineitems}

\index{\_set\_run\_date() (polo.windows.run\_updater\_dialog.RunUpdaterDialog method)@\spxentry{\_set\_run\_date()}\spxextra{polo.windows.run\_updater\_dialog.RunUpdaterDialog method}}

\begin{fulllineitems}
\phantomsection\label{\detokenize{polo.windows:polo.windows.run_updater_dialog.RunUpdaterDialog._set_run_date}}\pysiglinewithargsret{\sphinxbfcode{\sphinxupquote{\_set\_run\_date}}}{}{}
Set the \sphinxcode{\sphinxupquote{date}} attribute of the \sphinxcode{\sphinxupquote{Run}} referenced
by the {\hyperref[\detokenize{polo.windows:polo.windows.run_updater_dialog.RunUpdaterDialog.run}]{\sphinxcrossref{\sphinxcode{\sphinxupquote{run}}}}} attribute from 
the value in the \sphinxcode{\sphinxupquote{QDateEdit}} widget.

\end{fulllineitems}

\index{\_update\_plate\_id() (polo.windows.run\_updater\_dialog.RunUpdaterDialog method)@\spxentry{\_update\_plate\_id()}\spxextra{polo.windows.run\_updater\_dialog.RunUpdaterDialog method}}

\begin{fulllineitems}
\phantomsection\label{\detokenize{polo.windows:polo.windows.run_updater_dialog.RunUpdaterDialog._update_plate_id}}\pysiglinewithargsret{\sphinxbfcode{\sphinxupquote{\_update\_plate\_id}}}{}{}
Private method that updates the \sphinxtitleref{plate\_id} attribute of the 
Run instance references by the {\hyperref[\detokenize{polo.windows:polo.windows.run_updater_dialog.RunUpdaterDialog.run}]{\sphinxcrossref{\sphinxcode{\sphinxupquote{run}}}}} attribute based on the contents
of the plate ID \sphinxcode{\sphinxupquote{QLineEdit}} widget.

\end{fulllineitems}

\index{\_update\_run() (polo.windows.run\_updater\_dialog.RunUpdaterDialog method)@\spxentry{\_update\_run()}\spxextra{polo.windows.run\_updater\_dialog.RunUpdaterDialog method}}

\begin{fulllineitems}
\phantomsection\label{\detokenize{polo.windows:polo.windows.run_updater_dialog.RunUpdaterDialog._update_run}}\pysiglinewithargsret{\sphinxbfcode{\sphinxupquote{\_update\_run}}}{}{}
Private wrapper method that calls all other \sphinxtitleref{\_update} methods
and then closes the dialog.

\end{fulllineitems}

\index{\_update\_run\_cocktail\_menu() (polo.windows.run\_updater\_dialog.RunUpdaterDialog method)@\spxentry{\_update\_run\_cocktail\_menu()}\spxextra{polo.windows.run\_updater\_dialog.RunUpdaterDialog method}}

\begin{fulllineitems}
\phantomsection\label{\detokenize{polo.windows:polo.windows.run_updater_dialog.RunUpdaterDialog._update_run_cocktail_menu}}\pysiglinewithargsret{\sphinxbfcode{\sphinxupquote{\_update\_run\_cocktail\_menu}}}{}{}
Private method that updates the \sphinxtitleref{cocktail\_menu} attribute of the 
\sphinxtitleref{Run} instance referenced by the {\hyperref[\detokenize{polo.windows:polo.windows.run_updater_dialog.RunUpdaterDialog.run}]{\sphinxcrossref{\sphinxcode{\sphinxupquote{run}}}}} attribute based on the current 
{\hyperref[\detokenize{polo.utils:polo.utils.io_utils.Menu}]{\sphinxcrossref{\sphinxcode{\sphinxupquote{Menu}}}}} \sphinxcode{\sphinxupquote{QComboBox}} selection.

\end{fulllineitems}

\index{\_update\_spectrum() (polo.windows.run\_updater\_dialog.RunUpdaterDialog method)@\spxentry{\_update\_spectrum()}\spxextra{polo.windows.run\_updater\_dialog.RunUpdaterDialog method}}

\begin{fulllineitems}
\phantomsection\label{\detokenize{polo.windows:polo.windows.run_updater_dialog.RunUpdaterDialog._update_spectrum}}\pysiglinewithargsret{\sphinxbfcode{\sphinxupquote{\_update\_spectrum}}}{}{}
Private method that update the spectrum of the {\hyperref[\detokenize{polo.windows:polo.windows.run_updater_dialog.RunUpdaterDialog.run}]{\sphinxcrossref{\sphinxcode{\sphinxupquote{run}}}}} attribute 
and the images in that run based on the current selection of the 
spectrum  \sphinxcode{\sphinxupquote{QComboBox}}.

\end{fulllineitems}

\index{current\_menus() (polo.windows.run\_updater\_dialog.RunUpdaterDialog property)@\spxentry{current\_menus()}\spxextra{polo.windows.run\_updater\_dialog.RunUpdaterDialog property}}

\begin{fulllineitems}
\phantomsection\label{\detokenize{polo.windows:polo.windows.run_updater_dialog.RunUpdaterDialog.current_menus}}\pysigline{\sphinxbfcode{\sphinxupquote{property }}\sphinxbfcode{\sphinxupquote{current\_menus}}}
The {\hyperref[\detokenize{polo.utils:polo.utils.io_utils.Menu}]{\sphinxcrossref{\sphinxcode{\sphinxupquote{polo.utils.io\_utils.Menu}}}}} instances that are currently being displayed
to the user via the {\hyperref[\detokenize{polo.utils:polo.utils.io_utils.Menu}]{\sphinxcrossref{\sphinxcode{\sphinxupquote{Menu}}}}} :\sphinxcode{\sphinxupquote{QComboBox}} widget.
\begin{quote}\begin{description}
\item[{Returns}] \leavevmode
List of {\hyperref[\detokenize{polo.utils:polo.utils.io_utils.Menu}]{\sphinxcrossref{\sphinxcode{\sphinxupquote{polo.utils.io\_utils.Menu}}}}} instances

\item[{Return type}] \leavevmode
list

\end{description}\end{quote}

\end{fulllineitems}

\index{run() (polo.windows.run\_updater\_dialog.RunUpdaterDialog property)@\spxentry{run()}\spxextra{polo.windows.run\_updater\_dialog.RunUpdaterDialog property}}

\begin{fulllineitems}
\phantomsection\label{\detokenize{polo.windows:polo.windows.run_updater_dialog.RunUpdaterDialog.run}}\pysigline{\sphinxbfcode{\sphinxupquote{property }}\sphinxbfcode{\sphinxupquote{run}}}
The run being updated.
\begin{quote}\begin{description}
\item[{Returns}] \leavevmode
The run being updated

\item[{Return type}] \leavevmode
{\hyperref[\detokenize{polo.crystallography:polo.crystallography.run.Run}]{\sphinxcrossref{Run}}} or {\hyperref[\detokenize{polo.crystallography:polo.crystallography.run.HWIRun}]{\sphinxcrossref{HWIRun}}}

\end{description}\end{quote}

\end{fulllineitems}


\end{fulllineitems}



\subsubsection{polo.windows.secure\_dave\_dailog module}
\label{\detokenize{polo.windows:polo-windows-secure-dave-dailog-module}}

\subsubsection{polo.windows.settings\_dialog module}
\label{\detokenize{polo.windows:polo-windows-settings-dialog-module}}

\subsubsection{polo.windows.spectrum\_dialog module}
\label{\detokenize{polo.windows:module-polo.windows.spectrum_dialog}}\label{\detokenize{polo.windows:polo-windows-spectrum-dialog-module}}\index{module@\spxentry{module}!polo.windows.spectrum\_dialog@\spxentry{polo.windows.spectrum\_dialog}}\index{polo.windows.spectrum\_dialog@\spxentry{polo.windows.spectrum\_dialog}!module@\spxentry{module}}\index{SpectrumDialog (class in polo.windows.spectrum\_dialog)@\spxentry{SpectrumDialog}\spxextra{class in polo.windows.spectrum\_dialog}}

\begin{fulllineitems}
\phantomsection\label{\detokenize{polo.windows:polo.windows.spectrum_dialog.SpectrumDialog}}\pysiglinewithargsret{\sphinxbfcode{\sphinxupquote{class }}\sphinxcode{\sphinxupquote{polo.windows.spectrum\_dialog.}}\sphinxbfcode{\sphinxupquote{SpectrumDialog}}}{\emph{\DUrole{n}{loaded\_runs}}}{}
Bases: \sphinxcode{\sphinxupquote{PyQt5.QtWidgets.QDialog}}

Small dialog used to link runs together by image spectrum. This is
generally done when the same plate has been imaged using different
photographic technologies. Linking the runs together allows the user to
switch between the images in either run easily.
\begin{quote}\begin{description}
\item[{Parameters}] \leavevmode
\sphinxstyleliteralstrong{\sphinxupquote{loaded\_runs}} (\sphinxstyleliteralemphasis{\sphinxupquote{list}}) \textendash{} List of runs that have been loaded into Polo

\end{description}\end{quote}
\index{display\_suggestion() (polo.windows.spectrum\_dialog.SpectrumDialog method)@\spxentry{display\_suggestion()}\spxextra{polo.windows.spectrum\_dialog.SpectrumDialog method}}

\begin{fulllineitems}
\phantomsection\label{\detokenize{polo.windows:polo.windows.spectrum_dialog.SpectrumDialog.display_suggestion}}\pysiglinewithargsret{\sphinxbfcode{\sphinxupquote{display\_suggestion}}}{}{}
Show the link suggestion to the user by selecting suggested links.

\end{fulllineitems}

\index{get\_selections() (polo.windows.spectrum\_dialog.SpectrumDialog method)@\spxentry{get\_selections()}\spxextra{polo.windows.spectrum\_dialog.SpectrumDialog method}}

\begin{fulllineitems}
\phantomsection\label{\detokenize{polo.windows:polo.windows.spectrum_dialog.SpectrumDialog.get_selections}}\pysiglinewithargsret{\sphinxbfcode{\sphinxupquote{get\_selections}}}{}{}
Retrieve the runs that have been selected by the user or by
suggestion.
\begin{quote}\begin{description}
\item[{Returns}] \leavevmode
list of selected run names

\item[{Return type}] \leavevmode
list

\end{description}\end{quote}

\end{fulllineitems}

\index{get\_spectrum\_list() (polo.windows.spectrum\_dialog.SpectrumDialog method)@\spxentry{get\_spectrum\_list()}\spxextra{polo.windows.spectrum\_dialog.SpectrumDialog method}}

\begin{fulllineitems}
\phantomsection\label{\detokenize{polo.windows:polo.windows.spectrum_dialog.SpectrumDialog.get_spectrum_list}}\pysiglinewithargsret{\sphinxbfcode{\sphinxupquote{get\_spectrum\_list}}}{\emph{\DUrole{n}{run}}}{}
Returns the listwidget that a run should be assigned to based
on the run’s image type.
\begin{quote}\begin{description}
\item[{Parameters}] \leavevmode
\sphinxstyleliteralstrong{\sphinxupquote{run}} ({\hyperref[\detokenize{polo.crystallography:polo.crystallography.run.Run}]{\sphinxcrossref{\sphinxstyleliteralemphasis{\sphinxupquote{Run}}}}}) \textendash{} Run object to assign to a listWidget

\item[{Returns}] \leavevmode
QListWidget to place that run into

\item[{Return type}] \leavevmode
QListWidget

\end{description}\end{quote}

\end{fulllineitems}

\index{link\_current\_selection() (polo.windows.spectrum\_dialog.SpectrumDialog method)@\spxentry{link\_current\_selection()}\spxextra{polo.windows.spectrum\_dialog.SpectrumDialog method}}

\begin{fulllineitems}
\phantomsection\label{\detokenize{polo.windows:polo.windows.spectrum_dialog.SpectrumDialog.link_current_selection}}\pysiglinewithargsret{\sphinxbfcode{\sphinxupquote{link\_current\_selection}}}{}{}
Link the currently selected runs together. Creates a circular
linked list structure.

\end{fulllineitems}

\index{populate\_list\_widgets() (polo.windows.spectrum\_dialog.SpectrumDialog method)@\spxentry{populate\_list\_widgets()}\spxextra{polo.windows.spectrum\_dialog.SpectrumDialog method}}

\begin{fulllineitems}
\phantomsection\label{\detokenize{polo.windows:polo.windows.spectrum_dialog.SpectrumDialog.populate_list_widgets}}\pysiglinewithargsret{\sphinxbfcode{\sphinxupquote{populate\_list\_widgets}}}{}{}
Adds items to each image spectrum type list widget based on the
Run objects stored in the \sphinxtitleref{loaded\_runs} attribute.

\end{fulllineitems}

\index{show\_error\_message() (polo.windows.spectrum\_dialog.SpectrumDialog method)@\spxentry{show\_error\_message()}\spxextra{polo.windows.spectrum\_dialog.SpectrumDialog method}}

\begin{fulllineitems}
\phantomsection\label{\detokenize{polo.windows:polo.windows.spectrum_dialog.SpectrumDialog.show_error_message}}\pysiglinewithargsret{\sphinxbfcode{\sphinxupquote{show\_error\_message}}}{\emph{\DUrole{n}{message}\DUrole{o}{=}\DUrole{default_value}{\textquotesingle{}:(\textquotesingle{}}}}{}
Helper method for showing a QErrorMessage dialog to the user.
\begin{quote}\begin{description}
\item[{Parameters}] \leavevmode
\sphinxstyleliteralstrong{\sphinxupquote{message}} \textendash{} String. The message text to show to the user.

\end{description}\end{quote}

\end{fulllineitems}

\index{suggest\_links() (polo.windows.spectrum\_dialog.SpectrumDialog method)@\spxentry{suggest\_links()}\spxextra{polo.windows.spectrum\_dialog.SpectrumDialog method}}

\begin{fulllineitems}
\phantomsection\label{\detokenize{polo.windows:polo.windows.spectrum_dialog.SpectrumDialog.suggest_links}}\pysiglinewithargsret{\sphinxbfcode{\sphinxupquote{suggest\_links}}}{}{}
Suggest runs to link together based on their imaging dates. A link
suggestion will be made if the images were taken on the same day but the
runs are labeled as different image types.
\begin{quote}\begin{description}
\item[{Returns}] \leavevmode
Suggested links as list of tuples, each tuple containing two
runs that are suggested for linking.

\item[{Return type}] \leavevmode
list

\end{description}\end{quote}

\end{fulllineitems}

\index{validate\_selection() (polo.windows.spectrum\_dialog.SpectrumDialog method)@\spxentry{validate\_selection()}\spxextra{polo.windows.spectrum\_dialog.SpectrumDialog method}}

\begin{fulllineitems}
\phantomsection\label{\detokenize{polo.windows:polo.windows.spectrum_dialog.SpectrumDialog.validate_selection}}\pysiglinewithargsret{\sphinxbfcode{\sphinxupquote{validate\_selection}}}{\emph{\DUrole{n}{selected\_runs}}}{}
\end{fulllineitems}


\end{fulllineitems}



\subsubsection{polo.windows.time\_res\_dialog module}
\label{\detokenize{polo.windows:module-polo.windows.time_res_dialog}}\label{\detokenize{polo.windows:polo-windows-time-res-dialog-module}}\index{module@\spxentry{module}!polo.windows.time\_res\_dialog@\spxentry{polo.windows.time\_res\_dialog}}\index{polo.windows.time\_res\_dialog@\spxentry{polo.windows.time\_res\_dialog}!module@\spxentry{module}}\index{TimeResDialog (class in polo.windows.time\_res\_dialog)@\spxentry{TimeResDialog}\spxextra{class in polo.windows.time\_res\_dialog}}

\begin{fulllineitems}
\phantomsection\label{\detokenize{polo.windows:polo.windows.time_res_dialog.TimeResDialog}}\pysiglinewithargsret{\sphinxbfcode{\sphinxupquote{class }}\sphinxcode{\sphinxupquote{polo.windows.time\_res\_dialog.}}\sphinxbfcode{\sphinxupquote{TimeResDialog}}}{\emph{\DUrole{n}{available\_runs}}}{}
Bases: \sphinxcode{\sphinxupquote{PyQt5.QtWidgets.QDialog}}
\index{auto\_detect\_time\_links() (polo.windows.time\_res\_dialog.TimeResDialog method)@\spxentry{auto\_detect\_time\_links()}\spxextra{polo.windows.time\_res\_dialog.TimeResDialog method}}

\begin{fulllineitems}
\phantomsection\label{\detokenize{polo.windows:polo.windows.time_res_dialog.TimeResDialog.auto_detect_time_links}}\pysiglinewithargsret{\sphinxbfcode{\sphinxupquote{auto\_detect\_time\_links}}}{}{}
\end{fulllineitems}

\index{display\_runs() (polo.windows.time\_res\_dialog.TimeResDialog method)@\spxentry{display\_runs()}\spxextra{polo.windows.time\_res\_dialog.TimeResDialog method}}

\begin{fulllineitems}
\phantomsection\label{\detokenize{polo.windows:polo.windows.time_res_dialog.TimeResDialog.display_runs}}\pysiglinewithargsret{\sphinxbfcode{\sphinxupquote{display\_runs}}}{}{}
\end{fulllineitems}

\index{get\_HWI\_runs() (polo.windows.time\_res\_dialog.TimeResDialog method)@\spxentry{get\_HWI\_runs()}\spxextra{polo.windows.time\_res\_dialog.TimeResDialog method}}

\begin{fulllineitems}
\phantomsection\label{\detokenize{polo.windows:polo.windows.time_res_dialog.TimeResDialog.get_HWI_runs}}\pysiglinewithargsret{\sphinxbfcode{\sphinxupquote{get\_HWI\_runs}}}{}{}
\end{fulllineitems}

\index{sort\_available\_runs\_by\_date() (polo.windows.time\_res\_dialog.TimeResDialog method)@\spxentry{sort\_available\_runs\_by\_date()}\spxextra{polo.windows.time\_res\_dialog.TimeResDialog method}}

\begin{fulllineitems}
\phantomsection\label{\detokenize{polo.windows:polo.windows.time_res_dialog.TimeResDialog.sort_available_runs_by_date}}\pysiglinewithargsret{\sphinxbfcode{\sphinxupquote{sort\_available\_runs\_by\_date}}}{}{}
\end{fulllineitems}

\index{validate\_user\_selection() (polo.windows.time\_res\_dialog.TimeResDialog method)@\spxentry{validate\_user\_selection()}\spxextra{polo.windows.time\_res\_dialog.TimeResDialog method}}

\begin{fulllineitems}
\phantomsection\label{\detokenize{polo.windows:polo.windows.time_res_dialog.TimeResDialog.validate_user_selection}}\pysiglinewithargsret{\sphinxbfcode{\sphinxupquote{validate\_user\_selection}}}{}{}
\end{fulllineitems}


\end{fulllineitems}



\subsubsection{Module contents}
\label{\detokenize{polo.windows:module-polo.windows}}\label{\detokenize{polo.windows:module-contents}}\index{module@\spxentry{module}!polo.windows@\spxentry{polo.windows}}\index{polo.windows@\spxentry{polo.windows}!module@\spxentry{module}}

\section{Module contents}
\label{\detokenize{polo:module-polo}}\label{\detokenize{polo:module-contents}}\index{module@\spxentry{module}!polo@\spxentry{polo}}\index{polo@\spxentry{polo}!module@\spxentry{module}}\index{make\_default\_logger() (in module polo)@\spxentry{make\_default\_logger()}\spxextra{in module polo}}

\begin{fulllineitems}
\phantomsection\label{\detokenize{polo:polo.make_default_logger}}\pysiglinewithargsret{\sphinxcode{\sphinxupquote{polo.}}\sphinxbfcode{\sphinxupquote{make\_default\_logger}}}{\emph{\DUrole{n}{name}}}{}
\end{fulllineitems}



\chapter{Indices and tables}
\label{\detokenize{index:indices-and-tables}}\begin{itemize}
\item {} 
\DUrole{xref,std,std-ref}{genindex}

\item {} 
\DUrole{xref,std,std-ref}{modindex}

\item {} 
\DUrole{xref,std,std-ref}{search}

\end{itemize}


\renewcommand{\indexname}{Python Module Index}
\begin{sphinxtheindex}
\let\bigletter\sphinxstyleindexlettergroup
\bigletter{p}
\item\relax\sphinxstyleindexentry{polo}\sphinxstyleindexpageref{polo:\detokenize{module-polo}}
\item\relax\sphinxstyleindexentry{polo.crystallography}\sphinxstyleindexpageref{polo.crystallography:\detokenize{module-polo.crystallography}}
\item\relax\sphinxstyleindexentry{polo.crystallography.cocktail}\sphinxstyleindexpageref{polo.crystallography:\detokenize{module-polo.crystallography.cocktail}}
\item\relax\sphinxstyleindexentry{polo.crystallography.image}\sphinxstyleindexpageref{polo.crystallography:\detokenize{module-polo.crystallography.image}}
\item\relax\sphinxstyleindexentry{polo.crystallography.run}\sphinxstyleindexpageref{polo.crystallography:\detokenize{module-polo.crystallography.run}}
\item\relax\sphinxstyleindexentry{polo.marco}\sphinxstyleindexpageref{polo.marco:\detokenize{module-polo.marco}}
\item\relax\sphinxstyleindexentry{polo.marco.run\_marco}\sphinxstyleindexpageref{polo.marco:\detokenize{module-polo.marco.run_marco}}
\item\relax\sphinxstyleindexentry{polo.threads}\sphinxstyleindexpageref{polo.threads:\detokenize{module-polo.threads}}
\item\relax\sphinxstyleindexentry{polo.threads.thread}\sphinxstyleindexpageref{polo.threads:\detokenize{module-polo.threads.thread}}
\item\relax\sphinxstyleindexentry{polo.utils}\sphinxstyleindexpageref{polo.utils:\detokenize{module-polo.utils}}
\item\relax\sphinxstyleindexentry{polo.utils.dialog\_utils}\sphinxstyleindexpageref{polo.utils:\detokenize{module-polo.utils.dialog_utils}}
\item\relax\sphinxstyleindexentry{polo.utils.exceptions}\sphinxstyleindexpageref{polo.utils:\detokenize{module-polo.utils.exceptions}}
\item\relax\sphinxstyleindexentry{polo.utils.ftp\_utils}\sphinxstyleindexpageref{polo.utils:\detokenize{module-polo.utils.ftp_utils}}
\item\relax\sphinxstyleindexentry{polo.utils.io\_utils}\sphinxstyleindexpageref{polo.utils:\detokenize{module-polo.utils.io_utils}}
\item\relax\sphinxstyleindexentry{polo.utils.math\_utils}\sphinxstyleindexpageref{polo.utils:\detokenize{module-polo.utils.math_utils}}
\item\relax\sphinxstyleindexentry{polo.utils.unrar\_utils}\sphinxstyleindexpageref{polo.utils:\detokenize{module-polo.utils.unrar_utils}}
\item\relax\sphinxstyleindexentry{polo.widgets}\sphinxstyleindexpageref{polo.widgets:\detokenize{module-polo.widgets}}
\item\relax\sphinxstyleindexentry{polo.widgets.file\_browser}\sphinxstyleindexpageref{polo.widgets:\detokenize{module-polo.widgets.file_browser}}
\item\relax\sphinxstyleindexentry{polo.widgets.map\_box}\sphinxstyleindexpageref{polo.widgets:\detokenize{module-polo.widgets.map_box}}
\item\relax\sphinxstyleindexentry{polo.widgets.optimize\_widget}\sphinxstyleindexpageref{polo.widgets:\detokenize{module-polo.widgets.optimize_widget}}
\item\relax\sphinxstyleindexentry{polo.widgets.plate\_inspector\_widget}\sphinxstyleindexpageref{polo.widgets:\detokenize{module-polo.widgets.plate_inspector_widget}}
\item\relax\sphinxstyleindexentry{polo.widgets.plate\_viewer}\sphinxstyleindexpageref{polo.widgets:\detokenize{module-polo.widgets.plate_viewer}}
\item\relax\sphinxstyleindexentry{polo.widgets.plate\_visualizer}\sphinxstyleindexpageref{polo.widgets:\detokenize{module-polo.widgets.plate_visualizer}}
\item\relax\sphinxstyleindexentry{polo.widgets.run\_organizer}\sphinxstyleindexpageref{polo.widgets:\detokenize{module-polo.widgets.run_organizer}}
\item\relax\sphinxstyleindexentry{polo.widgets.run\_tree}\sphinxstyleindexpageref{polo.widgets:\detokenize{module-polo.widgets.run_tree}}
\item\relax\sphinxstyleindexentry{polo.widgets.slideshow\_inspector}\sphinxstyleindexpageref{polo.widgets:\detokenize{module-polo.widgets.slideshow_inspector}}
\item\relax\sphinxstyleindexentry{polo.widgets.slideshow\_viewer}\sphinxstyleindexpageref{polo.widgets:\detokenize{module-polo.widgets.slideshow_viewer}}
\item\relax\sphinxstyleindexentry{polo.widgets.table\_inspector}\sphinxstyleindexpageref{polo.widgets:\detokenize{module-polo.widgets.table_inspector}}
\item\relax\sphinxstyleindexentry{polo.widgets.table\_viewer}\sphinxstyleindexpageref{polo.widgets:\detokenize{module-polo.widgets.table_viewer}}
\item\relax\sphinxstyleindexentry{polo.widgets.unit\_combo}\sphinxstyleindexpageref{polo.widgets:\detokenize{module-polo.widgets.unit_combo}}
\item\relax\sphinxstyleindexentry{polo.windows}\sphinxstyleindexpageref{polo.windows:\detokenize{module-polo.windows}}
\item\relax\sphinxstyleindexentry{polo.windows.ftp\_dialog}\sphinxstyleindexpageref{polo.windows:\detokenize{module-polo.windows.ftp_dialog}}
\item\relax\sphinxstyleindexentry{polo.windows.image\_pop\_dialog}\sphinxstyleindexpageref{polo.windows:\detokenize{module-polo.windows.image_pop_dialog}}
\item\relax\sphinxstyleindexentry{polo.windows.log\_dialog}\sphinxstyleindexpageref{polo.windows:\detokenize{module-polo.windows.log_dialog}}
\item\relax\sphinxstyleindexentry{polo.windows.main\_window}\sphinxstyleindexpageref{polo.windows:\detokenize{module-polo.windows.main_window}}
\item\relax\sphinxstyleindexentry{polo.windows.pptx\_dialog}\sphinxstyleindexpageref{polo.windows:\detokenize{module-polo.windows.pptx_dialog}}
\item\relax\sphinxstyleindexentry{polo.windows.run\_importer}\sphinxstyleindexpageref{polo.windows:\detokenize{module-polo.windows.run_importer}}
\item\relax\sphinxstyleindexentry{polo.windows.run\_updater\_dialog}\sphinxstyleindexpageref{polo.windows:\detokenize{module-polo.windows.run_updater_dialog}}
\item\relax\sphinxstyleindexentry{polo.windows.spectrum\_dialog}\sphinxstyleindexpageref{polo.windows:\detokenize{module-polo.windows.spectrum_dialog}}
\item\relax\sphinxstyleindexentry{polo.windows.time\_res\_dialog}\sphinxstyleindexpageref{polo.windows:\detokenize{module-polo.windows.time_res_dialog}}
\end{sphinxtheindex}

\renewcommand{\indexname}{Index}
\printindex
\end{document}